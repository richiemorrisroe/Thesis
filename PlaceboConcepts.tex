\subsection{What is a placebo?}
\label{sec:what-placebo}



Some researchers would not classify the Oken and Shiv studies as looking at placebo effects at all. It is arguable whether or not these effects of cognitive improvement can really be classified as placebo effects, as they could more properly be termed expectancy effects. 

Others would also argue that the same is true of all placebo effects, as they appear to be mediated by expectancies \cite{Kirsch1985, Kirsch1997,Montgomery1997}. Recent research does seem to show that expectancies do not mediate all placebo effects \cite{benedetti2003a}, so the distinction between these two terms is still useful.  The only difference between the effects observed in these experiments and those seen in more typical placebo experiments appears to be that the latter can be conceptualised as treatments, while the former cannot. 

\subsection{Use of the term placebo}
\label{sec:use-term-placebo}



This highlights a huge problem with the placebo terminology, it has broadened to mean any effect mediated by information and meaning rather than through biochemical mechanisms. An example of this is a recent study of exercise and the "placebo effect" \cite{Crum2007} (described in Section \ref{sec:plac-cogn-perf}). This fascinating experimental study which demonstrated that significant improvements in fitness could result from information provided about the health benefits of work is nonetheless symptomatic of an obscuring trend in social science research; that of treating any influence of mental information on the body as a placebo effect. 

This kind of broadening of the concept is  useless to continued progress is the field, as it obscures the very real and significant differences between these effects. Some can be attributed to the effects of expectancies, others to meaning, still others to subtle communications on the part of the experimenter or treatment provider, still others to biological mechanisms not well understood, and the lack of clarity in definitions and terminology is holding back research tremendously. One is almost tempted to ditch the concept entirely, as was proposed in a recent issue of the British Medical Journal \cite{nunn2009s}. 

For all the appeal of starting again with a new term, such an approach would have a number of problems. The first of these is  that the phrase, however poorly defined, conveys a meaning to researchers about how an effect occurred. 
The second problem is what to do with clinical trials. In this situation, the placebo is a necessary control. Does Nunn suggest that we use only equivalence or non-inferiority trials? Obviously this would not work for new treatments, and even in cases where there is an accepted treatment, the statistical requirements with regard to sample size and other matters would cause large problems with the proving of the efficacy of a treatment \cite{Benedetti2008}. This follows as larger sample sizes would be required to determine that two treatments are not significantly different from one another at conventional levels of statistical power. %insert benedetti book reference here

Another issue with this jettisoning of the placebo concept is what to replace it with? The literature we have has established that  treatments lacking a currently understood biological basis can have real effects \cite{Meissner2007}  in pain \cite{Vase2002} , in depression \cite{Kirsch2002a}  and in Parkinson's disease \cite{Benedetti2004a} among others. How are we to conceptualise these effects, if not in terms of placebo? Perhaps a better approach would be that of Gotzsche \cite{Gotzsche1995} where he suggested that placebo effects be broken down into those attributable to patient-provider interaction, those attributable to a stimulus from the outside world and those attributable to the patients belief in the treatment. Hrobjartsson also gave a similiar definition where he described three types of placebo:

\begin{quotation}
change after a placebo intervention, effect of a placebo intervention and the effect of the patient provider interaction. 
\end{quotation}

Of these three definitions, only the third is not tautological, and it has been covered when the Di Blasi definition was discussed above. 

Even then, these effects are difficult to disentangle. Consider the case of a researcher giving a participant a placebo as a pain killing treatment, after which the participant reports less pain. While the administration of the inert substance and the patients belief in it certainly had an impact, so too did the presence of the researcher, who is the provider in this situation. It is difficult to see how these effects can be disentangled without specialised designs requiring the same participants being administered placebos by multiple researchers.   



I would propose that the term placebo effect should be reserved for effects which exert a beneficial change in health due to effects of internal and/or external context. Other effects should be termed expectancy effects, or mind-body effects, or effects of therapeutic relationship (which have a long history in psychology, being typically known as experimenter effects) \cite{rosenthal1969interpersonal,rosenthal1967covert,Rosenthal1956}.  

\section{Debates about the Nature and Existence of Placebo Effects}
\label{sec:nature-existence}

The placebo is quite a mysterious phenomenon and does not fit very well into the biomedical paradigm \cite{Kaptchuk1998} \cite{Caspi2002}. In this school of thought, specific treatments exert action on specific parts of the body and cause relief from sickness. The huge effectiveness of quinine, vaccination and pasteurisation on the health of European societies took place without any regard to the mental state of the patients receiving the treatment, and this encouraged a model of the body as machine, where certain inputs led to certain outcomes \cite{Caspi2002}. 

It was this very commitment to evidence and specific remedies that has eventually led to the restoration of the place of the placebo effect in medicine. The first clinical trial was conducted in 1913, but the method was not widely adopted until after world war 2 \cite{Kaptchuk1998}. The adoption of randomised controlled trials led to on the one side, the denigration of the placebo as a mere experimental tool, and on the other side, this development created an environment where its effectiveness in a wide variety of situations could be recognised. 

The placebo had concurrently fallen out of official use  by doctors as it was regarded as a deception, although recent surveys in this age of far greater attention on ethical prescription show that many medical practitioners do not take this advice \cite{hrobjartsson2003use,sherman2008academic,Sherman2008b,Sherman2003,Ross1983,Buckalew1981,Krugman1964,Ross1962}. 

% The placebo languished in the bowels of the developing controlled trials until in 1955, when Beecher published The Powerful Placebo and introduced a new excitement into the study of the field. Beecher also introduced some unfortunate canards into the field, with his research noting that around 1/3rd of patients reported a placebo effect. This statistic has been carried from article to article for over fifty years now, despite evidence that it is not even remotely true \cite{Kienle1997}.

The placebo remained a topic of interest to some scientists and psychologists for the next few decades, but many argued that it could not be real on principle, and others that while the effects seemed real, they were due to demand characteristics of the experimental situation or response biases. % Some even invoked signal detection theory in order to explain the effects away \cite{Allan2002}. 
This is despite evidence from the early 1960's of placebo effects in other mammals \cite{Herrnstein1962}, which obviously could not be attributed to either of these experimental artefacts. Some argued that if the placebo did have an effect, then it could only be on subjective outcomes \cite{hrobjartsson2001}.

A new phase of placebo research began in the late 1970's with the demonstration that placebo analgesia could be mediated by naloxone \cite{Levine1978a,Levine1984,Fields1981,Gordon1981,Levine1979}. This research was quickly contextualised by evidence which showed that although some placebo analgesic responses were mediated by endogenous opiates, not all could be \cite{Gracely1983,Levine1984}. These experiments represented perhaps a tipping point in the study of placebo. No longer could these effects be dismissed as mere response biases, and researchers in the field finally had a physiological mechanism to point at to convince doubters that the placebo effect was a real phenomenon and was worthy of study. 

The nineteen eighties were a good decade for placebo research. The first placebo conditioning experiments were performed on humans \cite{Voudouris1985}, and a new theory of response expectancy was proposed to account for the effects \cite{Kirsch1985}. These two theories of conditioning and expectancies were tested against one another, and the controversy continues in the field today, despite what many regard as proof positive of the supremacy of expectancies over and above conditioning \cite{Montgomery1997}. More recent research (noted in the placebo analgesia section above) has contextualised the conditions under which placebo effects can be created by conditioning and expctancies \cite{benedetti2003a}. 

In 1998, an article was released that purported to show that anti-depressants were not much more effective than placebo \cite{Kirsch1998}. This research attracted a furore of publicity, and many newspapers argued in favour of ditching medications entirely and resorting to placebo. 

While this is a misinterpretation of the research findings (if the patients were given placebo as placebo it is unlikely they would have recovered), it gave new impetus to the field. This research was further backed up in 2002 \cite{Kirsch2002a}  with an analysis of all the information submitted to the FDA to prove the efficacy of Selective Serotonin Re-uptake Inhibitors (SSRI's) was shown to reveal little benefit for active drug over placebo. In the field of depression at least, it appeared that the placebo was indeed powerful. 

However, in 2001, the field was thrown into controversy. Two well known researchers conducted a meta-analysis on all clinical trials which included both a placebo and a no treatment condition \cite{hrobjartsson2001}  and concluded that there was no evidence of placebo effects on objective paramters and only a minor effect on subjective parameters. This research caused a furore in academic circles, and was widely reported by the same media who had given the placebo such a warm welcome some years earlier. The meta-analysis was widely criticised \cite{Evans2003,Kirsch2001, Wickramasekera2001,Greene2001}  on the grounds that it ignored psychological studies of placebo, considered too wide a variety of clinical conditions and did not properly ensure that the no treatment condition was indeed a no treatment condition. Nonetheless, this meta-analysis probably focused placebo research into the area of pain for many years afterwards. An update of the review in 2005 and in 2010 for the Cochrane Library revealed no essential changes in the findings of these authors \cite{Hrobjartsson2004}. 

Predictably, a storm of research findings and counter finding ensued. Vase and colleagues collected data for experimental placebo studies and argued that the null results in the clinical trials studied were due to the lack of suggestion in the clinical trial setting \cite{Vase2002}.  The original authors argued that this meta-analysis was sub-standard and when the same methods were used as in the original study, much smaller effects were found \cite{hrobjartsson2003}. 

The debate continued in the pages of the Journal of Clinical Psychology \cite{Wampold2005}  where the noted authors argued that placebo effects were revealed in the situations where they were expected, and not elsewhere. Again, the response of Hrobjarrtson et al was forceful as they claimed that this was a post hoc rationalisation of the results, and argued in favour of dispensing with the placebo concept outside of clinical trials \cite{Hrobjartsson2007a} . Wampold et al replied to this accusation \cite{Wampol2007}  with counter-evidence and Horbjarrtson et al countered that their original conclusions remained solid \cite{Hrobjartsson2007}. 

%this section needs more description

The debate rested there for a time, with those researchers working with placebos convinced that there were working with a real effect, and possessing enough neuro-biological and physical evidence that they were not concerned with the by now infamous meta-analysis while those who did not accept the placebo theories were able to point to a large study that confirmed what they believed to be true. However, more recent research has cast new light on this complex issue. 

In a recent meta-analysis  \cite{Meissner2007}, Meissner and colleagues reviewed clinical trials using both placebo and no treatment conditions, and made a surprising discovery. In trials where the outcome measure was a physical parameter, there were large placebo effects but in trials where the outcome measure was hormone levels, there were no placebo effects. They re-analysed the studies looked at by Hrobjarrtsson and Goetzche and found that when this classification was utlised, placebo effects were observed in the data. The implications of this line of research and other research is explicated in Chapter \ref{cha:notes-towards-theory}. 
 %% this section appears quite a number of times

The debate about the reality and effectiveness of placebos appears to have subsided somewhat, though there are many unresolved issues. While the Meissner et al meta-analysis appears to resolve the major point of contention, more research is definitely needed to ascertain exactly why the effect sizes for placebo responses vary so much as a function of type of study.

One factor that this definition excludes is the interior experience of the person who has the response. Context refers to external things or processes (at least according to the Oxford English Dictionary), and there is evidence that when participants attend to their own internal somatic sensations, placebo effects increase \cite{Geers2006}. Obviously, the participants somatic sensations are not part of the context surrounding them, and yet they appear to have an impact upon the response. 


One of the most interesting definitions of the placebo response is that developed by Daniel Moerman, an anthropologist with his conception of the ``meaning response'' \cite{Moerman2002b,Moerman2002a}. 
This definition is as follows: 

\begin{quotation}
  We define the meaning response as the physiologic or psychological
  effects of meaning in the origins or treatment of illness; meaning
  responses elicited after the use of inert or sham treatment can be
  called the "placebo effect" when they are desirable and the "nocebo
  effect" when they are undesirable
\end{quotation}

This definition is particularly interesting as it focuses on the perception of the treatment by the individual, which is a factor often neglected in many definitions of placebo.  This definition, while it seems to be among the best that we have, still displays far too much of a focus on treatment and medical conditions to account for placebo effects on cognitive function or exercise performance \cite{Crum2007} . 

 While this is understandable, given the roots of the placebo concept, it is far too restrictive given that we have evidence of placebos in contexts far removed from medicine, such as sport and cognitive function \cite{Benedetti2007a,Oken2008}. 

One problem with this definition rests on our understanding of the word "meaning". The OED defines this as \textit{"what is meant by a word, idea or action"}, which is not particularly useful in this context. What Moerman seems to understand by the word is the interpretation placed on the treatment by the individual when they receive it. This again supposes that people are aware of all of the meanings which they possess, when the research would seem to indicate that this is not always the case, demonstrated by (amongst other evidence) the Shiv et al (2005) study noted below  \cite{Shiv2005a}.  

The Moerman definition is useful in that it points to the individual and cultural interpretations of treatments which appear to facilitate a placebo response, it is still quite imprecisely specified and not particularly useful in practice. % (meaning is internal, and some eleme
nts of meaning may reside at a non-verbal, implicit level so there is great difficulty in assessing these meanings in order to devise more effective forms of treatment). 

Another interesting study which points towards the need for a more clearly defined placebo concept is that of Shiv et al \cite{Shiv2005a} , where the response to an energy drink was heavily affected by the price. 

When participants believed that the energy drink had been discounted, they solved significantly less puzzles than when the were given the drink at its full price. An important issue to note in the context of the paragraph above is that when participants attention was drawn to the discounting, the effect disappeared. This implies that although the signal of the reduced price affected their abilities, this meaning only activated when their attention was not focused on it (suggesting that they responded to the signal in an implicit fashion, somewhat like priming). 

Again, this study cannot be conceptualised as a treatment, as we are looking at an improvement in cognitive performance rather than the relief of some illness or problem. 

\section{Conclusions}
\label{sec:conclusions}

To conclude, the concept of the placebo is wide-ranging, and requires careful elucidation if research is not to be held back by mistaken assumptions surrounding the concept. Therefore, there is a need for this confusion to be resolved in one of two manners. Firstly, the definitions of placebo used could be tailored for the specific experiment or condition studied. 

This would have the advantage of being able to more precisely define the expected effects and outcomes for this experiment. However, it would also create a large number of defined ``placebo effects'' and this would be problematic for the researcher who wishes to study these effects across a broad range of conditions and treatments. That being said, some researchers in the field continually make the point that there are many placebo effects, not just one and such a strategy would have the benefit of making this extremely clear \cite{Benedetti2008} 

The second option would be to look for an all-encompassing definition, and use this for all research in the field. While this would make matters easier for researchers interested in the broad concept of the placebo, it would create difficulties when new forms of placebo effects come to light in the course of research. 

Such an all-encompassing definition would also lose much of the precision that allows for future avenues of research to be discerned from it, and prove much less useful to specialised researchers in the field. To summate, the best approach here may be to define the placebo as broadly as possible within the confines of effects on health and illness, and encourage researchers to specify exactly what they believe constitutes a placebo effect for the purposes of each study.



%%% Local Variables: 
%%% mode: latex
%%% TeX-master: "PlaceboMeasurementByMultipleMethods"
%%% End: 


