\subsection{What is a placebo?}
\label{sec:what-placebo}



Some researchers would not classify the Oken and Shiv studies as looking at placebo effects at all. It is arguable whether or not these effects of cognitive improvement can really be classified as placebo effects, as they could more properly be termed expectancy effects. 

Others would also argue that the same is true of all placebo effects, as they appear to be mediated by expectancies \cite{Kirsch1985, Kirsch1997,Montgomery1997}. Recent research does seem to show that expectancies do not mediate all placebo effects \cite{benedetti2003a}, so the distinction between these two terms is still useful.  The only difference between the effects observed in these experiments and those seen in more typical placebo experiments appears to be that the latter can be conceptualised as treatments, while the former cannot. 

\subsection{Use of the term placebo}
\label{sec:use-term-placebo}



This highlights a huge problem with the placebo terminology, it has broadened to mean any effect mediated by information and meaning rather than through biochemical mechanisms. An example of this is a recent study of exercise and the "placebo effect" \cite{Crum2007} (described in Section \ref{sec:plac-cogn-perf}). This fascinating experimental study which demonstrated that significant improvements in fitness could result from information provided about the health benefits of work is nonetheless symptomatic of an obscuring trend in social science research; that of treating any influence of mental information on the body as a placebo effect. 

This kind of broadening of the concept is  useless to continued progress is the field, as it obscures the very real and significant differences between these effects. Some can be attributed to the effects of expectancies, others to meaning, still others to subtle communications on the part of the experimenter or treatment provider, still others to biological mechanisms not well understood, and the lack of clarity in definitions and terminology is holding back research tremendously. One is almost tempted to ditch the concept entirely, as was proposed in a recent issue of the British Medical Journal \cite{nunn2009s}. 

For all the appeal of starting again with a new term, such an approach would have a number of problems. The first of these is  that the phrase, however poorly defined, conveys a meaning to researchers about how an effect occurred. 
The second problem is what to do with clinical trials. In this situation, the placebo is a necessary control. Does Nunn suggest that we use only equivalence or non-inferiority trials? Obviously this would not work for new treatments, and even in cases where there is an accepted treatment, the statistical requirements with regard to sample size and other matters would cause large problems with the proving of the efficacy of a treatment \cite{Benedetti2008}. This follows as larger sample sizes would be required to determine that two treatments are not significantly different from one another at conventional levels of statistical power. %insert benedetti book reference here

Another issue with this jettisoning of the placebo concept is what to replace it with? The literature we have has established that  treatments lacking a currently understood biological basis can have real effects \cite{Meissner2007}  in pain \cite{Vase2002} , in depression \cite{Kirsch2002a}  and in Parkinson's disease \cite{Benedetti2004a} among others. How are we to conceptualise these effects, if not in terms of placebo? Perhaps a better approach would be that of Gotzsche \cite{Gotzsche1995} where he suggested that placebo effects be broken down into those attributable to patient-provider interaction, those attributable to a stimulus from the outside world and those attributable to the patients belief in the treatment. Hrobjartsson also gave a similiar definition where he described three types of placebo:

\begin{quotation}
change after a placebo intervention, effect of a placebo intervention and the effect of the patient provider interaction. 
\end{quotation}

Of these three definitions, only the third is not tautological, and it has been covered when the Di Blasi definition was discussed above. 

Even then, these effects are difficult to disentangle. Consider the case of a researcher giving a participant a placebo as a pain killing treatment, after which the participant reports less pain. While the administration of the inert substance and the patients belief in it certainly had an impact, so too did the presence of the researcher, who is the provider in this situation. It is difficult to see how these effects can be disentangled without specialised designs requiring the same participants being administered placebos by multiple researchers.   



I would propose that the term placebo effect should be reserved for effects which exert a beneficial change in health due to effects of internal and/or external context. Other effects should be termed expectancy effects, or mind-body effects, or effects of therapeutic relationship (which have a long history in psychology, being typically known as experimenter effects) \cite{rosenthal1969interpersonal,rosenthal1967covert,Rosenthal1956}.  


%%% Local Variables: 
%%% mode: latex
%%% TeX-master: "PlaceboMeasurementByMultipleMethods"
%%% End: 

