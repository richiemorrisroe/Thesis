

\section{General Discussion}

The central aim of this thesis was to examine the measurement of
treatment expectancies in the context of the placebo effect. This
thesis used explicit (self-report), implicit and physiological
variables to deliver greater understanding of these psychological
constructs and their relationship with the placebo effect.


The thesis was structured into a number of parts, each of which will
be discussed in turn, setting the discussions at the end of each
chapter into a fuller context. Following this, the overall aims and
objectives of the thesis will be reviewed and set into context.
Finally, some avenues for future research will be described.

The first section of the thesis was the review of the literature
around the placebo effect and implicit measures, along with any known
individual-level predictors of the effect. This review found that
placebo effects have been seen to be mediated by optimism (as
operationalised by the Life Orientation Test, Revised (LOT-R)
\cite{Scheier1994}) \cite{Geers2005,morton2009reproducibility} in both
pain and non-pain paradigms. This review also linked the constructs of
optimism, which can be defined as ``generalised outcome expectancies
around the future'' \cite{Carver2010}, with the construct of response
expectancies of Kirsch \cite{Kirsch1985,Kirsch1997}, which are
described as the ``expectation of a non-volitional response''. These
constructs both crucially rely on the participants' perception of the
likelihood of the event, and a secondary aim of the thesis developed
which was to examine the usefulness of these as two separate rather
than one combined predictor. The hypothesis was that these two
constructs should be significantly correlated, and that one of them
would mediate the impact of the other on the observed placebo response
\cite{Geers2005}.

Another important potential moderator variable came to light in the
literature review, which was the construct of mindfulness, as
operationalised by the Mindful Attention Absorption Scale (MAAS)
\cite{brown2003benefits}. This variable was found to moderate the
correlations between explicit and implicit measures in an experience
sampling study. Furthermore, implicit measures were found to be better
predictors of spontaneous responses, while explicit measures were
found to be better predictors of deliberate behaviour
\cite{Levesque2007}, a pattern known as the double-dissociation effect
and observed in IAT studies in a variety of domains
\cite{Asendorpf2002,Perugini2005,Grumm2007,Steffens2006}.

This research literature suggested that a good strategy would be to
replicate the LOT-R and MAAS on a number of samples from which the
eventual experimental sample would be drawn, both to develop better
models for these constructs' relationship (as they had not been
concurrently examined when this work was carried out) and to tailor
the measures towards the experimental sample, a strategy which was
hoped to increase the precision of estimation on the overall sample.
In this research, the RAND-MOS was used both to replicate the impact
of these primary constructs on health, and to provide a weak link to
health and treatment expectancies in preparation for the experimental
research.

One large problem with the development of an implicit measure of
treatment expectancies was that there existed no gold-standard measure
of explicit treatment expectancies, despite them having been shown to
materially affect the outcome of clinical trials
\cite{Linde2007,Bausell2005,Benedetti2005}. This was an issue both
because typical methods of IAT development rely on such a self-report
measure, and because without such a measure, the incremental
improvement (if any) granted by an explicit measure would not be
clear. Therefore, the development and testing of such a measure formed
part of the work of this thesis.

Another fact which became extremely clear during the course of the
literature review, is that current methods of both developing and
validating implicit measures (or more specifically, IATs), rely
extremely heavily on the existence of prior self-report instruments.
For the placebo effect which was the focus of primary investigation,
no such self-report measure existed, which meant that a new approach
needed to be tested. A review of methods in this space revealed the
work of George Kelly on personal construct theory \cite{Kelly1991},
and more specifically the repertory gird introduced in Book 1 of the
aforementioned reference. It was believed that such an approach might
prove useful in the current case, and might also prove useful with
other constructs of interest in the future\footnote{I am indebted to
Drs Jurek Kirakowski and Sean Hammond for originally suggesting this
idea.}.

Therefore, in order to develop these stimuli, it was decided to do
some qualitative research around health constructs with doctors,
alternative therapists and students, to gain an appreciation for how
they contextualised and understood issues around health, medicine and
sickness. The results of this procedure, along with a parallel
exercise asking participants to rate the five most important people or
qualities of people related to health, would then be used to develop
the Health Repertory Grid.

Finally, the development of the experiment was planned, and it was
decided to use a placebo analgesia design, inducing pain with the
sub-maximal torniquet technique \cite{moore1979submaximal}, as this
form of pain induction has been shown in a meta-analysis to produce
the largest effect sizes \cite{Sauro2005}.

Below, the results and findings from each of these sections are
described and placed in a context of larger research.

\section{Health, Optimism and Mindfulness}
\label{sec:health-optim-mindf}

This study consisted of approximately fifteen hundred ($N \approx
1500$) observations, of which 1100 were collected in an online
fashion, and four hundred were collected in pen and paper format.

The major aims of this part of the research was to collect useful
background data for the experimental sample, enabling psychometric
models to be built on large samples, which could then be applied to
the experimtal sample, as well as developing psychometric models which
could tease out the relationships between the health-related
constructs (RAND MOS), optimism (LOT-R) and mindfulness (MAAS). The
headline finding of this research, which was replicated across both
samples (and a further one which was not analysed for this thesis) was
that optimism was negatively related to both health and mindfulness, a
finding which has not been reported before.

This finding can be better understood in the light of some of the SEM
results on the same dataset. In these, it was determined that
emotional well-being moderated the impact of this suprising effect.


Additionally, this strange phenomenon replicated across both original
studies, and also in the pilot experiment. However, it did not
replicate in the main experimental study, raising the issue of what
the critical differences between the samples were. Upon examination of
the materials, the only difference between the setup for the
experimental study was the order in which the measures were
administered. In all the previous studies, the MAAS was filled out
before the LOT-R, while in the experimental study, the order was
opposite.

This finding links in with the other finding that emotional well-being
(a subscale of the RAND-MOS) moderated the effect of optimism on
health. To see this, merely look at the questions which make up the
Emotional Well-Being scale, and note its similarity to the LOT-R. It
appears that what occurred is that participants who would typically
respond highly to optimism (as evidenced by their Emotional Well-Being
score) had their response patterns altered by the act of completing
the MAAS. This may have occurred because the MAAS operationalises
mindfulness in terms of negative mindlessness. This may have caused
the participants high in Optimism to have realised that some of their
optimism was unjustified, which therefore depressed their responses on
the LOT-R, leading to the unexpected negative correlation with health.
However, this is somewhat circumstantial given that this hypothesis
was not directly tested in this research, but it was supported by the
SEM carried out in Chapter \ref{cha:health-for-thesis} which pointed
towards mindfulness and emotional well being as mediating the strange
optimism-health relationship.

Additionally, the cross-validation approaches employed in this section
of the research showed an interesting pattern. The factor models which
were developed using the reduced scales from IRT analyses showed
better predictive ability on unseen data than did those which were
developed from factor analysis alone. This suggests that an
interesting approach to the development of scales would be to use IRT
(specifically Mokken analyses) to reduce the number of items, and use
this reduced scale in SEM to determine if this effect is widespread or
merely circumstantial to the samples collected here.

\section{Explicit Expectancy Measure}
\label{sec:expl-expect-meas}

The next step in the research was the development of an explicit
expectancy measure, which would capture the multi-dimensional nature
of treatment expectancies more fully \cite{Stone2005}. Again, this
measure was developed across two samples, a small first sample which
aimed to validate the measure in terms of reliability, and a second
larger study which incorporated the results of the first study and
utilised the Beliefs About Medicine Questionnaire (BMQ)
\cite{Horne1999} to provide evidence of convergent validity for the
measure.

The results of the first study showed that the measure had extremely
good split-half reliability ($ \alpha=0.9$). Another finding was that
the first instrument broke down in factor analytic terms in terms of
Conventional (Pills, Creams, Injections) and Alternative (Acupuncture)
treatments. This was obvious both from the intercorrelations of each
of the questions, and from the results of factor analytic and IRT
modelling. This suggested that two other alternative treatments should
be added to the questionnaire, to determine if this was a valid
structure and to provide balance.

The second study showed the expected correlations with the BMQ measure
(positive for alternative treatments, negative for conventional
treatments) which suggested that the measure did have convergent
validity. Additionally, the notion of conventional and alternative
treatments being separate factors was replicated across both samples,
and using factor analytic and IRT methods. Again, the backtesting
strategy employed in the previous chapter, using a combination of IRT
and FA approaches, performed better than either of these approaches
alone, suggesting that this may be a useful methodology in general.

%%insert some interesting SEM results here, when available.

\section{Repertory Grids and Experimental Pilot}
\label{sec:qual-rese-rep}

The final portion of research conducted before the main experimental
approach was the development of the repertory grid and the development
of the IAT measures and their piloting, along with the piloting of the
experimental procedure.

% From the coding and analysis of the interviews, it was determined
that the major contrast between styles of healthcare (conventional and
alternative) was that the alternative therapists regarded themselves
as treating a person, whereas the GP's focused more on the symptoms
which this person presented with. Additionally, the GP's seemed to
focus more on external causes (pathogens, environment) whereas the
alternative therapist focused more on internal causes (body-mind
linkages, thought patterns). Note that both of the groups focused on
the nature of health as both a social and personal responsibility, but
that they disagreed on where the responsibility for changes lay. The
students in the sample provided either a moderately alternative
viewpoint, or a moderately conventional viewpoint, but in no case were
they as extreme as any of the practitioners. The major differential
theme appeared to be that of conventional versus alternative, and the
truth or falsity of health claims, and this was incorporated into the
development of the Health Repertory Grid (and later into the design of
the Treatment Credibility IAT).

The next step was the development of the Health Repertory Grid. This
was developed analogously to the original repertory grid, using health
related constructs taken both from the interviews and from the
questionnaire asking people to name the five most important
health-related people or qualities in their life. Unfortunately, it
became apparent in the course of piloting this instrument that
participants did not have enough examples of these constructs to
enable them to complete this task. Therefore, the treatment
credibility IAT was developed using the notion of true versus false
and conventional treatments versus alternative treatments, while the
optimism IAT was developed from the LOT-R, in line with current
practice in the field.

Next, the two implicit measures were administered to a small sample of
volunteers, and showed the expected correlations (small, but in the
right direction) with the explicit measures of the same constructs.

Additionally, when the experiment was piloted, there was a small
effect of the IAT's on the probability of placebo response in a
logistic regression. This pilot study also showed that the two IAT's
correlated quite well with the respective explicit measures, and that
a placebo response could be induced with the procedure. However, given
that a majority of the participants did not respond to placebo, it was
decided to use a priming procedure to increase the probability of
placebo response~\cite{Geers2005a}.

\section{Experimental Research}
\label{sec:exper-rese}

The final part of the thesis was the experimental research. One
hundred and eleven (N=111) participants were assigned to either a
Deceptive (told drug, got placebo), Open (told placebo, got placebo)
or No Treatment Condition. Additionally, participants were randomised
to either a Prime or No Prime condition. % The major finding in terms
of condition was that those participants who were in the Open
condition showed a greater placebo response than those who were in the
Deceptive Condition. This finding was extremely unexpected, and has
not been seen in the literature before. However, given that the
suggestions focused on ``clinically proven'' (in the Open condition)
and ``recently approved'' in the Deceptive condition, this may have
had an impact. There was a significant Prime by Condition interaction,
suggesting that priming was differentially impacted by by the
condition (Open Placebo participants showed a greater propensity to
respond to suggestion after being primed). This would seem to have
potentially large implications for medical practice, if replicated.

With respect to the major hypothesis of the thesis, no direct impact
of the explicit or implicit measures was shown to be significant in a
logistic regression, using a step-wise approach with a training and
test set to ensure that the estimated p-values were unbiased. Note
that essentially, while this study provides some preliminary evidence
for an impact of the TCQ IAT on placebo response, this effect was not
significant on the test set (though the z-value was quite large), and
so further research would need to be a larger study to confirm or
refute this effect.

% However, a three way interaction was shown to be significant,
however, this % % model suffered from issues of multi-collinearity and
so should not be taken as accurate without replication. Attempts were
made to utilise machine learning techniques for the prediction of
placebo response, and while some of these techniques were extremely
effective at predicting the lack of a placebo response, none performed
at above chance level at predicting positive placebo response.
However, it would be useful in terms of the design of clinical trials
if participants who would reliably respond to placebo could be
eliminated, and so this measure might prove useful if replicated.

Finally, the relationship between skin conductance and pain ratings
was examined. This analysis showed that for the two treatment
conditions, entirely different GSR responses were shown. This would
seem to indicate that the difference between certain and uncertain
expectancies can be examined using simple GSR equipment as opposed to
expensive fMRI equipment.

In conclusion, there was no significant relationship shown between
implicit measures and placebo in standard analyses, suggesting that
any effect, if it exists, is likely to be quite small.

\section{Methodological Contributions}
\label{sec:meth-contr}

Another contribution of this thesis was in the general methodological approach taken. 
As described throughout the thesis, a number of methods were employed to develop the instruments 
in this thesis which are not commonly used in psychological research.

The first contribution of this part was the collection of survey samples of all the explicit
measures used in the experiment from the same population as the experimental participants were sampled from. This is novel in the literature, and provides a useful check on the generalisability of the measures to this population, and insight into how the experimental sample differed from the survey samples. 

Another contribution was the pervavise use of cross-validation throughout both the psychometric analyses and the regression modelling. This is a better approach in terms of minimising the number of false decisions made as a result of over-fitting, and its use throughout this work has ensured that all models were tested on unseen data, so it is somewhat more certain that they reflect a real process of interest rather than sample variability. 

These two contributions are important in that this thesis can provide a case study for future work which can reuse these tools. 
\section{Conclusions and Further Research}

This thesis set out to achieve the following:

\begin{itemize}
\item To develop and test an implicit measure(s) to predict the
placebo response;
\item To develop and test a more substantial explicit expectancy
measure for use in future placebo-related research;
\item To test a number of theoretical models around the relationships
between explicit, physiological and implicit measures and the response
to placebo;
\item To apply better models and methods to the experimental and survey data-sets used.
\end{itemize}

At this point of the thesis, it remains to assess to what extent these
goals have been achieved. Briefly, there was weak to some evidence for
a potential effect of implicit measures on the response to placebo.
This should be tempered with the fact that a direct significant effect
only occurred in a training sample, and did not replicate on unseen
data. Additionally, the impact was only significant in the presence of
interactions, and may as such represent random sample variability.

Next, the explicit measure developed in Chapter~\ref{cha:tcq-thesis}
and used in Chapters~\ref{cha:devel-impl-meas}
and~\ref{cha:primary-research} appeared to possess both face and
construct validity. However, in the experimental portion of the
research it did not appear to possess any predictive validity, which
would seem to suggest that this measure is not particularly useful in
the prediction of placebo response in healthy volunteers.


However, this measure could still prove useful in assessing
treatment-related expectancies in general. The instrument has been
revised in this thesis to have a common scale for all questions, and
additionally has been subjected to rigorous psychometric validation.
This should allow the measure to be re-used by researchers interested
in the manner in which treatment-related expectancies are
conceptualised by participants from the general population. It may be
that the measure itself would prove useful if the size of placebo
effects had been larger in the experimental study, and this is
something that could be tested by future research.

In terms of the testing of theoretical models, a model including both
implicit and explicit measures provided a better fit to the data than
did either alone. Importantly from a theoretical perspective, a model
with a generalised expectancy factor (from Kirsch) provided the best
fit to the data. This should be contextualised with the finding that
the LOT-R interacted with the TCQ IAT to achieve significance on the
training set.

An interesting and unexpected finding from this research was that the
GSR of participants appeared to be affected by the Condition in which
they were in. In fact, the cross-correlations between the pain ratings
and GSR were differentially directional between the Deceptive
(positive) and Open (negative) placebo conditions. This would seem to
suggest that certain versus uncertain expectancies can impact
physiological variables differently. This is an interesting finding,
and one which has not been reported before.

So, in conclusion, this thesis has provided a comprehensive test of
the predictive ability of implicit measures (IAT's) and a new
treatment credibility self-report item. On balance, neither of these
measures provided useful predictive power in a large study of placebo
analgesia in healthy volunteers.

While the major hypothesis of this study was not demonstrated to be
true, some other contributions to the literature have been made
\begin{itemize}
\item Priming exerts significant effects on the response to placebo;
\item GSR appears to correlate appropriately with response to placebo;
\item Models using both IRT and FA approaches perform better than
models trained with either method alone;
\item The relationship between optimism and mindfulness appears to
vary based on the order in which each of the instruments is delivered.
\item An approach using cross-validation and replication provides better evidence for the models tested
\end{itemize}

%%% Local Variables: %%% TeX-master:
"PlaceboMeasurementByMultipleMethods" %%% End: