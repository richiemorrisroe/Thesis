% \subsubsection{Frequency of administation}
% \label{sec:freq-admin}

% In addition to the factors noted above, it does appear that the frequency of placebo administration can affect the response to placebo treatment \cite{Moerman2000}. This may occur due to the ritualistic nature of taking pills at particular times of the day, and indeed greater adherence to a regimen of placebos has been found to decrease mortality \cite{Chewning2006b}. 

% \subsection{Contextual Factors}
% \label{sec:contextual-factors}



% \subsection{Spatio-temporal factors}
% \label{sec:geogr-diff-plac}

% One fascinating result from a meta-analysis was found by Garud et al \cite{Garud2008}. This meta-analysis looked at the placebo response in ulcerative colitis and  found that  that there were significant differences between the same placebos based on the geographical location of the trials. 

% In the USA, placebo response rates were 10\% lower than in Europe, suggesting that some cultural force might be driving this difference. Its hard to see what this cultural difference could be though, given the substantial heterogenity which exists between countries in Europe, which is far more than the comparable differences between states in the USA. 

% Nonetheless, this meta-analysis points toward some kind of cultural or geographic factor which can influence the placebo response, though a more nuanced and systematic explanation is lacking at present. 

% This would link in with earlier work carried out by Moerman into variability of placebo response rates across countries and cultures \cite{Moerman2000}, where differences in ulcer rates and blood pressure across cultures were noted. While this original meta-analysis by Moerman was criticised for a lack of rigour, another analysis by  De Craen which did not suffer from these problems \cite{Craen1999a} confirmed these results. 

% Additionally, some research shows that placebo response rates in clinical trials have been increasing over time \cite{Enck2005a}. This paper suggested that rates of response to placebo may have increased by over 20\% in this time period. To some extent, this may be attributable to laxer criteria for entry into clinical trials, but the same phenomenon has been noted in schizophrenia \cite{Ravi2008}, wehre the same issue of expanding amounts of clinical trials necessitating looser criteria for entry does not apply. While these geographical and temporal factors should not be over-interpreted, it does suggest that the placebo response is extremely sensitive to small differences in the context in which it is administered. 

% \subsection{Proportion of participants assigned to placebo}
% \label{sec:effects-clin-trial}

%  As the placebo response is commonly regarded as a nuisance in clinical trials, and the most recent medical guidelines suggest that placebo controlled trials should only be used when there is no proven alternative \cite{temple2000placebo}. This often makes ethical review committees reluctant to allow placebo conditions, and if they are allowed, the placebo group is often very small, in order to minimise the risk. 

% A recent meta-regression suggests that this may be counterproductive \cite{Papakostas2009}.  This research found that as the size of the placebo group decreased, the size of the placebo response often increased, in some cases meaning that the trials could not show an advantage of drugs over placebo. The authors speculate that, as the participants were informed of the drug:placebo ratio as part of informed consent procedures, they had stronger expectancies on the likely results of treatment and therefore they reported a larger response to treatment, which caused the placebo response rates to increase. This situation is an excellent example of the law of unintended consequences, and is an interesting object of study in its own right. 

% One finding which provides cause of caution is that effect sizes for placebo tend to correlate with the size of the trial, suggesting that regression to the mean may be responsible for some of the effects observed in clinical trials.\cite{Enck2005a}. To a certain extent, given the nature of clinical trials this is unavoidable as the selection criteria will tend to select groups of participants most likely to demonstrate this phenomenon. Unfortunately, though no-treatment groups provide an excellent bulwark against this confounder, most clinical trials do not have them. 

% \subsubsection{Name of Placebo}
% \label{sec:name-placebo}

% Another factor which affects the response to placebo is the name of the placebo. A recent study found that placebo responses to a placebo of the same name remained almost entirely constant, while the same inert cream given a different name evoked different responses from the same group of participants \cite{Whalley2008}. The authors use this finding to argue that this demonstrates that the placebo effect is completely inconsistent. This is not a particularly strong argument, as the name given to a placebo is one of the most important features, as it is one of the few pieces of information given to the participants in a typical experimental study. 

% \subsubsection{Gender}
% \label{sec:gender}

% Gender appears to be an important factor in placebo effects, with differing results being noted depending on the gender interactions between experimenters and participants. A good example of this is the Oken \cite{Oken2008} study which looked at the  effects of placebo pills on cognitive functions in older adults. This study had female experimenters, and a large placebo effect was shown for male participants but not for female participants. Flaten and colleagues \cite{Flaten2006}  also demonstrated effects of gender on placebo response. In this study with female experimenters, males did not show a placebo response, but females did. The authors explained this in terms of males being less willing to admit that they were in pain to female experimenters.

%  Another study \cite{Zubieta2006} showed a enhancement of dopamine production in males following placebo administration but not in females \footnote{this study was fMRI based however, and as such had a very small sample}. The authors note that this may be because of physiological differences, but gender was not ruled out as a cause of this effect during this study. Another example of gender influences on placebo treatments was observed in Milling \cite{Milling2007}  which examined the effectiveness of hypnotic, CBT and placebo treatments for pain. They observed that there was a significant effect of gender, but fail to note the gender of the experimenters, which renders the effects of any interaction difficult to interpret. A final study showed that men, but not women responded to a glucose administered placebo \cite{Haltia2008}. These last two studies do not provide enough information to conclusively examine the effect of gender on placebo response, as there could have been exogenous variables which actually drove the observed effects. 

% In conclusion, there appears to be some evidence that interactions between the gender of the experimenter and the gender of the patient/participant may affect the response to placebo. Nonetheless, the research is quite tentative and there does not appear to be a consistent, replicable effect (like much else within the study of placebo). 




% \subsubsection{Psychopathology and catatrophising}
% \label{sec:psych-catatr}

% Some other personality characteristics have been linked to placebo response and also to active treatment response. Firstly, a recent controlled trial \cite{Wasan2005}  showed that participants with higher levels of psychopathology (as measured using self report scales) derived significantly less benefit from analgesic treatment, but significantly more benefit from placebo analgesia treatment. Interestingly, levels of optimism also correlate with psychopathology \cite{Carver2010}, which may be a potential explanation for this interesting finding.  

% A second finding in this area \cite{Sullivan2008}  showed that in a trial of amyltriptine and ketamine versus placebo, the extent to which participants catastrophized about pain determined their treatment response. High catastrophisers reported a large effect from placebo, but low effects from the active treatment while for those low in catastrophizing, the results were the opposite. It is worth mentioning here that this Sullivan trial was a secondary analysis of a null result, so some cautions should be taken in its interpretation. No such caveats apply to the Wasan \textit{et al} trial. 



% However, more recent research has failed to replicate this effect in a sample of patients with pain problems \cite{Knipschild2005}. This newer study worked with general practitioners in the Netherlands, and although a large sample size was used, they failed to find any significant effects based on the positive consultation. 

% A few caveats apply here. Firstly, while the Thomas study had just one doctor involved, there were over 40 in the Knipschild study. Secondly, Thomas dealt with many different kinds of patients while Knipschild dealt only with pain patients. Thirdly, the Knipschild study used normal general practice clinics while a student sample was used in the Thomas study. Knipschild and Arntz suggest that the charisma\footnote{which could perhaps be conceptualised as being particularly effective at suggestion} of Thomas may have had something to do with his effectiveness. 

% Knipschild \textit{et al} also note that many of the GP's did not like giving negative consultations and the tape recorded interviews suggested that they were far more comfortable dispensing the clear advice.  Although there is substantial evidence for the impact of good patient provider relationships in the literature~\cite{blasi2001influence}  the specific matter of whether or not a positive consultation improves medical outcomes must be regarded as open at present. The Di Blasi systematic review \cite{Blasi2001} looked at the impact of cognitive care and emotional care, and argued that these two features drive much of the patient provider effects on health outcomes. 

% A recent study \cite{Kaptchuk2008}  using an RCT design with patients suffering from Irritable Bowel Syndrome (IBS) appears to indicate that interaction between patient and provider is a critical part of the placebo response. This study utilised sham acupuncture and divided participants into three groups. One received no treatment, another had sham acupuncture with minimal interaction while the third group had sham acupuncture and large amounts of interaction. The results showed that Group 3 had much better recovery rates then either of the other two groups, which would seem to suggest that the patient-provider interaction drives much of the observed placebo response, at least in this setting.  

% In addition, the documented success of treatments in medicine which have later been shown to be no better than a control treatment definitely involves the placebo effect. This finding comes from Roberts (1993), and is reported in \cite{Moerman2000a}.  The most famous example of this is surgery for angina pectoris carried out in the 1950's which was shown to be completely ineffective in blinded trials. A more recent example was a study conducted on osteoarthritis of the knee, where there was no significant effect of the surgery \cite{horng2002placebo}. Nonetheless, this treatment was found effective by many patients before this trial, suggesting that there are significant effects deriving from the provider interaction with the patient.  

% Moerman argues that the reason that these treatments were effective is that physicians believed in them, and they communicated this belief to their patients. 
% This belief (either directly or indirectly) caused the patients to respond well to the treatment \cite{Moerman2000}. Some authors have suggested that these patient-provider effects are the result of neural patterns laid down by caregivers in early childhood \cite{Kradin2004}.

% More broadly, patient-provider effects on health outcomes can be conceptualised as a form of experimenter effect, whereby the provider exerts an influence on the results. This is the positive side of demand characteristics where the patient becomes aware of the beliefs of the provider, and responds to these. An extremely good study of these effects was carried out by Walach \textit{et al} \cite{Walach2002}. In this study, two students were recruited as experimenters, and induced to have either positive or negative beliefs about the efficiacy of placebo. These experimenters then carried out the same experiment, and achieved results in line with the induced expectancies. 


% \subsubsection{Type of Placebo}
% \label{sec:type-placebo}

% This section will examine the effects of different kinds of placebo on their effectiveness. A recent study in this area looked at the differential effects of two different kinds of placebo therapy \cite{Kaptchuk2006}. In this study two different kinds of placebo (a pill and a medical device) were used, and showed differential outcomes on the recovery of the patients involved in the trial. 

% These kinds of findings are one of the best proofs that placebos actually produce measurable effects, as if they do not, then the effects of two placebo therapies should not differ significantly from one another \cite{Kaptchuk2006}. In the Kaptchuk et al study, it was shown that a sham device for acupuncture produced much greater placebo responses than an inert pill. 

% Physical placebos such as ultrasound have also been shown to have an larger average placebo effect size \cite{Ernst1995b}.  The reasons for these differences are not clear at present% , but some thoughts on this matter are presented in Chapter \ref{cha:methodology}. 

% One interesting finding from a meta-analytic study comes from the work of De Craen \cite{Craen2000}. This study showed that injection placebos were more effective than pill placebos in the treatment of migraine. This could be the result of injections being typically associated with stronger painkillers while pills are often sold over the counter, thus leading to stronger response expectancies regarding the injection. This finding can also be explained by the effects of prior experience and therefore also compatible with a conditioned model of the placebo effect. This appears to relate to the findings noted above about sham devices and physical devices being more effective than inert pills. 


%%% Local Variables: 
%%% mode: latex
%%% TeX-master: "PlaceboMeasurementByMultipleMethods"
%%% End: 
