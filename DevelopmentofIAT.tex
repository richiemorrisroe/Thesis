
\section{Analysis of Health Construct Interviews}

\subsection{Methodology}

Eight interviews were conducted in a semi-strctured format. The participants in the interviews included 3 alternative/complementary therapists, 2 GP's and 3 ordinary people with experience of healthcare, both western and complementary.

The questions which were asked were as follows:
\begin{enumerate}
\item What does the term health mean to you? 
\item Have you had any health problems in the past that you would like to discuss?
\item How do you believe that health can be promoted and sustained?
\item What kinds of treatments do you feel are effective? 
\item How are they effective?
\item What has been your experience with the official medical sector?
\item What has been your experience with the complementary medicine sector?
\item When you are sick, what types of help or treatment do you find most useful?
\item What do you think are the major causes of sickness (for you or others)?
\item What people or qualities do you associate with health?
\end{enumerate}


The Interviews were conducted between October 2009 and January 2010. 

\subsection{Analysis}

The interviews were first coded using an inductive coding procedure. In this, the codes were developed from the interview transcripts themselves, allowing the participants to define the nature of the analysis (in some sense). Following completion of the interviews and first coding, the interviews were then recoded using the entire database of codes which had been developed throughout the analysis procedure.
During this phase redundant codes were also removed. Following the coding procedure, codes weree combined into thematic units where appropriate. For example, the many codes relating to health were combined into an overall health code.

This analysis will look at the material collected and analysed in the following ways:

\begin{enumerate}
\item Analysis on the level of the group – GP's, ordinary people, and alternative therapists. All of these participants seemed to have different perspectives on the matters concerned, and this section will highlight the commonalities and differences between them
\item  Analysis of coding structures, across groups
\item Analysis of themes across the entire sample 
\end{enumerate}

\subsection{Overall theme}

In these interviews, a number of important themes emerged from the data. These are noted here and further developed throughout the test. These major themes were as follows:
\begin{enumerate}
\item Health – internal or externally determined. This theme seemed to stratify respondents into groups. Some respondents (alternative therapists mostly) seemed to look at the determinants of health as being mostly cognitive and emotional, while others (GP's mostly) seemed to look at more external determinants of health, such as pathogens and social status. The ordinary people interviewed in this research project tended to incorporate elements from both of these perspectives.
\item Doctors as prescribers, alternative therapists as facilitators – the alternative therapists tend to talk about facilitating people towards health, while the doctors tend to talk about curing people. 
\item Energy versus biology – doctors tend to talk about biology, alternative therapists tend to talk about energy. 

\end{enumerate}



\subsection{Group Analysis}

\subsubsection{Alternative/Complementary Practitioners}

\paragraph{Participant Information}

Three participants were interviewed as part of this group. One was a shaitsu and Reiki practitioner, one was a spiritual healer and the third was an acupuncturist. They were all recruited from the Cork area following email contact through an alternative therapy website. 

\subsection{Major Themes}
\begin{enumerate}
\item Health is caused by internal rather than external factors
\item Energy model of health – reference to vital forces, meridians and s on
\item They tend to talk about being facilitators rather than healers. They put more of the responsibility on the patient (self-healing model)
\end{enumerate}

\subsection{Coding Analysis and Examples}

\subsubsection{Food as medicine}

This was the most widely utilised code in the entire data-set, appearing over 17 times. This code only appeared in the transcripts of the ordinary people and the alternative therapists, which is why it appears here. 

An example of this code is below:

``in China, they have even the most common person on the street he's a peasant working in the fields are born and and have and inherent knowledge of how to eat in accordance with nature right in their body for a cold condition they know to eat something warm corn soup with ginger or rice casserole, meat and if they've a very very hot condition they know to eat know to eat, you know, cool foods and its very much putting nature into the body looking at external and balancing between the two like I was mentioning a while ago- Alternative Therapist No. 3''

As we can see in from this code, the conception (shared by folk healers across the world) regarding hot and cold foods exists here in his description of health. We can also see how he constructs this as a complement to Western styles of medicine, that it is common knowledge in other, non-traditional societies (where his method is drawn from) but is ignored in our prevalent system of medicine here in the west. 

Another example (from the same participant) follows below:

``I personally do like herbal medicine and I think that its its as close to food as you can get a herb generally grows either above or below ground the roots are usually stronger versions of it so just knowing how herbs work I will use preventative medicines and supplements to protect say if I find working too late head full of information and details, just go for a walk and just try and clear''

Here the participant is talking about what he does when he is sick, and we see how he legitimises herbal medicines by linking them with food, which he has previously described as the best method for promotion of health. 

``good food – healthy food organic food''. - Alternative Therapist No. 1
We can see here (this participant tended to give very short answers to questions) that the emphasis on food is common to the alternative therapists here. We can also note the linkage of health food, organic food as possibly indicating some disquiet with the technocractic nature of herbal medicine. 

\subsection{Linking Body and Mind}

Again, this code only appears in the transcripts of alternative practitioners and students, and is conspicuous by its absence throughout the transcripts of the General Practitioners. 

Some examples of this code follow below. 

`` like I would press certain acupuncture points, work certain energy channels...
I: mmmm
P: Through- which are directly applied on to the body so the focus of the person goes to those places – Alternative Therapist 1''

As one can see from this quote, this participant links physical manipulations of the body as having correlations to mental and emotional states. This is interesting as it is, in one sense, an extension of the idea of a healthy mind in a healthy body. 

``Yeah. Well, in in in again with every organ in the body there's there's associated emotions and there's pluses and minuses with that. So for example, we take an organ like the liver the liver ahhhh If we're totally overworked and ammm ehhh the liver can become very excitable and the energy of the liver it it it its a huge organ its it stores almost all of our blood – Alternative Therapist 3''

From this quote above, we can see a similar process to that observed from the first quote under this code heading. For this participant, emotional awareness can be correlated with the state of particular physical organs. Its interesting in that this is something which would probably never be done in psychology, as emotions would be regarded as being neural events rather than being distributed throughout the body. 

``Same thing with the heart If the heart is healthy, you experience joy 
I:Mmmmm
P:We find things funny, we like people, you know we think the best of every single person we meet 
I:Mmmm
P:When the heart is downtrodden and its there's so many expressions – my heart was broken or my heart sank when I heard that news so (pause) even in the lairs of that – Alternative Therapist 3''

Again, we see a cognate description here, where emotions are embodied in the organs and orificies of the body, rather than dissociated from these in the brain/mind. Its perhaps a radical dissolution of the traditional dualism that pervades Western european thinking. For this respondent the locus of emotion is embodied in the very physical structures that make up the body, rather than being something that sits on top of it, in the psychological or medical conceptions of emotion. 

\subsection{Health as balance}

Again, this code occurred quite frequently, but only in the transcripts of the alternative therapists. Some examples and discussions follow below. 

``So these are the processes in the mind, so again, health in respect to that, to emotions, mental energy very much keeping on a balanced level – Alternative Therapist 3''

We see here that the participants believes balance to be key to all health, on an emotional, physical and mental level. We can also note the focus on energy, which will be examined further below. We can also see here a focus on processes, that these elements are constrantly moving, which is borne out in the quote below. 

``The same thing with the emotions, keeping them moving, keeping them flowing, ammmm – Alternative Therapist 3''
Again, we can see the focus on movement, that emotional expressions should be facilitated, and we can imply conversely that stagnation of emotions is unhealthy. This is interesting in that there is much psychological evidence that suggests that secrets and unexpressed emotions can link into physical health problems. 

``P:Same thing, the imbalance-
I:mmmm
P:Of emotions, imbalance of mental thinking imbalance of sleep not sleep to much sleep ahhh and then what you're putting into the body obviously, and also what you're putting into the body if you're sitting in front of a computer for too long, if you're sitting in front of a TV too long, you're putting too much electromagnetic energy on the mobile phone for too long so generally, ammm so you've cardio-vascular health we need balance there- - Alternative Therapist 3''

We can see from the above quote that while health is conceived of as balance, sickness is thought to result from either too much or too little of a particular substance or process. We can note the focus on both mental and physical imbalance here, this linking of levels of existence is a theme that pervades the entire transcripts, and  will be examined in detail later. 

\subsection{Energy Model of Health}

Another code which occurs throughout the transcripts of the alternative therapists is this notion of \textit{elan vital}, or life force. This is one of those ideas with a strong resonance thoughout human history, as multiple cultures appear to have developed in independently. We'll look at this in more detail using excerpts from the transcripts below. 
``how do I think they work? Well....from doing different treatments....like shiatsu, so which works on the energetic plane – Alternative Therapist 1''

We note here, that when the respondent was asked how some of the treatments used by him worked, he focuses immediately on shiatsu (a form of Japanese massage) which is the treatment he uses most often in his professional practice. We can also see that he constructs this treatment as operating on an energetic plane, which presumably means that it effects the person on another level from the physical. This theme continues throughout the transcripts of the alternative therapists, and appears to be the explanatory model utilised by the majority of them. 

``you actually work energy channels that ammm...remind....the person to put energy in certain places. Yeah, like I would press certain acupuncture points, work certain energy channels...- Alternative Therapist 1''

As we can see from this excerpt, the practitioner seems to link this energy body to the physical body, such that changes in physical pressure on certain points, suggesting that these two parts of a person are interlinked and occupy the same space. 

``it is too easy to say mind body soul but roughly its pretty much that, physical energy, taking the energy from your food and translating that into nutrition into energy and accommodating that and then expelling energy and expelling waste products efficiently that's the physical side of things energy then – Alternative Therapist 3''

However, in this excerpt we have a very different interpretation of energy from this respondent. The energy she describes links up exactly with what we would consider energy to be (the breakdown of food into sugars), so this definition appears to differ quite substantially from the one given by Alternative Therapist 1 above, which suggests some conceptual confusion. Further excerpts will probably make this clear, however. 

``There's emmm very common ahhh which my professor used to say – if ahhh if you want to do exams, if you're studying, eat well. He says the spleen produces your blood your red blood cells – Alternative Therapist 3''

This extract is interesting, in that the energy model is extended here to apply to mental efforts also. We can also see the linkages between physical and mental health here. Again, we see examples of organs in the body being associated with particular states and traits of the mind. 

`` In france they use digestifs, and apertifs, and thats that's the same thing, you know the same as preparing the energetics to assimilate....transform and bring forth the energy from that food very very carefully, and if you do that, then you're going to have Chi and if you've Chi the the character for Chi in Chinese medicine, or sorry in Chinese language ammm the I'll draw for you is is similar to that and basically what it symbolising is a fire, a pot with rice in the pot and steam rising from it so the pot is sitting over the fire cooking the soup – Alternative Therapist 3''

Again here we see the linkage of physical nutrition and energy. This respondent links the physical act of eating to the life force describes as Chi, suggesting that this Chi terminology may be a descriptor for the sugars and other prodcuts needed to keep humans alive. Again, we see the notion that food also gives us this energy, which is seperate from the physical value of food, and that this is the energy which can impact our health and the lack of which causes us to become sick. 



``So again, when you say how do I think it works thats how I think it works as well manipulating energy
I:mmmm
P: ammm, the energy has to be there first day or first place, so someone is very depleted then often I have to make sure I that we get them to a place where they can that I can manipulate energy because if it's not there, I'm only going to deplete them much more – Alternative Therapist No 3''

Here we see a construction of this energy, or Chi in that it can become depleted, and that the task of the therapist is to increase this energy in order to allow their methods to work on the patient. 

``am manipulating pockets of either static or depleted for replete excessively replete some energy in the body and ahhh....
I: that effects change?
P: Yeah, exactly, yeah, yeah. So thats I suppose my limited way of understanding it and the rest then is down to patient cooperation, really, go and do your rest, we've told you enough here – Alternative Therapist 3''

In this excerpt we can see that the problem is constructed as one of imbalance, either there is too much energy, or too little and that the role of the therapist is to balance these energy pockets across the person to ensure that health returns. 

\subsection{Linking Levels}

This code is quite similar to the Linking Body and Mind code, but there are some important differences which we will explore below. 

``trouble with knees, have often ehhhh, a [pause] a relationship to, to to, the what direction you take in life – Alternative Therapist 1''

``your hands 
I: mmmm
P: what have you, how you handle things – Alternative Therapist 1''

``You know like your digestive system often is affected of how you digest life – Alternative Therapist''


We can see from these extracts that the respondent is linking problems of a physical nature to mental and emotional states. This excerpt tends to use argument by analogy which is not usually accepted as proof, but it does provide some insight into the construction of health by this participant. These quotes imply that the entire person is considered as one entity, and that any change in emotions, mental and physical issues will have an impact on all of the other areas. 

``breathe our emotions – Alternative Therapist 3''

``Breathing emotions and leaving them out – Alternative Therapist 3''

We see here again evidence of an extremely holistic approach to the person, where emotional expression can be linked to breathing. It is perhaps interesting in this sense that an old word for life energy was prana, which was also the word for breath, A similar observation can be made about the Greek pneuma, which means both spirit and breath. The Hebrews also used the same word for breath as for emotions (Ruach), all of which points to the commonality of these ideas throughout history. 
``probably equally important I presume it seems like kindof one feeds off the other if you're physically sick, chances are you're not going to be too well mentally, or if you're mentally you're probably not going to be too well physically – Alternative Therapist 3''

Here we can see a recognition of the interdependence of the differing parts of the human organism and the idea that one of them can throw the others off their functioning. 

\subsection{Emotional Health}

This code was predominantly used by the alternative therapists, but also appeared a number of times in one of the GP transcripts. We'll look at and discuss some of the examples from the alternative therapists section below. 

``P: Ammm...so....a non-stressful lifestyle you know? And a ...fun, laughter, that you do things that you like to do. - Alternative Therapist 1''

``o its more preventative, in its focus more and its all about balance being in
P: that's all the physical side
I: oh?
P: and then I try to definitely you know, meditate, 
I: mmmm, mmmm
P: and ehhhh, focus on my emotions and my thoughts – Alternative Therapist 1''

We can see from these quotes that for many of the alternative therapists, emotional well being is an important factor in their construction of health. They tend to construct health as being situated in a context whereby emotional well being is extremely important and that it forms an integral part of health. This links in with the research suggesting that optimism is associated with health, something which will be discussed later and which appears to be a construction made by almost all of the participants in this study. 

``P: Most causes are the emotional-
I:mmmm, mmmm-
P:and so the body is really reacting and that
I:mmmm, mmmm
P: So thats – thats why we say mind and body, you know that kind of a balance – Alternative Therapist 2''

For this respondent, the emotions are the primary driver of sickness or health. She seems to construct health as being something primarily determined by one's emotional outlook, rather than resulting from pathogens or the environment. Its interesting that this is exact opposite opinion offered by the GP's who were included in this study. 

``emotional energy has to be very well balanced we have to breathe our emotions, we cannot suppress them am they have to (pause) be (pause) tuned in and out of the body very carefully. - Alternative Therapist 3''

As we can see here, the construction of the importance of emotions runs throughout all of the transcripts of the alternative therapists. This respondent links the emotional expression with the breathing, that they have to be taken in and let out very regularly. He seems to construct them as something which can have great physical impact, and should be treated with respect.


``ahhh it it if used in itself it may not answer the whole problem because people come not only with a condition, they come with with a condition attached to their own bodies its got feelings, its got thoughts preconceptions, its got worries, etcetera- - Doctor 2''

Although this section is dealing with the transcripts of the alternative therapists, this quote is included here to gain a different perspective on emotional health. We can see from this quote that this respondent possesses a radically different construction of emotions and their relationship to health. For him, the body is the primary part, and he constructs this using depersonalising language (it) the body is the important part, but it has all of these messy emotions and thoughts attached to it which complicate matters. The contrasting quotes here nicely illustrate the differences in construction between the alternative therapists and the general practitioners. 


\subsection{Personal Effort}

Again, this code appeared mostly in the transcripts of the alternative therapists, although it also appeared in the transcripts of one of the doctors. 

``reflex points, If a person is not willing to work on themselves and to get better, you can do whatever you want, and they- Alternative Therapist 1''

``I: so it requires their willingness to take part in the procedure aswell - Alternative Therapist 1''

These quotes serve to illustrate a point made again and again by the alternative therapists and to a lesser extent by doctors. That point is that people are ultimately responsible for their own health, regardless of what help they get from a healthcare professional. 

``Yeah, but again it will need a change in lifestyle, change in diet, a willingness for the person to actually change – Alternative Therapist 1''

Here we can see a restatement of the points made above, that while the practitioner can give advice, the person must ultimately act upon it, and if they are not willing, then they will not change. The truly cynical could claim that this is because all of these therapies are nothing more than placebo, and this abdication of responsibility is nothing more than a device to conceal their lack of efficacious treatments. But we are not quite so cynical. 

``Yeah, I don't fix people, they have to fix themselves then with support and inspiration from me – Alternative Therapist 2''

This is perhaps the most telling quote in this section. This respondent constructs patients as being totally responsible for their own health, that he merely acts as a facilitator and allows them to heal themselves. This is a very empowering view of the healing process, and contrasts quite strongly with the more paternal constructions of the doctors. 

``how much they'll try to put practices that could be suggested in the clinic into into their lifestyle ammm, you can also I think tell if if they're \ldots you know you point something out to someone, that they can make this subtle  change that will make a huge difference to their bowel habits  
I:mmmm
P:Suggest something for them to eat-
I:mmmm
P: almost on hearing it, I could there could almost be a placebo effect would not be surprised when they come back in two weeks and tell me that their bowel habits have completely changed but I would often almost be able to tell with a patient who's going to do that or who's going to be...... a little bit more open to that or ammm.......how else do I think it works. I I I think a lot of the time with any form of healing I mean, - Alternative Therapist 3''

Again, we have a construction here that points out the differences between people's response to their health. Some other respondents suggest that health is not valued by many, and this respondent argues that some patients will do the work required to get better, while others will not, and that it is this factor which accounts for broadly divergent outcomes. 

``because people think that everything going right comes without any effort and in actual fact, to make things right requires huge effort – Doctor 1''

We have this quote from one of the GP's here, and it supports what the therapists have been saying. One can say that health is being constructed as something which needs to be valued by people if they are to enjoy good health. 

\subsection{Body Knowledge of Health}

Again, this code appeared only in the transcripts of the alternative therapists, and only in two of three of those. Below are some relevant examples with some commentaries on their constructions of health. 

``Because you should more talk to your own wisdom but if you really sick of course you have to go to the doctor – Alternative Therapist 2''

This is an interesting quote, in that it seems to construct an awareness on the part of the person as to the causes and nature of their sickness. The body appears imbued with understanding of the problems facing the person, and this can be consulted. It seems however, that this is only a first resort, as the quote continues to state that serious problems should be dealt with by a doctor.

``By paying attention to what is (pause) going on with yourself. We've simple basic needs. We need to sleep and we need to eat. Basically, those are the two most important we also need to breathe. Otherwise we can – we can name ammm – Alternative Therapist 3''

Here again there is a focus on intenal knowledge. This time, however, it appears constructed in terms of awareness of the body's needs at a more basic level, that of sleeping and eating. This constrasts with the unspecified wisdom referenced in the first quote. 


``P:And to reassure them you know that, everything is OK with them in in some fashion as well and its sustained by (pause) listening to yourself, and listening to nature – Alternative Therapist 3''

Here, from the same respondent we have an expansion of the previous statement. Where first listening to yourself was important, now we have a focus on listening to nature also. While we can construct this as listening to the natural world, this construction could prove quite problematic as there are many different `natures', and this quote does not allow us to distinguish between them. 

\subsection{Environmental Factors}

This next code is interesting, in that it appears  with about the same regularity in the transcripts of both doctors and alternative therapists, which does not occur particularly often throughout the interviews. 

The first example comes from the transcript of Alternative Therapist 1, where they refer to ``a healthy non-toxic environment'' as a major determinant of health. ``And ehhh, gives eh – a safe environment that the person feels safe and releaxed''. We can see here that the conception of the environment here is more of a reference to the sensations and feelings aroused by the environment. 

\section{Construction of Doctors}
\label{sec:construction-doctors}

One of the major themes that emerged from all of the interviews was the nature of doctors, even apart from a more general conception of health practitioners, all the participants focused on the role of doctors in society, and their positive and negative impacts. 

One of the ordinary participants focused on the relationship between Doctors and Alternative therapists, with this excerpt ``I think if you have like serious [pause] health problem, you will need both''. This stresses and indeed this participant stressed throughout the interview, that doctors and alternative therapists should be viewed as complementary rather than opposing. 

\begin{quotation}
  she mentioned this to her GP who (pause) became very angry and asked me to phone him immediately 
I:Mmmm
P:and this type of thing doesn't happen very often. I rang him, and he was quite set, saying I dont want this woman to have more and more treatments, you know and I want I want her to find the right treatment. And I said, so do I, and this is it, and this is what I deem most good for her at the moment and he asked about the treatment and I tried to explain a little – in simple language – and the end of the conversation he was Ok, let's do this. 

\end{quotation}

This quotation came from Alternative therapist 3, and describes the relationship between doctors and alternative therapists. Note that he constructs the two as in a collaborative relationship rather than competing, and shows that they were able to find a good balance between bothe of their particular forms of treatment. 

It is interesting in that one of the doctors focused on their role as agents of social change, particularly in this excerpt

\begin{quotation}
   I think as doctors we have a job maybe as advocates you know to point out the issues, and point out the we may not be have the solutions but we should be able to sortof say these are part of the problems – these are the problems, and these are some of the determinants and really these need sorting out 

\end{quotation}

This quotation focuses on the role of doctors in society rather than their role as a individual health practitioner, making the point that they have a responsibility to their patients to focus on the problems that are actually affecting their abilities to live healthy lives. 

Interestingly, many of the alternative practitioners did not subscribe to this viewpoint of doctors, instead regarding them as prescribers 
\begin{quotation}
  Ehhh, a lot of doctors just ah rely on on on on eh their pharmacuetical companies 

\end{quotation}


The implication here seems to be that the commercial relationships between doctors and pharmacuetical companies get in the way of healing. 

However, this point was actually raised by one of the doctors themselves, in a slightly different context: 

\begin{quotation}
  I think doctors actually should have a ethical obligation to say enough is enough you know and I get actually quite frustrated when I get sent a patient that I actually feel has nohting wrong with them 

\end{quotation}

Here we can see a doctor's frustration at the way in which they are expected to dispense medicines to patients even when they feel there is nothing wrong with them. It is worth noting that the doctors frame this as the patients demands, while to the alternative therapists, this is a relationship in which the doctors have power over the patients. 

The second GP also expressed frustration with this state of affairs, saying that 
\begin{quotation}
  even though you try to spend as much time as you can in terms of health prevention such as immunisation or health promotion by giving advice to people on healthy diets exercise giving up smoking etcetera we do tend to spend most of our time reaching for the pen, prescribing 

\end{quotation}

The sense is that they would like to be able to focus more on the health promotion effects that would actually change the persons state of health, but are corraled by the system into prescribing, as that is something that can be done within the confines of the 30 minutes doctors appointment. 

Also, one of the students expressed some annoyance with elements of the system: 
\begin{quotation}
  well as I say like, just that kind of stuff ammm like plenty of times i've been sick and gone to a doctor and he's given antibiotic and i've taken it and gone thats not helping me at all 

\end{quotation}

The notion is that antibiotics are a default option for many doctors and patients, and for some of the respondents, this is not actually particularly useful. 

\begin{quotation}
not unless you're actually particularly sick like, I think that an awful lot of the time people tend to be a small bit sick and go off to a doctor and get their antibiotics and go this didnt help at all, whereas there was probably no need to go to a doctor in the first place 

\end{quotation}

The quotation above referenced another students answer to the question how often do you go to the doctor. Note that given the age (early twenties) and social status (student) of this respondent this is perhaps not a surprising response.  The theme that comes through here is that doctors can be associated with unnecessary treatments and methods, and that this is a conception shared by all three groups in the sample. 


Another theme that emerged around doctors was that doctors focused more on illness. 

The extract (from Doctor 1) illustrates the point: 

\begin{quotation}
  I suppose sometimes health and ill health, you know, there's not different sides of the same coin
I:mmmm
P: but I suppose we're more set up in our training to deal with ill-health as a concept
I:yeah
P:- than health and particularly how the health services are structured 

\end{quotation}

The participant notes that ill-health gets far more attention than does health, perhaps linking back to the idea that ill-health is a breaking down of something that is normal (i.e. health). 

Another quote from the other Doctor (No 2) supports this reading:

\begin{quotation}
  we'd only get so far, and I think that we essentyially would be more into the disease ahhh ammm diagnosis and cure-

\end{quotation}

It is interesting to note that the doctor first says that they are more into disease, but then shifts their words to diagnosis and care, displaying the focus on ill-health referenced above. 


Perhaps an interesting theme which emerged from the interviews with ordinary people and alternative therapists was that doctors were not that important. The interesting point about this is that by making the point, they were in fact reinforcing the notion of doctors as important. 

\begin{quotation}
  So just to have a regular checkup and stuff checkup and ah, well everything was fine but yeah I don't really know so much about doctors. I think doctors are great if you need them but if you dont need them you shouldn't go to them.  -Alternative Therapist 1

\end{quotation} 

The respondent here proudly proclaims their ignorance of doctors, and introduces some tautologies in that doctors are only necessary if you ``need'' them, but does not define what exactly it is to need doctors. 

\begin{quotation}
  not unless you're actually particularly sick like, I think that an awful lot of the time people tend to be a small bit sick and go off to a doctor and get their antibiotics and go this didnt help at all, whereas there was probably no need to go to a doctor in the first place 

\end{quotation}

Here the respondent focuses on doctors only being important when you are really sick, and argues that people tend to go to doctors for issues which they would not consider important enough. 

The next theme around doctors came from Alternative Therapist 3, where they recall an incident where doctors came to see them. 

\begin{quotation}
  initially came for pain a third came for tiredness very interestingly they all asked to come after-hours 
I:Because they didnt want to be seen
P: By their patients, they know my system and that I do one patient per hour and I dont do two or three rooms like a lot of people do and they were fine with that ammm (pause) one a patient (pause) didnt want necessarily anything other than the main complaint which I think was stress ammm but I asked could I treat for other matters 

\end{quotation}

The interesting part here is that he constructs this incident as something out of the ordinary (whereas the converse is broadly accepted by all participants) and frames it as a series of differences between the doctor's typical practice versus that of the alternative practitioner. Note also that the doctor only wanted treatment for the main presenting symptom, rather than a more general process which the alternative therapist would have preferred. 

Another intriuging theme that came through from one of the doctors was the following: 

\begin{quotation}
  the herbalists, who have almost taken on a bio-medical model then a bio-medical sortof way of acting they see patients they diagnose they have a pharmacy of herbs-

\end{quotation}

The participant was responding to a question around whether or not they had any faith in alternative medicines and treatments. Note how the ``good'' alternative therapists are defined by their adherence to the bio-medical model, and note that the other salient features are diagnosis and pharmacies. 

However, note the contrast here with how one of the alternative therapists describes their relationship with doctors. 

\begin{quotation}
  I've become much more in happier terms today with it because of with the fertility side of things I need to rely on blood results a lot. I need people to go and have their hormones checked and and something like sperm analysis, I need, I need medical science completely for this these reasons.
- Alternative Therapist 3
\end{quotation}

This respondent regards their relationship with medical science as non-adversarial, but rather complementary. The contrast with the attitude of the doctor is quite striking. 


\begin{quotation}
  they just prescribe medication that has a lot of side effects and they don't really spend much time with you and you go to the doctor there's very few doctors which actually spend time and actually talk to people and really see where the problem comes from-
Alternative Therapist 1
\end{quotation}

Here, the conception of doctors is as seen earlier, distant prescrivers who do not actually spend time with a person. The implied contrast with their own practice is apparent. 

Even some of the Doctor (No 1) respondents argue in this point:

\begin{quotation}
  I would disagree with what my colleagues might do or i've occasionally been shocked ammm by what I would perceive to be a lack of care,

\end{quotation}

Here it is constructed more in terms of professional disagreement, and the occassional problem, whereas the alternative therapists are more fortright about this (but less so about the problems with their own profession). 

And from one of the Doctor participants: 

\begin{quotation}
  ammm, a bit, I'm not trained in them but obviously I prescribe St John's wort

\end{quotation}

Note the unassuming tone, a bit, expressions of ignorance but the use of obviously to describe the practice of prescribing St John's wort, as though it were the most obvious thing in the world. 
%%% Local Variables: 
%%% mode: latex
%%% TeX-master: "PlaceboMeasurementByMultipleMethods"
%%% End: 
