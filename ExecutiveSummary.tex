\documentclass{article}
\usepackage{natbib}
% \usepackage{apacite}
\title{Executive Summary Implicit and Explicit Treatment Expectancies: Methodological Issues}
\author{Richard Morrisroe}

\begin{document}
\maketitle
% \author[supervisor]{Dr Z. DiBlasi\\ Prof. J. Groeger}

This document lays out the major aims of the thesis currently titled, ``The Placebo Effect: Implicit and Explicit Expectancies''. 

\section{Background}
\label{sec:background}



The placebo, as a phenomenon has remained somewhat unpredictable even though the effects occur in many situations, from clincial trials to the doctor's clinic. There are a number of reasons for this. Firstly, investigations of individual psychological characteristics related to the effect have been mostly fruitless. Secondly, the methods of analysis in placebo experimentation are not always entirely appropriate to the data which is collected.

The placebo effect can be usefully defined as the study of the contextual factors which affect the response to treatment \citep{Kaptchuk2008}. This definition implies that the response to placebo is part of every healing encounter, not just those which involve treatments with no biological effect (the classic inert pill). Placebos have been shown to evoke real changes in physiological parameters and have been demonstrated to activate the production of opioids and dopamine within the body \citep{Sauro2005}, which may be mediators of the observed effects.   

\section{Expectancies}
\label{sec:expectancies}

The currently accepted model for the effects of placebo in humans is the response expectancy model developed by Irving Kirsch \citep{Kirsch1997,Kirsch1985}. Response expectancies were defined in this theory as ``the expectation of a non-volitional response''. However, some authors have argued that these expectancies must necessarily be conscious \citep{Stewart-Williams2004b}. The major empirical contention of this thesis is that expectancies are not necessarily conscious, and can be measured using reaction time based measures (i.e. Implicit Association Tests).

Currently, expectancies tend to be measured on an 11 point scale asking participants to rate how much benefit they expect to derive from a treatment. This, while quick and easy, fails to capture the complex and multidimensional relationships between expectancies and the response to treatment \citep{Stone2005,Kaptchuk2008a}. Therefore, one part of the work in this thesis was to develop and test a better measure of treatment credibility using modern psychometric methods.    

Implicit Association Tests \citep{Greenwald1998} are a relatively new method for the measurement of so called ``implicit attitudes'' which are defined as ``the unidentified or inaccurately identified traces of past experience''. The measurement procedure takes place on a computer, and requires participants to sort words into predetermined catgeories. The speed at which this categorisation can be carried out is taken as an index of attitude strength. This research developed and applied a Treatment Credibility IAT to measure implicit reactions to treatment-related words. 

A secondary contribution of this thesis was to apply better measurement methods to the placebo effect. These measurement models included the use of item response theory and factor analytical models developed on large survey samples as predictors for the responses of experimental participants to the same instruments. This was a way of improving the current predictability of the placebo effect through a model-based measurement approach. 

The thesis involved the conducting of two large surveys, including the development and validation of a new treatment credibility measure, along with interviews which were used to develop stimuli for an Implicit Association Test (IAT) to measure treatment related implicit expectancies. The measures and models developed on larger survey samples were then applied to the responses of experimental participants in a study which used a placebo analgesia design with healthy volunteers. 

% The major contribution of this thesis is to determine to what extent current models of the placebo and the predictors thereof have been shaped, not by the phenomenon, but by artifacts of the measurement process. To this end, a number of methods were employed, which are described in this document. 

Previous studies have shown links from the placebo to acqueisiance, intelligence, suggestibility and other variables. However, typically these findings have not used the full capabilities of modern psychometric methods (especially item response theory), and have almost never been replicated. This work reviewed this previous research and selected the most useful predictors for inclusion in the empirical part of the thesis. 

\section{Measurement Issues}
\label{sec:measurement-issues}

Normally pain responses (on which this thesis focuses) are collected over time, and yet the standard tool for investigating of whether or not a placebo response occurred and its predictors are methods such as ANOVA and linear regression. This is problematic as each pain rating is likely to be correlated with the ones before and after it, and such correlations violate the assumptions of such models. Therefore, this thesis looked to apply more appropriate methods such as mixed models and time series analysis to the pain rating data collected in experimental studies. 

In addition, the psychometric models used in this thesis are prone to overfitting (where noise as well as signal is modelled), and so to reduce the impact of this problem, cross-validation was employed. This technique splits the data into multiple random sets, and estimates the model on one of these sets and tests them on another. This approach allows for more parsimonuous and more predictive models \citep{friedman2009elements}, and has been demonstrated to be effective in a wide number of areas. 

The approach taken in the thesis was as follows:
\begin{enumerate}
\item Conduct a thorough literature review to identify constructs which have an effect on the placebo response;

\item Run large surveys to develop good psychometric models for these variables;

\item In tandem with this, seven interviews were conducted with doctors, alternative health practitioners and lay-people to develop an understanding of constructs associated with health and treatment in order to develop stimuli for the treatment credibility IAT. 

\item Using the data garnered from interviews and ratings of the importance of these constructs taken from a large web-based survey to develop an implicit measure of treatment credibility (an Implicit Association Test);

\item Development of a more comprehensive self report instrument of treatment expectancies and to use large surveys to refine and model this effectively;

\item Testing of these models and approaches in an experimental study of the placebo response to an inert cream in an experimental study;

\item In addition, physiological data was collected from all of the experimental participants to assess the extent to which the placebo response and the self report measures could be modelled as a function of these reactions.
\end{enumerate}

In essence, the thesis took a comprehensive approach to the measurement of the placebo effect, and made use of modern statistical and psychometric methods to enable an in-depth investigation of the properties and measurement of the placebo from multiple points of view. 


\section{Literature Review}
\label{sec:literature-review}

The literature review covered the placebo effect and its predictors, along with an investigation of best practices in implicit measure development and testing. Optimism was identified to be an important predictor of the placebo effect \citep{Geers2005}, and mindfulness was identified as a variable which moderated the relationship between explicit and implicit measures of the same construct. 

\section{Health, Optimism \& Mindfulness}
\label{sec:health-optimism-}

The next step in the research involved the collection of data on the Life Orientation Test Revised (LOT-R) and the Mindful Attention Awareness Scale (MAAS) for mindfulness. In addition, the RAND Medical Outcomes Survey was administered to the same participants, in order to investigate the relationship of these constructs to self reported health. 

This study was carried out three times, once using a paper based collection method, and twice using an online approach. The results showed that a five item LOT-R scale provided the most accurate estimation of participant abilities, when a two parameter IRT model was fitted. A similiar procedure was carried out for the MAAS and showed that a reduced 13 item scale provided accurate and predictive estimates of ability. Both of the IRT models which provided a good fit to the data were Graded Response Models. 

In addition, an inverse relationship between health and optimism was found across all three samples, contrary to previous research. 

\section{Treatment Credibility Questionnaire}
\label{sec:treatm-cred-quest}

The next major stage in the research was the development of a questionnaire to measure explicit expectancies more effectively. The questionnaire was adapted from the Therapy Credibility Questionnaire \citep{Devilly2000}. The therapy credibility questionnaire consisted of six items, and the adaptation used in this research repeated these questions for six different forms of pain treatment (Pills, Creams, Injections, Acupuncture, Homeopathy, Reiki). Three of the questions in each section consisted of credibility questions, and the other three consisted of expectancy questions. The questions concern both cognitive and emotional aspects of credibility, which have been shown to be important in the placebo effect  \citep{Blasi2001}. 

This instrument was first applied to a small sample (collected online) to determine its psychometric properties. Following the results of this study, which showed acceptable reliability and clear psychometric structure, a second, larger study was conducted to assess the construct validity of the measure. This was achieved by triangulation with a measure of treatment beliefs (the Beliefs About Medicine scale). The Beliefs about Medicine questionnaire showed the predicted correlations with the Treatment Credibility Questionnaire (positive with the alternative treatment questions and negative with the conventional treatement questions). 

In addition, IRT modelling of the TCQ showed that it was best modelled as two scales, one conventional and one alternative. As the completed questionnaire was 36 questions long, an information based IRT approach was applied to shorten this to 18 questions to reduce the amount of time spent by participants filling out questionnaires during the experimental study. 

\section{Development of the IAT}
\label{sec:development-iat}

One major focus of the thesis was the development of the IAT. To develop stimuli for the Treatment Credibility IAT, seven interviews were carried out with doctors, alternative health practitioners and laypeople. These interviews were coded line by line using an inductive procedure, and then themes were extracted from these codes. 

The aim here was to elicit health related constructs to develop a repertory grid, which was hypothesised to provide a principled way of developing the stimuli for the Implicit Association Test. In addition, the first sample who filled out the Treatment Credibility Questionnaire were asked to provide the five figures in their lives who most represented health to them. These responses were then summed and analysed to develop a health repertory grid. 

This grid was then piloted with seven participants, where it was found that it was extremely difficult for most to complete, as many (6) participants did not have exemplars of all the categories used in the grid. Therefore, the content analysis of the interviews, along with a hypothesis that conventional and alternative treatments represented opposite ends of a dichotomy (as shown in the TCQ research through IRT methods), the IAT was created on this basis. In addition, an Optimism IAT was developed (using the self report measures as a base) to determine if an implicit measure of this demonstrated predictor of the placebo would prove to be useful. 

Both of these measures were piloted in a small sample (17) and they correlated with the explicit measures in the predicted way. In addition, they appeared to be associated (though not significantly) with the placebo effect in a pilot study (N=7) which tested out the experimental procedure.

\section{Experimental Study}
\label{sec:experimental-study}

The final and most important part of the thesis was the experimental study. This was an investigation into the predictive capability of the implicit and explicit measures in the context of a placebo analgesia experiment (N=111). 

The experiment had a randomised three group design (Deceptive Placebo (n=34), Open Placebo (N=33) and No Treatment (N=34)) and involved the priming of the majority of the participants (N=77) to increase the size of the placebo effect without requiring a lengthy conditioning procedure. 

The explicit measures then had the models developed with the Optimism, Mindfulness and Treatment Credibility studies applied to the responses of the experimental participants. These significantly increased the predictive capability of the model.

The results showed that the treatment credibility IAT was a significant (p=0.02) predictor of the placebo response, as assessed with a logistic regression. However, the major surprise was that the Open Placebo group had a signficantly higher response to placebo than did the Deceptive Placebo Group, which was contrary to the hypotheses of the experimenter. 

In addition, the IRT based abilities of participants estimated from the responses to the explicit measures (Optimism and Treatment Credibility) were significant predictors of the placebo response, while the mean and sum scores were not, providing a validation of the psychometric approach applied throughout this thesis. 

\section{Contributions}
\label{sec:contributions}



The major contributions of the thesis are as follows:

\begin{enumerate}
\item A demonstration of the usefulness of implicit measures in the prediction of placebo;

\item A validation of cross-validatory approaches towards the development and testing of psychometric models;

\item A demonstration of the effectiveness of the strategy of building psychometric models on large samples and applying these models to the responses of experimental participants;

\item A validation of a new scale to assess treatment related expectancies.

\end{enumerate}


\bibliographystyle{agms}
\bibliography{placebo,IAT,healthoptmind2,statistics} 
\end{document}

%%% Local Variables: 
%%% mode: latex
%%% TeX-master: t
%%% End: 
