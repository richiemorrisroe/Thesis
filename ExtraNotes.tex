\documentclass{article}

\begin{document}


\section{Bains Price Augmented Placebo}
\label{sec:bains-price-augm}

comment on deventer (2009).

Proposes that placebos be charged for and argues that they will have greater impact because of this.

Testable implication is that patients who pay for a drug should report greater relief than those who have them provided by a government/employer scheme. 

\section{Bakal Medically Unexplained Symptoms}
\label{sec:bakal-medic-unexpl}

Argues that somatic awareness is a useful tool to aid in the treatment of patients with MUS. 

Refers to Damasio (2008) - brain regions involved in bodily perception are are the basis for emotions. 

Argues that MUS are caused by prior negative experiences being held in the body as tension and stress. Suggets that this is due to the muscol-skeletal system (Langevin). 

Common expectancy understanding of placebo is disempowering, argues that if patient understands that this is a self regulatory process, then matters will improve. 

\section{Benedetti 2010 Disruption of placebo response by activation of cholecystokinin type-2 receptors}
\label{sec:bened-2010-disr}

Prior work on biochemical inhibition of placebo has used non-specific antagonists such as proglumide. Some CCK2 receptor agonists show links to cognitive and emotional outcomes.  

40 pp, 10 to each of four groups. 
Natural history
hidden pentagastrin
placebo
placebo + pentagastrin

All Pp were screened for physical illness before participation. 
All Pp abstained from coffee/tea for 48h before each session.
Placebo and Placebo + groups were given intraavenous morphine during conditioning phase. 
Drugs adminstered 15m before pain.
Double-blind design. 

Four consecutive sessions, with an inter-day interval of 3-4 days.
NH had no treatments
HP had pentagastrin on days 2 and 4
Placebo group tested over 5 non-consecutive days. Day 1, no treatment, 2 and 3 had morphine conditioning, placebo on day 4. Day 5 was the same as day 1.
Placebo + pentagastrin had same procedure, but added pentagstrin with the placebo on Day 4. 

No main effect of NH or pentagastrin. Pre-conditioning was effective, and placebo effects were obtained on Day 4 for the placebo group. 

High correlation between the response to morphine and the response to placebo (r=0.79). In the pentagastrin group, the correlation was r=0.49. 

\section{Benedetti 2009 lack of prefrontal control abolishes the placebo response}
\label{sec:benedetti-2009-lack}

Basbum and Field's model of the opiodergic descending pain control system appears to modulate the placebo response. 

This links to the lack of placebo responses in Alzheimer's patients, where the prefrontal cortex has been damaged. 

Refers to Krummenmacher paper:
Two main findings:
\begin{itemize}
\item sham rTMS can be used to induce placebo effects
\item verum rTMS abolishes these effects completely.
\end{itemize}

Prefrontal areas are extremely important for placebo responsiveness. 

One major problem with the study is that frequencies of 10-20Hz were used, and localisation for individual participants was not performed, so it is difficult to implicate the precise regions involved. 

\section{Bensing and Verheul - The Silent Healer}
\label{sec:bens-verh-silent}

argues that placebo and doctor communication research should be integrated. 

Argues that research supports the placebo construct, and that one of the problems with H\&G was their use of dichotomous outcomes, which can obscure larger effects. 

Little overlap between doctor-patient communication and placebo research until recently. References DiBlasi et al. 

Makes distinction between process and outcome expectancies (i really need to read that damn bandura paper!).

Affect manipulation also proposed as a mechanism through which placebos produce their effects. Argues that negative affect can increase pain (possible link to CCK?). Argues also that affect manipulations are stronger in experimental studies, and that this is the reason for differences in outcomes. Affect also proposed as a mediator of adherence and disclosure. 

Placebo research can learn how to effectively embed suggestions into consultations from doctor-patient communication research, while giving rigour to a field not noted for its methodological strengths. 

\section{Kong  et al 2010 Acupuncture vs Placebo Expectancy Analgesia}
\label{sec:kong-et-al}

Gracely Sensory and Affective scales used. Gracely, 1979. 


\section{Picharo et al 2011}
\label{sec:picharo-et-al}

NRS (Natural Rating Scale) 0 (no pain), 10 (worst pain imaginable). Highly correlated with the VAS, sensitive to change. 

Used a 5*2 ANOVA to manipulate expectancies and implicit learning. 

No significant result for the implicit learning part. However, 87.5\% of the Pp who had responded to placebo were able to accurately report the order in which placebo and treatment were applied, suggesting some evidence for implicit learning. 

\end{document}

%%% Local Variables: 
%%% mode: latex
%%% TeX-master: t
%%% End: 
