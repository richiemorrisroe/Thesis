\chapter{Introduction}

% Science is an ever evolving process. Isaac Newton is reputed to have said that "if I have seen further than other men, it is because I am standing on the shoulders of giants". If such an aphorism were true in the 16th century, it is even truer today, in these days of global communication links and thousands of peer reviewed journals covering almost every possible subject and interest. 

% It is through this evolving process that each generation of researchers can pose new problems, and test them against the cold rock of reality to see which will flounder and which will flourish.

The central aim of this thesis was to apply new psychometric methods to the prediction of the placebo response to pain in healthy volunteers. The new psychometric method applied to the prediction of placebo was the Implicit Association Test, or IAT, which is a reaction time based categorisation method developed in the last fifteen years. 

It is the contention of this research that the alleged unpredictability of the placebo effect is a function of our measurement tools, and that new and improved measurement tools can furnish us with a placebo effect which can be reliably assessed and predicted. 

Although some form of placebo effect has been an integral part of medicine for at least the last 2000 years, it is only over the past sixty years that it has been conceptualised and operationalised in such a form as to make my research possible. 

Indeed, the major tool proposed by this research for the prediction of the placebo response is itself a more recent development than the construction of a defined placebo response. Implicit measures are further (indeed, exhaustively) defined in the later sections of this work, but for now we can model them as associations which lie outside our awareness much of the time, the fast and frugal operations of our inner computer which colour many aspects of our social lives. In another age and time, these measures would have been called unconscious, but such a conceptualisation runs into the tricky problems of unfalsifiability and so, nowadays, most researchers prefer to substitute the term implicit. 

The aims of this research were as follows:
\begin{enumerate}
\item To use modern psychometric methods to allow for better prediction of the factors involved with the placebo response. 

\item To conduct one large experimental study which aim to test the utility of the measurement and psychometric approaches taken. 

\item To create an implicit measure to assess expectancies related to treatment and to the placebo response

\item To develop a multifaceted method of assessing explicit expectancies which can be linked to the placebo response

\item To collect physiological data during the course of this experiment to assess its validity for the prediction of placebo responses in healthy volunteers.
\end{enumerate}

% Of course, this short introduction merely scratches the surface of what has been a hugely rewarding, frustrating and interesting research experience. Mere words cannot convey all that has occurred, but given the format required of us by the scientific community at this time, they shall have to do. The plan for this section of the thesis is as follows. 
The structure of the thesis shall be as follows:

\begin{enumerate}
\item The literature on the placebo effect and implicit measures shall be reviewed, as will the literature of constructs which appear to have been predictive of the placebo effect in the past. 

\item The methodology for the thesis shall then be described.

\item The results of a preliminary study relating health, optimism and mindfulness (all of which have been related to placebo or implicit measures) shall be described. 

\item The development and assessment of a measure designed to tap response expectancies to pain treatments shall be described. 


\item The development of the implicit measures shall be described, including a process of interviewing health professionals and a method of developing IAT stimuli for arbitrary constructs. 


\item Next, the experimental data will be analysed, and the usefulness of the models and measures used assessed. As part of this, the inter-relationships between the implicit, explicit, physiological and behavioural measures will be examined.


\item Finally, the thesis will conclude with a discussion of what has been discovered, and plans for future research. 
\end{enumerate}



%%% Local Variables:
%%% TeX-master: "ThesisContents030511"
%%% End: