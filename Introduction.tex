
\chapter{Introduction}

\section{Preface}
\label{sec:preface}

The placebo effect is a phenomenon which is replicable in many different settings and contexts,  but for which individual level predictors are rare \cite{Shapiro1997,Kaptchuk2008a}. Implicit measures (such as the Implicit Association Test) have been found to predict responses to constructs which are difficult to assess using self report. This thesis assesses the efficacy of such an approach to a the prediction of the placebo response in healthy volunteers. 


The central aim of this thesis was to examine the measurement of explicit and implicit treatment expectancies in the context of the placebo effect. 
For the purposes of this thesis, an implicit measure is one which infers psychological attributes from non-self-report methods (sometimes called an indirect measure) \cite{Greenwald1995a}. An explicit measure is one which is collected using a questionnaire (terminology also developed in Greenwald and Banaji). Treatment expectancies are regarded as an ``expectation of a non-volitional response''~\cite{Kirsch1985} by some prominent authors in the field. 

 
This thesis used self-report, interview, implicit and physiological variables to provide greater understanding of these psychological constructs and their relationship with the placebo effect using both observational and experimental designs.  The experimental portion of the research concentrated on the response to a placebo treatment for pain with a healthy, non-clinical sample. 

This thesis focused both on the explicit expectancies common to much psychological research (i.e. self report scales) and the implicit expectancies measured by such tools as the Implicit Association Test \cite{Greenwald1998}, as have been defined above. 

This thesis developed three new measures to aid in this task. One was a more comprehensive and multi-dimensional instrument for the measurement of treatment related explicit expectancies (Chapter \ref{cha:tcq-thesis}). This measure was a generalisation of the Credibility/Expectancy Questionnaire developed by Devilly and Borkovec~\cite{Devilly2000}, and the Beliefs About Medications Questionnaire \cite{Horne1999} was used to assess the construct validity of the revised instrument. 

The other two measures developed were Implicit Association Tests (IAT) -- an Optimism Implicit Association Test (IAT) and a Treatment Credibility Implicit Association Test (Chapter \ref{cha:devel-impl-meas}). 

Additionally, self report measures of both optimism which has been shown to be related to the placebo effect and a construct commonly conceptualised as expectancy related~\cite{Carver2010}) and mindfulness which is a construct of interest in terms of treatment in its own right~\cite{kabat1982outpatient}. This construct has also been found to moderate the relationship between explicit and implicit measures. In this thesis, background samples of these measures  were collected from the same sample as from which the experimental population were drawn, to control for potential covariates (Chapter \ref{cha:health-for-thesis}) such as differences between the survey development and experimental sample population. 

The use of both implicit and explicit measures of the same construct, all of which were validated in the population from which the experimental sample was drawn allowed for the relative impact of both the constructs and the different forms of measurement of them to be assessed. 

The experiment (see Chapter \ref{cha:primary-research}) was a three condition design (Deceptive Placebo, Open Placebo and No Treatment) and collected implicit, self-report and physiological measures to assess the relationship between implicit and explicit treatment expectancies and the placebo response. 


% Another contribution of this thesis is that it uses large samples to develop psychometric models of responses to self-report instruments, and then applies these models to help estimate effects in an experimental sample more accurately. 



% Finally, this thesis  made use of more sensitive and appropriate statistical models and methods, and applied these methods to the more precise estimation of placebo effects in an experimental sample of healthy volunteers drawn from the sample population. 


The aims of this research were as follows:


\begin{enumerate}

% \item To use multiple methods to allow for better prediction of the factors involved with the placebo response. 

\item To create an implicit measure to assess expectancies related to treatment and to the placebo response;

\item To develop a multifaceted method of assessing explicit expectancies which can be linked to the placebo response;

\item To compare the predictive power of implicitly and explicitly measured constructs as predictors of the placebo response;

\item To conduct one  experimental study using a placebo analgesia paradigm to assess the relationship between implicit and explicit expectancies and the placebo response.
\end{enumerate}

The remainder of this chapter provides a short introduction to some of the major areas of focus of the thesis, and both the placebo and implicit measures are much more exhaustively described in Chapter~\ref{cha:literature-review}. 


\section{Introduction to Placebo \& Implicit Measures}
\label{sec:intr-plac-impl-meas}

\subsection{Placebo}
\label{sec:placebo}

The placebo is a complex topic of study in that the term stands as a proxy for those elements of human health which are not determined by  features specific to the treatment. Instead, the term placebo refers to the ``non-specific'' parts of the treatment~\cite{grunbaum1981placebo}, those which are not attributable to  specific biological or mechanical activities of the treatment, but are observed across varying treatments. In controlled experiments, the placebo effect is often defined as the response in the placebo group less the response in the no-treatment group. While this definition has problems (described in Section~\ref{sec:concept-placebo}), it is good for an intuitive understanding of the construct. 

The placebo effect has been studied in and of itself for approximately 60 years since Beecher at least~\cite{beecher1955powerful}. The research study of placebo arose as a result of the requirement for placebo controlled studies of medical interventions, and the first such trial was for the efficacy of streptomycin~\cite{concato2000randomized}. In the time since then, there have been many studies to test the relationships between individual level constructs and the phenomenon. For the most part, these attempts have been unsuccessful. A review of the problems with many of the earlier approaches is in Shapiro and Shapiro's book~\cite{Shapiro1997}.  The central idea of this research is that this lack of success in prediction is an artefact of the measures used. 

More recently, there have been some intriguing studies that suggest that there may be some relationships between optimism and placebo response \cite{Geers2005,morton2009reproducibility}.

The placebo effect is rarely the focus in randomised controlled trials, and yet that is where much of the research which has broadened our understanding of the phenomenon has taken place.
Despite years of research, there are few known individual level predictors of the effect. In fact, some researchers have argued that there is no such entity as a placebo responder~\cite{Kaptchuk2008a}. 


There is some confusion surrounding the definition and interpretation of placebo (c.f. Section~\ref{sec:history-concept}).  The term was retained throughout this research  as it does form a useful overarching construct for this research on the interactions between mental and physical states in health. A definition suitable for the purposes of this thesis was the following:

\begin{quotation}
  a placebo is a treatment believed inert for the specific condition
  concerned which has an effect due to context.
\end{quotation}

The rationale behind this definition (and an explanation of context) is given in Chapter \ref{cha:literature-review}, in Section \ref{sec:concept-placebo}. Note that typically a ``placebo'' refers to the treatment, the ``placebo effect'' is the effect in a placebo treatment arm of some kind, and a ``placebo response'' refers to the impact of a placebo on a particular individual.



\subsection{Treatment Expectancies}
\label{sec:placebo-expectancies}

The current prevailing theory around what causes the placebo effect in humans is the response expectancy model of Kirsch~\cite{Kirsch1985,Kirsch1997}, which suggests that placebo effects are directly mediated by these expectancies, which are defined as ``the expectation of a non-volitional response''.

For Kirsch, these expectancies are regarded as the ultimate causative force behind the placebo response in participants and patients, in that all other factors are mediated by them. This theory was regarded as being confirmed in relation to other theories (especially the conditioning theory of Voudouris~\cite{Voudouris1985}) by a 1997 paper which showed that the large positive effect of surreptitious reduction of pain to induce conditioning could be abolished by informing participants that this was occurring~\cite{Montgomery1997}.

Despite the importance of expectancies in the field at present, no standardised measure of expectancies exists, and indeed a recent study showed that of sixteen placebo trials, only two had an expectancy measure in common~\cite{myers2008patient}. 

Given the expectancy theory's superiority within the field at present, it is important to look more closely at what the term means. Some authors~\cite{Stewart-Williams2004a}  argue that expectancies are necessarily conscious, which is a position which seems improbable, given the lack of awareness that typically accompanies observed placebo effects, and the deception which appears endemic to the field of study~\cite{Miller2008a,Miller2008}, in that participants in experimental studies are deliberately lied to in order to induce a placebo effect. 

Expectancy is a catch-all phrase, and while it appears to have applications in a wide variety of areas~\cite{Montgomery2007} the term is far too broad to focus research specifically. Indeed, a similar ``conditioning'' paradigm has been used to produce ``placebo'' effects on sensory reports~\cite{Sterzer2008}, suggesting that these expectancy related mechanisms may be useful predictors outside the study of the placebo effect. 

% Stewart-Williams et al were criticised for their position on expectancies~\cite{Kirsch2004}  and later retracted it, at least in its strongest form~\cite{Stewart-Williams2004}. 





% Recent research has shown that conditioning and expectancies interact to produce much more sustained placebo effects than either alone . The implications of this, and other research findings discussed in this chapter will be expanded upon in Chapter \ref{cha:notes-towards-theory}. 

% There is some contradictory evidence on the relationship between conditioning and expectancies, as Klinger \textit{et al} \cite{Klinger2007a} found that participants who had been conditioned to respond to placebo actually reported greater pain relief when they were given neutral, rather than positive instructions. This may indicate that the expectancy subsystem can interfere with the conditioning subsystem, suggesting that the two methods may activate neurotransmitters and hormones which regulate one another. That being said, this is not a finding which has been replicated, and so this may just have resulted from the particular suggestions given to the participants in this study, as occurred in the study of Levine and colleagues \cite{Levine2006}. 

The value of the expectancy framework is that it has provided both a common vocabulary and a common mechanism for the measurement of placebo responses. The very broadness of the construct allows for it to be used in a variety of situations, which has ensured its survival in the field. This is also the worst part of the definition, as its wide-ranging applicability coupled with its lack of falsifiable predictions has meant that it is regarded as a theory in the abstract, but research on the determinants and measurement of expectancies is not prominent in the field at present.

One of the largest problems with expectancy research at present within the field of placebo is that the measurement of expectancies tends to be quite superficial. Typically, one question on an 11 point scale is asked and the response to this question is taken as the participants expectancy for the treatment. This conflicts with evidence that expectancies are far more multi-dimensional than this approach suggests \cite{Stone2005}. Indeed, one of the aims of this thesis was to address this problem by developing a better scale for the measurement of expectancies (Chapter \ref{cha:tcq-thesis}). 

Another issue with expectancies and their convergent validity is the relationship between the response expectancies of Kirsch  and the generalised outcome expectancies used in the study of optimism~\cite{Carver2010}. Both of these constructs are similar enough that the use of having both of them needs to be determined, and this analysis was carried out as part of this research (Chapter~\ref{cha:primary-research}). Kirsch also notes that self-efficacy~\cite{Bandura1977} and response expectancies correlate quite highly~\cite{Kirsch1985} and given this, the usefulness of a separate response expectancy framework needs to be examined, especially in the light of some of the findings discussed in Chapter \ref{cha:literature-review} (c.f.~\cite{Geers2005,Hyland2007}, and discussion in Chapter~\ref{cha:health-for-thesis}). Note that as expectancies were the primary focus of this research, discussion of other theories is postponed until the next chapter (see Section \ref{sec:theor-plac-effects}). 


Given that the placebo seems to rely upon the belief of a participant that they have received a real treatment, it seems likely that measures which examine conscious beliefs are not the appropriate kind with which to predict its occurrence. It is the contention of this research that implicit measures may provide a better metric for the prediction of placebo response. 

\subsection{Moving Beyond Self-Report}
\label{sec:implicit-measures}

Psychology as a science, and indeed the social sciences more generally, have a problem. They seek to understand the mind and behaviour of individuals in particular contexts and cultures. While behaviour is less problematic to observe scientifically, the observation of mind is fraught with problems. Almost all of the constructs of interest to psychologists (mind, love, experience) are observable with the naked eye, and require interpretation in order to be understood. 

In the case of many variables, psychologists measure what people think and believe from their answers to questions devised by the psychologist, normally referred to as self-report instruments. 

This approach has obvious advantages, in that it is quick, cost-effective and can produce results of interest. However, as psychology has matured, a number of problems have become apparent with this approach~\cite{Nisbett1977}.  The first problem is that, especially in controversial topics, people may attempt to conceal their true beliefs or attitudes. The second, somewhat more philosophical problem, is that people may be unaware of their true beliefs, or at least may profess to believe one thing while behaving in a manner consistent with a belief in another. 

The first of these problems is known as social desirability~\cite{Egloff2003}, and  self report scales that measure this construct have been developed~\cite{Giebel2008}. The second problem, first noted by Freud, is that of unconscious (or implicit) influences~\cite{Hofmann2008}. While the system built by Freud no longer forms part of the framework of modern psychology, the contradictions between reports of experience and behaviour remain, and are still relevant to the aims of psychology. %%insert allport research on chinese attitudes and behaviour here



\subsection{Older non-self report methods}
\label{sec:older-non-self}

Some methods have been developed to get around this problem. One of the first techniques used for this purpose was that of free association, where a client was asked to respond to a stimulus word or picture with the first word that came to mind, without censoring the experience \cite{Hofmann2008}. These associations could then be used by the therapist to gain access to material which the client did not consciously report being aware of. 

Another method which was used was that of Rorschach ink-blots, where ambiguous ink blots are shown to the client, who interprets them. This technique can also provide insight into the mind of the client, but again this requires interpretation on the part of the therapist. It is this interpretation process that causes these procedures to lack scientific validity in the eyes of many, as what one therapist understands by the clients words may differ completely from what another therapist takes from the same material. 

These approaches were abandoned following the rise of behaviourism as the dominant approach within academic psychology, and further eclipsed by the notions of Karl Popper regarding falsifiability as a criterion for scientific theory. It was argued that since unconscious influences could be used to explain any criticism of the theory (and indeed, Freud was prone to doing this) then the theory was not truly scientific. The development of psychometric theory also played a role in the decline of interest in such instruments, as these methods seemed to produce reliable and valid data and scores could be corrected for impression management and social desirability biases by statistical techniques. 

\subsection{Modern Indirect Measures}
\label{sec:turn-back-indirect}
In recent years, however, there has been a resurgence of interest in such techniques. This resurgence grew out of the work on implicit memory and learning, where participants would consciously deny awareness of some piece of information while their behaviour seemed to show signs of this knowledge.  

This phenomenon can be seen in  experiments like word completion tasks. If participants are given a list of words to memorise, and then distracted by another task, followed by the word completion task, they tend to far more frequently complete the word fragments with the words on the previous list which was to be memorised.  However, they will typically deny this influence on their responding if asked~\cite{Wittenbrink2007a}. 

These approaches, allied with the continuing failure of self reported attitudes to predict behaviour as accurately as might be desired caused some researchers to look for another way to measure these constructs~\cite{Greenwald1995a}. The result of these investigations was the Implicit Association Test, or IAT for short~\cite{Greenwald1998}. The IAT is a reaction time measure which makes inferences about attitudes from the time which it takes participants to categorise words or pictures on a related theme into one of two categories. The difference between the participants response latencies is taken to indicate the relative strength of associations, and this difference is known as an IAT effect.  


\subsection{Introduction to the Implicit Association Test (IAT)}
\label{sec:intr-impl-assoc}

The IAT was developed by Greenwald~\textit{et al}, and he suggested that because of its design, it might be more resistant to social desirability influences and demand characteristics~\cite{Greenwald1998}. 

Social desirability tendencies would lead people to deny prejudicial behaviour in self report instruments, while as a reaction time measure, the IAT is less easy to fake. Demand characteristics result when participants in an experiment give the answers a researcher wants, rather than their real beliefs or attitudes. Again, it seems intuitively harder to do this within an IAT methodology as it would require tremendous and consistent control of reaction times to stimuli (though not impossible~\cite{DeHouwer2007b}). 

\subsection{Description of the Procedure}
\label{sec:descr-proc}

The (IAT) is a computer administered procedure which purports to measure implicit associations not directly accessible to consciousness \cite{Greenwald1998}. The test was developed as a result of mounting evidence for learning without awareness in human participants, as discussed above. 

This research was reviewed by Greenwald \& Banaji \cite{Greenwald1995a}, where they introduced a distinction between direct and indirect measures of social cognition. They referred to self report instruments as direct measures, and to such techniques as semantic priming as indirect measures. Semantic priming is the tendency for participants in experiments to give answers to ambiguous tasks similar to ones they have recently observed in their environment \cite{Wittenbrink2007}.  They defined implicit associations in the following manner as \textit{the unidentified or inaccurately identified traces of past experience}, and this definition implied that self report measures were not the best tools with which to assess these associations. 

The procedure works as follows. Firstly participants sit down in front of a computer and are assigned two keys (typically the ``e'' and ``i'' keys) to respond to each word or image presented, which fall into one of two categories. The metric examined is reaction time, and an assumption of the method is that categories which are more strongly associated will be easier to combine than those which are less associated.  

The participant is asked to classify words into either pleasant or unpleasant categories as they appear on the screen at the front of the computer, for example love, hate, good  and bad are words typically used in this part of the procedure. In the most well known IAT, the Race-IAT, the procedure is as follows:

Firstly, participants are asked to categorise faces into as either being ``Black'' or ``White''\footnote{Following some research around this, the categories were later re-named to African-American and Caucasian-American}.
 Following this, the two categories are combined, with one key being pressed for White or Pleasant and another being pressed for Black or Unpleasant. Then, the labels are reversed, and the participant categorises White faces with Unpleasant and Black faces with Pleasant. The original authors described the White + Pleasant trials as congruent, and the Black + Pleasant/White + Unpleasant trials as incongruent.  These response times are summed and averaged for each participant for each of the critical blocks, and the two block scores (Incongruent - Congruent) are subtracted from one another to produce a difference score which is referred to as an IAT effect. 

The procedure is not limited to assessment of racial attitudes, and has been applied far more widely \cite{Craeynest2008,Greenwald2009, Schmukle2008,Walker2008}.  A general schema for the process follows.   

Firstly participants classify words as either belonging to Category X or Y, where X and Y are positively (love, flowers etc) or negatively (hate, insects etc) associated words or images.  Then they classify faces or words
as either belonging to Group A or Group B, where the words are often descriptive of groups of people.  In the next step, these two associations are combined, with one key being a response for A and X and the other key being used for responses of B and Y. 

In the fourth step, the keys for pleasant and unpleasant are reversed, and in the final step the two dimensions are combined in the opposite manner (A and Y or B and X). In practice, only the 3rd and the 5th steps are analysed, and the difference between mean response latencies on the different combination tasks is known as an IAT effect~\cite{Greenwald1998}. 

In essence, any differences in reaction time in the combination of the two categories are assumed to be due to underlying differences between the relative associations of the concepts. The authors claim that the use of difference scores allows them to prevent issues of processing and response speed variability across individuals from distorting the results. However, this assumption has been questioned \cite{Blanton2006}, who claimed that processing speed was a major moderator of the observed variance in scores across the population studied. Controls were taken in the analysis of IAT scores to ensure that such effects did not impact the results (c.f. Chapter~\ref{cha:primary-research}). 

\section{Structure of this Thesis}
\label{sec:struct-this-thes}



This thesis follows the following structure:

\begin{enumerate}
\item Chapter \ref{cha:literature-review} reviews the literature surrounding the placebo response and explicit and implicit expectancies;

\item Chapter \ref{cha:methodology} describes a model for how the placebo effect and implicit and explicit expectancies inter-relate, and describes the approaches taken in this thesis for both the construction of measures and the testing of these models;

% \item Chapter \ref{cha:notes-towards-theory} describes a perspective on the placebo response informed by Chapters \ref{cha:literature-review} and \ref{cha:methodology}. This perspective is operationalised in the form of a theoretical model which is then tested in Chapter \ref{cha:primary-research}

\item Chapter \ref{cha:health-for-thesis} reports the analysis of self report variables which have been associated with placebo effect (optimism) as well as with response to implicit measures (mindfulness), and the construction of tailored scales for the population at hand;

\item Chapter \ref{cha:tcq-thesis} details the development and validation of the Treatment Credibility Questionnaire, the self report method used to assess expectancies;

\item Chapter \ref{cha:devel-impl-meas} describes the approach taken to the development of the stimuli for the implicit association tests, using multiple methods;

\item Chapter \ref{cha:primary-research} reports upon the experimental portion of the research, and models these results incorporating the models developed for the self report measures in Chapters \ref{cha:health-for-thesis} and \ref{cha:tcq-thesis};

\item Finally, Chapter \ref{cha:general-discussion} summarises what has been learned from this thesis, and gives directions towards future research.

\end{enumerate}



% The central aim of this thesis was to apply new psychometric methods to the prediction of the placebo response to pain in healthy volunteers. The new psychometric method applied to the prediction of placebo was the Implicit Association Test, or IAT, which is a reaction time based categorisation method developed in the last fifteen years. 

% It is the contention of this research that the alleged unpredictability of the placebo effect is a function of our measurement tool and our statistical methods, and that new and improved measurement tools can furnish us with a placebo effect which can be reliably assessed and predicted. 

% Although some form of placebo effect has been an integral part of medicine for at least the last 2000 years, it is only over the past sixty years that it has been conceptualised and operationalised in such a form as to make my research possible. 

% Indeed, the major tool proposed by this research for the prediction of the placebo response is itself a more recent development than the construction of a defined placebo response. Implicit measures are further defined in the later sections of this work, but for now we can model them as associations which lie outside our awareness much of the time which colour many aspects of our social lives. In another age and time, these measures would have been called unconscious, but such a conceptualisation runs into the tricky problems of unfalsifiability and so, nowadays, most researchers prefer to substitute the term implicit. 


% Of course, this short introduction merely scratches the surface of what has been a hugely rewarding, frustrating and interesting research experience. Mere words cannot convey all that has occurred, but given the format required of us by the scientific community at this time, they shall have to do. The plan for this section of the thesis is as follows. 
% The structure of the thesis shall be as follows:

% \begin{enumerate}
% \item The literature on the placebo effect and implicit measures shall be reviewed, as will the literature of constructs which appear to have been predictive of the placebo effect in the past. 

% \item The methodology for the thesis shall then be described.

% \item The results of a preliminary study relating health, optimism and mindfulness (all of which have been related to placebo or implicit measures) shall be described. 

% \item The development and assessment of a measure designed to tap response expectancies to pain treatments shall be described. 


% \item The development of the implicit measures shall be described, including a process of interviewing health professionals and a method of developing IAT stimuli for arbitrary constructs. 


% \item Next, the experimental data will be analysed, and the usefulness of the models and measures used assessed. As part of this, the inter-relationships between the implicit, explicit, physiological and behavioural measures will be examined.


% \item Finally, the thesis will conclude with a discussion of what has been discovered, and plans for future research. 
% \end{enumerate}








% The central aim of this thesis was to examine of measurement of psychological constructs in the context of the placebo effect. This thesis used self-report, interview, implicit and physiological variables in an examination of usefulness of these different kinds of predictors for  the placebo effect. 

% % This thesis contributes a more comprehensive investigation of a psychological phenomenon from multiple methods of data collection, and could serve as a model for similiar approaches to other constructs. 

% The major contribution of this thesis is that it uses large samples to develop psychometric models of responses to self-report instruments, and then applies these models to help estimate effects in an experimental sample more accurately. 

% In addition, this thesis developed two new implicit measures -- an Optimism Implicit Association Test (IAT) and a Treatment Credibility Implicit Association Test. Self report measures of both of these constructs were collected also, and this allowed for the relative usefulness of these different forms of measurement to be determined in the prediction of the placebo response. 

% Finally, this thesis  made use of more sensitive and appropriate statistical models and methods, and applied these methods to the more precise estimation of placebo effects in an experimental sample of healthy volunteers drawn from the sample population. 


% The aims of this research were as follows:


% \begin{enumerate}

% \item To use multiple methods to allow for better prediction of the factors involved with the placebo response. 

% \item To compare the predictive power of implicitly and explicitly measured constructs as predictors the placebo response. 

% \item To develop a multifaceted method of assessing explicit expectancies which can be linked to the placebo response

% \item To create an implicit measure to assess expectancies related to treatment and to the placebo response

% \item To conduct one large experimental study to test the utility of the measurement and psychometric approaches taken. 

% % \item To integrate physiological data as another signal for the assessment of placebo response. 
% \end{enumerate}

% This thesis follows the following structure:

% \begin{enumerate}
% \item Chapter \ref{cha:literature-review} reviews the literature surrounding the placebo response, implicit measurement and self report. 

% \item Chapter \ref{cha:methodology} describes the approaches taken towards measurement in this thesis, as well as the research designs.

% \item Chapter \ref{cha:notes-towards-theory} describes a perspective on the placebo response informed by Chapters \ref{cha:literature-review} and \ref{cha:methodology}. This perspective is operationalised in the form of a theoretical model which is then tested in Chapter \ref{cha:primary-research}

% \item Chapter \ref{cha:health-for-thesis} describes the analysis of self report variables which have been associated with placebo effect (optimism) as well as with response to implicit measures (mindfulness). 

% \item Chapter \ref{cha:tcq-thesis} describes the development and validation of the Treatment Credibility Questionnaire, the self report method used to assess expectancies.

% \item Chapter \ref{cha:devel-impl-meas} describes the approach taken to the development of the stimuli for the implicit association tests, using multiple methods. 

% \item Chapter \ref{cha:primary-research} describes the experimental portion of the research, and models these results incorporating the models developed for the self report measures in Chapters \ref{cha:health-for-thesis} and \ref{cha:tcq-thesis}

% \item Finally, Chapter \ref{cha:general-discussion} summarises what has been learned from this thesis, and gives directions towards future research.

% \end{enumerate}



% % The central aim of this thesis was to apply new psychometric methods to the prediction of the placebo response to pain in healthy volunteers. The new psychometric method applied to the prediction of placebo was the Implicit Association Test, or IAT, which is a reaction time based categorisation method developed in the last fifteen years. 

% % It is the contention of this research that the alleged unpredictability of the placebo effect is a function of our measurement tool and our statistical methods, and that new and improved measurement tools can furnish us with a placebo effect which can be reliably assessed and predicted. 

% % Although some form of placebo effect has been an integral part of medicine for at least the last 2000 years, it is only over the past sixty years that it has been conceptualised and operationalised in such a form as to make my research possible. 

% % Indeed, the major tool proposed by this research for the prediction of the placebo response is itself a more recent development than the construction of a defined placebo response. Implicit measures are further defined in the later sections of this work, but for now we can model them as associations which lie outside our awareness much of the time which colour many aspects of our social lives. In another age and time, these measures would have been called unconscious, but such a conceptualisation runs into the tricky problems of unfalsifiability and so, nowadays, most researchers prefer to substitute the term implicit. 


% % Of course, this short introduction merely scratches the surface of what has been a hugely rewarding, frustrating and interesting research experience. Mere words cannot convey all that has occurred, but given the format required of us by the scientific community at this time, they shall have to do. The plan for this section of the thesis is as follows. 
% % The structure of the thesis shall be as follows:

% % \begin{enumerate}
% % \item The literature on the placebo effect and implicit measures shall be reviewed, as will the literature of constructs which appear to have been predictive of the placebo effect in the past. 

% % \item The methodology for the thesis shall then be described.

% % \item The results of a preliminary study relating health, optimism and mindfulness (all of which have been related to placebo or implicit measures) shall be described. 

% % \item The development and assessment of a measure designed to tap response expectancies to pain treatments shall be described. 


% % \item The development of the implicit measures shall be described, including a process of interviewing health professionals and a method of developing IAT stimuli for arbitrary constructs. 


% % \item Next, the experimental data will be analysed, and the usefulness of the models and measures used assessed. As part of this, the inter-relationships between the implicit, explicit, physiological and behavioural measures will be examined.


% % \item Finally, the thesis will conclude with a discussion of what has been discovered, and plans for future research. 
% % \end{enumerate}



%%% Local Variables:
%%% TeX-master: "PlaceboMeasurementByMultipleMethods"
%%% End:
