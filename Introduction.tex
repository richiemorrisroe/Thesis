\chapter{Introduction}

Science is an ever evolving process. Isaac Newton is reputed to have said that "if I have seen further than other men, it is because I am standing on the shoulders of giants". If such an aphorism were true in the 16th century, it is even truer today, in these days of global communication links and thousands of peer reviewed journals covering almost every possible subject and interest. 

It is through this evolving process that each generation of researchers can pose new problems, and test them against the cold rock of reality to see which will flounder and which will flourish. The prediction of the placebo effect is one such problem. Although some form of placebo effect has been an integral part of medicine for at least the last 2000 years, it is only over the past sixty years that it has been conceptualised and operationalised in such a form as to make my research possible. It is the contention of this research that the alleged unpredictability of the placebo effect is a function of our measurement tools, and that new and improved measurement tools can furnish us with a placebo effect which can be reliably assessed and predicted. 

Indeed, the major tool proposed by this research for the prediction of the placebo response is itself a more recent development than the construction of a defined placebo response. This tool is that of the implicit measure, and it burst upon the scientific scene less than 20 years ago. Implicit measures are further (indeed, exhaustively) defined in the later sections of this work, but for now we can model them as associations which lie outside our awareness much of the time, the fast and frugal operations of our inner computer which colour many aspects of our social lives. In another age and time, these measures would have been called unconscious, but such a conceptualisation runs into the tricky problems of unfalsifiability and so, nowadays, most researchers prefer to substitute the term implicit. 

The aims of this research were as follows:
1)	To create an implicit measure to assess expectancies related to treatment and to the placebo response
2)	To develop a multifaceted method of assessing explicit expectancies which can be linked to the placebo response
3)	To conduct two experimental studies which aim to test the utility of this approach and contribute to the sum of human knowledge

Of course, this short introduction merely scratches the surface of what has been a hugely rewarding, frustrating and interesting research experience. Mere words cannot convey all that has occurred, but given the format required of us by the scientific community at this time, they shall have to do. The plan for this section of the thesis is as follows. Firstly, the placebo concept shall be defined, and then research which has a bearing on this shall be examined and reviewed. Following this, implicit measures shall be analysed and reviewed in order that we may understand the approach taken by this researcher. Finally, the process of developing the implicit measure will be reviewed. 

%%% Local Variables:
%%% TeX-master: "ThesisContents030511"
%%% End: