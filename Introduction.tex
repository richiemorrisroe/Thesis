
\chapter{Introduction}



The central aim of this thesis was to examine the measurement of treatment expectancies in the context of the placebo effect. This thesis used self-report, interview, implicit and physiological variables to deliver greater understanding of these psychological constructs and their relationship with the placebo effect.

This thesis focused both on the explicit expectancies common to much psychological research (i.e. self report scales) and the implicit expectancies measured by such tools as the Implicit Association Test \cite{Greenwald1998}. 

This thesis developed three new measures to aid in this task. One was a more comprehensive and multi-dimensional instrument for the measurement of treatment related explicit expectancies,  while the other two were implicit measures -- an Optimism Implicit Association Test (IAT) and a Treatment Credibility Implicit Association Test. 

Additionally, self report measures of both optimism (related to the placebo effect and a construct commonly conceptualised as expectancy related \cite{Carver2010}; and mindfulness which is both a construct of interest in terms of treatment in its own right \cite{kabat1982outpatient} and one which has been found to moderate the relationship between explicit and implicit measures of the same construct were collected from the same sample as from which the experimental population were drawn, to control for potential covariates. 

The use of both implicit and explicit measures of the same construct, all of which were validated in the population from which the experimental sample was drawn allowed for the relative impact of both the constructs and the different forms of measurement of them to be assessed. 


Another contribution of this thesis is that it uses large samples to develop psychometric models of responses to self-report instruments, and then applies these models to help estimate effects in an experimental sample more accurately. 



% Finally, this thesis  made use of more sensitive and appropriate statistical models and methods, and applied these methods to the more precise estimation of placebo effects in an experimental sample of healthy volunteers drawn from the sample population. 


The aims of this research were as follows:


\begin{enumerate}

% \item To use multiple methods to allow for better prediction of the factors involved with the placebo response. 

\item To develop a multifaceted method of assessing explicit expectancies which can be linked to the placebo response

\item To create an implicit measure to assess expectancies related to treatment and to the placebo response

\item To compare the predictive power of implicitly and explicitly measured constructs as predictors the placebo response. 

\item To conduct one large experimental study using a placebo analgesia paradigm to test the major hypotheses of the thesis.


\end{enumerate}

This thesis follows the following structure:

\begin{enumerate}
\item Chapter \ref{cha:literature-review} reviews the literature surrounding the placebo response and explicit and implicit expectancies. 

\item Chapter \ref{cha:methodology} describes a model for how the placebo effect and implicit and explicit expectancies inter-relate, and descibes the approaches taken in this thesis for both the construction of measures and the testing of this model. 

% \item Chapter \ref{cha:notes-towards-theory} describes a perspective on the placebo response informed by Chapters \ref{cha:literature-review} and \ref{cha:methodology}. This perspective is operationalised in the form of a theoretical model which is then tested in Chapter \ref{cha:primary-research}

\item Chapter \ref{cha:health-for-thesis} describes the analysis of self report variables which have been associated with placebo effect (optimism) as well as with response to implicit measures (mindfulness), and the construction of tailored scales for the population at hand. 

\item Chapter \ref{cha:tcq-thesis} describes the development and validation of the Treatment Credibility Questionnaire, the self report method used to assess expectancies.

\item Chapter \ref{cha:devel-impl-meas} describes the approach taken to the development of the stimuli for the implicit association tests, using multiple methods (interviews and repertory grids).

\item Chapter \ref{cha:primary-research} describes the experimental portion of the research, and models these results incorporating the models developed for the self report measures in Chapters \ref{cha:health-for-thesis} and \ref{cha:tcq-thesis}

\item Finally, Chapter \ref{cha:general-discussion} summarises what has been learned from this thesis, and gives directions towards future research.

\end{enumerate}



% The central aim of this thesis was to apply new psychometric methods to the prediction of the placebo response to pain in healthy volunteers. The new psychometric method applied to the prediction of placebo was the Implicit Association Test, or IAT, which is a reaction time based categorisation method developed in the last fifteen years. 

% It is the contention of this research that the alleged unpredictability of the placebo effect is a function of our measurement tool and our statistical methods, and that new and improved measurement tools can furnish us with a placebo effect which can be reliably assessed and predicted. 

% Although some form of placebo effect has been an integral part of medicine for at least the last 2000 years, it is only over the past sixty years that it has been conceptualised and operationalised in such a form as to make my research possible. 

% Indeed, the major tool proposed by this research for the prediction of the placebo response is itself a more recent development than the construction of a defined placebo response. Implicit measures are further defined in the later sections of this work, but for now we can model them as associations which lie outside our awareness much of the time which colour many aspects of our social lives. In another age and time, these measures would have been called unconscious, but such a conceptualisation runs into the tricky problems of unfalsifiability and so, nowadays, most researchers prefer to substitute the term implicit. 


% Of course, this short introduction merely scratches the surface of what has been a hugely rewarding, frustrating and interesting research experience. Mere words cannot convey all that has occurred, but given the format required of us by the scientific community at this time, they shall have to do. The plan for this section of the thesis is as follows. 
% The structure of the thesis shall be as follows:

% \begin{enumerate}
% \item The literature on the placebo effect and implicit measures shall be reviewed, as will the literature of constructs which appear to have been predictive of the placebo effect in the past. 

% \item The methodology for the thesis shall then be described.

% \item The results of a preliminary study relating health, optimism and mindfulness (all of which have been related to placebo or implicit measures) shall be described. 

% \item The development and assessment of a measure designed to tap response expectancies to pain treatments shall be described. 


% \item The development of the implicit measures shall be described, including a process of interviewing health professionals and a method of developing IAT stimuli for arbitrary constructs. 


% \item Next, the experimental data will be analysed, and the usefulness of the models and measures used assessed. As part of this, the inter-relationships between the implicit, explicit, physiological and behavioural measures will be examined.


% \item Finally, the thesis will conclude with a discussion of what has been discovered, and plans for future research. 
% \end{enumerate}








% The central aim of this thesis was to examine of measurement of psychological constructs in the context of the placebo effect. This thesis used self-report, interview, implicit and physiological variables in an examination of usefulness of these different kinds of predictors for  the placebo effect. 

% % This thesis contributes a more comprehensive investigation of a psychological phenomenon from multiple methods of data collection, and could serve as a model for similiar approaches to other constructs. 

% The major contribution of this thesis is that it uses large samples to develop psychometric models of responses to self-report instruments, and then applies these models to help estimate effects in an experimental sample more accurately. 

% In addition, this thesis developed two new implicit measures -- an Optimism Implicit Association Test (IAT) and a Treatment Credibility Implicit Association Test. Self report measures of both of these constructs were collected also, and this allowed for the relative usefulness of these different forms of measurement to be determined in the prediction of the placebo response. 

% Finally, this thesis  made use of more sensitive and appropriate statistical models and methods, and applied these methods to the more precise estimation of placebo effects in an experimental sample of healthy volunteers drawn from the sample population. 


% The aims of this research were as follows:


% \begin{enumerate}

% \item To use multiple methods to allow for better prediction of the factors involved with the placebo response. 

% \item To compare the predictive power of implicitly and explicitly measured constructs as predictors the placebo response. 

% \item To develop a multifaceted method of assessing explicit expectancies which can be linked to the placebo response

% \item To create an implicit measure to assess expectancies related to treatment and to the placebo response

% \item To conduct one large experimental study to test the utility of the measurement and psychometric approaches taken. 

% % \item To integrate physiological data as another signal for the assessment of placebo response. 
% \end{enumerate}

% This thesis follows the following structure:

% \begin{enumerate}
% \item Chapter \ref{cha:literature-review} reviews the literature surrounding the placebo response, implicit measurement and self report. 

% \item Chapter \ref{cha:methodology} describes the approaches taken towards measurement in this thesis, as well as the research designs.

% \item Chapter \ref{cha:notes-towards-theory} describes a perspective on the placebo response informed by Chapters \ref{cha:literature-review} and \ref{cha:methodology}. This perspective is operationalised in the form of a theoretical model which is then tested in Chapter \ref{cha:primary-research}

% \item Chapter \ref{cha:health-for-thesis} describes the analysis of self report variables which have been associated with placebo effect (optimism) as well as with response to implicit measures (mindfulness). 

% \item Chapter \ref{cha:tcq-thesis} describes the development and validation of the Treatment Credibility Questionnaire, the self report method used to assess expectancies.

% \item Chapter \ref{cha:devel-impl-meas} describes the approach taken to the development of the stimuli for the implicit association tests, using multiple methods. 

% \item Chapter \ref{cha:primary-research} describes the experimental portion of the research, and models these results incorporating the models developed for the self report measures in Chapters \ref{cha:health-for-thesis} and \ref{cha:tcq-thesis}

% \item Finally, Chapter \ref{cha:general-discussion} summarises what has been learned from this thesis, and gives directions towards future research.

% \end{enumerate}



% % The central aim of this thesis was to apply new psychometric methods to the prediction of the placebo response to pain in healthy volunteers. The new psychometric method applied to the prediction of placebo was the Implicit Association Test, or IAT, which is a reaction time based categorisation method developed in the last fifteen years. 

% % It is the contention of this research that the alleged unpredictability of the placebo effect is a function of our measurement tool and our statistical methods, and that new and improved measurement tools can furnish us with a placebo effect which can be reliably assessed and predicted. 

% % Although some form of placebo effect has been an integral part of medicine for at least the last 2000 years, it is only over the past sixty years that it has been conceptualised and operationalised in such a form as to make my research possible. 

% % Indeed, the major tool proposed by this research for the prediction of the placebo response is itself a more recent development than the construction of a defined placebo response. Implicit measures are further defined in the later sections of this work, but for now we can model them as associations which lie outside our awareness much of the time which colour many aspects of our social lives. In another age and time, these measures would have been called unconscious, but such a conceptualisation runs into the tricky problems of unfalsifiability and so, nowadays, most researchers prefer to substitute the term implicit. 


% % Of course, this short introduction merely scratches the surface of what has been a hugely rewarding, frustrating and interesting research experience. Mere words cannot convey all that has occurred, but given the format required of us by the scientific community at this time, they shall have to do. The plan for this section of the thesis is as follows. 
% % The structure of the thesis shall be as follows:

% % \begin{enumerate}
% % \item The literature on the placebo effect and implicit measures shall be reviewed, as will the literature of constructs which appear to have been predictive of the placebo effect in the past. 

% % \item The methodology for the thesis shall then be described.

% % \item The results of a preliminary study relating health, optimism and mindfulness (all of which have been related to placebo or implicit measures) shall be described. 

% % \item The development and assessment of a measure designed to tap response expectancies to pain treatments shall be described. 


% % \item The development of the implicit measures shall be described, including a process of interviewing health professionals and a method of developing IAT stimuli for arbitrary constructs. 


% % \item Next, the experimental data will be analysed, and the usefulness of the models and measures used assessed. As part of this, the inter-relationships between the implicit, explicit, physiological and behavioural measures will be examined.


% % \item Finally, the thesis will conclude with a discussion of what has been discovered, and plans for future research. 
% % \end{enumerate}



%%% Local Variables:
%%% TeX-master: "ThesisContents030511"
%%% End:
