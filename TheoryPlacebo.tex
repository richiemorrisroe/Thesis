






\section{Introduction to the theory}

Theories form an indispensible part of science. They represent an attempt to generalise beyond particular forms of evidence and data and to derive some kinds of overall principles which lie behind the oberved events. The major theories behind the placebo and implicit measures were reviewed in the last chapter, and here there is an attempt to synthesise all of this information into a coherent whole. The major building blocks of this theory are as follows:
\begin{enumerate}
\item The results of implicit measures point towards there being at least two systems of attitude assessment and evaluation of stimuli in the human mind
\item The placebo effect is typically conceptualised as a conscious phenomenon, in spite of the experimental evidence
\item There appear to be feedback loops between bodily sensations and conscious perception - embodied cognition
\item These feedback loops are evidence against an additive model of drug-placebo interactions
\item Despite the centrality of such conceptions to the placebo effect, no theory has as yet incorporated these findings into their theories
\end{enumerate}

This section will briefly review the evidence in favour of the above propositions, then will elucidate how these could be combined into our theories of placebo and implicit measures, and will provide some testable hypotheses regarding the theory, in the spirit of Popperian falsification.

\subsection{Implicit Measures and Dual Process Models of Mind}

The notion of dual process models of mind is an old one, dating back within psychology to at least the time of Freud, and possibly before. However, the modern conception of dual process models is much more recent, and developed as a result of work with implicit measures and through the findings of cognitive psychology. Essentially, the modern theory suggests that there are two prevalant systems of reasoning inherent to humans, a slow, rational, conscious system, and a fast, frugal and implicit system. 

These two systems activate under different conditions and seem to perform different functions. One of the major hypotheses emerging from this theory is that under conditions of attentional strain, the implicit system takes over. This is, as we have seen in the previous chapter, borne out by much of the research into implicit attitudes. Asendorpf \cite{Asendorpf2002} demonstrated what came to be called the double-dissociation effect, where implicit measures (of shyness, in this case) were more predictive of performance on a spontaneous speaking task (the Trier social stress test), while explicit measures were more predictive of considered, deliberate behaviour. Thus, we can take the results of the IAT research and that into other implicit measures as pointers towards the operation of this system. The major point to take from this section is that this system appears to exist, and yet has only been touched upon in one or two articles over the past decade \cite{Geers2005}. 

\subsection{Conscious Conceptions of the Placebo}

The dominant model within placebo research at present is the response expectancy theory of Kirsch \cite{Kirsch1985, Kirsch1997a}. This theory conceptualises the placebo as resulting from response expectancies, which are defined as ``the conscious expectation of a non-volitional response''. This theory has had some success, displacing the then prevalent theory of conditioned placebo responses \cite{Vuodouris1985}. That being said, the very definition of placebo and its nature as occuring as a result of deception and belief that one is getting a real drug would seem to suggest that non-conscious systems must be centrally involved in the mediation between awareness and the documented physical responses. 

Indeed, the Geers et al study cited earlier \cite{Geers2005} demonstrated that semantic priming (by means of a scrambled sentence task) was an independent predictor of the response to a sleep placebo, which given that semantic priming does not effect conscious awareness, implies that such priming (and the implicit system more generally) can affect the response to placebos and indeed biologically active treatments. Another study which points in the same direction is the Shiv et al 2005 \cite{Shiv2005a} study which demonstrated an effect of price of an energy drink effecting the number of puzzles solved by participants in a particular time. This effect disappeared when participants attention was drawn to it, which again implies that implicit systems were involved. Despite this evidence, the dominant theoretical framework remains untouched. In this chapter, I will propose a new model that incorporates both of these systems, and makes a number of predictions that will be either supported or rejected by further research. 

\subsection{Embodied Cognition and Placebo}

Another issue with most of the conceptions of placebo current in today's research is that they are almost exclusively cognitive, a point made most forcefully by anthropology researchers \cite{} %Add this citation. 
This seems strange, given that it rests on assumptions regarding the relationship of mind and body which all of the evidence of placebo would seem to argue against. It seems that the theoretical perspective within the field is that the mind may effect the body, but not vice versa. Unortunately, this is an untenable proposition, as recent work on haptic cognition (where judgements made by participants are affected by the sensory input they are receiving at the time), published in Science in 2010 %get this paper too
Indeed, more recent work by Cwir \cite{Cwir2011} suggests that awareness of one's own interioceptive processes appears to be a good predictor of people's ability to predict the emotional states of others. 



Indeed, there has been a resurgance of interest in mind-brain-body feedback loops within psychology and the social sciences more generally of late. It seems that the dominant cognitive model of mind (a kind of splendid isolation for the brain) is being slowly worn down by experimental evidence. An approach such as this was also proposed by Meissner et al in 2009 \cite{Meissner2009} in their meta-analysis which showed large placebo effects in some areas and none in others. They suggested that the larger placebo effects may have occurred in some conditions but not in others as the places in which placebo effects occurred tended to be those systems which have large nervous system connections with the brain, as opposed to communication typically mediated through hormones, which are many orders of magnitude slower. One problem with Meissner's theory is that he posits that placebo effects do not occur in clinical trials when the outcome of interest is a hormone such as cortisol. Howevever, the field of psycho-neuro-immunology has noted that psychological variables (specifically optimism) exert major influences on antibody responses. There is also some emerging evidence that mindfulness based treatments appear to affect antibody responses, at least in cancer patients. This begs the question of why the analysis by Meisnner et al found no such effect. One explanation for these conflicting findings would be that the placebo effect is mostly determined by current state effects, while those effects investigated by psycho-neuro-immunology are the result of certain trait like characteristics of individuals.  

One psychological state which may have an impact on placebo responses, and should if my theory is to hold is that of mindfulness. Mindfulness is often defined as a moment to moment awareness of somatic and mental processes. If, as the results of Geers et al suggest, somatic focus can increase the size of placebo effects, then mindfulness (or another proxy marker for somatic focus) should also moderate the size of observed placebo responses. One problem with this hypothesis is that it is not clear in which direction the relationship should go. One could make an argument that mindfulness could reduce the size of placebo effects, as if the feedback loop is made conscious, then it should lose its power, or one could argue that higher levels of trait mindfulness would allow for more attention to be placed on the somatic sensation, thus increasing the placebo response. This presents a difficulty in the attempt to lay out bold conjectures and then attempt to refute them. 

If attention is conceptualised as a limited resource, then placebo effects should be enhanced by placing participants in conditions of low stimulation while the treatment is applied. However, this hypothesis causes problems when we considered (below) the effects of participant provider interaction, which appear to account for a significant portion of observed placebo effects. 

\subsection{Additive Models of Placebo}

The common approach throughout clinical trials, and the study of placebo effects more generally, is that the effects of drug and placebo are additive. This assumption leads nicely to the principle that placebo effects and drug effects can be seperated precisely. However, not all of the evidence points in this direction, and (apart from statistical convenience) there exists no a-priori reason why this should be the case. Indeed, the work of Geers et al on somatic focus would seem to suggest that paying attention to somatic experience can increase the size of placebo effects. This could be occuring because of a feedback loop whereby a treatment is applied, the patient's awareness of the treatment leads them to attend to sensations related to the treatment, which engages the body's own healing systems, which then increase the size of the effect over and above what would have happened without awareness of treatment on the part of the patient. It is perhaps for this reason that drugs which are adminstered by an automated process are less effective \cite{Benedetti2003}. 

Another issue to consider in terms of mathematical models of placebo response is the phenomenon of the active placebo. This is where an active drug is offered as a treatment for which it has no efficacy, and nonetheless this treatment produces larger healing effects than a typical sugar pill placebo. I would argue that this phenomenon occurs because the side-effects of the drug produce  a feedback loop whereby the treatment has a somatic impact, which alerts participants to the treatment, causes them to accept it as more credible, and thus activates the body's own healing systems. This process, over time, could easily create a conditioned response to the original non-effective treatment, and loops such as this could be responsible for the observations regarding the efficacy of conditioned placebo responses. In this sense, I am making the argument that conditioned placebo responses are turned into expectancies over time.   


\subsection{Social Aspects of Placebo}

In addition to the possibility of feedback loops between awareness and sensation contributing to the placebo effect, the role of the provider needs to be emphasised also. In most drug research, the treatment itself (the pill or cream) is only one element of the context in which the healing process takes place. In addition to the internal factors which shape response to treatment (optimism and expectancies more generally), there are also important social and environmental factors. For instance, the classic work of Gracely et al \cite{Gracely1985} demonstrated that the effects of the awareness of the provider can have a large impact on the outcome. In this study, half of the dentists were informed that half of the patients would receive placebo. 

In fact, all of them received the real painkiller. However, compared to the group treated by dentists who had not been told that placebo was a possibility, the other patients reported significantly higher pain. Indeed, a systematic review of healthcare interventions \cite{diblasi2001} provided evidence that provider characteristics accounted for a large proportion of the observed placebo effects. More recently \cite{Kaptchuk2008}, a three armed randomised control trial of acupuncture demonstrated that healing rates were greatly increased when the provider spent more time with the patient discussing symptoms and treatment (45 minutes as opposed to 15). This would seem to suggest that one of the reasons so many people use alternative treatments is not the efficacy of the particular form of treatment, but rather the chance to discuss their symptoms in detail and for longer period of time, while the average GP time per patient is currently around 5 minutes in the UK and Ireland. 

Another important feature of the context in which healing takes place is the social context, that is the shared beliefs and rituals that make up a culture. For instance, Valium has a powerful resonance in our culture, immortalised in songs by the Rolling Stones are glorified through media et al. Therefore, even when participants have never experienced the drug itself, they bring pre-conceived notions of what it can do, and apply these to their perception of treatment, which alters its efficacy. However, research using the open-hidden paradigm (discussed previously) would seem to suggest that Valium lacks any efficacy when participants are not aware that they are taking it. An additional example of this effect can be seen in the prescription of antibiotics for viral infections. Even though doctors know that they will have no effect, they are still administered to patients. It would be extremely interesting to give people placebo antibiotics and assess its impact on viral infections, relative to the efficacy of true antibiotics. My thesis would suggest that real antibiotics should be slightly more effective, given their obvious side-effects, but that this difference in standardized means would be less than 0.1. 

% A final (in my estimation) variable on placebo response would seem to be the effects of environmental factors. There is some evidence that suggests that a view of nature decreases healing times, and that patients in rooms with sunlight recover faster than those in rooms 
% without a direct view of the sun. It can be argued that this is as a result of environmental influences activating particular implicit cognitions relating to health. It is worth noting that the sun has been consistently associated with health in many cultures across the globe, and that people's mood tends to improve when the weather is nice. 

\subsection{Towards an Embodied Conception of Placebo}

Bearing the previous sections in mind, we can now move forward into proposing a new model for placebo, the embodied placebo model. This model, while not rejecting outright the findings of the expectancy theory, aims to enhance it with more recent research. Indeed, this theory is also compatible with that of conditioned placebo, and also with the motivational concordance approach of Hyland et al \cite{Hyland2007}.  
The essential features of the embodied model of placebo are as follows. Placebo effects are those healing effects which arise from the perception of treatment or caring on the part of the health care provider or from the context surrounding the treatment. They are mediated by four different kinds of factors.
\begin{itemize}
\item They are mediated by implicit cognitions operating extremely quickly
\item They are also mediated by awareness of the treatment and cognitive models of its efficacy
\item In addition, these implicit and explicit attitudes are enhanced or degrenated by somatic sensations
\item They are mediated by the communications exchanged between the participant and the provider
\item They are also mediated by the system of communications surrounding both the participant and the provider
\item Finally, they are mediated by somatic feedback from the surrounding environment.
\end{itemize}
 
In an effort to advance research and to conclusively demonstrate either my own percipience or ignorance, I shall argue that my theory makes a number of key predictions. 

\begin{itemize}
\item First, that placebo effects will be correlate with scores on implicit measures
\item Second, that there will be an interaction effect between explicit and implicit attitudes on placebo response
\item Third, that increased interioceptive awareness (or mindfulness) will increase the size of placebo responses
\item Fourth, that the implicit attitudes of the practitioner will exert a significant impact on the placebo response of the patient
\item Fifth, that adding sensory input to a placebo treatment will increase its effectiveness. 
\end{itemize}

Thus ends my prologmena towards a theory of placebo.