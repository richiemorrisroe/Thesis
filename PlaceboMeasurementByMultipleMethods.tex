\documentclass[apsych,phd,draft]{uccthesis}
\title{The Placebo Effect: Measurement by Multiple Methods}
\author[99473437]{Richard Patrick Morrisroe}
\qualifications{BA (Hons)}
\usepackage[pdftex]{lscape}% for landscaped tables
\usepackage{booktabs}
\usepackage{longtable}% for 
\usepackage{rotating}
\usepackage{Sweave}
% \includeonly{TCQThesis}
% \usepackage[T1]{fontenc}
\usepackage[utf8x]{inputenc}
\newcommand{\bftab}{\fontseries{b}\selectfont}
\makeindex
\begin{document}
\supervisors{Dr Z. Di Blasi \\ Dr Sean Hammond}
\professor{Prof John McCarthy}
\sponsor{Irish Research Council for Humanities and Social Sciences}

\setkeys{Gin}{width=0.9\textwidth}

\maketitle




\begin{dedication}
To my parents, without whom none of this would have been possible
\end{dedication}

\begin{abstract}
  This thesis investigated the relationship of explicit (self-report), implicit (IAT) and physiological variables to the placebo effect. The thesis consisted of three main parts. The first collected background data and developed models for two constructs (Optimism and Mindfulness) associated with the placebo effect and implicit attitudes, respectively. The second part of the thesis consisted of the development of an explicit measure of treatment expectancies, and the development of two IATs, one for Optimism and the other for Treatment Credibility. The final portion of the thesis was an experimental study (N=111) which tested these new measures in a sample of healthy volunteers. The primary hypothesis of the thesis, that there would be a relationship between the placebo effect and implicit measures, was not supported. Major findings include an effect of semantic priming on placebo response mediated by condition (Deceptive versus Open Placebo), an unexpected negative relationship between Optimism and self-reported Health, and a physiological relationship between pain ratings and GSR data, which was also mediated by Condition in the experiment. 
\end{abstract}

\begin{acknowledgements}
  Firstly, I'd like to thank all of the people who supervised me over the course of this rather long journey. These are Dr Zelda Di Blasi, Dr Dylan Evans, Dr Jurek Kirakowski, Prof John Groeger and Dr Sean Hammond. All of you have helped immeasurably in making this thesis better, and for that I thank you. 

Next, I'd like to thank all of my friends who stuck with me throughout this long and winding road, covering three cities, four jobs and two relationships. On that note, I'd like to thank Nicola Ziedler for being there for the first four years. 

Additionally, I'd like to thank all of the people who contributed to the open source software that helped make this thesis. 

Finally, I'd like to thank Elizabeth Ann Doyle for putting up with a difficult ending to a really long process. 
\end{acknowledgements}

% \chapter{Introduction to the Research}
\label{cha:intr-rese}


\chapter{Introduction}

\section{Preface}
\label{sec:preface}

% The placebo effect is a phenomenon which is replicable in many different settings and contexts,  but for which individual level predictors are rare \cite{Shapiro1997,Kaptchuk2008a}. Implicit measures (such as the Implicit Association Test) have been found to predict responses to constructs which are difficult to assess using self report. This thesis assesses the efficacy of such an approach to a the prediction of the placebo response in healthy volunteers. 


% The central aim of this thesis was to examine the measurement of explicit and implicit treatment expectancies in the context of the placebo effect. 
% For the purposes of this thesis, an implicit measure is one which infers psychological attributes from non-self-report methods (sometimes called an indirect measure) \cite{Greenwald1995a}. An explicit measure is one which is collected using a questionnaire (terminology also developed in Greenwald and Banaji). Treatment expectancies are regarded as an ``expectation of a non-volitional response''~\cite{Kirsch1985} by some prominent authors in the field. 

 
% This thesis used self-report, interview, implicit and physiological variables to provide greater understanding of these psychological constructs and their relationship with the placebo effect using both observational and experimental designs.  The experimental portion of the research concentrated on the response to a placebo treatment for pain with a healthy, non-clinical sample. 

% This thesis focused both on the explicit expectancies common to much psychological research (i.e. self report scales) and the implicit expectancies measured by such tools as the Implicit Association Test \cite{Greenwald1998}, as have been defined above. 

% This thesis developed three new measures to aid in this task. One was a more comprehensive and multi-dimensional instrument for the measurement of treatment related explicit expectancies (Chapter \ref{cha:tcq-thesis}). This measure was a generalisation of the Credibility/Expectancy Questionnaire developed by Devilly and Borkovec~\cite{Devilly2000}, and the Beliefs About Medications Questionnaire \cite{Horne1999} was used to assess the construct validity of the revised instrument. 

% The other two measures developed were Implicit Association Tests (IAT) -- an Optimism Implicit Association Test (IAT) and a Treatment Credibility Implicit Association Test (Chapter \ref{cha:devel-impl-meas}). 

% Additionally, self report measures of both optimism and mindfulness were collected. Optimism been shown to be related to the placebo effect and a construct commonly conceptualised as expectancy related~\cite{Carver2010}). Mindfulness measures were also taken, which is a construct of interest in terms of treatment in its own right~\cite{kabat1982outpatient}, and has been found to moderate the relationship between explicit and implicit measures. In this thesis, background samples of these measures  were collected from the same sample as from which the experimental population were drawn, to control for potential covariates (Chapter \ref{cha:health-for-thesis}) such as differences between the survey development and experimental sample population. 

% The use of both implicit and explicit measures of the same construct, all of which were validated in the population from which the experimental sample was drawn allowed for the relative impact of both the constructs and the different forms of measurement of them to be assessed. 

% The experiment (see Chapter \ref{cha:primary-research}) was a three condition design (Deceptive Placebo, Open Placebo and No Treatment) and collected implicit, self-report and physiological measures to assess the relationship between implicit and explicit treatment expectancies and the placebo response. 

The central aim of this thesis was to examine the measurement of
treatment expectancies in the context of the placebo effect. This
thesis used explicit (self-report), implicit and physiological
variables to deliver greater understanding of these psychological
constructs and their relationship with the placebo effect.

The more specific aims of this research were as follows:


\begin{enumerate}

% \item To use multiple methods to allow for better prediction of the factors involved with the placebo response. 

\item To create an implicit measure to assess expectancies related to treatment and to the placebo response;

\item To develop a multifaceted method of assessing explicit expectancies which can be linked to the placebo response;

\item To compare the predictive power of implicitly and explicitly measured constructs as predictors of the placebo response;

\item To conduct one  experimental study using a placebo analgesia paradigm to assess the relationship between implicit and explicit expectancies and the placebo response.
\end{enumerate}




The thesis was structured into a number of parts, each of which will
be discussed in turn. Following this, the overall aims and
objectives of the thesis will be reviewed and set into context. 
Next, the specific contributions of this research to the literature will be elucidated. 
Finally, some avenues for future research will be described.

The first section of the thesis was the review of the literature
around the placebo effect and implicit measures, along with any known
individual-level predictors of the effect. This review found that
placebo effects have been seen to be mediated by optimism (as
operationalised by the Life Orientation Test, Revised 
(LOT-R)~\cite{Scheier1994})~\cite{Geers2005,morton2009reproducibility} in both
pain and non-pain paradigms. This review also linked the constructs of
optimism, which can be defined as ``generalised outcome expectancies
around the future'' \cite{Carver2010}, with the construct of response
expectancies of Kirsch \cite{Kirsch1985,Kirsch1997}, which are
described as the ``expectation of a non-volitional response''. These
constructs both crucially rely on the participants' perception of the
likelihood of the event, and a secondary aim of the thesis developed
which was to examine the usefulness of these as two separate rather
than one combined predictor. The hypothesis was that these two
constructs should be significantly correlated, and that one of them
would mediate the impact of the other on the observed placebo response
\cite{Geers2005}.

Another important potential moderator variable which came to light in the
literature review was the construct of mindfulness, as
operationalised by the Mindful Attention Absorption Scale (MAAS)
\cite{brown2003benefits}. This variable was found to moderate the
correlations between explicit and implicit measures in an experience
sampling study. Furthermore, implicit measures were found to be better
predictors of spontaneous responses, while explicit measures were
found to be better predictors of deliberate behaviour
\cite{Levesque2007}, a pattern known as the double-dissociation effect
and observed in IAT studies in a variety of domains
\cite{Asendorpf2002,Perugini2005,Grumm2007,Steffens2006}.

This research literature suggested that a good strategy would be to
replicate the LOT-R and MAAS on a number of samples from which the
eventual experimental sample would be drawn, both to develop better
models for these constructs' relationship (as they had not been
concurrently examined when this work was carried out) and to tailor
the measures towards the experimental sample, a strategy which was
hoped to increase the precision of estimation on the overall sample.
In this research, the RAND-MOS was used both to replicate the impact
of these primary constructs on health, and to provide a weak link to
health and treatment expectancies in preparation for the experimental
research.

One large problem with the development of an implicit measure of
treatment expectancies was that there exists no gold-standard measure
of explicit treatment expectancies, despite them having been shown to
materially affect the outcome of clinical trials
\cite{Linde2007,Bausell2005,Benedetti2005}. This was an issue both
because typical methods of IAT development rely on such a self-report
measure, and because without such a measure, the incremental
improvement (if any) granted by an implicit measure would not be
clear. Therefore, the development and testing of such a measure formed
part of the work of this thesis.

Another fact which became extremely clear during the course of the
literature review, is that current methods of both developing and
validating implicit measures (or more specifically, IATs), rely
extremely heavily on the existence of prior self-report instruments.
For the placebo effect which was the focus of primary investigation,
no such self-report measure existed, which meant that a new approach
needed to be tested. A review of methods in this space revealed the
work of George Kelly on personal construct theory \cite{Kelly1991},
and more specifically the repertory gird introduced in Book 1 of the
aforementioned reference. It was believed that such an approach might
prove useful in the current case, and might also prove useful with
other constructs of interest in the future\footnote{I am indebted to
Drs Jurek Kirakowski and Sean Hammond for originally suggesting this
idea.}.

Therefore, in order to develop these stimuli, it was decided to do
some qualitative research around health constructs with doctors,
alternative therapists and students, to gain an appreciation for how
they contextualised and understood issues around health, medicine and
sickness. The results of this procedure, along with a parallel
exercise asking participants to rate the five most important people or
qualities of people related to health, would then be used to develop
the Health Repertory Grid.

Finally, the development of the experiment was planned, and it was
decided to use a placebo analgesia design, inducing pain with the
sub-maximal torniquet technique \cite{moore1979submaximal}, as this
form of pain induction has been shown in a meta-analysis to produce
the largest effect sizes \cite{Sauro2005}.

% Below, the results and findings from each of these sections are
% described and placed in a context of larger research.



% Another contribution of this thesis is that it uses large samples to develop psychometric models of responses to self-report instruments, and then applies these models to help estimate effects in an experimental sample more accurately. 



% Finally, this thesis  made use of more sensitive and appropriate statistical models and methods, and applied these methods to the more precise estimation of placebo effects in an experimental sample of healthy volunteers drawn from the sample population. 




The remainder of this chapter provides a short introduction to some of the major areas of focus of the thesis, and both the placebo and implicit measures are much more exhaustively described in Chapter~\ref{cha:literature-review}. 


\section{Introduction to Placebo \& Implicit Measures}
\label{sec:intr-plac-impl-meas}

\subsection{Placebo}
\label{sec:placebo}

The placebo is a complex topic of study in that the term stands as a proxy for those elements of human health which are not determined by  features specific to the treatment. Instead, the term placebo refers to the ``non-specific'' parts of the treatment~\cite{grunbaum1981placebo}, those which are not attributable to  specific biological or mechanical activities of the treatment, but are observed across varying treatments. In controlled experiments, the placebo effect is often defined as the response in the placebo group less the response in the no-treatment group. While this definition has problems (described in Section~\ref{sec:concept-placebo}), it is good for an intuitive understanding of the construct. 

The placebo effect has been studied in and of itself for approximately 60 years since Beecher at least~\cite{beecher1955powerful}. The research study of placebo arose as a result of the requirement for placebo controlled studies of medical interventions, and the first such trial was for the efficacy of streptomycin~\cite{concato2000randomized}. In the time since then, there have been many studies to test the relationships between individual level constructs and the phenomenon. For the most part, these attempts have been unsuccessful. A review of the problems with many of the earlier approaches is in Shapiro and Shapiro's book~\cite{Shapiro1997}.  The central idea of this research is that this lack of success in prediction is an artefact of the measures used. 

More recently, there have been some intriguing studies that suggest that there may be some relationships between optimism and placebo response \cite{Geers2005,morton2009reproducibility}.

The placebo effect is rarely the focus in randomised controlled trials, and yet that is where much of the research which has broadened our understanding of the phenomenon has taken place.
Despite years of research, there are few known individual level predictors of the effect. In fact, some researchers have argued that there is no such entity as a placebo responder~\cite{Kaptchuk2008a}. 


There is some confusion surrounding the definition and interpretation of placebo (c.f. Section~\ref{sec:history-concept}).  The term was retained throughout this research  as it does form a useful overarching construct for this research on the interactions between mental and physical states in health. A definition suitable for the purposes of this thesis was the following:

\begin{quotation}
  a placebo is a treatment believed inert for the specific condition
  concerned which has an effect due to context.
\end{quotation}

The rationale behind this definition (and an explanation of context) is given in Chapter \ref{cha:literature-review}, in Section \ref{sec:concept-placebo}. Note that typically a ``placebo'' refers to the treatment, the ``placebo effect'' is the effect in a placebo treatment arm of some kind, and a ``placebo response'' refers to the impact of a placebo on a particular individual.



\subsection{Treatment Expectancies}
\label{sec:placebo-expectancies}

The current prevailing theory around what causes the placebo effect in humans is the response expectancy model of Kirsch~\cite{Kirsch1985,Kirsch1997}, which suggests that placebo effects are directly mediated by these expectancies, which are defined as ``the expectation of a non-volitional response''.

For Kirsch, these expectancies are regarded as the ultimate causative force behind the placebo response in participants and patients, in that all other factors are mediated by them. This theory was regarded as being confirmed in relation to other theories (especially the conditioning theory of Voudouris~\cite{Voudouris1985}) by a 1997 paper which showed that the large positive effect of surreptitious reduction of pain to induce conditioning could be abolished by informing participants that this was occurring~\cite{Montgomery1997}.

Despite the importance of expectancies in the field at present, no standardised measure of expectancies exists, and indeed a recent study showed that of sixteen placebo trials, only two had an expectancy measure in common~\cite{myers2008patient}. 

One of the largest problems with expectancy research at present within the field of placebo is that the measurement of expectancies tends to be quite superficial. Typically, one question on an 11 point scale is asked and the response to this question is taken as the participants expectancy for the treatment. This conflicts with evidence that expectancies are far more multi-dimensional than this approach suggests \cite{Stone2005}. Indeed, one of the aims of this thesis was to address this problem by developing a better scale for the measurement of expectancies (Chapter \ref{cha:tcq-thesis}). 

Given the expectancy theory's superiority within the field at present, it is important to look more closely at what the term means. Some authors~\cite{Stewart-Williams2004a}  argue that expectancies are necessarily conscious, which is a position which seems improbable, given the lack of awareness that typically accompanies observed placebo effects, and the deception which appears endemic to the field of study~\cite{Miller2008a,Miller2008}, in that participants in experimental studies are deliberately lied to in order to induce a placebo effect. 

Expectancy is a catch-all phrase, and while it appears to have applications in a wide variety of areas~\cite{Montgomery2007} the term is far too broad to focus research specifically. Indeed, a similar ``conditioning'' paradigm has been used to produce ``placebo'' effects on sensory reports~\cite{Sterzer2008}, suggesting that these expectancy related mechanisms may be useful predictors outside the study of the placebo effect. 

% Stewart-Williams et al were criticised for their position on expectancies~\cite{Kirsch2004}  and later retracted it, at least in its strongest form~\cite{Stewart-Williams2004}. 





% Recent research has shown that conditioning and expectancies interact to produce much more sustained placebo effects than either alone . The implications of this, and other research findings discussed in this chapter will be expanded upon in Chapter \ref{cha:notes-towards-theory}. 

% There is some contradictory evidence on the relationship between conditioning and expectancies, as Klinger \textit{et al} \cite{Klinger2007a} found that participants who had been conditioned to respond to placebo actually reported greater pain relief when they were given neutral, rather than positive instructions. This may indicate that the expectancy subsystem can interfere with the conditioning subsystem, suggesting that the two methods may activate neurotransmitters and hormones which regulate one another. That being said, this is not a finding which has been replicated, and so this may just have resulted from the particular suggestions given to the participants in this study, as occurred in the study of Levine and colleagues \cite{Levine2006}. 

The value of the expectancy framework is that it has provided both a common vocabulary and a common mechanism for the measurement of placebo responses. The very broadness of the construct allows for it to be used in a variety of situations, which has ensured its survival in the field. This is also the worst part of the definition, as its wide-ranging applicability coupled with its lack of falsifiable predictions has meant that it is regarded as a theory in the abstract, but research on the determinants and measurement of expectancies is not prominent in the field at present.



Another issue with expectancies and their convergent validity is the relationship between the response expectancies of Kirsch  and the generalised outcome expectancies used in the study of optimism~\cite{Carver2010}. Both of these constructs are similar enough that the use of having both of them needs to be determined, and this analysis was carried out as part of this research (Chapter~\ref{cha:primary-research}). Kirsch also notes that self-efficacy~\cite{Bandura1977} and response expectancies correlate quite highly~\cite{Kirsch1985} and given this, the usefulness of a separate response expectancy framework needs to be examined, especially in the light of some of the findings discussed in Chapter \ref{cha:literature-review} (c.f.~\cite{Geers2005,Hyland2007}, and discussion in Chapter~\ref{cha:health-for-thesis}). Note that as expectancies were the primary focus of this research, discussion of other theories is postponed until the next chapter (see Section \ref{sec:theor-plac-effects}). 


Given that the placebo seems to rely upon the belief of a participant that they have received a real treatment, it seems likely that measures which examine conscious beliefs are not the appropriate kind with which to predict its occurrence. It is the contention of this research that implicit measures may provide a better metric for the prediction of placebo response. 

\subsection{Moving Beyond Self-Report}
\label{sec:implicit-measures}

Psychology as a science, and indeed the social sciences more generally, have a problem. They seek to understand the mind and behaviour of individuals in particular contexts and cultures. While behaviour is less problematic to observe scientifically, the observation of mind is fraught with problems. Almost all of the constructs of interest to psychologists (mind, love, experience) are not observable with the naked eye, and require interpretation in order to be understood. 

In the case of many variables, psychologists measure what people think and believe from their answers to questions devised by the psychologist, normally referred to as self-report instruments. 

This approach has obvious advantages, in that it is quick, cost-effective and can produce results of interest. However, as psychology has matured, a number of problems have become apparent with this approach~\cite{Nisbett1977}.  The first problem is that, especially in controversial topics, people may attempt to conceal their true beliefs or attitudes. The second, somewhat more philosophical problem, is that people may be unaware of their true beliefs, or at least may profess to believe one thing while behaving in a manner consistent with a belief in another. 

The first of these problems is known as social desirability~\cite{Egloff2003}, and  self report scales that measure this construct have been developed~\cite{Giebel2008}. The second problem, first noted by Freud, is that of unconscious (or implicit) influences~\cite{Hofmann2008}. While the system built by Freud no longer forms part of the framework of modern psychology, the contradictions between reports of experience and behaviour remain, and are still relevant to the aims of psychology. %%insert allport research on chinese attitudes and behaviour here



\subsection{Older non-self report methods}
\label{sec:older-non-self}

Some methods have been developed to get around this problem. One of the first techniques used for this purpose was that of free association, where a client was asked to respond to a stimulus word or picture with the first word that came to mind, without censoring the experience \cite{Hofmann2008}. These associations could then be used by the therapist to gain access to material which the client did not consciously report being aware of. 

Another method which was used was that of Rorschach ink-blots, where ambiguous ink blots are shown to the client, who interprets them. This technique can also provide insight into the mind of the client, but again this requires interpretation on the part of the therapist. It is this interpretation process that causes these procedures to lack scientific validity in the eyes of many, as what one therapist understands by the clients words may differ completely from what another therapist takes from the same material. 

These approaches were abandoned following the rise of behaviourism as the dominant approach within academic psychology, and further eclipsed by the notions of Karl Popper regarding falsifiability as a criterion for scientific theory~\cite{popper1954conjectures}. It was argued that since unconscious influences could be used to explain any criticism of the theory (and indeed, Freud was prone to doing this) then the theory was not truly scientific. The development of psychometric theory also played a role in the decline of interest in such instruments, as these methods seemed to produce reliable and valid data and scores could be corrected for impression management and social desirability biases by statistical techniques. 

\subsection{Modern Indirect Measures}
\label{sec:turn-back-indirect}
In recent years, however, there has been a resurgence of interest in such techniques. This resurgence grew out of the work on implicit memory and learning, where participants would consciously deny awareness of some piece of information while their behaviour seemed to show signs of this knowledge.  

This phenomenon can be seen in  experiments like word completion tasks. If participants are given a list of words to memorise, and then distracted by another task, followed by the word completion task, they tend to far more frequently complete the word fragments with the words on the previous list which was to be memorised.  However, they will typically deny this influence on their responding if asked~\cite{Wittenbrink2007a}. 

These approaches, allied with the continuing failure of self reported attitudes to predict behaviour as accurately as might be desired caused some researchers to look for another way to measure these constructs~\cite{Greenwald1995a}. The result of these investigations was the Implicit Association Test, or IAT for short~\cite{Greenwald1998}. The IAT is a reaction time measure which makes inferences about attitudes from the time which it takes participants to categorise words or pictures on a related theme into one of two categories. The difference between the participants response latencies is taken to indicate the relative strength of associations, and this difference is known as an IAT effect.  


\subsection{Introduction to the Implicit Association Test (IAT)}
\label{sec:intr-impl-assoc}

The IAT was developed by Greenwald~\textit{et al}, and he suggested that because of its design, it might be more resistant to social desirability influences and demand characteristics~\cite{Greenwald1998}. 

Social desirability tendencies would lead people to deny prejudicial behaviour in self report instruments, while as a reaction time measure, the IAT is less easy to fake. Demand characteristics result when participants in an experiment give the answers a researcher wants, rather than their real beliefs or attitudes. Again, it seems intuitively harder to do this within an IAT methodology as it would require tremendous and consistent control of reaction times to stimuli\footnote{though not impossible~\cite{DeHouwer2007b}}. 

\subsection{Description of the Procedure}
\label{sec:descr-proc}

The (IAT) is a computer administered procedure which purports to measure implicit associations not directly accessible to consciousness \cite{Greenwald1998}. The test was developed as a result of mounting evidence for learning without awareness in human participants, as discussed above. 

This research was reviewed by Greenwald \& Banaji \cite{Greenwald1995a}, where they introduced a distinction between direct and indirect measures of social cognition. They referred to self report instruments as direct measures, and to such techniques as semantic priming as indirect measures. Semantic priming is the tendency for participants in experiments to give answers to ambiguous tasks similar to ones they have recently observed in their environment \cite{Wittenbrink2007}.  They defined implicit associations in the following manner as \textit{the unidentified or inaccurately identified traces of past experience}, and this definition implied that self report measures were not the best tools with which to assess these associations. 

The procedure works as follows. Firstly participants sit down in front of a computer and are assigned two keys (typically the ``e'' and ``i'' keys) to respond to each word or image presented, which fall into one of two categories. The metric examined is reaction time, and an assumption of the method is that categories which are more strongly associated will be easier to combine than those which are less associated.  

The participant is asked to classify words into either pleasant or unpleasant categories as they appear on the screen at the front of the computer, for example love, hate, good  and bad are words typically used in this part of the procedure. In the most well known IAT, the Race-IAT, the procedure is as follows:

Firstly, participants are asked to categorise faces into as either being ``Black'' or ``White''\footnote{Following some research around this, the categories were later re-named to African-American and Caucasian-American}.
 Following this, the two categories are combined, with one key being pressed for White or Pleasant and another being pressed for Black or Unpleasant. Then, the labels are reversed, and the participant categorises White faces with Unpleasant and Black faces with Pleasant. The original authors described the White + Pleasant trials as congruent, and the Black + Pleasant/White + Unpleasant trials as incongruent.  These response times are summed and averaged for each participant for each of the critical blocks, and the two block scores (Incongruent - Congruent) are subtracted from one another to produce a difference score which is referred to as an IAT effect. 

The procedure is not limited to assessment of racial attitudes, and has been applied far more widely \cite{Craeynest2008,Greenwald2009, Schmukle2008,Walker2008}.  A general schema for the process follows.   

Firstly participants classify words as either belonging to Category X or Y, where X and Y are positively (love, flowers etc) or negatively (hate, insects etc) associated words or images.  Then they classify faces or words
as either belonging to Group A or Group B, where the words are often descriptive of groups of people.  In the next step, these two associations are combined, with one key being a response for A and X and the other key being used for responses of B and Y. 

In the fourth step, the keys for pleasant and unpleasant are reversed, and in the final step the two dimensions are combined in the opposite manner (A and Y or B and X). In practice, only the 3rd and the 5th steps are analysed, and the difference between mean response latencies on the different combination tasks is known as an IAT effect~\cite{Greenwald1998}. 

In essence, any differences in reaction time in the combination of the two categories are assumed to be due to underlying differences between the relative associations of the concepts. The authors claim that the use of difference scores allows them to prevent issues of processing and response speed variability across individuals from distorting the results. However, this assumption has been questioned \cite{Blanton2006}, who claimed that processing speed was a major moderator of the observed variance in scores across the population studied. Controls were taken in the analysis of IAT scores to ensure that such effects did not impact the results (c.f. Chapter~\ref{cha:primary-research}). 

\section{Structure of this Thesis}
\label{sec:struct-this-thes}



This thesis follows the following structure:

\begin{enumerate}
\item Chapter \ref{cha:literature-review} reviews the literature surrounding the placebo response and explicit and implicit expectancies;

\item Chapter \ref{cha:methodology} describes a model for how the placebo effect and implicit and explicit expectancies inter-relate, and describes the approaches taken in this thesis for both the construction of measures and the testing of these models;

% \item Chapter \ref{cha:notes-towards-theory} describes a perspective on the placebo response informed by Chapters \ref{cha:literature-review} and \ref{cha:methodology}. This perspective is operationalised in the form of a theoretical model which is then tested in Chapter \ref{cha:primary-research}

\item Chapter \ref{cha:health-for-thesis} reports the analysis of self report variables which have been associated with placebo effect (optimism) as well as with response to implicit measures (mindfulness), and the construction of tailored scales for the population at hand;

\item Chapter \ref{cha:tcq-thesis} details the development and validation of the Treatment Credibility Questionnaire, the self report method used to assess expectancies;

\item Chapter \ref{cha:devel-impl-meas} describes the approach taken to the development of the stimuli for the implicit association tests, using multiple methods;

\item Chapter \ref{cha:primary-research} reports upon the experimental portion of the research, and models these results incorporating the models developed for the self report measures in Chapters \ref{cha:health-for-thesis} and \ref{cha:tcq-thesis};

\item Finally, Chapter \ref{cha:general-discussion} summarises what has been learned from this thesis, and gives directions towards future research.

\item The Appendices supply supporting tables (~\ref{cha:append-1:-supp} for the quantitative research) and qualitative analysis (~\ref{cha:append-ii:-qual}) which was not core to the rest of the thesis.

\end{enumerate}



% The central aim of this thesis was to apply new psychometric methods to the prediction of the placebo response to pain in healthy volunteers. The new psychometric method applied to the prediction of placebo was the Implicit Association Test, or IAT, which is a reaction time based categorisation method developed in the last fifteen years. 

% It is the contention of this research that the alleged unpredictability of the placebo effect is a function of our measurement tool and our statistical methods, and that new and improved measurement tools can furnish us with a placebo effect which can be reliably assessed and predicted. 

% Although some form of placebo effect has been an integral part of medicine for at least the last 2000 years, it is only over the past sixty years that it has been conceptualised and operationalised in such a form as to make my research possible. 

% Indeed, the major tool proposed by this research for the prediction of the placebo response is itself a more recent development than the construction of a defined placebo response. Implicit measures are further defined in the later sections of this work, but for now we can model them as associations which lie outside our awareness much of the time which colour many aspects of our social lives. In another age and time, these measures would have been called unconscious, but such a conceptualisation runs into the tricky problems of unfalsifiability and so, nowadays, most researchers prefer to substitute the term implicit. 


% Of course, this short introduction merely scratches the surface of what has been a hugely rewarding, frustrating and interesting research experience. Mere words cannot convey all that has occurred, but given the format required of us by the scientific community at this time, they shall have to do. The plan for this section of the thesis is as follows. 
% The structure of the thesis shall be as follows:

% \begin{enumerate}
% \item The literature on the placebo effect and implicit measures shall be reviewed, as will the literature of constructs which appear to have been predictive of the placebo effect in the past. 

% \item The methodology for the thesis shall then be described.

% \item The results of a preliminary study relating health, optimism and mindfulness (all of which have been related to placebo or implicit measures) shall be described. 

% \item The development and assessment of a measure designed to tap response expectancies to pain treatments shall be described. 


% \item The development of the implicit measures shall be described, including a process of interviewing health professionals and a method of developing IAT stimuli for arbitrary constructs. 


% \item Next, the experimental data will be analysed, and the usefulness of the models and measures used assessed. As part of this, the inter-relationships between the implicit, explicit, physiological and behavioural measures will be examined.


% \item Finally, the thesis will conclude with a discussion of what has been discovered, and plans for future research. 
% \end{enumerate}








% The central aim of this thesis was to examine of measurement of psychological constructs in the context of the placebo effect. This thesis used self-report, interview, implicit and physiological variables in an examination of usefulness of these different kinds of predictors for  the placebo effect. 

% % This thesis contributes a more comprehensive investigation of a psychological phenomenon from multiple methods of data collection, and could serve as a model for similiar approaches to other constructs. 

% The major contribution of this thesis is that it uses large samples to develop psychometric models of responses to self-report instruments, and then applies these models to help estimate effects in an experimental sample more accurately. 

% In addition, this thesis developed two new implicit measures -- an Optimism Implicit Association Test (IAT) and a Treatment Credibility Implicit Association Test. Self report measures of both of these constructs were collected also, and this allowed for the relative usefulness of these different forms of measurement to be determined in the prediction of the placebo response. 

% Finally, this thesis  made use of more sensitive and appropriate statistical models and methods, and applied these methods to the more precise estimation of placebo effects in an experimental sample of healthy volunteers drawn from the sample population. 


% The aims of this research were as follows:


% \begin{enumerate}

% \item To use multiple methods to allow for better prediction of the factors involved with the placebo response. 

% \item To compare the predictive power of implicitly and explicitly measured constructs as predictors the placebo response. 

% \item To develop a multifaceted method of assessing explicit expectancies which can be linked to the placebo response

% \item To create an implicit measure to assess expectancies related to treatment and to the placebo response

% \item To conduct one large experimental study to test the utility of the measurement and psychometric approaches taken. 

% % \item To integrate physiological data as another signal for the assessment of placebo response. 
% \end{enumerate}

% This thesis follows the following structure:

% \begin{enumerate}
% \item Chapter \ref{cha:literature-review} reviews the literature surrounding the placebo response, implicit measurement and self report. 

% \item Chapter \ref{cha:methodology} describes the approaches taken towards measurement in this thesis, as well as the research designs.

% \item Chapter \ref{cha:notes-towards-theory} describes a perspective on the placebo response informed by Chapters \ref{cha:literature-review} and \ref{cha:methodology}. This perspective is operationalised in the form of a theoretical model which is then tested in Chapter \ref{cha:primary-research}

% \item Chapter \ref{cha:health-for-thesis} describes the analysis of self report variables which have been associated with placebo effect (optimism) as well as with response to implicit measures (mindfulness). 

% \item Chapter \ref{cha:tcq-thesis} describes the development and validation of the Treatment Credibility Questionnaire, the self report method used to assess expectancies.

% \item Chapter \ref{cha:devel-impl-meas} describes the approach taken to the development of the stimuli for the implicit association tests, using multiple methods. 

% \item Chapter \ref{cha:primary-research} describes the experimental portion of the research, and models these results incorporating the models developed for the self report measures in Chapters \ref{cha:health-for-thesis} and \ref{cha:tcq-thesis}

% \item Finally, Chapter \ref{cha:general-discussion} summarises what has been learned from this thesis, and gives directions towards future research.

% \end{enumerate}



% % The central aim of this thesis was to apply new psychometric methods to the prediction of the placebo response to pain in healthy volunteers. The new psychometric method applied to the prediction of placebo was the Implicit Association Test, or IAT, which is a reaction time based categorisation method developed in the last fifteen years. 

% % It is the contention of this research that the alleged unpredictability of the placebo effect is a function of our measurement tool and our statistical methods, and that new and improved measurement tools can furnish us with a placebo effect which can be reliably assessed and predicted. 

% % Although some form of placebo effect has been an integral part of medicine for at least the last 2000 years, it is only over the past sixty years that it has been conceptualised and operationalised in such a form as to make my research possible. 

% % Indeed, the major tool proposed by this research for the prediction of the placebo response is itself a more recent development than the construction of a defined placebo response. Implicit measures are further defined in the later sections of this work, but for now we can model them as associations which lie outside our awareness much of the time which colour many aspects of our social lives. In another age and time, these measures would have been called unconscious, but such a conceptualisation runs into the tricky problems of unfalsifiability and so, nowadays, most researchers prefer to substitute the term implicit. 


% % Of course, this short introduction merely scratches the surface of what has been a hugely rewarding, frustrating and interesting research experience. Mere words cannot convey all that has occurred, but given the format required of us by the scientific community at this time, they shall have to do. The plan for this section of the thesis is as follows. 
% % The structure of the thesis shall be as follows:

% % \begin{enumerate}
% % \item The literature on the placebo effect and implicit measures shall be reviewed, as will the literature of constructs which appear to have been predictive of the placebo effect in the past. 

% % \item The methodology for the thesis shall then be described.

% % \item The results of a preliminary study relating health, optimism and mindfulness (all of which have been related to placebo or implicit measures) shall be described. 

% % \item The development and assessment of a measure designed to tap response expectancies to pain treatments shall be described. 


% % \item The development of the implicit measures shall be described, including a process of interviewing health professionals and a method of developing IAT stimuli for arbitrary constructs. 


% % \item Next, the experimental data will be analysed, and the usefulness of the models and measures used assessed. As part of this, the inter-relationships between the implicit, explicit, physiological and behavioural measures will be examined.


% % \item Finally, the thesis will conclude with a discussion of what has been discovered, and plans for future research. 
% % \end{enumerate}



%%% Local Variables:
%%% TeX-master: "PlaceboMeasurementByMultipleMethods"
%%% End:


\chapter{Literature Review}
\label{cha:literature-review}
\part{The Placebo Effect}

\section{Introduction to the Placebo Effect}
\label{sec:intr-plac-effect}


The major focus of this thesis was measurement, specifically a multi-method approach to the measurement of the placebo effect. The placebo effect is typically regarded as a nuisance parameter in clinical trials, and yet that is where much of the research which has broadened our understanding of the pheneomenon has taken place. Despite years of research, there are few known individual level predictors of the effect. In fact, some researchers have argued that there is no such entity as a placebo responder \cite{Kaptchuk2008a} . 

The placebo is an interesting topic of study in that the term stands as a proxy for those elements of human health which are not determined by  features specific to the treatment. Instead, the term placebo refers to the ``non-specific'' parts of the treatment, those which are not attributable to  specific biological or mechanical activities forming the basis of modern medicine. 

Despite the confusion surrounding the definition and interpretation of placebo (discussed more fully in Section \ref{sec:concept-placebo} below), the term will be retained throughout this  as it does form a useful overarching construct for this research and other research on the interactions of mental and physical states in health. A definition suitable for the purposes of this thesis will be given in Section \ref{sec:defin-this-thes}.

\section{The Concept of the Placebo}
\label{sec:concept-placebo}

The placebo effect has a long history in medicine, and some researchers believe that almost every treatment developed before the 20th Century may have relied primarily on this effect for their curative properties \cite{Shapiro1997,Macedo2003} . Despite this, the concept is not particularly well defined , and different researchers use the same phrase to mean different things \cite{Ernst1995b,hrobjartsson1996uncontrollable} . This section will examine some of these definitions, and look for markers towards which  a more comprehensive definition of the effect can be outlined.  

\subsection{History of the Concept} 
\label{sec:history-concept}

Placebos have a long and colourful past, some elements of which are still relevant today. The word itself is taken from the Latin for "I will please" and referred originally to the cries of paid mourners at funeral ceremonies \cite{Macedo2003}. Over time, it became associated with any medicine which was given more to please a patient than to actually relieve the symptoms \cite{Kaptchuk1998}. Placebos were considered ethically acceptable, if somewhat dubious for many years. However, as more effective and tailored treatments were developed, the use of placebos declined. Following the development of randomised double-blind trials, the  placebo increased in importance once more \cite{Kaptchuk1998}, as it became a required part of the process for establishing the efficacy of all new drugs.  

The first serious attempts at studying of the placebo as an effect in itself  were begun by Beecher's article "The Powerful Placebo" which argued that placebos were useful in many differing kinds of ailments and diseases, and that their effectiveness did not depend on the intelligence of the patient \cite{beecher1955powerful,Kaptchuk1998}.  This article aroused interest in the placebo, and is generally regarded as the first paper which focused on the effect in its own right, rather than as a device to disguise a doctor's lack of effective treatments. 

Beecher made many strong claims in his 1955 article, including an assertion that placebos were effective 35\% of the time. Later research has demonstrated that many of these claims were  inaccurate or misleading \cite{Kienle1998}. The main criticisms of Kienle and Keine will be covered below, but in brief, they found that much of the reported improvement in the studies examined by Beecher could have been due to natural history of the disease and regression to the mean.  

Much of the lack of clarity found within the use and definition of placebo \cite{Macedo2003,Kaptchuk1998} results from its use in two contradictory situations. In the first, that of the randomised controlled trial (RCT), placebos are a control for all effects of treatment not related to the substance or procedure under test \cite{Vickers2000}. In this setting, the aim is that they should be minimised. In the second, the setting of clinical practice, the placebo is imbued with all the authority of medicine and utilised in order to effect changes that may result from mindset or to placate a troublesome patient \cite{Bootzin2003,Sherman2008a}. Macedo and Kaptchuk (in seperate articles) argue that both of these approaches give too much power to the concept and contribute to the confusion surrounding its definition and explication. 

Much of the confusion results from the terms ``specific'' and ``non-specific'' effects, introduced by Grunabum in 1981 \cite{grunbaum1981placebo}. These particularly confusing terms have lead to much agonising and debate over the years. Some have even suggested that these terms should be abandoned \cite{Caspi2002}. The specific parts of a treatment are typically defined as the biologically or theoretically effective agent, while the non-specific factors are all other parts of the treatment. Price and Benedetti argue that there are three sources of non-specific effects - the patient, the provider and the relationship between them \cite{Price2008}, a distinction also made by other authors \cite{Finniss2005}.

Given that placebos can exert extremely specific changes \cite{Amanzio2001,Caspi2002}, they are not non-specific effects, and to the extent that they do not exert changes in outcomes, then the concept is irrelevant \cite{Moerman2003,Barrett2006}.  

\subsection{Placebos in Randomised Trials}
\label{sec:plac-rand-trials}

Placebos are most well known for their use in a clinical trial setting. The FDA (Food and Drug Administration) in the United States of America  requires all drugs to demonstrate usefulness over and above placebo in order to be licensed. This followed a number of scandals in the late nineteen fifties and early nineteen sixties where treatments which had been in use for many years showed little or no effect under double-blind conditions \cite{Moerman2000a}. 

In a typical randomised trial, neither patients nor physicians know whether a particular person receives drug or placebo.  If 50\% of those given placebo feel  better (with respect to the primary outcome measure of the study), even in the absence of medication, this is called a placebo effect, and the inert procedure or pill used is a placebo. More technically, any mean improvement in the control group can be classified as a placebo effect. However, this is not entirely accurate, as will be seen below.   



% The most common situation in which a placebo is used is as in the example above, as a control for the effects of treatment apart from the specific medication being tested (the non specific effects). Here the placebo is typically regarded as a pill or procedure which appears real, but in fact is not.  In this situation, a placebo effect is often defined as the effects observed in the placebo arm of a clinical trial. 

This definition runs into immediate problems, as the effects in a placebo arm of a clinical trial result from a combination of the placebo effect and other factors, such as regression to the mean \cite{Morton2003}, demand characteristics of clinical trials \cite{HrAbjartsson2001} and other factors such as the natural history of the disease.   

Regression to the mean is the tendency for an extreme score measured at time 1 to be closer to the mean at time 2. This tendency is a property of all measuring tools which are not perfectly reliable \cite{Morton2003}, and can be controlled for by sampling from the general population, as using participants with high scores on the outcome variable to be measured tends to exacerbate the effect. 

Sampling from the general population, while a good strategy, is not practical for many trials of new medicines, given the exclusion and inclusion criteria for clinical trials. These criteria normally require that participants in trials suffer from the condition, and have no other effective treatment \cite{Daugherty2008}.   

Demand characteristics \cite{Fernandez1994,weber1972subject} refer to tendency for participants in research to give the answer which they believe that the researcher wants to hear. This can result in an over-exaggeration of symptoms at the first assessment and a minimising tendency for the same symptoms at the end of the study \cite{Vase2005}.  

Another factor which can effect the results of a clinical trial is unidentified parallel interventions \cite{Ernst1995b}. These occur when participants in a clinical trial change other factors as the result of being in the trial, thus biasing the results. One example of this could be if participants in a clinical trial for hypertension reduce their salt intake (potentially as a result of the increased salience of the risks associated with hypertension caused by the process of enrollment in the trial, where typically large amounts of information are provided regarding the condition under treatment). 

It is important to note that without a no-treatment control group the effects of these confounders cannot be separated from the true placebo response.  The no treatment group serves this purpose as factors such as regression to the mean and natural history should apply equally to both the placebo and the no-treatment group.  The definition of the placebo typically used is:
\begin{quotation}
 the placebo response is the response to treatment in the placebo group less the response to treatment in the no-treatment group.  
\end{quotation}

A more precise definition in the context of clinical trials is given by Knipschild et al \cite{Knipschild2005}

\begin{quotation}
   the placebo effect ... [is] ... the difference in effect between the placebo group and the spontaneous course in a randomized clinical trial 
\end{quotation}

This definition relies upon an understanding of the spontaneous course of an illness in a controlled trial, which can be operationalised as the progress of the no-treatment control group. It is somewhat limiting, but is clearly operationalised for a particular setting, which is useful to increase clarity around the construct  being measured. 

\subsection{Placebos and Cognitive Performance}
\label{sec:plac-cogn-perf}

Within the confines of the clinical trial, the two definitions above are workable definitions of the placebo effect. The important part of a clinical trial is the test of the active medication, and the placebo is important only insofar as it relates to the testing of this medication. 

However, clinical trials are not the only context where placebos are administered, and in other situations these definitions run into problems. Consider some recent work of Oken et al \cite{Oken2008}. In this experiment, healthy seniors were administered placebos which they were told would improve cognitive performance. There were two active groups (given different instructions) and one no-treatment group. The participants were tested for cognitive abilities at the beginning, middle and end of the placebo treatment.  The seniors given the placebo pills showed significant increases in cognitive ability over the course of the study, and many were disappointed when debriefed and were told that they had been given placebos.

While the definitions given above can, at a stretch, account for these results, it does indicate a need to more carefully define the concept of placebo. Whether or not one accepts the clinical trial definition depends crucially on ones definition of treatment. 

For many, this is some device or procedure that restores the organism to optimal function. Alternatively, this view could be described as believing that treatment restores homeostasis to the organism. Indeed, the Oxford English Dictionary agrees with this definition describing treatment as \textit{"medical care for an illness or injury"}.

In the Oken study above however, this was not the case, as the researchers were examining the positive effects of placebos, rather than attempting to cure a deficit. Treatment could be defined as something which improves the performance or health of an organism, and such a definition would not encounter these problems.

It is relatively easy to see how placebos can reduce pain, but is more difficult to see how such pills can improve cognitive performance. One can argue that decreases in pain result from a response bias \cite{Allan2002}, but the measurements of cognitive performance in the Oken et al study were not subject to these kinds of bias. 

 One can argue that the expectancy of the participants for improvement led them to actually improve, but this argument begs the question as we then need to define how expectancies exert such effects. 

A more plausible explanation for these findings is stereotype threat \cite{schmader2003converging,spencerclaude1999stereotype}. Stereotype threat occurs when a member of a particular group performs badly as the result of their worry surrounding being judged badly for their performance. 

This phenomenon could have accounted for the results observed in the Oken et al study.  It may be that the belief that the pills were enabling improved performance compensated for the effects of stereotype threat. An interesting experiment would be to investigate the effects of placebo cognitive pill administration on the performance of women and African Americans in more traditional academic environments.  

Another study where placebo effects were demonstrated in a non clinical setting was that of Crum \cite{Crum2007}. In this study attendents in hotels were cluster randomised (using the hotel as the unit of sampling) and half of the attendents were informed of how many calories they burned by engaging in their roles as hotel attendents. One month later, the informed group had lost significantly more weight and had improved on both self report and externally measured dimensions related to weight and health. Again, this effect is difficult to conceptualise in terms of treatment.  Crum, in this paper defined the placebo effect as

\begin{quotation}
  any effect of treatment which is mediated more by the participants
  beliefs and expectancies rather than the physiological actions of
  the treatment.
\end{quotation}

This definition is more widely applicable than our first definition above, but at the cost of introducing two new terms which are not clearly defined, namely beliefs and expectancies. While the second of these terms has a number of specific meanings in psychological thought (for example, process, outcome and response expectancies), belief is such a commonly used word that it has not been precisely defined in any psychological theory. This definition also suffers from the issues surrounding the definition of treatment that were noted above, in Section \ref{sec:plac-rand-trials}. 

% One problematic point here is that while we have some evidence for the physiological pathways followed by painkilling placebos \cite{Benedetti2005a,Mayberg2002,Benedetti} we have little to no idea how placebos can exert change on cognitive variables. Such research, especially if these cognitive gains were long lasting, could result in a substantial rethinking of how we measure and approach the concept of intelligence, and particularly its genetic and environmental components. 

Another definition (which avoids the treatment definition problem) was quoted from Ross \& Olson, 1981 by Flaten and colleagues \cite{Flaten1999}
This definition is as follows:

\begin{quotation}
  A placebo or nocebo may be defined as an inactive
substance or a procedure that is administered with
suggestions that it will modify a symptom or sensation
\end{quotation}

This definition avoids the problem of defining treatment, and also avoids the use of the terms beliefs and expectancies. However, it can be seen from the definition that this is achieved only at the cost of introducing suggestion as a possible mediator of the placebo effect. On the face of it, this is perhaps not a term without merit. In studies of hypnosis, suggestion is commonly regarded as the driver of the observed effects \cite{Kirsch1994} and hypnosis has been proposed as an ethical method to induce placebo responses in participating individuals. This definition is perhaps the best of those that have been examined so far, but it does require us to limit ourselves in the study of placebo to sensory phenomena. Both the Crum \& langer study and the Oken et al study indicate that effects termed as placebos have a wider sphere of effect. 

A definition which can be regarded as quite similiar to the one given by Crum \& Langer above is the following:
\begin{quotation}
the nocebo effect is  a causation of sickness by expectation of sickness and by associated emotional states
\end{quotation}

While this definition refers to the nocebo effect (negative placebo effects) it focuses on the expectation portion of placebo. However, Benedetti et al have demonstrated that there are at least some placebo effects which cannot be mediated by expectancies \cite{Benedetti2003a}, so this definition does not account for all placebo effects. 

One interesting feature (not seen in other definitions) is the focus on emotional states as a cause of the nocebo effect. This does seem to make physiological states, as placebo responses can be inhibited by a chemical known as CCK, which also acts as an anxiogenic \cite{Benedetti2006c}. However, emotional states are typically measured with the same or less precision than placebo, so this definition runs into problems when we attempt to operationalise (and therefore measure) emotional states. 

% A further definition of the placebo effect was given by Shapiro and Shapiro \cite{Shapiro1997}, in which they defined a placebo effect as ``the effect caused by a placebo treatment''. This definition is wide enough to encompass all the possible placebo effects, which is useful. However, this definition is so wide that it tells us very little about the placebo itself. It is merely a rephrasing of the term placebo effect and tells very little and does not point to useful directions for further research. It is also worth noting that this definition excludes the influence of the practitioner involved, and some research indicates that this is the most important part of the effect \cite{Blasi2001,Kaptchuk2008}.


Another definition of the placebo effects,  proposed by Price et al \cite{Price2008} is that

\begin{quotation}
  a placebo is any substance or procedure that simulates a treatment.
\end{quotation}

This definition fails to resolve our problem discussed above with relation to cognitive abilities. 
Kienle \& Keine give two definitions of placebo in their critique of Beecher:

\begin{quotation}
Placebo is defined in two separate ways, firstly as the imitation of a
therapy, and secondly as any self-healing effect. 
\end{quotation}

The first definition of imitation of a therapy is extremely similiar to the Price definition quoted above, and the second, while it sounds plausible is far too vague to be of any use in research. Definition of self-healing alone could take many papers, and there is no guarantee that the concept would prove useful in the end. 

Another definition \cite{Blasi2001}, does resolve the problems encountered with the Price \& Benedetti  definitions above. This definition is that

\begin{quotation}
  placebos are inert substances that have an effect due to context
\end{quotation}

This definition has some good points, in that it allows for placebo effects in any areas in which they are found, it allows for patient-provider effects and it does not prejudge the causes of the effect. 

One major difficulty which can be observed with this definition is that it does not account for active placebos, where a substance which is a medication in one context is administered for a condition where it is not expected to have any pharmacological effect\cite{Kirsch1998} . 

These active placebos would contradict the definition of placebo as inert, and yet the research demonstrates that these active placebos can be as effective as the regular inert pill \cite{Flaten2004} and sometimes more effective \cite{Kirsch2002a}. 

Another issue with this definition was elided to above, to the extent which a placebo induces specific changes, it is not inert, and therefore, using this definition would cease to be a placebo, which is clearly nonsensical \cite{Moerman2002b}.

A similiar definition (in the context of clinical trials) is given by Knipschild et al \cite{Knipschild2005}

\begin{quotation}
  [the placebo effect] ... is the effect of co-interventions in a
treatment study connected to the doctor--patient relationship.
\end{quotation}

Again, this definition is quite precisely operationalised, but it assumes that the active ingredient of placebo is the relationship between provider and patient, which has not been demonstrated to be the case at all. 


A more recent definition of the effect, formulated by Colloca and Benedetti
 \cite{Colloca2005}  says that 

\begin{quotation}
the study of the placebo effect is the study of the psychosocial context
  surrounding the patient
\end{quotation}


This appears to be an update of the Di Blasi et al  definition and seems quite useful. One factor that this definition excludes is the interior experience of the person who has the response. Context refers to external things or processes (at least according to the Oxford English Dictionary), and there is evidence that when participants attend to their own internal somatic sensations, placebo effects increase \cite{Geers2006}. Obviously, the participants somatic sensations are not part of the context surrounding them, and yet they appear to have an impact upon the response. 

% I don't know about the argument above, seems a little weak

One of the most interesting definitions of the placebo response is that developed by Daniel Moerman, an anthropologist with his conception of the ``meaning response'' \cite{Moerman2002b,Moerman2002a}. 
This definition is as follows: 

\begin{quotation}
  We define the meaning response as the physiologic or psychological
  effects of meaning in the origins or treatment of illness; meaning
  responses elicited after the use of inert or sham treatment can be
  called the "placebo effect" when they are desirable and the "nocebo
  effect" when they are undesirable
\end{quotation}

This definition is particularly interesting as it focuses on the perception of the treatment by the individual, which is a factor often neglected in many definitions of placebo.  This definition, while it seems to be among the best that we have, still displays far too much of a focus on treatment and medical conditions to account for placebo effects on cognitive function or exercise performance \cite{Crum2007} . 

 While this is understandable, given the roots of the placebo concept, it is far too restrictive given that we have evidence of placebos in contexts far removed from medicine, such as sport and cognitive function \cite{Benedetti2007a,Oken2008}. 

One problem with this definition rests on our understanding of the word "meaning". The OED defines this as \textit{"what is meant by a word, idea or action"}, which is not particularly useful in this context. What Moerman seems to understand by the word is the interpretation placed on the treatment by the individual when they receive it. This again supposes that people are aware of all of the meanings which they possess, when the research would seem to indicate that this is not always the case, demonstrated by (amongst other evidence) the Shiv et al (2005) study noted below  \cite{Shiv2005a}.  

The Moerman definition is useful in that it points to the individual and cultural interpretations of treatments which appear to facilitate a placebo response, it is still quite imprecisely specified and not particularly useful in practice. % (meaning is internal, and some elements of meaning may reside at a non-verbal, implicit level so there is great difficulty in assessing these meanings in order to devise more effective forms of treatment). 

Another interesting study which points towards the need for a more clearly defined placebo concept is that of Shiv et al \cite{Shiv2005a} , where the response to an energy drink was heavily affected by the price. 

When participants believed that the energy drink had been discounted, they solved significantly less puzzles than when the were given the drink at its full price. An important issue to note in the context of the paragraph above is that when participants attention was drawn to the discounting, the effect disappeared. This implies that although the signal of the reduced price affected their abilities, this meaning only activated when their attention was not focused on it (suggesting that they responded to the signal in an implicit fashion, somewhat like priming). 

Again, this study cannot be conceptualised as a treatment, as we are looking at an improvement in cognitive performance rather than the relief of some illness or problem. 

\subsection{Definition for this thesis}
\label{sec:defin-this-thes}


A definition which can be widened to fit these kinds of placebo effects and mind-body interactions is the definition of Di Blasi et al \cite{Blasi2001}, if we take account of the issue arising from active placebos. Perhaps the definition would work better if we removed the word 'inert' and replaced with 'believed inert for the specific condition concerned', so that it reads

\begin{quotation}
  a placebo is a treatment believed inert for the specific condition
  concerned which has an effect due to context.
\end{quotation}

This definition would need to be supplemented with a more precise defintion of context. One attempt at this would be that context is

\begin{quotation}
   the internal states, external environment and relationship of the individual to these states and environment and other individuals in their presence 
\end{quotation}

\subsection{What is a placebo?}
\label{sec:what-placebo}



Some researchers would not classify the Oken and Shiv studies as looking at placebo effects at all. It is arguable whether or not these effects of cognitive improvement can really be classified as placebo effects, as they could more properly be termed expectancy effects. 

Others would also argue that the same is true of all placebo effects, as they appear to be mediated by expectancies \cite{Kirsch1985, Kirsch1997,Montgomery1997}. Recent research does seem to show that expectancies do not mediate all placebo effects \cite{Benedetti2003a}, so the distinction between these two terms is still useful.  The only difference between the effects observed in these experiments and those seen in more typical placebo experiments appears to be that the latter can be conceptualised as treatments, while the former cannot. 

\subsection{Use of the term placebo}
\label{sec:use-term-placebo}



This highlights a huge problem with the placebo terminology, it has broadened to mean any effect mediated by information and meaning rather than through biochemical mechanisms. An example of this is a recent study of exercise and the "placebo effect" \cite{Crum2007}. This fascinating experimental study which demonstrated that significant improvements in fitness could result from information provided about the health benefits of work is nonetheless symptomatic of an obscuring trend in social science research; that of treating any influence of mental information on the body as a placebo effect. 

This kind of broadening of the concept is  useless to continued progress is the field, as it obscures the very real and significant differences between these effects. Some can be attributed to the effects of expectancies, others to meaning, still others to subtle communications on the part of the experimenter or treatment provider, still others to biological mechanisms not well understood, and the lack of clarity in definitions and terminology is holding back research tremendously. One is almost tempted to ditch the concept entirely, as was proposed in a recent issue of the British Medical Journal \cite{nunn2009s}. 

For all the appeal of starting again with a new term, such an approach would have a number of problems. The first of these is  that the phrase, however poorly defined, conveys a meaning to researchers about how an effect occurred. 
The second problem is what to do with clinical trials. In this situation, the placebo is a necessary control. Does Nunn suggest that we use only equivalence or non-inferiority trials? Obviously this would not work for new treatments, and even in cases where there is an accepted treatment, the statistical requirements with regard to sample size and other matters would cause large problems with the proving of the efficacy of a treatment \cite{Benedetti2008}. This follows as larger sample sizes would be required to determine that two treatments are not significantly different from one another at conventional levels of statistical power. %insert benedetti book reference here

Another issue with this jettisoning of the placebo concept is what to replace it with? The literature we have has established that  treatments lacking a currently understood biological basis can have real effects \cite{Meissner2007}  in pain \cite{Vase2002} , in depression \cite{Kirsch2002a}  and in Parkinson's disease \cite{Benedetti2004a} among others. How are we to conceptualise these effects, if not in terms of placebo? Perhaps a better approach would be that of Gotzsche \cite{Gotzsche1995} where he suggested that placebo effects be broken down into those attributable to patient-provider interaction, those attributable to a stimulus from the outside world and those attributable to the patients belief in the treatment. Hrobjarttsson also gave a similiar definition where he described three types of placebo:

\begin{quotation}
change after a placebo intervention, effect of a placebo intervention and the effect of the patient provider interaction. 
\end{quotation}

Of these three definitions, only the third is not tautological, and it has been covered when the Di Blasi definition was discussed above. 

Even then, these effects are difficult to disentangle. Consider the case of a researcher giving a participant a placebo as a pain killing treatment, after which the participant reports less pain. While the administration of the inert substance and the patients belief in it certainly had an impact, so too did the presence of the researcher, who is the provider in this situation. It is difficult to see how these effects can be disentangled without specialised designs requiring the same participants being administered placebos by multiple researchers.   



I would propose that the term placebo effect should be reserved for effects which exert a beneficial change in health due to effects of internal and/or external context. Other effects should be termed expectancy effects, or mind-body effects, or effects of therapeutic relationship (which have a long history in psychology, being typically known as experimenter effects) \cite{rosenthal1969interpersonal,rosenthal1967covert,Rosenthal1956}.  

\subsection{Problems with establishing true placebo effects}

The focus of this section is on the definition of placebo, but the definition of placebo effects is without consequence if it is impossible to establish that a placebo effect has or has not occurred in a controlled setting. One of the largest problems with establishing this is when the outcome is a subjective one, such as pain (which is the domain where the placebo was examined in this thesis). Pain is a subjective experience, and equivalent stimulus levels may cause entirely different reactions  in two different people \cite{Kirsch1997}. 

This issue is often controlled for by calibration of the stimulus levels to the individual participant in an experiment. This allows the researcher to assess to what extent these pain levels are altered by the placebo treatment. Normally, pain intensity and unpleasantness levels are assessed on an eleven point Visual Analogue Scale (VAS)  and these ratings are used as the input data for statistical analysis of different groups. 

However, the use of these self report instruments carries with it a number of problems. Firstly, the issue of demand characteristics arises as a consequence of this. Demand characteristics refer to the subtle pressures faced by participants in experiments to live up to the expectations of the researcher \cite{weber1972subject} and were discussed in Section \ref{sec:plac-rand-trials}. The idea is that if a researcher really wants to demonstrate something, then he or she will communicate this subtly to the participants in the study, and they will (theoretically) respond in the manner in which the researcher prefers. This factor is one of the major reasons why tests of new drugs must blind the experimenter as well as the participants. 

Another large factor which affects placebo analgesia research is the issue of response biases. The problem is that if participants receive a treatment which they believe to be effective, then their thresholds of perception may be altered without any actual biochemical changes. This is the theory put forward by Allan \& Siegel's \cite{Allan2002} analysis of the placebo phenomenon in terms of signal detection theory. These two factors were proposed to account for the entirety of the placebo effect in pain by Hrobjarrstsson \& Goetzche \cite{HrAbjartsson2001}. This interpretation of placebo seems less likely given that placebo effects have been observed on biochemical levels, and given that placebo analgesia can be removed by the administration of naloxone \cite{Levine1979} \cite{Benedetti2003}. 

%there needs to be a section linking this part to the conclusion of this part of the chapter.


To conclude, the concept of the placebo is wide-ranging, and requires careful elucidation if research is not to be held back by mistaken assumptions surrounding the concept. Therefore, there is a need for this confusion to be resolved in one of two manners. Firstly, the definitions of placebo used could be tailored for the specific experiment or condition studied. 

This would have the advantage of being able to more precisely define the expected effects and outcomes for this experiment. However, it would also create a large number of defined ``placebo effects'' and this would be problematic for the researcher who wishes to study these effects across a broad range of conditions and treatments. That being said, some researchers in the field continually make the point that there are many placebo effects, not just one and such a strategy would have the benefit of making this extremely clear \cite{Benedetti2008} 

The second option would be to look for an all-encompassing definition somewhat like Shapiro's, and use this for all research in the field. While this would make matters easier for researchers interested in the broad concept of the placebo, it would create difficulties when new forms of placebo effects come to light in the course of research. 

Such an all-encompassing definition would also lose much of the precision that allows for future avenues of research to be discerned from it, and prove much less useful to specialised researchers in the field. To summate, the best approach here may be to define the placebo as broadly as possible within the confines of effects on health and illness, and encourage researchers to specify exactly what they believe constitutes a placebo effect for the purposes of each study.

% \section{Common Features of Placebo Effects}

\section{Theories of Placebo Effects}
\label{sec:theor-plac-effects}

Following the discussion of the definitions of placebo and some problems arising from the different contexts within which it has been developed, the next step in this review is to examine the theories which have been proposed to account for this phenomenon. 

In the case of the placebo effect, there are a few major theories, each of these will be described in turn  and examined for the ways in which they account for the effects, and those features of the effect which they fail to explain (or reject as being illegitimate). A useful run-down of all of these theories appeared recently \cite{Stewart-Williams2004b} and that review has helped to inform the arguments presented here. 

\subsection{Conditioning}
\label{sec:conditioning}



The first major theory which attempted to account for placebo effects was that of conditioning. Building off the demonstration of placebo effects in non human mammals \cite{Herrnstein1962} the conditioning theory argued that placebo effects resulted from the learned association between a contingency in the environment (the doctor, pill or medical setting) and healing. This contingency lead to the activation of healing mechanisms based on previous experience with the pill. 

The conditioning theory has a number of advantages. Firstly, it can account for placebo effects in all mammals, as all seem to capable of learning through reinforcement. Secondly, it is parsimonious, as it allows us to explain the placebo phenomenon without invoking any new processes or mechanisms. Thirdly, it appears to account for much of the effects. 

However, the fatal flaw in the conditioning explanation is that it cannot account for placebo effects from a product which a participant  has not experienced before. Given the nature of clinical trials, this destroys  conditioning as an explanation for all placebo effects.  One could assume that the observed placebo effects in clinical trials result from generalised associations with medical treatments more generally \cite{pearce1987model}. This does retain aspects of the theory, and provides a testable hypothesis, which is the following - to the extent that there is commonalities between the learning environment and the clinical trial environment, placebo responses will be observed.  

This hypothesis would be difficult to test, as the learning environment would need to be controlled, but this is possible if using a longitudinal design where a new treatment is developed and presented to participants on a number of occassions before systematically varying the environment in which the treatment is delivered in order to test this theory. 

There do appear to be some placebo responses which are totally mediated by conditioning \cite{Amanzio1999}, but not all of them can be \cite{Benedetti2003a}. Experimental research has elucidated some of the connections here, in that motor movement in participants with Parkinson's disease and pain can be modulated by expectancies while changes in hormone secretion appear to be modulated by conditioning exclusively (or at least have not been shown to be affected by expectancies) \cite{Benedetti2003a}.

Strangely enough, even though conditioning appears to induce stronger placebo responses than does expectancies, nocebo suggestions can completely reverse the effects of positive placebo conditioning \cite{Benedetti2008}. Additionally, in one study, conditioned participants showed a greater placebo response when given neutral, rather than positive instructions \cite{Klinger2007a}. 

Some authors have argued \cite{Stewart-Williams2004b} that conditioning is not a theory of what causes placebo effects, but rather a mechanism through which other variables, such as expectancies, exert their influence. This is an interesting idea, and has some merit, but it does suggest that all placebo effects are mediated by expectancies, and this is known to not be the case. 

\subsection{Expectancies}
\label{sec:expectancies}

The competing theory to conditioning for the past few decades was the expectancy theory, as expressed by Kirsch \cite{Kirsch1985}. Kirsch coined the term ``response expectancy'' to describe what he called ``the expectation of a non-volitional response''. A ten year review \cite{Kirsch1997}  suggests that this theory has applications in hypnosis and placebo effects. Recent research has shown that expectancies can also modulate sensory experience \cite{Benedetti2008}. 

This theory competed with the conditioning theory for over a decade, but the issue was mostly resolved by a 1997 paper \cite{Montgomery1997}, which pitted the expectancy and conditioning explanations against one another. This study used the conditioning manipulation devised by Voudouris \cite{Voudouris1985} where the painful stimulus is reduced after application of a placebo cream to increase the size of the placebo effect. 

One group was told of the pain reduction, while the other was not. The group who were told showed no enhanced placebo response, which supported the expectancy theory. A multiple regression also carried out as part of the study indicated that the effects of conditioning were completely mediated by expectancies. 

This seemed to be convincing evidence in favour of the expectancy theory. However, it is worth noting that some authors \cite{Stewart-Williams2004a}  argue that conditioning is a mechanism, not a theory, and they claim that conditioning is one method through which expectancies are formed. This theory does not appear to be plausible given the existence of placebo responses which have only been demonstrated with conditioning mechanisms \cite{Benedetti2003a}.   

Given the expectancy theory's superiority within the field at present, it is important to look more closely at what the term means. Some authors \cite{Stewart-Williams2004a}  argue that expectancies are necessarily conscious, which is a position which seems improbable, given the lack of awareness that typically accompanies observed placebo effects, and the deception which appears endemic to the field of study \cite{Miller2008a,Miller2008}.  

Stewart-Williams et al were criticised for their position on expectancies \cite{Kirsch2004}  and later retracted it, at least in its strongest form \cite{Stewart-Williams2004}. Expectancy is a catch-all phrase, and while it appears to have applications in a wide variety of areas \cite{Montgomery2007} the term is far too broad to focus research specifically. 

The value of the expectancy framework is that it has provided both a common vocabulary and a common mechanism for the measurement of placebo responses. The very broadness of the construct allows for it to be used in a variety of situations, which has ensured its survival in the field. This is also the worst part of the definition, as its wide-ranging applicability coupled with its lack of falsifiable predictions has meant that it is regarded as a theory in the abstract, but research on the determinants and measurement of expectancies has not progressed much in the past decade. 

Recent research has shown that conditioning and expectancies interact to produce much more sustained placebo effects than either alone . The implications of this, and other research findings discussed in this chapter will be expanded upon in Chapter \ref{cha:notes-towards-theory}. 

There is some contradictory evidence on the relationship between conditioning and expectancies, as Klinger \textit{et al} \cite{Klinger2007a} found that participants who had been conditioned to respond to placebo actually reported greater pain relief when they were given neutral, rather than positive instructions. This may indicate that the expectancy subsystem can interfere with the conditioning system, suggesting that the two methods may activate neurotransmitters and hromones which regulate one another. That being said, this is not a finding which has been replicated, and so this may just have resulted from the particulat suggestions given to the participants in this study, as occurred in the study of Levine and colleagues \cite{Levine2006}. 

One of the largest problems with expectancy research at present within the field of placebo is that the measurement of expectancies tends to be quite superficial. Typically, one question on an 11 point scale is asked and the response to this question is taken as the participants expectancy for the treatment. This conflicts with evidence that expectancies are far more multi-dimensional than this approach suggests \cite{myers2008patient}. Additionally some research has shown that there is substantial heterogenity in the framing and analysis of expectancy questions \cite{myers2008patient}. This is worrying, as this heterogenity may mean that measures reported as expectancy scales may be substantially different from one another, which will lead to theoretical and conceptual confusion in the research. Indeed, one of the aims of this thesis was to address this problem by developing a better scale for the measurement of expectancies. 

Another issue with expectancies and their convergent validity is the relationship between the response expectancies of Kirsch  and the generalised outcome expectancies used in the study of optimism \cite{Carver2010}. Both of these constructs are similiar enough that the use of having both of them needs to be determined, and this analysis was carried out as part of this research. Kirsch also notes that self-efficacy \cite{Bandura1977} and response expectancies correlate quite highly \cite{Kirsch1985} and given this, the usefulness of a seperate response expectancy framework needs to be examined, especially in the light of some of the findings discussed below (cf \cite{Geers2005} \cite{Hyland2007}, and discussion in Section \ref{sec:behav-plac-motiv}). 

\paragraph{Suggestion} 
\label{sec:suggestion}

Suggestion is a feature which while prominent in explanations of hypnotic phenomena, is often neglected in studies of the placebo. This is despite the fact that often the placebo phenomenon is brought about by suggesting to participants that they have received an effective treatment. Recently, Kirsch proposed that  placebos could be fruitfully considered in terms of suggestion rather than expectancies  \cite{Kirsch1999}. This viewpoint seems illuminating, as there are large differences in the size of placebo effects depending on the type of suggestion used. 

This line of research begain with Kirsch \cite{kirsch1988double} when he looked at the effects of either telling participants that they would receive coffee, or that they might receive coffee. This clever design  mirrors the difference between placebo studies and double blind trials. This experimental study found that when coffee was deceptively administered, there was a much larger effect. This finding has been replicated in a clinical setting using analgesics following surgery, with the same results \cite{Amanzio2001}, though another author failed to replicate this finding using student experimenters \cite{Walach2002}. 

 These research findings argue in favour of the suggestion in placebo being one of the major factors in driving the effect. Additionally, other authors have suggested that suggestion and placebo have much in common, and the lack of linkages between them may be due to lack of clarity in definitions \cite{DePascalis2002}. 

One experimental study which shows the subtle effects of suggestion was the work of Levine \textit{et al}  on motion sickness \cite{Levine2006}.  In this study, participants were told that one placebo pill would reduce motion sickness (placebo suggestion) while the other would reduce spinning, but would increase the other effects (nocebo suggestion). Contrary to the hypothesis, participants in the nocebo condition showed the greatest reduction in symptoms. This was probably because of the caveat in the nocebo suggestion that spinning was felt by others to be the worst symptom. This example illustrates the need to be extremely careful when giving suggestions, lest an opposite result be obtained. This point is further discussed in Chapter \ref{cha:primary-research}, and in Chapter \ref{cha:general-discussion}. 

Furthermore, some research in hypnosis indicates that even the features associated with hypnosis (lack of memory, lack of volition etc) are themselves the result of personal and cultural suggestions \cite{Kirsch1999}. This does seem to suggest that hypnosis and its effects are just as subject to suggestion as any other interpersonal phenomenon.  It may be that placebo effects are merely the result of suggestions (conscious or unconscious) which are given in the domain of health, while hypnosis merely refers to suggestions given in the context of hypnotic treatment or entertainment. 

Suggestion also appears to be able to override conditioning in some situations \cite{Benedetti2008}. In this study, after pre-conditioning with ketrolac (a non-opioid painkiller), analgesia could be induced with positive suggestions, while if negative suggestions were given, then they were able to override the prior conditioning. This may be an effect of salience asymmetry \cite{Rothermund2004}, where negative stimuli (in this case, suggestion) are more salient, or it could be related to the non-opiod nature of the conditioning. A contradictory finding, discussed above, was that in a trial of placebo analgesia, conditioned participants demonstrated a greater response to placebo following neutral, rather than positive suggestions \cite{Klinger2007a}. 

In conclusion, the conceptual landscape around the response expectancy construct is quite confused, with elements of this construct being conflated with self efficacy, optimism and suggestion. This thesis contends that this confusion will not be resolved until after a better measure of expectancies is developed, which will allow for a more accurate estimation of what unique explanatory power is possessed by this construct. 

\subsection{Motivational Theories}
\label{sec:behav-plac-motiv}

A competing perspective on the placebo has been advanced recently by Michael Hyland \cite{Hyland2006}. Hyland's theory is called motivational concordance, and it regards the behaviours which people engage in and the meanings that they attach to these as primary,  rather than the cognitive focus of response expectancy theory.  

His research seems to show that depending on how a particular therapy is framed, different variables can predict the placebo response.  In the study cited above, spirituality predicted the placebo response to Bach herbal essences, while expectancy was not an independent predictor. In further research \cite{Hyland2007} he established that this is only the case when flower essences were framed as a spiritual treatment, and not when they were described as motivational tools. These findings argue against the contention of Kirsch that expectancies mediate the effects of placebo on the body directly. 

He also noted that when a placebo sleep therapy (which involved writing down things which participants were grateful for) was utilised, gratitude was the best predictor, and again, expectancy added nothing to the results. 

These findings are quite interesting, as they imply that although expectancies may contribute to the placebo effects, they do not account for it totally. 

% Some evidence which would argue for this theory's usefulness may be the effects of psychotherapy. 

% Psychotherapy is perhaps the very definition of a healing ritual, and some recent evidence suggests that perceived assignment to a real therapy is a much stronger predictor of improvement than actual assignment is \cite{Bausell2005,Linde2007}.

One issue with the theory of Hyland is that it has not undergone extensive testing, and has never been analysed in a double-blind design, so it is unknown at this point to what extent it will generalise across the various conditions of placebo administration. For example, his results could be due to experimenter effects and demand characteristics. This is less likely as interactions between the researcher and the participants were minimised, but the possibility has yet to be seriously examined. 

Another recent theory regarding placebo effects is that of Geers \cite{Geers2005a}. Geers et al notes that  motivational approaches to placebos were popular in the past.  He suggests that this perspective may prove fruitful for an analysis of placebo effects.   His research used priming techniques in order to influence the desires of  participants to respond to the treatment. 

The major finding of Geers \textit{et al} across a number of studies \cite{Geers2007,Geers2005a} was that placebo effects were significantly greater after participants had been primed with cooperative goals. His research also showed that expectancies had an impact, but again it was not independently significant after motivation was controlled for in a stepwise multiple regression procedure \footnote{But see section \ref{sec:regress-models} for some discussion of problems with this approach}. Effects of motivation were also demmonstrated by Jenson and Karoly, \cite{Jensen1991}. The Jensen \textit{et al }   research found that motivation was a predictor of placebo response while expectancies were not, while the Geers \textit{et al } studies found that motivation and expectancies interacted to produce the observed effects.

Some research does suggest that goal directed behaviour may be associated with endogenous dopamine release \cite{Scott2007a} which could provide a plausible mechanism through which  goals and motivation help to activate placebo effects. Some other evidence that would support the theory of Geers is that patients suffering from illness typically show much greater placebo responses to experimental pain than do healthy controls \cite{Klinger2007a}. The research of Geers \textit{et al } used various different priming manipulations to increase motivation to respond, which also suggests that implicit (or unconscious) motivations may be able to influence the response to placebo. 

\subsection{Other theories of placebo}
\label{sec:other-theor-plac}

Some researchers, arguing from an anthropological perspective \cite{Thompson2009}  have claimed that all of the current theories are far too cognitively focused, and argue for a conception of the placebo response as residing in embodied experience, rather than the constructs used to describe it currently. 

This seems like an interesting hypothesis, but has not undergone much testing. It does however, fit with recent conceptions of cognition as embodied \cite{wilson2002six}, especially the conception that the body is intimately involved in cognition.  

An experimental study \cite{Geers2006} demonstrated increased placebo responses when participants were asked to attend to bodily symptoms, which suggests some role for somatic awareness in the effect. This theory would also fit nicely with the recent meta-analysis \cite{Meissner2007} which noted large placebo effects on peripheral outcome parameters in organs, but very little in hormones. 

This would fit the data as there are feedback mechanisms from organs to brain (through the central and peripheral nervous system), but the hormone system does not have as immediate feedback links to the parts of the brain involved in placebo effects, which would suggest that this theory deserves some credence. This theory is developed further and the implications expanded upon in Section \ref{cha:notes-towards-theory}.

A final theory concerning the placebo effect is the framing of Daniel Moerman \cite{Moerman2000a,Moerman2003}, who conceptualises the effect as a meaning response, which is a useful idea as it brings awareness to the important intra-individual factors which underlie the observed placebo effects. 

However, research designs and ways of distinguishing meaning from expectancies are sorely lacking, so at present this theory is little more than a clever name change for the same old effect.  This theory  could be tested by requiring participants in clinical trials to keep diaries of their experience and analysing them using a structured approach to determine the individual construction of experience which presumably underlies the construct of meaning. 


\section{Moderators of the Placebo Effect}
\label{sec:moder-plac-effect}

Moving on from theories about the nature of the effect, the next step is to examine factors which can moderate the placebo responses observed in research and clinical practice, covering both experimental data and the results of large meta-analyses. 

% The placebo response is interesting in that it pointed towards cognitive influences on physical health long before there was any interest in such matters in other parts of medicine.

This section will address potential moderators of the effect.
These moderators will be divided into factors inherent to the participant, factors relating to the health care provider, and factors relating to the nature of the placebo and the study design. 


%This stuff can probably go into the measurement section. 
% Single studies can often be biased by the researcher, the setting and factors such as the precise wording of suggestions. Meta--analysis should, in theory, remove these issues as problems, as (assuming the differences follow a normal distribution) they will cancel each other out when an average effect is calculated. However, this can cause problems of its own, as study selection (which is a subjective process) takes on huge importance in the understanding achieved by the meta-analytic procedure.  

% As an example of these problems, two meta-analyses were conducted on the placebo response in Irritable Bowel Syndrome \cite{Patel2005,Enck2005}. One found that the number of study visits increased the placebo response, while the other claimed that the number of visits decreased the effect.  These meta-analyses both included one hundred studies, but only twenty six were common to both \cite{Klosterhalfen2008}.   This example highlights the critical importance of looking at both single studies and meta-analysis  if  a comprehensive picture is desired. 


\subsection{Patient/Participant Characteristics}
\label{sec:psych-char}

The first kind of moderators of placebo which will be reviewed are psychological characteristics of the individuals under study. Both state and trait variables may be involved here, though most of the research has focused on traits, as they tend to be easier to measure. 

\subsubsection{Optimism}
\label{sec:optimism}

The first trait to examine is that of dispositional optimism, which is often defined as \textit{generalised outcome expectancies about the future}. When using this definition, it seems relatively likely that there may be a relationship between optimism and placebo response, and yet this has only been investigated in recent years. 

Dispositional optimism appears to exert some influence on placebo effects, in some situations \cite{Geers2005,morton2009reproducibility}. The effect seems to be that those higher in optimism respond better to positive suggestions, while those higher in pessimism respond better to negative suggestions. Another study found that general (but not specific) expectancies had a significant impact on the response to placebo in a meta-analysis of randomised controlled trials of chronic back pain \cite{myers2008patient}. Generalised expectancies is essentially how optimism is defined, given that all expectancies around future states of health are essentially outcome expectancies. 

Other studies appear to  show that this optimism effect is not general, but rather depends on the context in which the experiment takes place \cite{Hyland2006}. In this experimental study, spirituality rather than optimism was a predictor of the response.  

Crucially, this only occurred when the treatment was classified as spiritual. When a gratitude based treatment was used, gratitude acted as a predictor. These results suggest that any trait which predicts placebo response will likely only be effective in certain situational settings \cite{Kaptchuk2008a}. 

Moerman's meaning response theory would seem to be the most apt theory to use as a framework for understanding these effects, as the only element which differs in these experimental designs is the meaning which participants assign to the stimulus which acts as a placebo. Alternatively, one could argue that expectancies drive these effects by mediating the impact of other contextually relevant variables. However, to take this position would require that the theory of Kirsch, that expectancies exert direct physiological effects, would need to be abandoned \cite{Kirsch1985}.  

A recent study \cite{morton2009reproducibility} in a placebo analgesia paradigm argues for a stronger interpretation of the role of optimism. This experimental study used a repeated measures design, and utilised a preconditioning method in the first session which is known to increase the size of the placebo response \cite{Voudouris1985}. 

While in the first session there was no effect of optimism on the results, in the second study dispositional optimism was significantly correlated with placebo analgesia, explaining 55\% of the variance. This would suggest that while optimism may not produce a placebo response in itself, once a response has been produced it can be effective in maintaining it over time. 

Hyland suggests that the optimism effects on placebo response are mediated through expectancies, and that when these are not a factor, neither is optimism. This sounds plausible, but the relationship could easily go the other way in that optimism could drive the observed effects of expectancies. This is not a question which can be answered without further empirical research, which was conducted as part of this research (see Chapter \ref{cha:primary-research}). 

\subsubsection{Gender}
\label{sec:gender}

Gender appears to be an important factor in placebo effects, with differing results being noted depending on the gender interactions between experimenters and participants. A good example of this is the Oken \cite{Oken2008} study which looked at the  effects of placebo pills on cognitive functions in older adults. This study had female experimenters, and a large placebo effect was shown for male participants but not for female participants. A Flaten \cite{Flaten2006} study also showed effects of gender on placebo response, as in this study with female experimenters, males did not show a placebo response, but females did. The authors explained this in terms of males being less willing to admit that they were in pain to female experimenters.

 Another study \cite{Zubieta2006} showed a enhancement of dopamine production in males following placebo administration but not in females. The authors note that this may be because of physiological differences, but gender was not ruled out as a cause of this effect during this study. Another example of gender influences on placebo treatments was observed in Milling \cite{Milling2007}  when they were looking at the effectiveness of hypnotic, CBT and placebo treatments for pain. They observed that there was a significant effect of gender, but fail to note the gender of the experimenters, which renders the effects of any interaction difficult to interpret. A final study showed that men, but not women responded to a glucose administered placebo \cite{Haltia2008}. These last two studies do not provide enough information to conclusively examine the effect of gender on placebo response, as there could have been exogenous variables which actually drove the observed effects. 

In conclusion, there appears to be some evidence that interactions between the gender of the experimenter and the gender of the patient/participant may affect the response to placebo. Nonetheless, the research is quite tentative and it does not appear to be a consistent effect (like much else within the study of placebo). 



\subsubsection{Somatic Focus}

Somatic focus, or the focus on internal bodily sensations (the proprioceptive sense ) appears to have an impact on the response to placebo, though this finding has only been demonstrated in a small number of studies. 

This finding arises from the work of Geers et al \cite{Geers2006}  on somatic focus and its effect on the placebo response. In summary, this experimental study asked half the participants to attend to their somatic sensations following placebo administration, and gave the other half no such instructions. A similiar finding was made by Rainville \textit{et al} with regard to hypnotic suggestions \cite{Price2008}. 

The participants who focused on their bodily sensations showed an increased placebo effect, which is an interesting finding for many reasons. Firstly, it suggests that the effectiveness of a treatment can be increased by asking participants to pay attention \footnote{this may also be a potential mechanism through which MBSR exerts its beneficial health effects}. 

Secondly, it links in with an explanation given for differences in placebo response across treatments following a meta-analytic review \cite{Meissner2007} where the authors provide evidence that placebo effects are not common where the outcome measure is a hormone level, while they are common where the outcome measure is a peripheral disease parameter. They suggest that this occurs because nervous system feedback loops are available for the second kind of outcome, but not for the first. This finding is discussed further in Chapter \ref{cha:notes-towards-theory}.  

\subsubsection{Psychopathology and catatrophising}
\label{sec:psych-catatr}



Some other personality characteristics have been linked to placebo response and also to active treatment response. Firstly, a recent controlled trial \cite{Wasan2005}  showed that participants with higher levels of psychopathology (as measured using self report scales) derived significantly less benefit from analgesic treatment, but significantly more benefit from placebo analgesia treatment. Interestingly, levels of optimism also correlate with psychopathology, which may be a potential cause of this interesting finding \cite{Carver2010}.  

A second finding in this area \cite{Sullivan2008}  showed that in a trial of amyltriptine and ketamine versus placebo, the extent to which participants catastrophized about pain determined their treatment response. High catastrophisers reported a large effect from placebo, but low effects from the active treatment while for those low in catastrophizing, the results were the opposite. It is worth mentioning here that this Sullivan trial was a secondary analysis of a null result, so some cautions should be taken in its interpretation. No such caveats apply to the Wasan \textit{et al} trial. 




\subsection{Provider Factors}

Another factor which is often claimed to be of importance in placebo effects is the patient-provider relationship. The classic study in this field was performed by Thomas \cite{thomas1987general}  trial where  patients suffering from unclear symptoms were given either either a positive or negative consultation.  The results of this study showed that 2/3rds of the patients given a positive consultation improved, while only 1/3rd of the patients given a neutral consultation had. This finding influenced many researchers in the field over the next few decades. 

However, more recent research has failed to replicate this effect in a sample of patients with pain problems \cite{Knipschild2005}. This newer study worked with general practitioners in the Netherlands, and although a large sample size was used, they failed to find any significant effects based on the positive consultation. 

A few caveats apply here. Firstly, while the Thomas study had just one doctor involved, there were over 40 in the Knipschild study. Secondly, Thomas dealt with many different kinds of patients while Knipschild dealt only with pain patients. Thirdly, the Knipschild study used normal general practice clinics while a student sample was used in the Thomas study. Knipschild and Arntz suggest that the charisma of Thomas may have had something to do with his effectiveness. 

They also note that many of the GP's did not like giving negative consultations and the tape recorded interviews suggested that they were far more comfortable dispensing the clear advice.  Although there is substantial evidence for the impact of good patient provider relationships in the literature \cite{blasi2001influence}  the specific matter of whether or not a positive consultation improves medical outcomes must be regarded as open at present. The Di Blasi systematic review above looked at the impact of cognitive care and emotional care, and argued that these two features drive much of the patient provider effects on health outcomes. 

A recent study \cite{Kaptchuk2008}  using an RCT design with patients suffering from Irritable Bowel Syndrome (IBS) appears to indicate that interaction between patient and provider is a critical part of the placebo response. This study utilised sham acupuncture and divided participants into three groups. One received no treatment, another had sham acupuncture with minimal interaction while the third group had sham acupuncture and large amounts of interaction. The results showed that Group 3 had much better recovery rates then either of the other two groups, which would seem to suggest that the patient-provider interaction drives much of the observed placebo response, at least in this setting.  

In addition, the documented success of treatments in medicine which have later been shown to be no better than a control treatment definitely involves the placebo effect. This finding comes from Roberts (1993), and is reported in Moerman (2000) \cite{Moerman2000a}.  The most famous example of this is surgery for angina pectoris carried out in the 1950's which was shown to be completely ineffective in blinded trials. A more recent example was a study conducted on osteoarthritis of the knee, where there was no significant effect of the surgery \cite{horng2002placebo}. Nonetheless, this treatment was found effective by many patients before this trial, suggesting that there are significant effects deriving from the provider interaction with the patient.  

Moerman argues that the reason that these treatments were effective is that physicians believed in them, and they communicated this belief to their patients. 
This belief (either directly or indirectly) caused the patients to respond well to the treatment \cite{Moerman2000}. Some authors have suggested that these patient-provider effects are the result of neural patterns laid down by caregivers in early childhood \cite{Kradin2004}.

More broadly, patient-provider effects on health outcomes can be conceptualised as a form of experimenter effect, whereby the provider exerts an influence on the results. This is the positive side of demand characteristics where the patient becomes aware of the beliefs of the provider, and responds to these. An extremely good study of these effects was carried out by Walach \textit{et al} \cite{Walach2002}. In this study, two students were recruited as experimenters, and induced to have either positive or negative beliefs about the efficiacy of placebo. These experimenters then carried out the same experiment, and achieved results in line with the induced expectancies. 

% This would tie in with findings that PhD scientists are far more likely to find a significant placebo effect than are medical doctors  can't find a cite for this, but I definitely remember reading about it. 

\subsection{Treatment Factors}

\subsubsection{Type of Placebo}
\label{sec:type-placebo}

This section will examine the effects of different kinds of placebo on their effectiveness. The most recent study in this field looked at the differential effects of two different kinds of placebo therapy \cite{Kaptchuk2006}. In this study two different kinds of placebo (a pill and a medical device) were used, and showed differential outcomes on the recovery of the patients involved in the trial. 

These kinds of findings are one of the best proofs  that placebos actually produce measurable effects, as if they do not, then the effects of two placebo therapies should not differ significantly from one another \cite{Kaptchuk2006}. In the Kaptchuk et al study, it was shown that a sham device for acupuncture produced much greater placebo responses than an inert pill. 

Physical placebos such as ultrasound have also been shown to have an larger average placebo effect size \cite{Ernst1995b}.  The reasons for these differences are not clear at present, but some thoughts on this matter are presented in Chapter \ref{cha:notes-towards-theory}. 

One interesting finding from a meta-analytic study comes from the work of De Craen \cite{Craen2000}. This study showed that injection placebos were more effective than pill placebos in the treatment of migraine. This could be the result of injections being typically associated with stronger painkillers while pills are often sold over the counter, thus leading to stronger response expectancies regarding the injection. This finding can also be explained by the effects of prior experience and therefore also compatible with a conditioned model of the placebo effect. This appears to relate to the findings noted above about sham devices and physical devices being more effective than inert pills. 

\subsubsection{Name of Placebo}
\label{sec:name-placebo}



Another factor which affects the response to placebo is the name of the placebo. A recent study found that placebo responses to a placebo of the same name remained almost entirely constant, while the same inert cream given a different name evoked different responses from the same group of participants \cite{Whalley2008}. The authors use this finding to argue that this demonstrates that the placebo effect is completely inconsistent. This is not a particularly strong argument, as the name given to a placebo is one of the most important features, as it is one of the few pieces of information given to the participants in a typical trial. A much older study \cite{Morris1974} found that the familiarity of a drug name had no impact on the response to placebo. Some contrary evidence related to branding and the response to named placebos is presented in Section \ref{sec:price} below. 


\subsubsection{Price of placebo}
\label{sec:price}

Another factor which may affect the placebo response is price \cite{Shiv2005a}. The Shiv et al study utilised an energy drink distributed to college students in an on campus gym under two conditions. In the first, they merely received the drink and were asked to solve a number of puzzles. In the second, they received the same drink, but were told that the price had been discounted (without being given a reason). The participants in the second condition solved significantly less puzzles than those in the first, suggesting an impact of perceived price on the effectiveness of the energy drink. 

This is extremely topical, given the nature of the patenting process on pharmaceutical drugs and the proliferation of generic drugs following the expiry of the original patent. This finding probably reflects cultural associations of price with value, and one could hypothesise that in other cultures, items perceived as being of greater value would invoke similiar effects.  

This finding has been replicated in placebo analgesia \cite{Waber2008}, which is perhaps more relevant to the discussion about pharmaceutical drugs. It is worth noting that in neither study were the participants actually required to pay for the drugs, and as such, inferences cannot be made directly regarding the real world effects of these results. Such a study would have much greater external validity and relevance to health care policy makers. 

This research also ties into the classic paper by Branthwaite on branded and unbranded pills for the treatment of headaches \cite{Branthwaite1981}. This experimental study, using an extremely large sample, found that branded placebos were more effective than unbranded placebos, suggesting that either advertising or prior learning can affect the effectiveness of two identical preparations. 

Again, the findings discussed in this section can be interpreted in terms of suggestion and expectancies. Price is typically taken as a signal for quality in Western societies, and particular brands of pharmaceuticals and medicines can be associated with relief. That being said, the branding experiments are equally conducive to being explained in terms of conditioning, while the price findings are certainly expectancy driven. Older research did not find any effects of familiarity, which would suggest that the effects of branding have either increased in recent times or that familiarity alone is not enough \cite{Morris1974}.  

\subsubsection{Frequency of administation}
\label{sec:freq-admin}

In addition to the factors noted above, it does appear that 

\subsection{Contextual Factors}
\label{sec:contextual-factors}



\subsection{Spatio-temporal factors}
\label{sec:geogr-diff-plac}

One fascinating result from a meta-analysis was found by Garud et al \cite{Garud2008}. This meta-analysis looked at the placebo response in ulcerative colitis and  found that  that there were significant differences between the same placebos based on the geographical location of the trials. 

In the USA, placebo response rates were 10\% lower than in Europe, suggesting that some cultural force might be driving this difference. Its hard to see what this cultural difference could be though, given the substantial heterogenity which exists between countries in Europe, which is far more than the comparable differences between states in the USA. 

Nonetheless, this meta-analysis points toward some kind of cultural or geographic factor which can influence the placebo response, though a more nuanced and systematic explanation is lacking at present. 

This would link in with earlier work carried out by Moerman into variability of placebo response rates across countries and cultures \cite{Moerman2000}, where differences in ulcer rates and blood pressure across cultures were noted. While this original meta-analysis by Moerman was criticised for a lack of rigour, another analysis by  De Craen which did not suffer from these problems \cite{Craen1999a} confirmed these results. 

Additionally, some research shows that placebo response rates in clinical trials have been increasing over time \cite{Enck2005a}. This paper suggested that rates of response to placebo may have increased by over 20\% in this time period. To some extent, this may be attributable to laxer criteria for entry into clinical trials, but the same phenomenon has been noted in schizophrenia \cite{Ravi2008}, wehre the same issue of expanding amounts of clinical trials necessitating looser criteria for entry does not apply. While these geographical and temporal factors should not be over-interpreted, it does suggest that the placebo response is extremely sensitive to small differences in the context in which it is administered. 

\subsection{Proportion of participants assigned to placebo}
\label{sec:effects-clin-trial}

The placebo response is commonly regarded as a nuisance in clinical trials, and the most recent medical guidelines suggest that placebo controlled trials should only be used when there is no proven alternative \cite{temple2000placebo}. This often makes ethical review committees reluctant to allow placebo conditions, and if they are allowed, the placebo group is often very small, in order to minimise the risk. 

A recent meta-regression suggests that this may be counterproductive \cite{Papakostas2009}.  This research found that as the size of the placebo group decreased, the size of the placebo response often increased, in some cases meaning that the trials could not show an advantage of drugs over placebo. The authors speculate that, as the participants were informed of the drug:placebo ratio as part of informed consent procedures, they had stronger expectancies on the likely results of treatment and therefore they reported a larger response to treatment, which caused the placebo response rates to increase. This situation is an excellent example of the law of unintended consequences, and is an interesting object of study in its own right. 

One finding which provides cause of caution is that effect sizes for placebo tend to correlate with the size of the trial, suggesting that regression to the mean may be responsible for some of the effects observed in clinical trials.\cite{Enck2005a}. To a certain extent, given the nature of clinical trials this is unavoidable as the selection criteria will tend to select groups of participants most likely to demonstrate this phenomenon. Unfortunately, though no-treatment groups provide an excellent bulwark against this confounder, most clinical trials do not have them. 


\subsubsection{Certain and Uncertain Expectations}
\label{sec:cert-uncert-expect}
Perhaps the most important feature of trial design affecting the response to placebo  is the influence of suggestion/expectancies. In most clinical trials, participants are informed that they will receive either active treatment or placebo. Some authors have suggested \cite{kirsch1988double}  that this process diminishes expectancies related to the treatment efficacy, which in turn reduces the effects \cite{Kleijnen1994}. The Kirsch study noted above looked at the effects of differing instructions on the results of ingesting placebo caffeine, and showed larger effects when participants were given placebo coffee with suggestions that it was real than when they were told there was a 50\% chance they would receive placebo. A replication attempt by Walach \textit{et al} did not confirm this finding, which may suggest that such an effect is contingent on the cultural background of the participants. 

More recently,  Amanzio and colleagues \cite{Amanzio2001} replicated this finding with patients recovering from thoracic surgery. The principal finding was that those patients who believed they were getting a real medicine required much less analgesia than those who believed that they might receive placebo. Such an effect could account for the differences in effect sizes seen between experimental and clinical studies of placebo. Indeed, Amanzio \textit{et al } suggested that variations in placebo response were responsible for much of the variability in the response to analgesics in general.

A second, related factor may be the use of suggestions in the experimental research. Participants are typically told that they will receive a powerful painkiller before placebo administration, whereas in the clinical trial, no such instructions are given. This finding regarding certain and uncertain expectations was also replicated in a test of a placebo sleep therapy by Geers  \cite{Geers2005a}.

Linking to the discussion above regarding certain and uncertain expectations, it may also be important to examine a paper by Ploghaus et al \cite{Ploghaus2003} where the authors argue that certain expectations of aversive events are associated with fear, while uncertain expectations are associated with anxiety. Anxiety is associated with both the nocebo effect and the production of the hormone CCK, which may be why uncertain expectancies appear to lead to lower placebo effects \cite{Colloca2008b}. These two emotions activate differing parts of the brain, and given the finding that dopamine systems are activated differentially by certain and uncertain expectancies \cite{Scott2007a}, this may point towards some important future avenues for research. This new focus on the brain and body correlates of placebo effects has contributed much to the field, as we will see below in section \ref{sec:neur-plac-effect} .

\subsubsection{Treatment Preference}
\label{sec:treatment-preference}


There is some evidence that benefits accruing from clinical trials may result from the patients expectancies about whether or not they have received the real treatment \cite{Bausell2005}. This study showed no difference between sham and real acupuncture but showed large differences between the outcomes of those who believed they received real treatment versus those who did not. This factor is typically ignored in clinical trials, although prominent commentators have argued that it should be taken more into account \cite{Benedetti2007}. This finding was later replicated \cite{Linde2007}  where four clinical trials of acupuncture were pooled. This study suggested that although real acupuncture showed similar improvements regardless of expectancies, the minimal acupuncture groups improvement was dependent on their expectancies around acupuncture and perceived treatment assignment. In another trial, belief in receiving real nicotine was an extremely good predictor of successfully stopping smoking \cite{Benedetti2008}. 

These results do suggest that pre-existing expectancies and treatment preferences and beliefs about assignment are an important moderator of the observed placebo responses. These results would seem to argue that placebo controlled trials are not valid unless these treatment beliefs are controlled for, as they could easily bias the outcome of a study. 


In conclusion,the placebo response is a complex phenomenon and can be impacted by internal  participant factors, features of the patient-provider relationship, features of the treatment itself, and also features of the setting in which the treatment is administered. Very few studies control all of these factors, and this may contribute to some of the confusion and controversy surrounding the construct. 

\subsection{Just how powerful is the placebo?}

One well known meta-analysis suggested that the benefits of placebo were negligible in most areas \cite{HrAbjartsson2001}, with the exception of pain trials. While this meta-analysis has been critiqued for its ignorance of psychological studies of placebo \cite{Evans2003, Stewart-Williams2004b}  and for the combining of placebo effects across 200 plus treatments \cite{Wickramasekera2001}, there is no denying its large effects on the field. 

One of the major reasons for the popularity of pain studies in placebo research is probably the large effect sizes, as measured by Cohen's d. While effects in some areas range from about $d=.15--.25$, the effect sizes in pain studies tend to be much larger, ranging from $d=.45--.95$ \cite{Vase2002}. Given that the $d$ measure  expresses effect sizes in terms of standard deviations, an effect of between a half and one standard deviation is quite respectable, and allows for smaller studies to examine effects of interest.  But again, it has been argued that these effect sizes are illusory, and result from lack of blinding, inadequate controls and poor randomisation \cite{Hrobjartsson2003,Kienle1997} . 

These two viewpoints can be reconciled, at least in the opinion of Vase et al  (2002) \cite{Vase2002}. In this meta-analysis, Vase and colleagues looked at the sample of placebo pain trials in both the clinical area and in the experimental area. What they found was that effect sizes tended to be small when the placebo was used in a clinical trial, and much larger in experimental studies of the placebo effect. This analysis was disputed by Hrobjarrtson and Goetzche \cite{Hrobjartsson2003} who noted problems with the methods of analysis chosen by Vase et al. Even using the more conservative estimates of Hrobjarrtsson and Goetzche, the effect size from experimental research (d=0.5) is still twice as large as those observed in clinical trials. There are a number of factors which differ in these two contexts which could be responsible for these observed differences. 

 % although the two types of research seem to give us different information, they are mutually interdependent, as when enough single studies are conducted, this data becomes ripe pickings for a meta-analysis.
% The major conclusion we can draw from this section is that the placebo response is a complex phenomenon, upon which many factors exert an influence and that we need to bear in mind the individual level factors, the interaction factors, the treatment factors and those factors which only become apparent after looking at aggregated results of many similar studies. 

If there is a gap in the literature, it probably results from a paucity of meta-analytic studies on non clinical trials of placebo (with some notable exceptions \cite{Wampol2007,Vase2002}), but this is a matter that could be easily addressed by future research. It is however, a matter worth addressing, as what little studies we do have indicate that there are a number of major differences between the data revealed by each of these methodologies. 

The above review of the  research relating to features associated with response to placebo has established the following. 
\begin{itemize}
\item Placebo effects appear to be related to expectancies and through these, optimism
\item Gender may be a factor in the response to placebo
\item The relationship between patient and provider appears to drive some of the observed placebo effects
\item Study design and the expectancies induced by different study designs appears to have a large impact on the outcome of the trials. 
\item The perceived value of a placebo (as expressed through price and branding) effects the response
\item Treatment preference and beliefs about assignment to treatment appear to be more important than actual treatment assignment.
\end{itemize}


\section{Debates about the Nature and Existence of Placebo Effects}
\label{sec:nature-existence}

The placebo is quite a mysterious phenomenon and does not fit very well into the biomedical paradigm \cite{Kaptchuk1998} \cite{Caspi2002}. In this school of thought, specific treatments exert action on specific parts of the body and cause relief from sickness. The huge effectiveness of quinine, vaccination and pasteurisation on the health of European societies took place without any regard to the mental state of the patients receiving the treatment, and this encouraged a model of the body as machine, where certain inputs led to certain outcomes \cite{Caspi2002}. 

It was this very commitment to evidence and specific remedies that has eventually led to the restoration of the place of the placebo effect in medicine. The first clinical trial was conducted in 1913, but the method was not widely adopted until after world war 2 \cite{Kaptchuk1998}. The adoption of randomised controlled trials led to on the one side, the denigration of the placebo as a mere experimental tool, and on the other side, this development created an environment where its effectiveness in a wide variety of situations could be recognised. 

The placebo had concurrently fallen out of official use  by doctors as it was regarded as a deception, although recent surveys in this age of far greater attention on ethical prescription show that many medical practitioners do not take this advice \cite{hrobjartsson2003use,sherman2008academic,tilburt2008prescribing,Sherman2008,Sherman2008a,Sherman2008b,Sherman2003,Ross1983,Buckalew1981,Krugman1964,Ross1962}. 

% The placebo languished in the bowels of the developing controlled trials until in 1955, when Beecher published The Powerful Placebo and introduced a new excitement into the study of the field. Beecher also introduced some unfortunate canards into the field, with his research noting that around 1/3rd of patients reported a placebo effect. This statistic has been carried from article to article for over fifty years now, despite evidence that it is not even remotely true \cite{Kienle1997}.

The placebo remained a topic of interest to some scientists and psychologists for the next few decades, but many argued that it could not be real on principle, and others that while the effects seemed real, they were due to demand characteristics of the experimental situation or response biases. % Some even invoked signal detection theory in order to explain the effects away \cite{Allan2002}. 
This is despite evidence from the early 1960's of placebo effects in other mammals \cite{Herrnstein1962}, which obviously could not be attributed to either of these experimental artefacts. Some argued that if the placebo did have an effect, then it could only be on subjective outcomes \cite{Hrobjartsson2001}.

A new phase of placebo research began in the late 1970's with the demonstration that placebo analgesia could be mediated by naloxone \cite{Levine1978a,Levine1984,Fields1981,Gordon1981,Levine1979}. This research was quickly contextualised by evidence which showed that although some placebo analgesic responses were mediated by endogenous opiates, not all could be \cite{Gracely1983,Levine1984}. These experiments represented perhaps a tipping point in the study of placebo. No longer could these effects be dismissed as mere response biases, and researchers in the field finally had a physiological mechanism to point at to convince doubters that the placebo effect was a real phenomenon and was worthy of study. 

The nineteen eighties were a good decade for placebo research. The first placebo conditioning experiments were performed on humans \cite{Voudouris1985}, and a new theory of response expectancy was proposed to account for the effects \cite{Kirsch1985}. These two theories of conditioning and expectancies were tested against one another, and the controversy continues in the field today, despite what many regard as proof positive of the supremacy of expectancies over and above conditioning \cite{Montgomery1997}. More recent research (noted in the placebo analgesia section above) has contextualised the conditions under which placebo effects can be created by conditioning and expctancies \cite{Benedetti2003a}. 

In 1998, an article was released that purported to show that anti-depressants were not much more effective than placebo \cite{Kirsch1998}. This research attracted a furore of publicity, and many newspapers argued in favour of ditching medications entirely and resorting to placebo. 

While this is a misinterpretation of the research findings (if the patients were given placebo as placebo it is unlikely they would have recovered), it gave new impetus to the field. This research was further backed up in 2002 \cite{Kirsch2002a}  with an analysis of all the information submitted to the FDA to prove the efficacy of Selective Serotonin Re-uptake Inhibitors (SSRI's) was shown to reveal little benefit for active drug over placebo. In the field of depression at least, it appeared that the placebo was indeed powerful. 

However, in 2001, the field was thrown into controversy. Two well known researchers conducted a meta-analysis on all clinical trials which included both a placebo and a no treatment condition \cite{hrobjartsson2001placebo}  and concluded that there was no evidence of placebo effects on objective paramters and only a minor effect on subjective parameters. This research caused a furore in academic circles, and was widely reported by the same media who had given the placebo such a warm welcome some years earlier. The meta-analysis was widely criticised \cite{Evans2003,Kirsch2001, Wickramasekera2001,Greene2001}  on the grounds that it ignored psychological studies of placebo, considered too wide a variety of clinical conditions and did not properly ensure that the no treatment condition was indeed a no treatment condition. Nonetheless, this meta-analysis probably focused placebo research into the area of pain for many years afterwards. An update of the review in 2005 and in 2010 for the Cochrane Library revealed no essential changes in the findings of these authors \cite{Hrobjartsson2004}. 

Predictably, a storm of research findings and counter finding ensued. Vase and colleagues collected data for experimental placebo studies and argued that the null results in the clinical trials studied were due to the lack of suggestion in the clinical trial setting \cite{Vase2002}.  The original authors argued that this meta-analysis was sub-standard and when the same methods were used as in the original study, much smaller effects were found \cite{Hrobjartsson2003}. 

The debate continued in the pages of the Journal of Clinical Psychology \cite{Wampold2005}  where the noted authors argued that placebo effects were revealed in the situations where they were expected, and not elsewhere. Again, the response of Hrobjarrtson et al was forceful as they claimed that this was a post hoc rationalisation of the results, and argued in favour of dispensing with the placebo concept outside of clinical trials \cite{Hrobjartsson2007a} . Wampold et al replied to this accusation \cite{Wampol2007}  with counter-evidence and Horbjarrtson et al countered that their original conclusions remained solid \cite{Hrobjartsson2007}. 

%this section needs more description

The debate rested there for a time, with those researchers working with placebos convinced that there were working with a real effect, and possessing enough neuro-biological and physical evidence that they were not concerned with the by now infamous meta-analysis while those who did not accept the placebo theories were able to point to a large study that confirmed what they believed to be true. However, more recent research has cast new light on this complex issue. 

In a recent meta-analysis  \cite{Meissner2007}, Meissner and colleagues reviewed clinical trials using both placebo and no treatment conditions, and made a surprising discovery. In trials where the outcome measure was a physical parameter, there were large placebo effects but in trials where the outcome measure was hormone levels, there were no placebo effects. They re-analysed the studies looked at by Hrobjarrtsson and Goetzche and found that when this classification was utlised, placebo effects were observed in the data. The implications of this line of research and other research is explicated in Chapter \ref{sec:notes-towards-theory}. 
 %% this section appears quite a number of times

The debate about the reality and effectiveness of placebos appears to have subsided somewhat, though there are many unresolved issues. While the Meissner et al meta-analysis appears to resolve the major point of contention, more research is definitely needed to ascertain exactly why the effect sizes for placebo responses vary so much as a function of type of study.

\section{Physiology of the Placebo Effect}
\label{sec:neur-plac-effect}


The placebo effect is an interesting phenomenon in that it straddles the boundaries of psychological and physical. This section will examine the research demonstrating the effects of placebo on a neurological level, and then examine other physiological impacts and correlates of placebo administration. 




\subsection{Brain Correlates of Placebo}
\label{sec:brain-corr-plac}

A strand of placebo research which has become more and more important with time has been the increasing focus on brain correlates of placebo responses. 

% \subsection{Problems with fMRI studies of placebo}
% \label{sec:problems-with-fmri}

Much of the recent research on physiological correlates of placebo effects has been carried using using functional magnetic resonance imaging. This method is extremely expensive and time consuming, and so typically sample sizes are small. Despite this, many researchers have reported surprisingly large effects. 
A potential reason for this pattern was proposed by Vul \textit{et al} in a paper which caused much controversy in the field \cite{vul2009puzzlingly} . 

Essentially Vul and colleagues pointed out that methods of analysing brain data were prey to issues of multiple comparisons, where voxels were chosen after the fact based on significance levels. This statistical error occurs in many contexts (indeed it was also a major problem in genomics) and leads to inferences which tend not to replicate. Some proposed methods of preventing these errors (borrowed from computer science) are discussed in Chapter \ref{cha:methodology}. Vul also pointed out that the maximum correlations that were possible between two measurements were bounded by the product of their reliabilities, and provided evidence that this restriction was violated in many fMRI studies. It is important to bear in mind that many of the studies reported here were conducted before the publication of this paper, and some of them may suffer from these same flaws. Issues with particular papers are noted as and when they appear. 

\subsection{Neurobiology of self reported pain}
\label{sec:neur-self-report}

Pain is the area where much of the recent neurological work has been done, as this is where the majority of placebo research has been carried out in recent years. In this section, the correlates and findings from this area of research are summarised. 

There appears to be a neural dissociation of the somatic and affective components of pain in the brain with the affective parts activating the dorsal anterior cingulate cortex and the sensory parts involving the somato-sensory cortex \cite{Lieberman2004}. This dissociation is also reflected in the measurement of pain in placebo studies, with intensity and unpleasantness typically being rated separately \cite{Price2008}. Lieberman et al claimed that there were two regions in the brain which inhibited one another and that this contributed to placebo analgesia, but this theory was not supported by the data. 

Another interesting suggestion was that placebo analgesia experiments which show altered brain activity in the rostal anterior cingulate cortex (rACC) and orbito-frontal cortex (OFC) demonstrate the existence of a generalised expectancy network. This hypothesis received some support from a recent experimental study which used either true or false sound cues to create expectancies for particular aversive tastes. This study showed that the rACC and OFC and to a lesser extent, the dorso-lateral pre-frontal cortex (DLPFC) activated in response to these expectations, suggesting that these parts of the brain may well be associated with the expectancies  \cite{Sarinopoulos2006}. 

Futher evidence for this viewpoint is research which shows that the rACC and the RLPFC are related to willed behaviour, which would seem to be associated with expectancies \cite{Beauregard2007a}.  However, Craggs \textit{et al} showed that these correlations of the rACC with placebo analgesia were not significant after correcting for multiple comparisions \cite{Craggs2008a}. Indeed, more recent research \cite{krummenacher2010prefrontal} demonstrates that when the prefrontal cortex is inhibited by repetitive transcranial magnetic stimulation (rTMS) then no placebo response occurs. This almost certainly indicates that expectancy related placebo effects are mediated by this part of the brain. Another study showed correlations between the dorso-lateral prefrontal cortex and expectancies, which appeared to be correlated with the placebo response \cite{Zubieta2006}. 

Additionally, placebo effects on emotional processing activated the VLPFC, the rACC and the ROLFC, suggesting that there is a common substrate for placebo effects in this area \cite{Beauregard2007a}. It is important to note that all of these studies examined the neural correlates of expectancy induced placebo effects, and so it remains to be determined whether or not these brain regions are part of all placebo effects. 
 
An interesting finding arose from an experimental study into patients suffering from Irritable Bowel Syndrome (IBS) \cite{Lieberman2004}. This study looked at placebo using a disruption theory account, which accounts for neural changes due to placebo in terms of inhibition. The authors found that although the right ventro-lateral pre-frontal cortex was activated by expectancies of analgesia, this activity was totally mediated by the dorsal anterior cingulate cortex which argues that this part of the brain is foundational to the placebo response.  

There is some evidence to suggest that some of the effects may involve both descending and ascending pathways within the brain, judging from the results of a study on mechanical hyperalgesia \cite{Goffaux2007}.This study used a counter-irritation technique and the use of a basin of water to act either as a placebo or nocebo. The authors suggested that the reflexes in the arm should not change if the placebo effect was completely cortically mediated, but the results suggest that descending pathways are equally as important in placebo analgesia. These pathways are controlled from the mid-brain and these findings suggest that the placebo effect exerts changes in large portions of the body, and is not exclusively a cortical phenomenon. This finding would seem to support a more embodied conception of placebo. This finding, and others like it are discussed more fully in Section \ref{sec:sec:non-brain-effects} below.

The lateral orbito-frontal cortex has also been associated with placebo analgesia in some studies \cite{Petrovic2002}, and additionally has been associated with the cognitive control of pain. This part of the brain may be associated with the generalised expectancy network suggested above, and this would seem to fit the evidece. The link between the upward and downward pathways suggested by Goffaux \textit{et al} may be in this brain area. 

Additionally, these areas (the ACC and PFC) have large amounts of projections to the periaqueductal gray (PAG) area, which is extremely rich in opioid receptors  \cite{Colloca2008a}. There are also dopamine receptors in these areas \cite{Fuente-FernAndez2004}, suggesting that multiple neurotransmittters may exert their impact through this area. This would make sense as this part of the brain has previously been noted to be involved in the regulation of pain. 

Further evidence in favour of the idea that placebo effects are mediated by both upward and downward pathways to and from the brain comes from the study of Matre \cite{Matre2006a} who noted large differences in mechanical hyperalgesia between placebo and control areas of the body, again suggesting the involvement of the whole body in the response. In this context, the results of Roelofs \textit{et al.} \cite{Roelofs2000} are worth considering. Using similar techniques to the two other studies referenced in this Matre and Goffaux, they found no evidence that placebo effects cause changes in spinal reflex activity. However, this study also found no evidence for a placebo effect in general, which weakens their conclusions. It is worth mentioning that even though they found no significant effects, they did find a correlation between the brain activity and spinal reflexes, which suggests that they found an effect, but their study was either underpowered or used a badly designed expectancy manipulation (most likely the latter) \cite{Goffaux2007}.

An interesting finding which has come about through placebo research is what is known as the uncertainty principle in analgesia \cite{Colloca2005} , where it is argued that the effects of any analgesic can not be accurately measured in a clinical situation as the awareness of being given this substance will activate the opioid system which will further reduce pain. This finding arises from work done previously, where it was shown that open injections of painkillers or placebo registered far more variability than hidden injections \cite{Benedetti2003c}, suggesting that while physiological responses to analgesia may be similar across people, the awareness of treatment may invoke differential activation of endogenous painkilling systems which cause the total effects to appear to vary quite substantially \cite{Amanzio2001} .Research has also confirmed that placebo and opioid analgesia share the same neural patterns of activation in the brain \cite{Petrovic2005}.

\subsubsection{Depression and Placebo}
\label{sec:depression-placebo}

Much research  has also been done in the area of depression and placebo response.  A fascinating study \cite{Hunter2006}  suggests that prior to treatment, placebos may induce changes in neurophysiology which predict later treatment response. This is an extremely interesing finding, however the authors used a new measure (that of EEG cordance) developed by themselves and to date, there have been no replications of the study. Another useful study of placebo neural activity in depression has also been conducted comparing the activation of particular brain regions following treatment with either prozac or placebo \cite{Mayberg2002}.

 This experimental, double blind Positron Emission Topography (PET) study showed that placebo and Prozac both activated common brain regions in the prefrontal cortex, premotor cortex, posterior insula, posterior cingulate, subgenual cingulate, hypothalamus, thalamus, insula and parahippocampus. Prozac additionally activated areas of the striatum hippocampus and anterior insula. These findings are intriguing as they support the recent meta-analytic evidence that the placebo response accounts for much of the effect in antidepressants \cite{Kirsch2002a}.

 One fascinating finding of the Mayberg et al study is that areas of the striatum were activated, and this region of the brain is known to be rich in dopamine receptors \cite{DeLaFuente-Fernandez2002}, which may suggest that while the placebo response in depression is primarily opioid mediated, the effects of SSRI's may also influence the dopamine systems, which may account for their superior effectiveness overall. However, some research shows that psychotherapy activates different brain regions in the treatment of depression, which argues against the existence of a common depression treatment pathway in the brain \cite{Benedetti2008}.

\subsection{Non Brain Effects and Correlates of Placebo response}
\label{sec:non-brain-effects}

\subsection{Opioids and Placebo}
\label{sec:opiods-placebo}

The biochemical history of placebo begins with Levine \cite{Levine1978a} and the demonstration that naloxone blocks many placebo pain responses. Induced from this is the notion that placebo pain relief is mediated by the endogenous opioid system. 

This finding has been qualified by research over the past thirty years, suggesting that both opioid and non-opioid systems can be involved in the placebo pain responses depending on the the method of inducing placebo responses and the biological system involved. \cite{Amanzio2001, Benedetti2003a}. The lasting contribution of this research is that it paved the way for the placebo to come in from the fringes of medical science.

In this area, the work of Benedetti and his colleagues has been instrumental in unveiling the biochemical pathways through which placebos exert their effects, and much of this work is summarised in his book.  It appears that both the opioid and dopaminergic systems are involved in the placebo effect.  Benedetti and colleagues have demonstrated that respitory depression can be indcued by placebo administration \cite{Benedetti1999a}. 

A further discovery with regard to placebo analgesia is that it can be directed at specific sites in the body \cite{Benedetti1999}. This study induced expectancies of placebo responses at either the right or the left hand, and demonstrated the expected placebo effects. These effects were completely antagonised by naloxone, which suggests that they were mediated by the endogenous opioid system. 

This finding is interesting as it suggests that the opioid systems can be activated at specific parts of the body, and not just globally as some former theorists have claimed. A more recent finding \cite{Watson2006} found that perhaps 50\% of participants in a placebo analgesia study generalised a placebo response across both arms, even though cream was only applied to one arm for each person. This study would suggest that the placebo analgesia phenomenon is quite malleable and subject to individual interpretation. 

Further research on the blockade of opioid receptors by naloxone has established that proglumide can be used to increase the size of placebo analgesic effects \cite{Benedetti1995}. Additionally, CCK, a chemical which tends to produce anxiety in human participants, has been shown to increase the size of the nocebo effect \cite{Benedetti1996}. 

A recent meta-analytic review \cite{Sauro2005} seems to argue that placebo effects in pain are quite large (d=.89) and that naloxone is quite effective in reducing them (d=.55), pointing towards an interpretation of placebo effects in pain being substantially mediated by endogenous opioids. 

Unfortunately, Sauro \textit{et al} do not report what kinds of receptors these endogenous opioids bound to, as this was not the primary focus of their meta-analysis. They did find a significant difference between the effect sizes in post-operative and experimental pain, with post-operative pain showing an average effect size of $d=0.65, 95CI(0.37-0.87)$ while the experimental studies showed an average effect size of from $d=0.53, 95CI(0.02-1.04)$ for shock induced pain, to $d=0.72, 95CI(0.34-1.16)$ for capsician induced pain, to $d=0.62, 95CI(1.00-1.46)$ for ischemic pain, suggesting that ischemic pain is the best way to invoke a substantial placebo effect. Note that all $d$ measures above, are Cohen's $d$, and are weighted effect sizes based on a the inverse of the variance in the sample from which the estimates were drawn. 

The issue of which receptors are implicated in placebo analgesia is important, as they have different effects. Up to eighteen types have been reported, but there are three main types, the mu, kappa and delta receptors. These receptors have differing sites of action, and information regarding which ones are activated by placebo analgesia will presumably improve our understanding of how and when this phenomena is likely to occur.  The anterior cingulate cortex, as discusssed above appears to be involved in placebo analgesia, and is also involved in the opioid system in the brain. This area is rich in mu receptors, which would seem to suggest that this kind of receptor is important in placebo analgesia.

One study which looked at patients suffering from IBS found that naloxone did not reduce the size of placebo effects, which would suggest that these were not opioid mediated \cite{Vase2005}. It remains to be determined why placebo effects in IBS are not opioid mediated, and understanding this may give some insight into the phenomenon. 

The ACC area of the brain is involved in the brain opioid system, and additionally is also associated with nitrous oxide (see Section \ref{sec:nitr-oxide-plac}). 

\subsubsection{Dopamine and Placebo}
\label{sec:dopamine-placebo}

While Benedetti and others have done much of the research into the opioid system, De La Fuente Fernandez \cite{DeLaFuente-FernAndez2002} has published a large amount of research looking at the dopaminergic system. It has been observed that the dopamine system activates not just to reward, but rather the expectancy of reward, and that this release varies as a factor of the certainty of the expectancies \cite{Scott2007}. In one study, the activation of the dopaminergic systems during placebo analgesia was correlated with activity observed during a monetary reward task, suggesting that the mechanisms of reward are a common feature of placebo effects \cite{Scott2007a}. This finding also may provide a mechanism through which certain and uncertain expectancies exert their effects (see Section \ref{sec:cert-uncert-expect}.   

De La Fuente Fernandez has argued that there is a descending link from the OFC to the Periaqueductal Gray Area (PAG) and from here to the amygdala, and that this link is responsible for the observed placebo effects \cite{Fuente-Fernandez2002}. These areas (along with the substantia nigra, which has also been linked with placebo response), produce large amounts of dopamine, and this may play a large role in the mediation of placebo effects and the physical body.Additionally, Fuente-Fernandez has argued that all of the parts of the brain that activate during the response to placebo receive dopaminergic projections \cite{Fuente-Fernandez2002}.  

Other research has shown that the expectation of drug triggers large releases of dopamine in the brains of patients with parkinson's disease, and this release of dopamine directly causes improvement \cite{Pollo2002}. The really interesting question that arises from this research is why, if patients with Parkinson's can release this dopamine when treatment is expected, do their brains not release these levels of dopamine naturally?. Additionally, this study also found that the amount of dopamine released was related to the amount of improvement, in a dose-response fashion \cite{Fuente-Fernandez2002}. 

This finding should be tempered with other research that indicated that there was no correlation between the amount of dopamine released and the size of the placebo response \cite{Scott2007a}. However, correlations only test for linear relations, and many relationships in the body are non-linear, suggesting that this finding may not be particularly robust. 

Additionally, the expectancies surrounding motor improvement have been found to correlate with actual motor improvement in both verum and placebo sub-cranial thalamic stimulation for Parkinson's disease \cite{Benedetti2004a}. 

Expectations of receiving caffeine have been associated with dopamine release, which would seem to provide further evidence that dopaminergic systems are involved in the difference in outcomes between certain and uncertain expectations Kassien (2004), cited in \cite{Beauregard2007a}. 

Additionally, there is some evidence that the mid-brain dopamine cells associated with addiction and reward project on to areas which are involved with motor and emotional function \cite{DeLaFuente-Fernandez2002}.  This ties in with the effects that CCK has on the nocebo effect (as it additionally induces anxiety), and the involvement of motor function again ties in with a conception of the placebo as a much more embodied phenomenon than is currently thought. The implications of these findings will be examined in Chapter \ref{cha:notes-towards-theory}. 

It is important to realise that dopamine and opioid systems may interact, and the limbic system (which the PAG is a part of) within the brain appears to be the site where they do so \cite{Fuente-FernAndez2002}. This section of the brain is associated with both mood and movement, and it may be here that the effects of CCK are exerted, along with those of proglumide, either reducing or increasing the size of the placebo response. 

\subsubsection{Nitrous Oxide and Placebo}
\label{sec:nitr-oxide-plac}

It has also been hypothesised \cite{Stefano2001,Fricchione2005} that Nitrous Oxide (NO) is involved in the placebo effect. These authors argue that the placebo effect is similar to the relaxation response, and they present a substantial amount of evidence that links Nitrous Oxide to various health promoting systems in the body. However, all of their evidence is quite circumstantial and no empirical study has tested the involvement of the NO system in the placebo effect.  

NO does regulate the production of both dopamine and norepiphedrine, and also disinhibits the actions of striatal neurons, which have been associated with placebo effects in Parkinson's and in placebo analgesia \cite{Fricchione2005,Fuente-FernAAndez2001}. Additionally, mu opioid receptors are activated by NO synthesis, additionally suggesting that NO may play an important role in the placebo response. 

 The rostral anterior cingulate cortex has been associated with placebo analgesia in many studies, and this part of the brain does produce large quantities of  NO, which may suggest that nitrous oxide does have some input to placebo analgesia. The findings noted above with regards to the rACC also include that this area of the brain is associated with hypnotic analgesia also. This may be because both placebo and hypnotic analgesia are phenomena of suggestion, and thus fall under the control of the generalised expectancy network described above.  


% One intriguing finding, noted above is that conditioning can affect hormonal and endocrine responses, while expectancies cannot \cite{Benedetti2003a}. This finding is important, as hormonal and endocrine responses were those found by Meissner  \textit{et al} to not show significant placebo effects in a review of randomised controlled trials. This would seem to suggest that the only placebo effects which are controlled for in clinical trials are those related to expectancies. 

\subsubsection{Placebos and the heart}
\label{sec:placebos-heart}

Heart rate and patterns across heart beats are altered by opiods, and also by endogenous opioids released during the placebo effect. While heart rate and the placebo have not been studied as extensively in recent times as neural or endocrine correlates of the placebo, there has been some work done, mostly in an experimental setting. 

One study found that the heart rate was reduced by placebo analgesia, and this response was blocked by naloxone, suggesting that this effect was mediated by endogenous opioids \cite{Benedetti2008}. Matre and colleagues found no effects of placebo administration on blood pressure or heart rate \cite{Matre2006a}, but they appear to have conducted a between groups ANOVA, which is an inefficient means of examining the changes evoked by placebo (as these variables would have changed continuously over time, and would probably ne best modelled as time series). Indeed, this problem is common in placebo releated research, and the effects of this problem and some possible resolutions are discussed in Chapter \ref{cha:methodology}. 

Additionally, one of the studies investigated earlier showed an effect of placebo administration on blood pressure \cite{Shiv2005a}, but only when participants were motivated to solve puzzles (which was the task in that study). This particular effect seems more likely as it was examining effects of a energy drink placebo, and increased blood pressure is a side effect of consuming caffeinated drinks. Flaten in a 1999 study, found no effect of placebo on any of the physiological outcome variables studied \cite{Flaten1999}

Heart rate variability is one measure which has been examined in some studies. Alasken \textit{et al } (2008) argued that placebo administration would decrease the ratio of High frequency to low frequency waves, and this hypothesis was supported in their experiment \cite{Aslaksen2008}. 

\subsubsection{Placebo and Skin COnductance}
\label{sec:plac-skin-cond}



Skin conductance is another measure sometimes used in placebo experiments. Some authors have reported no difference between groups on this measure \cite{Flaten1999}, but these authors also examined differences between groups using an ANOVA method, which is not appropriate for this time series data. One extremely interesting study claimed that pain ratings could be derived from the measurement of skin conductance, and that active drugs changed the response patterns, while placebo administration did not \cite{Fujita2000}.

The Crum et al exercise study discussed earlier also showed physiological effects of their exercise ``placebo'', arguing that suggestion and expectancy effects can have sizeable impacts on physiological outcomes \cite{Crum2007}.

Placebos have been noted to effect hormone levels also, but these placebo responses appear to be completely insensitive to expectancies, and can only be induced by conditioning \cite{Benedetti2003a}. 

These kinds of physiological markers of response to placebo are extremely useful as they can be used to determine if a physiological placebo effect is occurring, or if the change in self rated pain is driven by more cognitive re-appraisals of the situation. If more cognitively driven, it would be expected that these changes would lag the changes in self reported pain, whereas if the placebo were mediated locally then it would be expected that the physiological changes would occur in advance of a reported drop in self rated pain. One cautionary point is that biological mediators and correlates of placebo are not the cause of placebo effects, rather they are examples of them, and while the study of physiological correlates of placebo can help to understand how they are mediated, it will not remove the need for social and psychological explanations for how they occur \cite{Stewart-Williams2004b}. 


\section{Review of the Placebo Effect}
\label{sec:revi-plac-effect}

In this section, the major themes which have emerged from the literature review of the placebo effect are recapped. There are a number of major points to bear in mind. 

Firstly, the placebo effect is a difficult phenomenon to define precisely, and there is wide variability in what is considered to constitute a placebo effect. Perhaps the best conceptualisation of the effect is that proposed by Goetzche \cite{Gotzsche1995}, where it is argued that placebo effects should be broken down into three parts: those attributable to the patient-provider interaction, those attributable to the administration of the medicine, and those attributable to the context in which the treatment was delivered. 

The next major theme to emerge from this review is that that there is one theoretical perspective shared by most researchers in the field, that of response expectancies \cite{Kirsch1997,Kirsch1985}. However, this theory, while still core to the conceptualisation of the effect, has been shaken by a number of demonstrations that expectancies are not always the best predictor of placebo \cite{Hyland2006,Geers2005a}. This theory also suffers from the lack of clear terminology and measurement tools to measure expectancies, as deficency which this research hopes, in some small way, to remedy. 

The next theme to emerge from this review is that the placebo effect is a complex phenomenon which has been difficult to study accurately. However, there appear to be a number of characteristics of both patient, provider and context which can either  facilitate or reduce placebo effects.  

The major factors relating to context appears to be the setting in which placebo is administered (clinical trial or experimental), along with the participants beliefs about treatment assignment. Another major contextual factor appears to be the rationale given for the placebo's effects.  In addition, provider factors appear to include the level of suggestion given (certain versus uncertain) and the charisma and authority of the provider. Patient characteristics which have been shown to effect the response to placebo include dispositional optimism and pain catastrophising. 

A further theme which emerges from this review is that the various placebo responses observed in clinical and non-clinical populations appear to recruit a number of neurobiological systems, at the very least the opioid and dopamine systems, and potentially the serotonin system also. It also appears that placebo effects can display a high level of specific action at particular parts of the body, and involve both the central and peripheral nervous system. 

The meta-analytic evidence, though conflicting, appears to indicate that placebo effects occur when the outcome variable is under the control of the central nervous system, and do not occur nearly as much in the endocrine system. However, this finding is based on clinical trial data and contradicts the successful conditioning of humans to respond to endocrine placebos \cite{Benedetti2003a}. The size of placebo effects is also a matter of some dispute, and appears to significantly differ as a function of study context, as noted above. 

In conclusion, the placebo effect is a complex phenomenon which appears to provide a link between the psychological and physiological  experience of the world, and which is associated  with some psychological and physiological variables. 


\part{The Implicit Association Test}
\label{part:impl-assoc-test}

\section{Introduction to Implicit Measures}
\label{sec:intr-impl-meas}

Psychology as a science, and indeed the social sciences more generally, have a problem. They seek to understand the mind and behaviour of individuals in particular contexts and cultures. While behaviour is less problematic to observe scientifically, the observation of mind is fraught with problems. Almost all of the constructs of interest to psychologists (mind, love, experience) are unobservable with the naked eye, and require interpretation in order to be understood. 

These constructs are still measured however, but the means to get at them are more indirect and subtle than the microscope. In the case of many variables, psychologists measure what people think and believe from their answers to questions devised by the psychologist, normally referred to as self-report instruments. 

This approach has obvious advantages, in that it is quick, cost-effective and can produce results of interest. However, as psychology has matured, a number of problems have become apparent with this approach \cite{Nisbett1977}.  The first problem is that, especially in controversial topics, people may attempt to conceal their true beliefs or attitudes. The second, somewhat more philosophical problem, is that people may be unaware of their true beliefs, or at least may profess to believe one thing while behaving in a manner consistent with a belief in the opposite. 

The first of these problems is called social desirability \cite{Egloff2003}, and indeed psychologists have developed other self report scales that purport to measure this construct. The second problem, first noted by Freud, is that of unconscious (or implicit) influences \cite{Hofmann2008}. While the system built by Freud no longer forms part of the framework of modern psychology, the contradictions between reports of experience and behaviour remain, and are still relevant to the aims of psychology. %%insert allport research on chinese attitudes and behaviour here

\subsection{Older non-self report methods}
\label{sec:older-non-self}

Some methods have been developed to get around this problem. One of the first techniques used for this purpose was that of free association, where a client was asked to respond to a stimulus word or picture with the first word that came to mind, without censoring the experience \cite{Hofmann2008}. These associations could then be used by the therapist to gain access to material which the client did not consciously report being aware of. 

Another method which was used was that of Rorschach ink-blots, where ambiguous ink blots are shown to the client, who interprets them. This technique can also provide insight into the mind of the client, but again this requires interpretation on the part of the therapist. It is this interpretation process that causes these procedures to lack scientific validity in the eyes of many, as what one therapist understands by the clients words may differ completely from what another therapist takes from the same material. 

These approaches were abandoned following the rise of behaviourism as the dominant approach within academic psychology, and further eclipsed by the notions of Karl Popper regarding falsifiability as a criterion for scientific theory. It was argued that since unconscious influences could be used to explain any criticism of the theory (and indeed, Freud was prone to doing this) then the theory was not truly scientific. The development of psychometric theory also played a role in the decline of interest in such instruments, as these methods seemed to produce reliable and valid data and scores could be corrected for impression management and social desirability biases by statistical techniques. 

\subsection{The turn back to indirect measures}
\label{sec:turn-back-indirect}
In recent years, however, there has been a resurgence of interest in such techniques. This resurgence grew out of the work on implicit memory and learning, where participants would consciously deny awareness of some piece of information while their behaviour seemed to show signs of this knowledge .  

This phenomenon can be seen in  experiments like word completion tasks. If participants are given a list of words to memorise, and then distracted by another task, followed by the word completion task, they tend to far more frequently complete the word fragments with the words on the previous list which was to be memorised.  However, they will typically deny this influence on their responding if asked \cite{Wittenbrink2007a}. 

These approaches, allied with the continuing failure of self reported attitudes to predict behaviour, caused some researchers to look for another way to measure these constructs \cite{Greenwald1995a}. The result of these investigations was the Implicit Association Test, or IAT for short \cite{Greenwald1998}. The IAT is a reaction time measure which makes inferences about attitudes from the time which it takes participants to categorise words or pictures of a group into one of two categories. 

\section{Introduction to the Implicit Association Test (IAT)}
\label{sec:intr-impl-assoc}

The IAT was developed by Greenwald, and he suggested that because of its design, it might be more resistant to social desirability influences and demand characteristics \cite{Greenwald1998}. 

Social desirability tendencies would lead people to deny prejudicial behaviour in self report instruments, while as a reaction time measure, the IAT is less easy to fake. Demand characteristics result when participants in an experiment give the answers a researcher wants, rather than their real beliefs or attitudes. Again, it seems intuitively harder to do this within an IAT methodology as it would require tremendous and consistent control of reaction times to stimuli (though not impossible as discussed in Section \ref{sec:iat-contr-faking}. 

\subsection{Description of the Procedure}
\label{sec:descr-proc}

The (IAT) is a computer administered procedure which purports to measure implicit associations not directly accessible to consciousness \cite{Greenwald1998}. The test was developed as a result of mounting evidence for learning without awareness in human participants (see above section \ref{sec:intr-impl-meas} for some examples). 

This research was reviewed by Greenwald \& Banaji \cite{Greenwald1995a}, where they introduced a distinction between direct and indirect measures of social cognition. They referred to self report instruments as direct measures, and to such techniques as semantic priming as indirect measures. Semantic priming is the tendency for participants in experiments to give answers to ambiguous tasks similar to ones they have recently observed in their environment.  They defined implicit associations in the following manner \textit{the unidentified or inaccurately identified traces of past experience}, and this definition implied that self report measures were not the best tools with which to assess these associations. 

The procedure works as follows. Firstly participants sit down in front of a computer and are assigned two keys (typically the ``e'' and ``i'' keys) to respond to each word or image presented, which fall into one of two categories. The major measure is reaction time, and an assumption of the method is that categories which are more strongly associated in consciousness will be easier to combine than those which are less associated.  The participant is asked to classify words into either pleasant or unpleasant categories as they appear on the screen at the front of the computer, for example love, hate, good  and bad are words typically used in this part of the procedure. 

Then, participants are asked to categorise faces into either black or white categories. Following this, the two categories are combined, with one key being pressed for White or Pleasant and another being pressed for Black or Unpleasant. Then, the labels are reversed, and the participant categorises White faces with Unpleasant and Black faces with pleasant. These response times are summed and averaged for each participant for each of the critical blocks, and the two block scores (Incongruent - Congruent) are subtracted from one another to produce a difference score which is referred to as an IAT effect. 

The procedure is not limited to assessment of racial attitudes, and has been applied far more widely \cite{Craeynest2008,Greenwald2009, Schmukle2008,Walker2008}.  A general schema for the process follows.   

Firstly participants classify words as either belonging to Category X or Y, where X and Y are positively (love, flowers etc) or negatively (hate, insects etc) associated words or images.  Then they classify faces or names as either belonging to Group A or Group B, where the labels are typically descriptive of groups of people.  In the next step, these two associations are combined, with one key being a response for A and X and the other key being used for responses of B and Y. 

In the fourth step, the keys for pleasant and unpleasant are reversed, and in the final step the two dimensions are combined in the opposite manner (A and Y or B and X). In practice, only the 3rd and the 5th steps are analysed, and the difference between mean response latencies on the different combination tasks is assumed to represent an IAT effect \cite{Greenwald1998} . 

In essence, any differences in reaction time in the combination of the two categories are assumed to be due to underlying differences between the relative associations of the concepts. The authors claim that the use of difference scores allows them to prevent issues of processing and response speed variability across individuals from distorting the results. However, this assumption has been questioned recently \cite{Blanton2006}, who claimed that processing speed was a major moderator of the observed variance in scores across the population studied.  

\section{Psychometric Analysis of the IAT}
\label{sec:uses-psych-feat}

The method has become very popular, and has been applied to many areas of social psychology, such as attitudes towards fatness \cite{Ahern2008}, towards disability \cite{Pruett2006} and towards smoking \cite{Kahler2007}. A few common features of the measurements taken with this instrument seem to be the following. 

\subsection{Convergent Validity}
\label{sec:convergent-validity}



\subsubsection{Relationships with Explicit Measures}
\label{sec:relat-with-expl}



IAt scores are typically weakly correlated with explicit measures of similar attitudes. These correlations average ($r=0.39$) \cite{Nosek2005},and so one could be justified as regarding the two as distinct constructs \cite{Nosek2007a}. Indeed, this is the approach taken by many of the originators and early workers in the field \cite{Greenwald2000,Nosek2007a}. 

Secondly, they tend to reveal stronger associations than explicit measures when the topic is politicised or controversial \cite{Greenwald2009}. 

Work on self-esteem has suggested that implicit and explicit self-esteem have a synergistic effect, in that participants with high implicit and explicit self esteem show much better resistance to a forced failure task \cite{Meagher2004}. This relationship for the constructs under study in  this thesis was examined in this thesis, to assess whether or not it was more general than self-esteem (see Chapter \ref{cha:primary-research}). 

It was proposed, based on theoretical considerations, that implicit measures should be more resistant to change than explicit measures. However, recent research has shown this not to be the case \cite{Meagher2004,Gschwendner2008}. In these experiments, IAT scores were shown to change much more than explicit measures in response to experimental manipulations. This may suggest that the measure captures state rather than trait variance, a position which will be discussed further throughout this chapter. Dasgupta \cite{Dasgupta2001} determined that implicit evaluations could be significantly changed by displaying either positive or negative exemplars of the construct under study. Dasgupta \textit{et al} suggested that this may mean that implicit associations are more effected by shallow processing techniques, which is a theory entirely consistent with the notion that they capture state variance. 

One major requirement for convergent validity is that the measure should correlate with other measures known to tap similar constructs. This requirement has been fulfilled for the IAT. One study showed that mindfulness (as measured by the MAAS) was a mediator of the relationship between explicit and implicit attitude scores (in the particular study concerned, attitudes towards autonomy) \cite{Levesque2007}. Additionally, private self consciousness, an analogous construct was found to also act as a moderator of the relationship betwen implicit and explicit attitudes \cite{Gschwendner2006}. Introspection was negatively related to the correlation between implicit and explicit measures \cite{Hofmann2005}, while spontaneity was positively associated with this correlation. 

Another feature which moderates the relationship between implicit and explicit attitudes is that of attitude importance. As attitudes increase in importance, the correlation between explicit and implicit attitudes increases, suggesting that implicit attitudes tap awarenesses that can be, but are not necessarily, conscious \cite{Karpinski2005}. 

Additionally, attitude elaboration increases the size of the correlation between implicit and explicit measures \cite{Karpinski2005}, this was demonstrated in a study of soda choice where participants asked to write about their attitudes towards soda showed a higher correlation between their explicit and implicit attitudes towards the topic. 

It does appear that IAT's may predict spontaneous behaviour better than do explicit measures \cite{Asendorpf2002,Richetin2007,Perugini2005}. This has been demonstrated in a number of domains, and appears to suggest a so-called double-dissociation model, whereby (in a food example) IAT measures predict spontaneous food choice while explicit measures predict food diary measures. 

The double dissociation model has been tested using Structural Equation Modelling \cite{Nosek2007a,Perugini2005} and appears to provide a better fit than a pure explicit or pure implicit measure. Again, this form of modelling will be applied in this thesis to examine the measures developed and used in both the preliminary and primary research. 

The form of the explicit measure also appears to affect the relationship between explicit and implicit measures. It appears from a meta-analysis that relative explicit measures tend to have a far greater correlation with implicit measures than do explicit ones \cite{Hofmann2005}. This makes sense given that the IAT is a relative measure. 

\subsubsection{Relationship with other implicit measures}
\label{sec:relat-with-other}


Thirdly, they tend to reveal similar effects to other techniques such as semantic priming, although IAT's seem to be more sensitive to variations within the construct \cite{Wittenbrink2007a}. Fourthly, they have low test-retest reliability. This seems to average around ($r=.59$) which, while permissible in a psychometric instrument for theoretical purposes, is far too low for making clinical or legal judgements \cite{Greenwald2000, Blanton2006d}. It does appear that implicit measures may be capturing different facets of the implicit construct under study as some research suggests that two implicit measures of self esteem (the IAT and the self-apperception test) did not correlate \cite{Meagher2004}. However, when all potential confounders were controlled for, measures of implicit self esteem did converge. The major question for future research in this area is to determine why the correlations for measures ostensibly tapping the same construc t are so low. 

\subsection{Reliability of the IAT}
\label{sec:reliability-iat}



Split half reliabilities are typically higher, averaging around ($r=0.80$), but test re-test reliabilities are much lower, even one day afterwards. It is worth noting however, that the test-retest reliabilities do not drop much farther than this, even over periods as long as one year \cite{Egloff2005}. 

This may suggest that the IAT response is composed of both state and trait portions, and that the trait portion of the measure is relatively invariant across temporal distance. Some authors have argued that this low test-retest reliability is due to both error variance and person by situation interactions \cite{Gschwendner2008}.  Manipulation of accessibility of the constructs measured in the IAT has been shown to improve the temporal stability of the IAT scores, suggesting that particular situations may make the constructs more or less likely to be expressed \cite{Gschwendner2008}. 

Additionally, some research has suggested that the picture of implicit attitudes as developed early in life and resistant to change, may be incorrect \cite{Gschwendner2008}. This, and the evidence given below in Section \ref{sec:iat-control-fake} may point towards implicit measures as being more state based than early theory surrounding dual-process models would suggest. 

Some controversy has surrounded the use of difference scores as a metric \cite{Blanton2006d}. Blanton \textit{et al} argue that the only condition under which difference scores make sense is when positive and negative stimuli are equally valenced, and they further argue that this condition is not met for many of the most popular implicit association tests (specifically, the racism IAT). They further argue that, unless the IAT score can be linked to an observable outcome, then it is an arbitrary metric \cite{Blanton2006}. This contention of Blanton and Jaccard rests on modelling the IAT as two seperate measures which are then combined. In addition, they appear to argue that successive responses within an IAT trial are independent, which seems like an unjustified assumption. Given that the procedure requires participants to respond to multiple stimuli in quick succession, the responses within blocks are likely to be auto-correlated \cite{mccleary1980applied}, and even the independence of responses across a single IAT is in doubt.

The psychometric characteristics of the IAT are still not completely defined, and no single model has provided a coherent explanation for how and why the effects will occur. However, the measure does appear to have predictive validity (see Section \ref{sec:pred-valid-iat}) so the proper question for psychometric analysis is not whether the IAT has any effects, but rather how these effects occur. 



%% Perhaps the best explanation for the effects of IAT's is that the result from individual differences in task-switching abilities, and that these task-swtiching skills are -

\subsection{IAT, Controllability and Faking}
\label{sec:iat-contr-faking}

While it appears that while the IAT resists demand characteristics, it does not prevent them entirely \cite{DeHouwer2007b}. De Houwer's research demonstrated that the IAT procedure can be faked when applied to novel attitudes. This research  demonstrated a large IAT effect when participants were given positive or negative information about fictitious social groups. 

De Houwer \textit{et al} also obtained IAT effects when participants were asked to respond in the opposite way from the information which they had been given, although the effects were smaller, and not all participants were capable of this feat.  

The major contribution of this research is that it demonstrates that participants can alter IAT effects by choice, at least when applied to novel or unfamiliar attitudes. The authors suggest that this may restrict the ability of the IAT to study the development of novel attitudes, but it would not affect the study of well developed attitudes, as are looked at in most IAT research.

However, there is a more serious problem, as the research on training participants to fake the IAT indicates \cite{Fiedler2005}. This research carried out three experiments, one on-line and two in a controlled experimental set-up. The results indicated that given prior experience with the IAT, and some instructions on how the measure works, participants were capable of reversing the sign of the IAT effect (which equates to showing an attitude opposite to their own). 

Clearly, this is an issue for researchers, especially with such well known IAT measures as the Race IAT. However, the results of this study also showed that mere experience was not enough to allow faking. Participants who were given no information on how to fake were not able to change the direction of their IAT effect. Worse still the authors sent the faked data to a number of researchers in the field, and only one of them was able to identify the faked results, and this was done with an accuracy of 58\% using an algorithmic procedure.   

That being said, the results of Foroni and others \cite{Foroni2005}  suggest that there are some interesting features involved in the modulation of IAT effects. This experiment utilised a flower-insect IAT \cite{Greenwald1998} and two conditions. In the first, participants read a story about how the world had changed and insects were now a major source of nutrition for humans, while flowers were poisonous. 

The other condition presented the same information, but not in narrative form. The story proved successful at changing the IAT effect to insects positive and flower negative, while the same information presented in a descriptive fashion was unable to induce these changes. This study also examined the effects of telling participants to fake the IAT, and noted that this was not effective at all. 

These two contradictory results present us with a problem to explain. There are two possible explanations which I will address in this paragraph. The first is that without instructions, the IAT is very difficult to fake. The second is that participants may be better able to fake attitudes towards social groups rather than flowers and insects. 

The first explanation runs into difficulties as the Fiedler \textit{et al}  study showed small effects of faking even without instructions. The second seems intuitively plausible, as people may have more incentive to conceal negative attitudes towards out-groups rather than insects, and thus may learn how to do so more quickly. Neither of these explanations are particularly satisfying however, so the issue of how and when the IAT can be faked remains and open question for future research. 

 
\subsection{Predictive Validity of the IAT}
\label{sec:pred-valid-iat}


The major proposed advantage for the use of implicit measures is that they would act as better predictors of behaviour, or allow for more insight into hidden cognitions that were associated with behaviours yet either not accessible or not reported by participants \cite{Greenwald1998}. This section examines the extent to which these hopes have been fulfilled. While the IAT does not appear to be a better predictor of behaviour overall, it does possess some ability to predict behaviours which are typically hard to predict using self report measures. 

The classic demonstration of the difference in prediction between implicit and explicit measures relates to \cite{Asendorpf2002} who investigated the attitude of shyness. In this study, spontaneous shy behaviour was predicted by implicit associations, while controlled shy behaviour was predicted by explicit attitudes. This pattern has become known as double dissociation, and has been observed in a number of studies \cite{Perugini2005}, and  then supported by  a meta-analysis \cite{Hofmann2005}. 

A recent review of implicit measures of self esteem suggests that implicit and explicit self esteem are entirely distinct constructs \cite{Rudolph2008}. Implicit self esteem has been shown to predict response to success or failure \cite{Greenwald2000}. Additionally, the research on implicit and explicit self esteem seems to indicate that individuals can be classified as having particular types of self esteem based on their relative levels of implicit and explicit self esteem, where participants who have high levels of both implicit and explicit self esteem are classified as having genuine self esteem \cite{Meagher2004}. These participants tended to be more resilient and suffer less negative outcomes following a false feedback manipulation designed to reduce self esteem. Although the effect size for the explicit measures was much higher in this study, the implicit measure (the Self Apperception Test and the IAT) appeared to be more sensitive to the emotional tone of the feedback. 


In the domain of personality, implicit measures of all Big 5 traits have been correlated with spontaneous behaviour which reflected these traits \cite{Steffens2006} (see Section \ref{sec:iat-personality}). 
%%expand on this paragraph

There is some evidence that IAT's can predict spontaneous behaviour better than explicit measures \cite{Conner2005,Perugini2005,Grumm2007} ,In the Conner \textit{et al} (2005) study, using an experience sampling methodology, the IAT measured attitudes predicted how the participants felt on a day to day basis far better than the explicit (self-report) measures, but the explicit measures predicted global ratings better. The extra predictive power afforded by the IAT  was only demonstrated for negative affect,  while the explicit measures were equally predictive for both positive and negative affect.  This is an interesting finding, as it suggests that there are differences in the effects of implicit attitudes on emotion dependent on the valence. As this pattern was not observed with the explicit measures, it may be an avenue for future research aimed at determining how implicit associations predict experience and behaviour. 

%%this paragraph makes no sense without introducing the mindfulness findings.
% Interestingly enough, the Hofmann meta-analysis \cite{Hofmann2005} showed that introspection was negatively correlated with implicit-explicit consistencies, which seems strange in light of the relationship to mindfulness observed in the Connor and Brown study. % An important point 
% to make here is that mindfulness is pre-reflexive , while introspection reflects on thought, mindfulness merely notes it without attempting to analyse it (even in vipassana meditation, the thought is labelled and then let go). 

There exists substantial evidence that IAT measured preference for males over females  can result in prejudicial behaviour against females in a simulated interview setting \cite{Greenwald2000,Heider2007}. The magnitude of the IAT scores was correlated with the observer-reported prejudicial behaviour scores. 

Some have critiqued these findings as both IAT and behavioural assessments were carried out in the same session, and this may falsely inflate the attitude-behaviour correlation. Another study \cite{McConnell2001} showed substantial correlations between IAT assessed bias and ratings of friendliness given to each participant in a scripted interaction. More recent research has demonstrated that even when separated by a week, attitude assessments using the IAT are significant predictors of verbal and non verbal friendliness with a compatriot of opposite race \cite{Heider2007}. 

The predictive validity of the IAT may also vary as a function of domain. An Italian study \cite{Arcuri2008} showed that the IAT was able to predict the future voting behaviour of people based on the results of an IAT measuring attitudes towards left and right wing candidates. This finding is quite impressive, given the difficulties political scientists normally find in predicting the behaviour of undecided voters. However, this should be qualified with the fact that it was self reported voting behaviour was measured, as opposed to actual voting patterns. 

Linking to the notion that the predictive power of the IAT varies across domains is a recent meta-analysis \cite{Greenwald2009} which compared the predictive validity of explicit and implicit measures across a large number of domains. In general, explicit measures appeared to be more predictive, except in the cases of inter-group behaviour and race attitudes. While this meta-analysis appeared to use some unclear selection criteria for studies, in general it was well conducted and the results show that the IAT has good predictive validity, especially in domains where there are social or awareness difficulties with the use of explicit measures. 

\subsection{IAT: Personal or Cultural?}
\label{sec:iat:-personal-or}


Another issue in the field of IAT research has been the relationship between implicit attitudes and cultural knowledge. Some have claimed that it is these extra-personal associations which influence response on the IAT \cite{Olson2004}. However, a recent study \cite{Nosek2007a} casts doubt on this interpretation. They found that consistently across domains measured by the IAT the relationship between cultural knowledge and implicit attitudes was almost completely accounted for by variability in explicit attitudes. Using a sample size of over 100,000 they found weak relationships between cultural knowledge and IAT effects. This would seem to indicate that whatever the IAT measures, it is more personal than extra-personal. 

The IAT is a new and exciting measure which has opened up new avenues for research by psychologists.   The measure can reliably differentiate between groups possessing different attitudes, and seems to produce results which while correlating with explicit attitudes, appear to reflect a different underlying process \cite{Nosek2007a}. 

It can predict behaviour quite well in certain situations, especially in matters of importance to the participants and when they are working with limited cognitive resources. It appears to predict spontaneous behaviour much better than explicit measures, which is certainly of interest. This prediction of spontaneous behaviour, coupled with low test-retest reliabilities, would seem to suggest that there are major state components to the measure. 

However, the measure still has its problems. There is still no generally accepted theoretical rationale for its effects, it can be contaminated by such issues as task switching costs\cite{Klauer2005}, processing speed\cite{Blanton2006} and the context in which it is administered \cite{Boysen2006}. 

Although it was proposed as a potential ``true pipeline'' to the attitudes of persons, it seems as sensitive to social desirability concerns and the presence of others as explicit measures. Finally, although the measure is not perfect, it has noticeably increased our understanding of human social cognition and is still stimulating new and interesting research, which of course is the major criterion of success for any measure. 

\section{Uses of the IAT}%%this title needs to be changed
\label{sec:uses-iat}

The IAT is a useful measure, and we have touched upon some of these uses in previous sections. In this section, we will look at some of the major areas to which it has been applied over the last decade. While the IAT began as a tool to assess attitudes towards social groups, its uses have broadened to include measurement of personality, pain associations and evaluative conditioning. We will look at the success or otherwise of these attempts in the following section. 

\subsection{Consumer Research and the IAT}
\label{sec:cons-rese-iat}

Another area where the IAT has seen much use is in the area of consumer research \cite{Lane2007,Maison2001}. Research in this area has focused on the effects of implicit attitudes on consumer choices. The IAT has shown some predictive validity here \cite{Maison2004}  but not significantly above the predictive power of explicit attitudes \cite{Greenwald2009}. One area where the IAT has proven useful is in examining the effects of evaluative conditioning on mature brand preferences \cite{Gibson2008} which reported that the conditioning could change attitudes (as measured with the IAT) towards brands for which the consumer had no pre-existing preference.

\subsection{IAT and Personality}
\label{sec:iat-personality}
One of the most interesting outgrowths from the IAT has been the measurement of implicit personality. In a recent study, Extraversion and Neuroticism IAT's were found to converge with the Extraversion and Conscientiousness traits as assessed by the NEO-FFI \cite{Grumm2007}. However, another recent article, which attempted to assess trait anxiousness and angriness, found that there was a significant interaction effect between the order of administration of the IAT's \cite{Schnabel2006}. 

When the angriness IAT was administered first, the results of the anxiousness IAT were highly correlated. The converse, however, did not occur. The authors suggest that this may be because the participants applied a coding strategy to the first IAT which they then generalised to the second. This problem may also have arisen because Schnabel et al administered the two IAT's directly after one another, rather than with other tasks in between as was done in the Grumm et al paper. 

Another recent paper \cite{Boldero2007} which used the Go/No Go Association Test (GNAT) showed that implicit Extraversion and Neuroticism were able to predict reaction time in the experiment. More generally, the implicit attitudes were able to predict scores on the explicit attitude measure. 

However, the Big 5 traits predicted were different from the findings of other researchers, suggesting that there may be some method variance involved. This method variance has also caused problems for the IAT \cite{Mierke2003,Greenwald2003a} . Again, this highlights the issue that these new implicit measures may have confounding factors within them which can only be highlighted by further detailed research.

It does appear that implicit personality measurement can predict spontaneous behaviours related to  personality traits \cite{Steffens2006}. This study predicted behaviour based on the Big 5 personality traits. There was no effect for extraversion, but the experimental manipulation may have been poorly designed. This findings lends more credence to the double dissociation theory \cite{Asendorpf2002} which has been shown to apply in a wide variety of tasks since then \cite{Perugini2005,Conner2005}. 

\subsection{Clinical Use of the IAT}
\label{sec:clinical-use-iat}

The IAT has been applied to clinical psychological research recently enough, and this has been somewhat successful \cite{DeHouwer2002}. However, the test retest reliability of the IAT averages around 0.6, which is far too low to be used in clinical decision making. This has not stopped many researchers however, but these results should be treated with caution as the precision of measurement may not be adaquete for the purposes for which the IAT is being used. 

One interesting study \cite{Grumm2008} points to a useful area in which the IAT has been applied. In this Grumm study, participants were recruited with chronic pain conditions. Their associations were measured using a self +pain design for the IAT, and they then underwent psychotherapy coupled with mindfulness programs to examine if these treatments were capable of changing their implicit associations. 

The study found that they did, with the authors reporting significant drops in self+pain associations over the course of the treatment. This finding is important as it provides validity for the conception of the IAT as measuring something useful, as the scores on the measure tracked the progress of patients through the treatment program, and this is unlikely to have occurred by chance.  

 The IAT has also been successful in distinguishing between spider and snake phobics and the general population \cite{Egloff2002,Lane2007}, and one interesting avenue of research would be a conceptual replication of the Grumm et al 2008 study using phobia sufferers and CBT to examine whether or not the same pattern of changes in automatic responses would be found. 

% Another area where the IAT has been applied is that of evaluative conditioning \cite{Mitchell2003}. This area has not received that much attention, as the results tend to be less spectacular than some of those in social cognition. 

A recent paper \cite{Boschen2007} used an IAT and measures of skin conductance to look at the development of fear responses in phobics. 

These authors found that while the skin conductance measures responded immediately to the intervention, this change was not reflected in the implicit associations until quite some time afterwards. This would seem to suggest that different mechanisms unerlie the information processing biases (measured by the IAT) and the autonomic responses (measured by SCR), which is a useful finding that may contribute to our understanding of the genesis of phobias. 



\section{Moderators of the IAT}
\label{sec:moderators-iat}

\subsection{Contextual Moderators of IAT Effects}
\label{sec:cont-moder-iat}



Some contextual factors have been noted to affect IAT measures, even though these were some of the problems which the measure were designed to avoid. A study asking participants to complete the IAT under conditions when they believed that the experimenter would or would not know their scores (the so-called bogus pipeline) \cite{Boysen2006} showed a notable diminution in IAT effects in a measure of attitudes towards homosexuals when participants believed that the experimenter would know their scores. Further investigations revealed that this did not arise because of social desirability issues as the effects were similar under a bogus pipeline condition. Another study examining the attitudes of Italian students to Turkish immigrants replicated this finding, with IAT scores being reduced when in the presence of others \cite{Castelli2008}. 

\subsection{Cognitive Moderators of IAT effects}
\label{sec:cogn-moder-iat}



Another factor which appears to moderate the observed IAT effects is that of memory resources. A study looking at attitudes towards Blacks and Turks \cite{Hofmann2008a} found that the IAT acted as a far better predictor of behaviour when participants had been asked to remember a list of words than when they were untaxed. This finding was replicated by the same authors using a different sample and IAT, suggesting that it is quite robust.  This suggests that the attitudes measured by the IAT are the result of more automatic processes, and will have predictive power to the extent that the matter involved is not the subject of deep processing. 

The Hofmann (2008) study described above involved interaction with an experimenter of the out-group measured in the IAT, and the second study separated the IAT from the behaviour assessment by one week, so these results are both good measures of behaviour and unaffected by  issues of attitude-behaviour consistency . 

One recent study \cite{Perugini2007} showed that the predictive ability of the IAT increases when under conditions of self activation. This was operationalised in these experiments by asking participants to either circle self or non-self related words in a passage of text. This finding was replicated across four domains and was found to be valid in all of them. In one case the correlation between IAT and behaviour was raised from .36 to .76 under conditions of self activation, which is a huge change. 

Some research \cite{Dasgupta2001} suggests that showing exemplars of groups typically the subject of negative associations on the IAT measures (such as Black Americans or Females) can reduce the size of these associations. However, as described below, difficulty in recalling such exemplars can lead to larger IAT effects. 

It also appears that IAT effects are influenced by ease of retrieval mechanisms \cite{Gawronski2005}. This extremely well conducted study examined a number of implicit measures and the mechanisms through which they are influenced by context. 

Gawronski \textit{et al} class the IAT as a response compatibility measure, and argue that these measures are affected by ease of retrieval from memory. In support of this, participants who generally liked African Americans showed higher levels of implicit preference against this group when asked to generate a high number of either liked or disliked African Americans. Conversely, participants who generally disliked African Americans showed lower levels of prejudice when they generated a lower number of exemplars. The authors explicate this effect in terms of ease of retrieval. The subjective difficulty of generating the exemplars seems to alter the attitudes which the participants report \cite{Kahneman2002}.  %%insert Kahneman and Tversky study here. 
This is probably affected by attitude-behaviour consistency effects, so it would interesting to examine whether or not these changes remain stable over time.  

The IAT also appears to be affected by attitude importance \cite{Karpinski2005}. In this study, the authors demonstrated that both scores on a Republican/Democrat IAT and a Coke/Pepsi IAT became more predictive of self reported behaviour as attitude importance increased. This would suggest that the measure is more useful in matters where the participants have a investment into the attitude being measured. However, this study used two domains where explicit measures are normally better predictors than implicit measures \cite{Nosek2007d}, so the findings here should be treated with some caution. 

Another study \cite{Levesque2007} looked at the effects of mindfulness on expression of implicit attitudes and argued that high mindfulness can stop implicit attitudes from being expressed and over time, causes them to become more in tune with self reported attitudes. This study used an experience sampling methodology and examined attitudes towards autonomy and heteronomy. The findings suggested that participants high in mindfulness seemed to show higher levels of autonomy in general, and that mindfulness could act as a protective factor against the expression of unwanted implicit attitudes.  This finding has also been supported by other recent research \cite{Gschwendner2006}, when they noted that Private Self Consciousness seemed to correlate with the expression of implicit attitudes towards Germans and Turks. In addition, the self reported habit scale, a measure which has been found to be negatively correlated with mindfulness, was found to be associated with stronger implicit attitudes and less congruence between explicit and implicit attitudes \cite{Conner2007}. 

\subsection{Task Switching and the IAT}
\label{sec:task-switching-iat}



One psychometric characteristic correlated with IAT responses appears to be task-switching ability \cite{Mierke2003}. Mierke and Klauer established that IAT's which should not have been correlated showed substantial variance in common.  Through a series of experiments they demonstrated that task switching abilities appear to be the cause for this. They also reviewed this work in later research, and established that these differences could be controlled for by using the new $D$ algorithm developed by Greenwald et al \cite{Greenwald2003,Klauer2005}. 

The important question for future research in this area is whether or not task switching ability is independent of IAT scores, in that the distribution of IAT scores is similiar across all levels of the trait, or whether or not it affects the IAT scores significantly. If the former, then if this variable is controlled for, then there should be no problems, whereas if task switching ability is not independent of IAT scores, then the measure and scoring methods will need to be revised.  



\section{Criticisms of the IAT}
\label{sec:crit-contr-iat}

The IAT is a relatively recent measurement tool, and yet in its short life it has been involved in a number of disputes and controversies, far out of proportion to the number of criticisms which a measure usually faces \cite{VonHippel2004}. Some of this may be due to the  extreme popularity of the measure and its lack of any firm theoretical foundations, while other issues may have more to do with the political nature of some of the results claimed for the measure, especially in the United States.  The aim of this section is to examine these issues in chronological order, present the arguments for each side and to examine the changes which have taken place in the use of the IAT since these disputes. 

For the first three years of the existence of the IAT, the measure did not seem to have many detractors. Many papers were published, and there seemed a general air of excitement around the measure, which offered a new approach to assessing attitudes. 

This time of peace did not last too long however. In 2002, \cite{McFarland2002} published their paper, which showed correlations between ostensibly unrelated IAT measures. They argued that the incongruent associations (e.g. Self +terrible) required greater task-switching ability, and that those who were weak on this skill were biased towards appearing to have lower self-esteem scores and greater prejudice on IAT measures. This paper was quite seminal, and the repercussions of it are still being investigated. Without going into all the messy detail at this point, it has now been firmly established that differential costs of task switching in the IAT can bias the results \cite{Mierke2001,Mierke2003} (and see Section \ref{sec:task-switching-iat}). 

This Mierke and Klauer paper demonstrated correlations between a geometrical shapes IAT, and a Race IAT. They exhaustively examined the details of this effect, and concluded that it could effect most IAT's. However, these authors also examined the new scoring measures developed by Greenwald \cite{Greenwald2003a}. This new scoring algorithm, called the D algorithm appears to control for these differential task switching costs. It does this by dividing the mean response times of each participant by their standard deviations which allows us to eliminate the kinds of biases that arise from task switching. It is also worth noting that this task switching dispute has also led to attempts to understand the IAT through the use of these processes, a method which appears to have had some success \cite{Klauer2005}. 

\subsection{Arkes and Tetlock: Prejudice or Rationality?}
\label{sec:arkes-tetl-prej}

In 2004, Psychological Inquiry published a criticism of the IAT procedure \cite{Arkes2004}. The remainder of the issue was devoted to articles either supporting the original critique or arguing against it, with about equal contributors on either side. Their criticisms have three major strands. The first is that the IAT results may reflect cultural stereotypes rather than personal associations, the second is that these negative associations may not be due to prejudice, and finally, that these negative associations may be the result of perfectly rational behaviour. They also make the point that many researchers have moved too quickly from the discovery of implicit associations to the notion of implicit prejudice, which is a position that has some merit. 

Their first strand of argument is that the associations revealed by the IAT represent cultural stereotypes. They argue that since some African Americans show implicit prejudice against their own race, then this cannot be the result of a personal attitude, but must be the result of a culturally held belief. They do, however, fail to account for the similar percentage of White IAT participants who show similar negative associations to their own race, and the explanation of shared cultural stereotypes does not seem to hold water in this case. 

An article responding to the criticisms \cite{Sears2004} presents the results of survey research which indicates that approximately ten percent of respondents show a preference to a race other than their own. These figures are similar to those obtained with implicit association research, and would seem to cast doubt upon the arguments of Arkes \& Tetlock in this case. 

Arkes  \textit{et al} also argue that the low correlations between implicit and explicit measures of attitudes are the result of this shared cultural stereotype, while the explicit measures look at the extent to which the participant agrees with this cultural stereotype. The authors suggest that the time limit in IAT experiments forces participants to rely upon these shared cultural stereotypes, and that this factor is responsible for the effects. This seems like a plausible explanation, however there has been work which has linked individual Race IAT scores onto prejudiced behaviour \cite{Heider2007}, which argues against this interpretation of the IAT scores. The only possible way in which these two ideas can be reconciled is if we assume that knowledge of a cultural stereotype is correlated with prejudiced behaviour, which would presumably make prejudice researchers the most discriminatory people in the world. 

The next part of their argument relies upon the \cite{Olson2004} study where the authors examined a  personalised  IAT. They then argue from analogy that this metric could cause Jesse Jackson to fail the IAT. This argument is somewhat unconvincing, and seems to fly in the face of research that indicates that IAT measures personal associations better than cultural ones \cite{Nosek2008a}. Arkes and Tetlock also argue against the studies which use body language as a measure of prejudice  claiming that these behaviours could also be interpreted as shame or sorrow. Unfortunately, they provide no evidence in support of their claims. 

% Their final argument is perhaps the most seductive, especially to readers of a statistical nature. Briefly, they argue that because most crime in America is committed by black people, it is perfectly rational for White Americans to associate negative words with Black Americans, and thus the associations revealed with the IAT are not prejudiced, because they are rational. Leaving aside the matter of whether individuals untrained in higher level mathematics are capable of assessing utility in this fashion, and if so, whether or not they do, this argument neglects to mention that class and income are also significantly influential in crime statistics and we do not have IAT results suggesting that poor people are prejudiced against rich people, and vice versa. Their rational calculation of crime statistics is also fatally flawed by the use of convictions, given the expense of lawyers and the likelihood that most poorer people will plead guilty in America in order to gain a reduced sentence (plea-bargaining). 

In summation, the Arkes and Tetlock article appears to present no new research or evidence against the IAT but merely restate old problems and argue for doubt on the issue. The two points they do make cogently is that the leap from associations to attitudes has been made too quickly, and their argument that participants in the dispute make reputational bets following design of experimental studies to conclude the disagreement in a Bayesian fashion. Unfortunately, this offer does not appear to have been taken up at this point. 

\subsection{Rothermund and Wentura: Salience or Associations?}
\label{sec:roth-went-sali}
In 2004, perhaps the most compehensive critique of the IAT was published \cite{Rothermund2004}. This critique focused on the validity of the association model presumed to underlie the observed IAT effects \cite{Greenwald1998}. Rothermund and Wentura proposed another model which they claimed could account for the effects and they called this the figure-ground model. 

They argue that salience asymmetries between the different stimuli could be driving the observed effects. They claim that negatively valent words are more noticeable and thus become the figure, while positively valent words become the ground. This then drives the observed  effect. In the paper noted above, Rothermund and Wentura produce results which appear to confirm this model, using non-words and strings of numbers as stimuli and producing typical IAT effects. 

They note that this cannot be due to pre-existing associations, and claim that these results support the salience asymmetries model of the IAT. This paper is extremely well conducted, and has  cast a shadow over  IAT research. These criticisms have some merit, however, they note in passing that valence and familiarity probably drive these effects, so logically, these effects of salience can be controlled for by controlling the valence and familiarity in IAT research, which has since become common practice. 

They also recommend that all IAT research should involve a word non-word task in order to assess the extent to which salience asymmetries contribute to the observed IAT effects. Greenwald  et al \cite{Greenwald2005} responded to this critique, and argued that the salience asymmetry explanation conflicted with the literature showing impressive correlations with explicit behaviour in known groups studies and meta-analyses \cite{Greenwald2009}. 

While this is a very important point, nonetheless this does not rule out a salience asymmetry explanation totally. Perhaps the most important element to take from this debate is that salience asymmetries need to be controlled for (by making sure valence and familiarity are matched across stimuli) so that we can be certain that any associations revealed are not spurious and reflect real differences in people's conceptions of the matters under study. 

\subsection{Blanton and Jaccard: Associations versus Attitudes}
\label{sec:blant-jacc-assoc}

The most recent scrutiny has mostly emanated from Hart Blanton and James Jaccard, along with some of their associates. The first critique began as a commentary on Greenwald's \cite{greenwald2002}  \textit{Unified theory of implicit attitudes, stereotypes, self esteem and self concept}, and expanded into a critique of multiplicative models within psychology more generally. 

The major problems for Blanton and Jaccard \cite{Blanton2006a} were the following. Greenwald's theory posits that attitudes are associated with the self through a network comprising discrete concepts, many of which are either positively or negatively associated with one another. So far, so good. However, as a result of the theory, Greenwald makes a number of predctions which he proposes to test using multiple regression. It is here that problems arise in the view of Blanton and Jaccard. 

Multiple regression and multiplicative models tend to require a rational zero point on the scale used. Blanton and Jaccard argue that this requirement has not been met for either the explicit or implicit measures used by Greenwald et al. They point out that Greenwald et al \cite{greenwald2002} assumes that the mid point of a scale measuring explicit attitudes represents a zero point. This requires that the scale be a perfect representation of the underlying construct, which seems somewhat unlikely. Greenwald often uses difference scores to avoid this kind of problem, but Blanton and Jaccard argue that this is only permissible if the positive and negative items on a scale are equally valent, which is an assumption which cannot be met for most scales used. However, it is an assumption that can be met for much of Greenwald's work, where stimuli are matched for valence. 

Greenwald also tends to use identical stimuli for self report and IAT instruments so at least this assumption is met across both methods \cite{Farnham1999,Greenwald1998}. Blanton and Jaccard conclude by suggesting an alternative method of analysis for the data reported in Greenwald et al's (2002) article. This led to a reply by Greenwald et al \cite{Greenwald2006b}, which used his acquired data to test against the model suggested by Blanton and Jaccard. He discerned some problems with his own model, but also serious flaws with the strategy suggested by Blanton and Jaccard. Following simulation and meta analysis, he produces evidence which supports his multiplicative model, at least in a somewhat weaker form. 

% If this were the limit of the issues raised by Blanton and Jaccard, it would seem not much more than an abtruse debate about relevant statistical models, with little to no relevance for practical use of the instrument, except perhaps for a greater awareness of its limitations. 


Later that year Blanton and Jaccard released an article \cite{Blanton2006} in which they claimed to have identified a number of important confounds in the entire IAT procedure. The first, and perhaps most important of these, was a problem with general processing speed. They re-analysed previous data supplied by Greenwald, and using the practice steps as a measure for general processing speed, found that implicit preferences were strongly correlated with one another across different domains. However, when processing speed was controlled for, there were no significant associations. This raises the disturbing possibility that previous associations reported using the IAT may have been the result of processing speed rather than true effects of implicit attitudes.

One major flaw in this argument presented by Gonzalez \textit{et al}  is that they assume that all responses in the IAT are independent of one another, across blocks. This is an important assumption as if the blocks were not independent, then blocks 2 and 4 could not be used as an index of general processing speed, as these authors did. Given the design of the IAT (a speeded response task where one block follows another almost immediately), this independence appears to be an assumption which needs further empirical testing. 

Another major problem identified by this re-analysis concerned the definition of the IAT as an instrument which measured relative attitudes rather than absolute preferences \cite{Greenwald1998} . This assumption was so important that it was reflected in the scoring, where an IAT effect is defined as the difference between response latencies in the two conditions. 

Re-analysing previous studies on the preference for maths over arts, Blanton et al discovered that the attitudes towards maths and arts were independent of one another, suggesting that the IAT may not be measuring relative preferences but rather independent attitudes. In the specific case above this may seem intuitively obvious, but this finding was replicated for the Race and Gender IAT's, where this result may seem a little less likely. Again, this re-analysis relies on the idea that the responses to one category are independent of one another, which is an even less jusitfiable assumption than that the responses of different blocks are independent. 

However, Nosek \& Srinam \cite{Nosek2007} replied to this analysis with a spirited critique of their own. They argue that Blanton et al's (2006) conception of the two IAT conditions as representing parallel measures of an underlying construct is faulty, and that the true conception of them is as interdependent measures, like a Stroop task. 

They re-analysed the data provided by Blanton et al, and found that when this assumption was made, they could produce models of the phenomenon which had good fit, and did not show noticeable effects even when processing speed was assumed to cause an effect. Blanton et al critiqued this comment mercilessly (though very politely) \cite{Blanton2007}, when they observed that many of the assumptions they had made had also been made by Nosek in the past. 

Essentially, Blanton and Jaccard argue that the IAT is poorly understood and an arbitrary metric, and also believe that general processing speed of individuals may be contaminating the results. 

These critiques have some merit, and will probably result in some changes to the methodology, as the problems seem to arise from the use of attitudes as bi-polar constructs. This, despite, the flaws in the analysis of Blanton \textit{et al}  may be an issue for future research. % This would mean that variations of the IAT such as the Go/No Go Association Test and the Single Category IAT would be unaffected, but much previous research would need to be re-confirmed.

However, it is worth noting that many psychometric scales can be regarded as arbitrary in Blanton and Jaccard's sense, yet they do not condemn the use of these measures. 

The work of Blanton and Jaccard has provided important caveats and insights into potential moderators of the IAT, but their explanation fails to account for the demonstrated predictive validity of IAT measures\cite{Richetin2007,Arcuri2008,Greenwald2000}.  

Their proposal of general processing speed as the driver of the effects implies that processing speed is also the driver of the predictive power of IAT's, which suggests that processing speed can also predict behaviour in a wide variety of settings and is differentially predictive of behaviours based on the stimuli which are used for measuring it. This is a far more complicated explanation of the experimental findings, which violates accepted scientific norms of parsimony. 

Even if general processing speed did drive the effects, it should not affect IAT measures scored using the new algorithm \cite{Greenwald2003a}, as this algorithm divides by the standard deviation of each participants responses in the pooled blocks.  


In summation, the Implicit Association Test is a new and fascinating measure. It has been embroiled in a number of spirited debates since its publication, and much progress has been made in the understanding of the measure as a result of this. Due to this previous work, we know the following important facts about the IAT. 

Firstly, the measure is sensitive to task-switching costs, and these need to be controlled for. The D algorithm seems perfectly adapted to this, fortunately. Secondly, valence and familiarity need to be controlled for in order to avoid contaminations with general processing speed and salience asymmetries. Thirdly, the measure does not have a theoretical foundation, and this is something that needs to be addressed. 

One can argue that if it works then we should use it, but a cogent theoretical account of the IAT would allow for more detail and precision in theorising  than is the case at present, and for this reason such a development is essential if the IAT is to become a permanent part of research in experimental psychology.

\part{Measurement in Psychology}
\label{part:meas-psych}

\section{Introduction}
\label{sec:introduction}



One of the features which both the placebo effect and implicit measures have is common is that they are both fields of study where measurement is controversial. In the case of the placebo, there are multiple definitions and situations where it is attempted to be measured, and in the case of implicit measures (more specifically, the implicit association test) there is a measure, but no clear model for how the measure achieves its predictive validity. 

One of the major aims of this thesis is to examine the placebo effect using a variety of different kinds of predictors, and to use these multiple kinds of measures to examine their differential usefulness and areas of strength and weakness. In order to have another form of measurement, it was decided to use self report measures which have been associated both with the placebo effect (optimism, expectancies) and the IAT (mindfulness). 

The reason for the use of these measures was threefold:
\begin{enumerate}
\item Firstly, it was believed to be important to have both an explicit and implicit measure both of treatment credibility and optimism as this would allow for a comparison of the relatively efficacy of both of these forms of measurement in the prediction of placebo
\item Secondly, the use of self report measures (where there is much research on methods for establishing the validity of a measure) would provide a consistency check on the development of the implicit measures
\item Finally, the use of self report measures allowed for large samples to be collected on some of the predictors of interest in order to develop better models for the measures which could then be applied to the experimental sample. 
\end{enumerate}

The remainder of this section shall be structured as follows:
\begin{enumerate}
\item Firstly, measurement methods in psychology shall be briefly reviewed, along with some general measurement method proposed for use in this thesis 
\item Secondly, the measurement of the placebo shall be reviewed, with a focus on statistical problems with current methods and on expectancy measures.
\item Next, the rationale for attempting to measure the placebo with implicit measures will be described. 
\item Finally, an approach to modelling the placebo response and the relative contributions of different measures shall be described. 
\end{enumerate}

% \subsection{Psychometric Considerations}

% This project involved a considerable amount of psychometric work, as the major part of it was the development and testing of two measures, one implicit association test and one expectancy questionnaire. This section will detail the approaches taken to the modelling of this data.

\section{Review of Measurement Methods in Psychology}
\label{sec:revi-meas-meth}

\subsubsection{Explicit Measures}

The use of self report measures of personality and attitude has been standard practice in psychology for over one hundred years \cite{spearman1904general}. In this time, well developed methodologies have been developed for the design and analysis of these measures.

The primary concerns for these measures were their validity and reliability.
Validity is typically taken to mean the extent to which a measure actually does measure what it purports to, while reliability is the extent to which the same measure applied to the same individuals will give consistent results \cite{raykov2010introduction}.  The validity of a measure can be assessed by correlational analyses with other measures which are theoretically related to the measure under study (convergent validity), but the ultimate test of the validity of the measures comes from its association with an indpendently measured outcome of interest. 

The reliability of the self report measures was assessed by the use of reliability indices such as cronbach's alpha \cite{cronbach1951coefficient}. In addition, factor analysis, structural equation modelling and item response theory methods were applied to the data to investigate latent structure underlying the observed responses. Each of these will be dealt with in turn.  A number of different scoring methods were applied to the self report data arising from these pieces of research and these methods  are discussed below.


\subsubsection{Latent Variables}

Latent variables are a primary focus within psychometrics and psychology more generally \cite{bollen2002latent} \cite{borsboom2006attack}. 
Latent variables have a number of both formal and informal definitions in the field \cite{bollen2002latent} and the one most useful for this research is the local independence definition. This defines latent variables as the cause of the correlations between observed variables, and asserts that, conditional on the latent variable, the correlations between observed variables not significantly different from zero (i.e. the observed variables are locally independent). This definition has the advantage of being equally applicable to both factor analytic and item response theory approaches, whereas other definitions (such as the expected value definition, where the latent variable is referred to as the true score of classical test theory \cite{bollen2002latent}) do not apply as easily to the methodologies employed in this thesis. 


The essence of the latent variable concept is that psychological tools such as self report measures or implicit measures are impure measures \cite{edwards2000nature}. They are not pure measures of whatever construct is under investigation, they also tap into elements such as context, social desirability, response patterns and a myriad of other biases and heuristics \cite{borsboom2006attack}. The latent variable approach is extremely common because of this, and forms the core of factor analysis, item response theory and structural equation modelling techniques \cite{bollen2002latent}. 

Some psychometricians would argue that factor analysis is a data reduction technique rather than a latent variable technique \cite{borsboom2006attack}. This critique has its merits, but is more applicable to principal components analysis (PCA) where all of the variance in a matrix is divided into components. Factor Analysis only examines the common variance between items, and as such is a latent variable approach.  In classical test theory (factor analysis, reliability analysis), a latent variable is often referred to as a ``true score'', that is, the score that would be obtained for a participant given an infinite number of replications of the study \cite{bollen2002latent,edwards2000nature}. Indeed, the error terms in a multiple regression model can also be regarded as latent variables \cite{bollen2002latent} in that they are variables which are conditional on a model which has been applied to a set of data. 

 The latent variable modelling approach necessitates that a number of impure measures of a construct are collected (for example, items on a self report measure or stimuli from an IAT) \cite{edwards2000nature}. Impure in this sense means that there is no one to one mapping between observable outcomes of interest and the responses to a particular measure. This means that there is at least some residual variance left unexplained between the criterion (our measure, for example of extraversion) and the outcome (extraverted behaviour).  By examining what these items have in common (by either their correlations or a non-linear function of the response patterns) a better estimation of values on the  construct can be derived \cite{borsboom2006attack}. This derived measure is then used as a predictor for the outcome variable. 

There are two main perspectives on latent variables. The first is the reflective model of latent variables, where the measures are believed to reflect the underlying construct. The second model is the formative model which suggests that latent variables are formed of the measures observed \cite{bollen2002latent,edwards2000nature}.

The first model reflects a positivistic concept of latent variables (i.e. that they exist within people and psychological measures elicit them) while the second represents a more constructivist approach (latent variables are constructed from our measures, and do not necessarily correspond to anything that exists within individuals) \cite{borsboom2005measuring}.

This research assumes the formative model of latent variables, as this is more useful for practical modelling of psychological constructs. 

\subsubsection{Factor Analysis}

Factor analysis has a long history in psychology, and is now over one hundred years old. It is the most commonly used latent variable modelling technique in psychology, and more pages of \textit{Psychometrika} have been devoted to it than to any other technique \cite{henson2006use}.  Despite this, there are still a number of issues and controversies which surround the technique \cite{sass2010comparative}.  Essentially, factor analysis is an attempt to approximate a correlation matrix with a smaller matrix of parameters.  These hypothesised latent variables tend to be called factors or components.

 % One of the first controversies surrounding factor analysis is the dispute between Factor Analysis proper and Principal Components Analysis\cite{henson2006use}. The major difference between Factor Analysis and Principal Components Analysis is that in Factor Analysis, only the variance common across observed response to items (or communalities) is analysed, while in PCA, all of the variance (including variance only found in one item, or unique variance) is analysed. PCA tends to work better for data reduction, and indeed this is the reason why it was developed \cite{borsboom2006attack}.

Throughout this research only factor analytic methods were used for psychometric purposes, as Factor Analysis provides a true latent variable modelling approach, while PCA does not. 

The most critical issue surrounding factor analysis concerns determination of the number of factors to extract\cite{zwick1986comparison}.  This is an important issue, as theory and practice are likely to be held back if an incorrect choice is made.  The issue is not that there are no criteria on which to base a principled decision, but rather that the different criteria often do not agree, and it is thus ultimately left to the informed opinion of the researcher which factor solution is to be preferred.  All of the decision criteria will be reviewed in turn, and their advantages and disadvantages will be discussed \cite{henson2006use}. 

\begin{quotation}
  "Solving the number of factors problem is
     easy, I do it everyday before breakfast.  But knowing the right
     solution is harder" (Kaiser, 1954).
\end{quotation}

The choice of criterion for retention of factors is extremely important for this thesis. This is because if an incorrect number of factors are extracted, then the predictions for the experimental portion of the research will be biased, and thus will not prove as useful as the method could otherwise be. 

The first, and most popular criterion is surprisingly the least useful \cite{zwick1986comparison}. This rule is called eigenvalues greater than one criterion and recommends keeping all factors whose eigenvalues are greater than one. The rationale behind this approach is that eigenvalues less than one explain less of the variance in the matrix than one item, and as such should not be retained.  More recent research appears to put the minimal criterion for retention of eigenvalues at approximately 0.7. 

The second criterion often used is the scree plot technique, which was popularised by Raymond Cattell. This criterion recommends that the eigenvalues of all factors should be plotted against their number, and only factors before the drop off in eigenvalues should be used. As the process of factor analysis ensures that the first factor will have the largest eigenvalue, followed by the second and so forth, this criterion looks for the point where the eigenvalues are very close to one another.  This criterion has a number of advantages.  It is available in all statistical packages, it can be used without any special training and it tends to give results which are somewhat, if not totally accurate \cite{zwick1986comparison}.  Its major disadvantage is that it relies upon the interpretation of the researcher, but given the strong emphasis on interpretation throughout factor analytic literature this should not be regarded as too much of a handicap.

The next criterion which can be used is that of parallel analysis. Parallel analysis is a Monte Carlo  technique which simulates a data matrix of equal size and shape to the matrix under study, and calculates the eigenvalues of these simulated matrix against those of the real matrix \cite{horn1965rationale}. All factors are retained up to the point where the simulated eigenvalues are greater than the true eigenvalues.  Parallel analysis is one of the better techniques for assessment of the number of factors to extract , and it can often give very accurate results \cite{zwick1986comparison}.

Its major disadvantage is that it tends not to be available in many statistical packages, and that it can often over factor the data-set. However, there is some evidence \cite{beauducel2001problems} that parallel analysis can underextract factors when there is pronounced oblique structure, so both possibilities need to be kept in mind. It does produce some of the most accurate results in simulation studies so it is a useful tool nonetheless \cite{zwick1986comparison}.

Another useful criterion is that of the Minimum Average Partial Criterion (MAP) which extracts factors from the data-set until only random variance is extracted \cite{revelle1979very}. Again, this is an accurate criterion  \cite{zwick1986comparison} which is little used as it is not available in popular statistical programs. The only problem that  has been found with this criterion is that it tends to under-extract factors. %However, with the use of this criterion as a lower bound, and parallel analysis as an upper bound, then the decision of how many factors to retain can be made much easier. [I found this, but the literature does not appear to have as many examples of it] This leads on to the major point and issue with much factor analytic research today, whereby one decision rule is used to the exclusion of all others. Many researchers have recommended the use of multiple decision criteria, but this does not appear to be am approach utilised by many in the literature \cite{henson2006use}\cite{sass2010comparative}. However, this is the approach which has been taken in this research. This work will use parallel analysis, the MAP criterion and examination of scree plots to ensure that all relevant factor solutions are examined.

However, the ultimate test of a factor solution (without using other methodologies, such as Structural Equation Modelling) \cite{joreskog1978structural} is its theoretical clarity and interpretability, and this will be the first test used for all proposed factor solutions.

Another area of dispute amongst researchers in the factor analytic field is which method of rotation to use\cite{sass2010comparative}. As the eigenvalues are only defined up to an arbitrary constant, these rotations do not have any substantive impact on the factor matrix, except that they can make it easier to interpret (which is a very useful function). 

Rotations are commonly applied to factor solutions in order to reduce items loading on multiple factors, and to aid in the discovery of simple structure \cite{henson2006use}. Rotations can be divided into two classes, orthogonal and oblique \cite{sass2010comparative}. Orthogonal rotations return uncorrelated factors, while oblique rotations allow the factors to be correlated. Given that most psychological measures are correlated with one another, one would expect oblique rotations to be more common. However, the default appears to be orthogonal rotations, as they are apparently easier to interpret \cite{henson2006use}. Oblique rotations were applied throughout this research, as if the factors are truly uncorrelated, then the oblique rotation will show that, while the converse is not true for orthogonal rotations 



\subsubsection{Structural Equation Modelling}

Structural equation modelling is regarded by many as an adjunct technique for evaluating the results of particular factor solutions\cite{fabrigar1999evaluating}. As has been explained above, the factor analytic procedure is full of interpretative procedures where no principled choice can be made, and structural equation modelling (hereafter SEM) is an attempt to compensate for some of these deficiences.

SEM was developed by Joreskog in the 1970's \cite{joreskog1978structural}. It provides a means of testing hypothesised relationships between latent and manifest variables. In practice, the result of a factor analysis is regarded as a measurement model of the data.

This is combined with a structural model (which describes how the latent variables relate to one another and to the manifest variables). The two of these models are then used to construct a covariance matrix which is then compared with the observed data, and a number of indices of model misfit are calculated. Foremost among these is the $\chi^2$, which estimates the degree of model misfit. The desired result is a p-value of greater than 0.05, which shows that the two matrices are not significantly different.

However, the $\chi^2$  is extremely sensitive to sample size, and tends to be rejected in almost every case \cite{henson2006use}, given the sample sizes needed for accurate factor analysis and structural equation modelling tend to be quite large. As a result of this, many other fit indices have been developed. Foremost amongst these are the Non Normed Fit Index (NNFI), which is also known as the Tucker Lewis Index \cite{bentler1990comparative}, the Root Mean Square Error of Approximation (RMSEA) \cite{rigdon1996cfi} and the Bayesian Information Criterion (BIC) \cite{schwarz1978estimating} and the Aikike Information Criterion (AIC) \cite{akaike1974new}. These all have different strengths and weaknesses and are typically used in a complementary way.

% There are also some requirements for SEM models which are typically not met for much psychological and psychometric data. These are as follows:
% \begin{itemize}
% \item The distribution must be approximated well by the first and second order moments.
% \item Sample size is required to be large (>300)
% \item The covariance matrix must be strictly positive definite across the entire parameter space.
% \end{itemize}

% Of these, the multivariate normality assumption  (assumption 1 above) is often difficult to meet in practice. However, there are a number of distribution free methods in SEM, of which the most common is a Weighted Least Squares approach. This proceeds similarly to a Weighted Least Squares approach in linear regression, where points are assigned weights depending on how closely they meet the assumptions of the model. Sample size, by contrast, is typically easy to increase (at least for non-clinical populations).

Identification of the model is one issue in practice, though as Joreskog notes, this can often be achieved by fixing a number of parameters to 0 or 1 (the inter-factor variances are often scaled in this fashion) \cite{joreskog1978structural}.

Another, more theoretical issue is that no set of data is uniquely determined by an SEM model. This is known as the problem of rotation in factor analysis \cite{maccallum2000applications}. Given a covariance matrix $W$, and a set of data $D$, there are many solutions which provide the same fit indices of the model to the data. This can lead to a similiar problem as occurs to factor analysis, where the researcher must make a choice between models which are quantitatively identical. One approach for resolving this problem is discussed below, in Section \ref{sec:probl-sample-infer}.

\subsubsection{Item Response Theory}

Item response theory (IRT) is often called model-based measurement\cite{fischer1995rasch}, and is also referred to as Rasch modelling, is a newer approach to analysing test data, developed both by the Danish mathematician Rasch in work for the Danish army, and also seperately by Lord and Novick in the US, while working for the Educational Testing Service (ETS) in the 1950's and 60's \cite{van1997handbook}.
The fundamental premise of IRT is that the properties and scores on a psychometric test can be modelled as functions of both the items on the test and the ability of the people taking the test.

The IRT approach suggests that conditional on both the ability of the person and the difficulty of the test, the responses of each participant can be predicted probabilistically. As the latent ability of the participant rises, they tend to choose alternative responses which are more reflective of this latent ability. One example of this might be an item for extraversion ``I am always the life and soul of the party``, those respondents who had a higher latent score on the extraversion construct would tend to choose the agree or strongly agree options (on a typical five point scale). 

For instance, if one was modelling extraversion using a set of ten items, the participants who scored highest in extraversion would be most likely to respond strongly agree to the items.  IRT also assumes a property called local independence, which states that conditional on the ability measured by the test, the scores of each participant are independent of one another.

There are a number of different approaches taken to IRT \cite{van1997handbook,fischer1995rasch}. The  Rasch models are the simplest, and have a number of extremely appealing mathematical properties. These models assume that only one trait is measured by the items, that all items are equally predictive of the trait, and that there is no guessing \cite{van1997handbook}.

Because of these assumptions, it is possible to seperate out person abilities and item difficulties perfectly. However, another approach (normally referred to as a two parameter model, or IRT proper) claims that items are differentially predictive of the ability being measured, in a manner analogous to different strength of loadings of items on a construct in factor analysis. Another model the three parameter model \cite{lord1968statistical}, allows for correct responses through a process of guessing, but this model is not normally applied to polytomous (many valued, like a Likert scale) items \cite{van1997handbook,Mair2010}.

The Rasch model is the simplest of these three general forms of models, and will be discussed first, followed by a discussion of two parameter models, after which I discuss three parameter models. Finally, this section ends with a discussion about non-parametric item response theory and multidimensional IRT.

In general, IRT models are represented by the logistic function, and are estimated iteratively through procedures of numerical optimisation (maximum likelihood). The function used to describe the data is a logistic one, where ability is estimated from the probability of answering the question correctly.

The difficulty of an item is conventionally defined as the ability of participants who answer the question with 50\% accuracy. The parameter $\alpha$ is defined as the difficulty of the item. In two parameter models, another parameter $\beta$ is defined and is used for the discrimination of the item (the slope of the curve). In the more complex 3 parameter model, $\theta$ is used to measure guessing (the probability of a correct answer given low ability)\cite{van1997handbook}.

IRT was developed in the context of ability tests, and this leads to much of the vocabulary fitting uneasily within psychology. For instance, in the context of a credibility questionnaire about various treatments, the questions on homeopathy are categorised as most difficult (see Chapter 4). This does not mean that they are harder to answer, just that the probability of a respondent endorsing them is lower than the proability of a participant endorsing an item on the efficacy of painkilling pills.

Below, all of the models will be described in terms of their original use in ability testing, and then conclude with a discussion of polytomous IRT, which is the methodology exclusively relied upon in this research as this research focused on personality testing rather than ability testing.

% It is important to note that the names of the models are slightly deceiving, while they are called 1, 2 \& 3 parameter models, they actually involve the estimation of 1,2, or 3 parameters for each item. This normally means that a test of ${1,2,3\ldots, n}$ items will require the estimation of either ${n, 2n, 3n}$ parameters. The parameters are estimated by an iterative maximum likelihood approach, as so issues of optimisation and ensuring that a global maximum has been found are often important in practical applications\cite{gill2002bayesian}.


\subsubsection{IRT Models}

Rasch models are the simplest of the different kinds of IRT models. The model makes a number of assumptions, here restated in less mathematical language than is typical\cite{fischer1995rasch}.

\begin{enumerate}
\item The latent variable is normally distributed
\item Each of the items provides equal information with regard to the latent trait
\item Given a set of responses to a question, $k$ and a latent ability, for every k there is a $\theta$ which is identical in rank (monotonicity)
\item Given the latent variable, all of the $n$ items are independent (local indpendence)
\end{enumerate}

The first and third assumptions are common to most IRT models, while the second assumption is the defining characteristic of the Rasch model, and the feature that allows for some of its appealing mathematical properties.

In essence, all of the information about a participant garnered from a Rasch model can be represented by the ability score (it is a sufficient statistic for the distribution) \cite{fischer1995rasch}. This is a characteristic which does not generalise to the more complex models discussed below.

Two parameter (IRT) models are defined by two parameters, the difficulty of the item and the discrimination of the item. The difficulty represents the probability of correctly answering the question, while the discrimination represents the change in difficulty for a given ability (the slope of the curve). In the Rasch model, these slopes are set to 1, so that all items provide equal information about the abilities of participants throughout the sample. This, while allowing for exact estimation of the difficulty of the test and strict sample invariance, is an assumption which is often not met for psychological testing instruments \cite{embretson2000item}.

Two parameter models maintain all of the assumptions of the Rasch model, except for 2, which is generalised \cite{van1997handbook}. The two parameter model recognises that different items provide different amounts of information regarding different participants. This alters the overall model by allowing the slopes of the item to differ, which disallows the property of conjoint additivity, which is the feature of Rasch models that ensures that participant ability and item difficulty can be seperated perfectly.

Non parametric IRT was developed by Mokken \cite{mokken1997nonparametric}, and is often called Mokken scale analysis\cite{van2007mokken}. It maintains the assumptions of all of the previous models, except for the parametric form of the latent variable. Mokken scaling is often performed as a prelude to parametric IRT, as it provides useful checks of all the assumptions noted above. In addition, Mokken scaling can be used in order to assess whether a scale should be broken into two or more subscales, which can then be modelled using parametric IRT. This is the manner in which mokken scale analysis was used in this research. 

% \subsubsection{Two Parameter Models}

% Two parameter models were developed by Lord \& Novick while they worked at Educational Testing Services (ETS)\cite{van1997handbook}. . However, in return this model allows for greater complexity and fidelity to the responses made by the participant.

% \subsubsection{Three Parameter Models}

% The three parameter model was developed by Birnbaum and is identical to the two parameter model, except for one crucial difference\cite{van1997handbook}. Both the one and two parameter models assume that participants only get the correct answer if they know it, that is, there is no guessing. This is a difficult assumption to meet in practice, as in an educational testing setting (assuming that there is no negative marking) participants are likely to guess if they do not know the answer.

%  Even if there is a negative mark penalty, as long as it is not equal to the mark gained from a correct answer, any rational participant will guess if they have narrowed the options down to two of four possible responses. Therefore, Birnbaum's model incorporates this into the test, and assumes that there will be a non-zero level of guessing throughout the test. This means that it cannot be assumed that a participant with extremely low ability will answer a difficult question incorrectly, as so the lower bound on the probability of answering is estimated from the data, and will be greater than 0 (where it is in both the one and two parameter models). Apart from that, the model is exactly the same as the two parameter model.

% \subsubsection{Non Parametric Item Response Theory}



% \subsubsection{Multi Dimensional Item Response Theory}

% So far, all of the IRT models we have considered have been univariate IRT models, where there is assumed to be a single latent variable $\theta$ responsible for the different patterns of responses. However, it is easy to see that often, this assumption will not be met.

% In terms of personality testing, there is a long history of stable inter-individual differences in the manner in which people respond to personality test items. Some participants will tend to utilise the extreme scores of the scale, while others will stick to the middle. These kinds of behaviours occur stably across divergent instruments, and need to be accounted for.

% As yet, unfortunately, there is no useful mathemtatical model \cite{borsboom2009end} which can be used to model these kinds of individual differences, which is essential if the measurement of human ability is to progress. However, what can be done is fit a multidimensional model which accounts for some of these factors in a coherent  fashion\cite{doran2007estimating}.

% Indeed, multi-level IRT models can allow us to examine the intercorrelations between latent variables, and can lead to much more stable estimates of ability and difficulty, though at the cost of a loss of parsimony and extensive computing requirements (though far less computation than is needed for even the simplest Markov Chain Monte Carlo model). Therefore, as many of the scales used in this research seem to measure multiple latent variables, ML IRT approaches will be employed after an acceptable  unidimensional model has been built for each of the sub scales.

\subsubsection{Polytomous Item Response Theory}
\label{sec:polyt-item-resp}
In contrast the dichotomous IRT models presented above, almost all of the IRT models used in psychology are IRT models for polytomous (many responses) data. This follows as with a typical Likert scale, there are five possible responses, arranged  in a montonically increasing fashion.

% Indeed, failures of monotonicity are common when fitting simpler models, and this needs to be checked before estimation of parameters. However, such failures tend to be obvious after the estimation of parameters, and the model can then be improved in an iterative process.

Common polytomous models used in psychology are the Graded Response Model \cite{van1997handbook} (two parameter) and the Partial Credit Model (one parameter)\cite{fischer1995rasch}. The generalised partial credit model is also used in research, and represents a generalisation of the PCM to more flexible two and three parameter IRT models. 

All of these models assume that the data is ordinal (as per a Likert scale). There is another model, the Rating Scale Model \cite{bock1997nominal}, which is used to fit nominal data, or when the assumptions of ordinal data have been breached. The model fitting procedure typically followed a path from simplest to most complex, as described below.

\subsubsection{Scoring Self Report Measures}

Scoring of self report measures is an area which has received surprisingly little research within psychology. Almost all scales seem to use either a sum or mean as their method for calculating a score \cite{borsboom2006attack}. This use of sums and means makes a number of assumptions, some of which may not be justified in this research:

\begin{enumerate}
\item All scores contribute equally to the construct;
\item All questions tap the same construct;
\item All questions are scored in the same direction.
\end{enumerate}

Of these assumptions, many of them will be violated for many measures. The first, the notion that all questions contribute equally to the construct will be false in all cases where the factor loadings are significantly different from one another (as assessed by correlations between them at conventional levels of significance). The second assumption will be violated where a scale measures more than one factor. 

It would seem obvious that scales which measures multiple factors need to have a seperate score for each one, but this does not appear to occur in many cases. As an example, the factor structure of the Life Orientation Test -Revised is a matter of controversy, with some studies have found a one factor structure while others have found a two factor structure. However, the official scoring guidelines produce only a single mean score for all items.  The third assumption is the only one which appears to be consistently followed with the use of reverse scoring.

There are some reasons why means and sums are preferred over the more complex methods described below throughout psychology. There are two main alternatives to the use of means, medians or sum-scores (with many variations within each of these two broad types), but both of these methods require large samples, which (especially in clinical work) may not be readily available. 
The two methods are:


\begin{enumerate}
\item Factor scores;
\item Participant abilities and item difficulties, as estimated from an IRT model.
\end{enumerate}

These two methods have much in common, and also some important differences. The first commonality is that they both require large amounts of participant data (perhaps 300 or more) \cite{van1997handbook,henson2006use}. The second commonality is that they are both model-based, in that a model is fitted to the observed responses, and then this model is used to turn these responses into scores for each of the participants. Given the paucity of experimental research using over 300 participants in most areas of psychology, it is not surprising that means and summations have remained the standard within the field. Another commonality between the two methods is that they both weight particular items differently, based either on their correlations with the factors (for factor analysis) or the probability of a particular response (for IRT). For Rasch models, the assumption of equal discrimination for each item implies that sum scores are a perfect representation of the model, but Rasch models are difficult to fit to psychological instruments, especially those developed using classical test theory \cite{borsboom2006attack}.   

There are also some differences between the two methods, which are described below. Firstly, factor analysis is a technique which falls squarely under the rubric of linear models\cite{venables2002modern}, so there is an assumed linearity between the responses and the factor scores. In contrast to this, IRT utilises a non linear approach to scoring the items (generally a logistic function), although these can also be classified under the rubric of linear models, albeit generalised \cite{venables2002modern}

One issue with the use of factor scores is that they come from an unidentified model, in the sense that there are an infinite number of factor scores consistent with the data \cite{grice2001computingit}, and the validity of these cannot be assessed through factor analysis alone. The resolution of this issue is discussed below in Section \ref{sec:probl-sample-infer}. There are two major types of factor scores, coarse and fine, and both of them will be assessed for their predictive ability on cross-validation test data \cite{grice2001computingit}.

Another difference concerns the assumptions about the parameters estimated from the sample. In factor analysis, they are assummed to be sample dependent, in that the same instrument given to a different population could produce entirely different factors.   For this research the models were developed and applied to the same sample, so theoretically, this should not be an issue.  In item response theory, the item difficulties are assumed to be independent of the sample which is used to estimate them. This can be rigorously proven for the Rasch model, and is often extended to the more complex 2 and 3 parameter models also, though this is still controversial \cite{van1997handbook}.

The approach taken in this research  was the following. Firstly, all measures were administered to a random sample of the population from which experimental participants were to be drawn (students at University College Cork). These large datasets ($n$ between 800 and 1500) were then used to build and test models based on both factor analysis and item response theory.

N fold Cross-validation \cite{friedman2009elements} (see also Section \ref{problems-sample-infer}) approaches were employed to decide amongst the different methods of scoring. N-fold cross-validation involves splitting the data into N sets, fitting the models on N-1 of these sets and testing the model on the remaining split. This process is repeated for all splits. These models were then  applied to the response patterns of the experimental participants, and estimates of their abilities and scores were created, along with a determination of the best model(s), which was then applied to the experimental data. 

 This allows for the psychometric tools developed for analysis of large samples to be applied to experimental data where there would not have been enough respondents to fit a model on if this was the only data available. This is important, as many measures (including the ones described in Chapter \ref{cha:health-for-thesis} and \ref{cha:tcqthesis}) have structure which implies that the use of means, medians or sum-scores would be invalid, and the collection of large amounts of data allowed for models to be applied to the scales which can then be used to predict the response to placebo more accurately in Chapter \ref{cha:primary-research}. 

\section{Measurement of Placebo}
\label{sec:measurement-placebo}

The measurement of placebo has improved greatly over the past fifty years since Beecher started examining the phenomenon. However, the measurement of placebo still suffers from a number of problems.

The first problem is the consistent use of ANOVA and regression methods to examine differences between groups. While there is nothing intrinsically wrong with this method (as long as it is not within groups differences which are examined), it is a waste of statistical power. The issue here is that in placebo analgesia research (which this thesis made use of) typically only examines group differences in scores. The problem with this approach is that typically, pain scores are collected over time, and so the responses of one participant at time $t$ are not independent of the responses of this participant at $t+1$. This violates the assumptions of ANOVA (and of linear regression). While a simple binary responder/non responder classification is possible, this approach throws away large amounts of information.

More appropriate methods for these analyses are time series analyses and survival analysis, both of which are discussed in Chapter \ref{cha:methodology}. 

The other major issue with current placebo research relates to expectancies. The first issue is that expectancies are typically measured with a simple one question scale. One of the aims of this thesis was to apply psychometric methods to the devlopment of a more sophisticated measure of treatment credibility and expectancies (see Chapter \ref{cha:tcq-thesis}). The other problem is that expectancies are assummed to be conscious, even though they appear to have far more in common with unconscious responses than they do with controlled processes. In the next section, the rationale behind this approach is explained. 

\section{Implicit Measures and Placebo}
\label{sec:impl-meas-plac}
So, having reviewed the placebo effect and the implicit association test above, we are now in a position to examine the main hypothesis of this thesis: that the placebo effect can be predicted by the use of reaction time based implicit measures. Below, I shall set out the reasons for believing this, and in the next section I shall present research evidence which supports this original hypothesis. 

The placebo effect has resisted prediction in experimental work for over 50 years. Many psychological variables have been tested to see if they can predict the response, but all have failed \cite{Shapiro1997}. Indeed, some argue that placebo responders do not even exist \cite{Kaptchuk2008a}. My thesis is that the lack of results in the prediction of the placebo effect is an artifact of the methods used rather than the unpredictability of the effect. My idea is that the Implicit Association Test is a useful method with which to predict the response. 


\subsection{The Research Base for the combination of the two measures}
\label{sec:rese-base-comb}
The placebo is a badly defined but widely used concept in medicine \cite{Kaptchuk1998} . Many theories have been put forth to account for its effects, including expectancy, conditioning, emotional change and meaning \cite{Stewart-Williams2004b}. However, none of these theories can account for all of the effects observed in research. The conditioning approach gained a following at first \cite{Voudouris1985}  but was quickly opposed by the response expectancy theory of Kirsch \cite{Kirsch1985,Kirsch1997}. Research on these two competing perspectives lead to a number of findings which may help us elucidate some of the features of this strange effect. In this section, I shall first review the expectancy versus conditioning debate insofar as it relates to my main point, which is to examine whether or not the use of implicit measures of expectancies will allow us to predict the placebo response in individuals with greater accuracy. Then I shall review some other lines of evidence which support my case, and finally sum up the evidence in favour of my hypothesis.

The general format developed by Voudouris and still in use today in the research of Benedetti \cite{Benedetti2006c} and others, consists of three stages. 

Firstly, the participant has their pain thresholds calibrated and is given a number of blocks of painful stimuli. Secondly, the participants are either given the same stimuli again following the application of placebo cream or the pain is reduced for the second stage after the application of the cream. The second group are then said to be conditioned by this stage. In the third block, pain is increased again for the conditioned participants, and they typically show a much larger placebo response than those who are merely given verbal suggestions of analgesia.  These results were believed by their developer to argue in favour of a conditioned approach to the placebo effect. However, an experiment by Montgomery and Kirsch \cite{Montgomery1997} demonstrated that these conditioned effects resulted solely from expectancies,and after a regression in which expectancies were partialled out, there were no significant effects arising from conditioning. This would seem to argue that expectancies are the prime method through which conditioning has an impact, and indeed this is the position of Kirsch. 

Kirsch's \cite{Kirsch1985,Kirsch1997} theory of response expectancies specifies that these are the expectation of a non-volitional response, and he argues that they play a role in placebos and hypnosis. This theory  relies upon the measurement of self report expectancies to determine this, and this seems to me to be somewhat incompatible to what happened in the Montgomery and Kirsch research described in the paragraph above. In the experiment that confirmed the effects of expectancies, there were two groups who received lowered stimuli to condition them. One of these groups was informed of this pairing, and the other was not. Contrary to the predictions of the conditioned response model, those in the informed pairing group did not show an enhanced placebo effect, while those in the uninformed pairing group did. This seems to indicate that the effects of the conditioning procedure were inhibited by awareness. In other words, what occurred here was an example of implicit learning (learning without conscious awareness).

This conclusion is further reinforced by the work of Shiv and Carmon \cite{Shiv2005a} in the placebo effects of energy drinks given to participants at a lowered price. This research used an outcome measure of the number of problems solved in a specified period, and it demonstrated that those who believed that the price had been discounted solved less puzzles than those who received the drink at the normal price. The interesting feature of this research (for my purposes, at least) is that when participants attention was drawn to the discounted price, the difference between the groups was reduced. This would seem to argue in favour of an implicit learning situation in this experiment also. 

So, reviewing these research findings, we can note that in some cases, the placebo effect seems to emerge without conscious awareness. The Shiv et al (2005) study noted above showed that when participants awareness was drawn to the discounted price, the effects disappeared. A similar phenomenon occurred in the Montgomery and Kirsch (1997) study. These research findings suggest that the placebo response is at least partially determined by factors outside conscious awareness, and as such, it would make more sense to use implicit measures, of which the IAT is the most prominent. 

Implicit learning is a phenomenon which has come to the attention of psychologists in the past three decades. The research mostly focused on learning of patterns in random letter fragments and in the effects of sub-threshold sounds and pictures upon participants in a research setting. The investigation of these implicit attitudes was given a huge leap forward by the development of the Implicit Association Test \cite{Greenwald1998}. This computer administered instrument requests the participants to classify words into either trait (race, gender etc) or evaluative (self versus other, pleasant versus unpleasant) categories. This categorisation is performed separately at first, and then one of each type is paired together. This pairing is then reversed in the final step. The test works primarily based on reaction time (in milliseconds) and the assumption underlying the test is that items which are associated are easier to classify together, and that the difference in mean response latencies for classifications of each concept reflect the attitude towards that object. 


Another line of evidence which supports this thesis is the finding that the IAT is better at predicting spontaneous behaviour than explicit measures \cite{Conner2005,Hofmann2005}. The IAT outperforms explicit measures in some domains \cite{Greenwald2009} and these domains tend to be where there is little conscious deliberation or reflection upon the matter concerned. The placebo effect is the example par excellence of a un-deliberated and spontaneous  type of phenomenon, as no one chooses to have such a response and it seems dependent upon the lack of awareness of a participant that the treatment which they are receiving is not an active one. Thus, we can take this as supporting evidence that implicit measures should predict the placebo response more effectively. 

A final line of evidence is suggestive of this link, and it arises from the neurological patterns of activation associated with the placebo effects and the implicit association test. These patterns have been studied greatly over the last decade in placebo research, and the major findings are that a network involving the dorsolateral prefrontal cortex and the rostral anterior cingulate cortex appear to be activated during placebo related cognitions \cite{Mayberg2002,Zubieta2006}. While neurological research into the IAT has not been as prominent, two recent studies  \cite{Knutson2007,Knutson2006} has shown that both of these areas are activated during the association task that is the IAT. While this cannot be used to prove that the two phenomena are related, it does suggest that there may be some common neurological substrate underlying the kinds of cognitions involved in both of these effects. 

In conclusion, I believe that the expectancies underlying the placebo effect can be measured by means of the Implicit Association Test for the following reasons. Firstly, some placebo effects seem to require a lack of conscious awareness in order to occur. Secondly, the placebo effect seems to be best modelled as a spontaneous phenomenon and the IAT has been shown to predict these kinds of behaviours better. Thirdly, the placebo and the IAT task seem to share some common patterns of neurological activation. For these reasons, I believe that  development of  an IAT which can measure these expectancies is a worthy contribution to huma knowledge. 

\section{Modelling Placebo by Multiple Methods}
\label{sec:modell-plac-mult}

Finally, this section will present the overall approach and rationale for the work carried out in the course of this thesis.

The placebo effect is a complex phenomenon. It has been established that it can be predicted by some variables which are typically measured using self report approaches. These variables include expectancies and optimism. Optimism was measured using the Life Orientation Test, Revised (LOT-R - see Chapter \ref{cha:health-for-thesis} for details on this instrument), and a new measure of expectancies and treatment credibility was developed (Chapter \ref{cha:tcq-thesis}). 

Additionally, these variables only explain a small proportion of the variance in the observed placebo response. It is the contention of this researcher that some of the residual variance can be predicted by the use of implicit measures (specifically the IAT). Therefore, two implicit measures, one of Treatment Credibility and the other of Optimism were developed (see Chapter \ref{cha:development-of-iats} for details). 

The relationship between the observed placebo response, and these explicit and implicit measures was of primary importance to this research, and so a measure of Mindfulness (Mindful Attention Awareness Scale) was also used. 

In order to make use of psychometric modelling, large samples of the self report instruments were collected in the same population from which the experimental participants were drawn. This allowed for factor score and IRT models to be built for each of the measures. 

In the experiment itself, (see Chapter \ref{cha:primary-research}) physiological (ECG and GSR) was collected in order to examine the physiological characteristics of placebo and to allow for another form of measurement to be included in the final model.

The collection of these various forms of data, along with a behavioural criterion (the observed placebo response) allowed for psychometric models (Structural equation models) to be examined to seperate out the effects of the various predictor type. In essence, this thesis seeks to marry the strengths of psychology in psychometric modelling to its complementary strengths in experimental design, with the aim of establishing these methods and measure's relative usefulness in the prediction of placebo. 

%%% Local Variables:
%%% TeX-master: "PlaceboMeasurementByMultipleMethods"
%%% End:


\chapter{Theory \& Methodology}
\label{cha:methodology}

\section{Introduction}

This research assessed the differential contributions of implicit and explicit expectancy measures of treatment credibility and optimism to the prediction of the placebo response. 
In this section, the theory underlying this primary research question will be explicated, and followed by a description of the methods used to answer the primary research question. 


This chapter will consist of the following sections, representing the core parts of the thesis.

\begin{enumerate}
\item The theoretical model of the thesis will be explained;
% \item The approach taken towards the development of the IAT(s) will be explained;
\item The methods used for the development and validation of the scales used in the research will be described.
\end{enumerate}

The development of the implicit measures is described in Chapter \ref{cha:devel-impl-meas}. 

\section{Introduction to the theory}

Theories form an indispensable part of science. They represent an attempt to generalise beyond particular forms of evidence and data and to derive some kinds of overall principles which lie behind the observed events. The major theories behind the placebo and implicit measures were reviewed in  Chapter~\ref{cha:literature-review}, and in this chapter provides  an attempt to synthesise all of this information into a coherent whole. The major building blocks of this theory are as follows:
\begin{enumerate}
\item The results of implicit measures point towards there being at least two systems of attitude assessment and evaluation of stimuli in the human mind% ~\footnote{perhaps more, but certainly at least two}
\item The placebo effect is typically conceptualised as a conscious phenomenon, in spite of the experimental evidence
\item There appear to be feedback loops between bodily sensations and conscious perception --- embodied cognition
\item These feedback loops are evidence against an additive model of drug-placebo interactions
\item Despite the centrality of such conceptions to the placebo effect, no theory has as yet incorporated these findings into their theories
\end{enumerate}

This section will briefly review the evidence in favour of the above propositions, then will elucidate how these could be combined into our theories of placebo and implicit measures, and will provide some testable hypotheses regarding the theory, in the spirit of Popperian falsification.

\subsection{Implicit Measures and Dual Process Models of Mind}

The notion of dual process models of mind is an old one, dating back within psychology to at least the time of Freud, and possibly before. However, the modern conception of dual process models is much more recent, and developed as a result of work with implicit measures and through the findings of cognitive psychology \cite{Kahneman2002,Greenwald1995a,Gigerenzer2011,Klauer2007}. Essentially, the modern theory suggests that there are two prevalent systems of reasoning inherent to humans, a slow, rational, conscious system, and a fast, frugal and implicit system~\cite{Kahneman2002}. 

These two systems activate under different conditions and seem to perform different functions. One of the major hypotheses emerging from this theory is that under conditions of attentional strain, the implicit system takes over. This is, as we have seen in the previous chapter, borne out by much of the research into implicit attitudes. Asendorpf~\cite{Asendorpf2002} demonstrated what came to be called the double-dissociation effect, where implicit measures (of shyness, in this case) were more predictive of performance on a spontaneous speaking task (the Trier Social Stress Test), while explicit measures were more predictive of considered, deliberate behaviour. These findings have been further replicated by other authors   Thus, we can take the results of the IAT research and that into other implicit measures as pointers towards the operation of this system. The major point to take from this section is that this system appears to exist, and yet has only been touched upon in one or two articles over the past decade~\cite{Geers2005}. 

\subsection{Conscious Conceptions of the Placebo}

The dominant model within placebo research at present is the response expectancy theory of Kirsch~\cite{Kirsch1985, Kirsch1997}. This theory conceptualises the placebo as resulting from response expectancies, which are defined as ``the conscious expectation of a non-volitional response''. This theory has had some success, displacing the then prevalent theory of conditioned placebo responses~\cite{Voudouris1985,Voudouris1989}. That being said, the very definition of placebo and its nature as occurring as a result of deception and belief that one is getting a real drug would seem to suggest that non-conscious systems must be centrally involved in the mediation between awareness and the documented physical responses. 

Indeed, the Geers et al study cited earlier~\cite{Geers2005} demonstrated that semantic priming (by means of a scrambled sentence task) was an independent predictor of the response to a sleep placebo, which given that semantic priming tends not to be recalled by participants as having an impact on behaviour \cite{Wittenbrink2007}, implies that such priming (and the implicit system more generally) can affect the response to placebos and indeed biologically active treatments. Another study which points in the same direction is the Shiv et al 2005~\cite{Shiv2005a} study which demonstrated an effect of price of an energy drink effecting the number of puzzles solved by participants in a particular time. This effect disappeared when participants attention was drawn to it, which again implies that implicit systems were involved. Despite this evidence, the dominant theoretical framework remains untouched. In this chapter, I will propose a new model that incorporates both of these systems, and makes a number of predictions that will be either supported or rejected by further research. 

\subsection{Embodied Cognition and Placebo}
\label{sec:embod-cogn-plac}
Another issue with most of the conceptions of placebo current in today's research is that they are almost exclusively cognitive, a point made most forcefully by anthropology researchers~\cite{Thompson2009}  %Add this citation. 
This seems strange, given that it rests on assumptions regarding the relationship of mind and body which placebo-related research would seem to argue against. It seems that the theoretical perspective within the field is that the mind may effect the body, but not vice versa. Unfortunately, this is an untenable proposition, as recent work on haptic cognition (where judgements made by participants are affected by the sensory input they are receiving at the time), published in Science in 2010~\cite{ackerman2010incidental}
Indeed, more recent work by Cwir~\cite{cwir2011your} suggests that awareness of one's own interioceptive processes appears to be a good predictor of people's ability to predict the emotional states of others. 


There has been a resurgence of interest in mind-brain-body feedback loops within psychology and the social sciences more generally of late. It seems that the dominant cognitive model of mind (a kind of splendid isolation for the brain) is being slowly worn down by experimental evidence. An approach such as this was also proposed by Meissner et al in 2007~\cite{Meissner2007} in their meta-analysis which showed large placebo effects in some areas and none in others. They suggested that the larger placebo effects may have occurred in some conditions but not in others as the places in which placebo effects occurred tended to be those systems which have large nervous system connections with the brain, as opposed to communication typically mediated through hormones, which are many orders of magnitude slower. One problem with Meissner's theory is that he posits that placebo effects do not occur in clinical trials when the outcome of interest is a hormone such as cortisol. 

However, the field of psycho-neuro-immunology has noted that psychological variables (specifically optimism) exert major influences on antibody responses~\cite{Carver2010}. There is also some emerging evidence that mindfulness based treatments appear to affect antibody responses, at least in cancer patients~\cite{Ledesma2009}. This begs the question of why the analysis by Meisnner et al found no such effect. One explanation for these conflicting findings would be that the placebo effect is mostly determined by current state effects, while those effects investigated by psycho-neuro-immunology are the result of certain trait like characteristics of individuals.  

One psychological state which may have an impact on placebo responses, and should if my theory is to hold is that of mindfulness. Mindfulness is often defined as a moment to moment awareness of somatic and mental processes. If, as the results of Geers et al suggest, somatic focus can increase the size of placebo effects, then mindfulness (or another proxy marker for somatic focus) should also moderate the size of observed placebo responses. If my hypothesis is correct, then mindfulness should be correlated with placebo response, as well as acting as a mediator between implicit and explicit measures of the same construct. 

If attention is conceptualised as a limited resource, then placebo effects should be enhanced by placing participants in conditions of low stimulation while the treatment is applied. However, this hypothesis causes problems when we considered (below) the effects of participant provider interaction, which appear to account for a significant portion of observed placebo effects. 

\subsection{Additive Models of Placebo}

The common approach throughout clinical trials, and the study of placebo effects more generally, is that the effects of drug and placebo are additive. This assumption leads nicely to the principle that placebo effects and drug effects can be separated precisely. However, not all of the evidence points in this direction, and (apart from statistical convenience) there exists no 
{\it apriori\/} reason why this should be the case. Indeed, the work of Geers et al on somatic focus would seem to suggest that paying attention to somatic experience can increase the size of placebo effects. This could be occurring because of a feedback loop whereby a treatment is applied, the patient's awareness of the treatment leads them to attend to sensations related to the treatment, which engages the body's own healing systems, which then increase the size of the effect over and above what would have happened without awareness of treatment on the part of the patient. It is perhaps for this reason that drugs which are administered by an automated process are less effective~\cite{benedetti2003}. 

Another issue to consider in terms of mathematical models of placebo response is the phenomenon of the active placebo. This is where an active drug is offered as a treatment for which it has no efficacy, and nonetheless this treatment produces larger healing effects than a typical sugar pill placebo. I would argue that this phenomenon occurs because the side-effects of the drug produce  a feedback loop whereby the treatment has a somatic impact, which alerts participants to the treatment, causes them to accept it as more credible, and thus activates the body's own healing systems. This process, over time, could easily create a conditioned response to the original non-effective treatment, and loops such as this could be responsible for the observations regarding the efficacy of conditioned placebo responses. In this sense, I am making the argument that conditioned placebo responses are turned into expectancies over time.   


\subsection{Social Aspects of Placebo}

In addition to the possibility of feedback loops between awareness and sensation contributing to the placebo effect, the role of the provider needs to be emphasised also. In most drug research, the treatment itself (the pill or cream) is only one element of the context in which the healing process takes place. In addition to the internal factors which shape response to treatment (optimism and expectancies more generally), there are also important social and environmental factors. For instance, the classic work of Gracely et al~\cite{Gracely1985} demonstrated that the effects of the awareness of the provider can have a large impact on the outcome. In this study, half of the dentists were informed that half of the patients would receive placebo. 

In fact, all of them received the real painkiller. However, compared to the group treated by dentists who had not been told that placebo was a possibility, the other patients reported significantly higher pain. Indeed, a systematic review of health care interventions~\cite{DiBlasi2001} provided evidence that provider characteristics accounted for a large proportion of the observed placebo effects. More recently~\cite{Kaptchuk2008}, a three armed randomised control trial of acupuncture demonstrated that healing rates were greatly increased when the provider spent more time with the patient discussing symptoms and treatment (45 minutes as opposed to 15). This would seem to suggest that one of the reasons so many people use alternative treatments is not the efficacy of the particular form of treatment, but rather the chance to discuss their symptoms in detail and for longer periods of time.

Another important feature of the context in which healing takes place is the social context, that is the shared beliefs and rituals that make up a culture. For instance, Valium has a powerful resonance in our culture, immortalised in songs by the Rolling Stones are glorified through media et al. Therefore, even when participants have never experienced the drug itself, they bring pre-conceived notions of what it can do, and apply these to their perception of treatment, which alters its efficacy. However, research using the open-hidden paradigm (discussed previously) would seem to suggest that Valium lacks any efficacy for anxiety relief when participants are not aware that they are taking it~\cite{benedetti2003}. An additional example of this effect can be seen in the prescription of antibiotics for viral infections. Even though doctors know that they will have no effect, they are still administered to patients. It would be extremely interesting to give people placebo antibiotics and assess its impact on viral infections, relative to the efficacy of true antibiotics. My thesis would suggest that real antibiotics should be slightly more effective, given their obvious side-effects, but that this difference in standardised means would be less than 0.1. 

% A final (in my estimation) variable on placebo response would seem to be the effects of environmental factors. There is some evidence that suggests that a view of nature decreases healing times, and that patients in rooms with sunlight recover faster than those in rooms 
% without a direct view of the sun. It can be argued that this is as a result of environmental influences activating particular implicit cognitions relating to health. It is worth noting that the sun has been consistently associated with health in many cultures across the globe, and that people's mood tends to improve when the weather is nice. 

\subsection{Towards an Embodied Conception of Placebo}
\label{sec:towards-an-embodied}
Bearing the previous sections in mind, we can now move forward into proposing a new model for placebo, the embodied placebo model. This model, while not rejecting outright the findings of the expectancy theory, aims to enhance it with more recent research. Indeed, this theory is also compatible with that of conditioned placebo, and also with the motivational concordance approach of Hyland et al~\cite{Hyland2007} and the motivational model of Geers \cite{Geers2005}.
The essential features of the embodied model of placebo are as follows. Placebo effects are those healing effects which arise from the perception of treatment or caring on the part of the health care provider or from the context surrounding the treatment. They are mediated by four different kinds of factors.


\begin{itemize}
\item They are mediated by implicit cognitions operating extremely quickly
\item They are also impacted by awareness of the treatment and cognitive models of its efficacy
\item In addition, these implicit and explicit attitudes are enhanced or denigrated by somatic sensations
\item They are enhanced or reduced by the communications exchanged between the participant and the provider
\item They are also affected by the system of communications surrounding both the participant and the provider
\item Finally, they are mediated by somatic feedback from the surrounding environment.
\end{itemize}
 
In an effort to advance research and to conclusively demonstrate either my own percipience or ignorance, I shall argue that my theory makes a number of key predictions. 

\begin{itemize}
\item First, that placebo effects will be correlated with scores on implicit measures
\item Second, that there will be an interaction effect between explicit and implicit attitudes on placebo response
\item Third, that increased interioceptive awareness (or mindfulness) will increase the size of placebo responses
\item Fourth, that the implicit attitudes of the practitioner will exert a significant impact on the placebo response of the patient~\footnote{due to researcher constraints, this hypothesis was not tested in this thesis}
\item Fifth, that adding sensory input to a placebo treatment will increase its effectiveness. 
\item Sixth, that the extent of changes in physiological measures in participants will be correlated (as a lagged variable) with the size of the next reported pain rating.
\end{itemize}

However, in order to properly test this model, it needs to be compared to other plausible models. The first of these is the model of Kirsch, noted in his 1985 paper. This model claims that expectancies are the major (indeed only) mediating factor between consciousness and placebo responses. Therefore, this model suggests that there is a direct effect of expectancies on placebo response, that physiological outcomes should not be predictive of the placebo response and that any effect of optimism and mindfulness will be mediated through expectancies. This model claims that there will be no direct effects of any other explicit or implicit measure on placebo response, but that they will all be mediated by expectancies. 

Another, equally plausible model is that optimism is the driver of the placebo response, and that the effects of expectancies are mediated by levels of optimism. This model would suggest that both implicit and explicit optimism will have direct effects on the placebo response, and that expectancies (implicit and explicit) shall be mediated by levels of optimism. This model makes no predictions about the effects between physiological response variables and placebo response. 

Additionally, a variation of each of these models would suggest that either explicit or implicit expectancies could be the sole mediator of the placebo response, and these models were also tested. 

Finally, a null model where none of the variables measured had a relationship to the placebo response was examined, to ensure that the other models were at least somewhat more plausible. 

These models will be examined using a structural equation modelling approach, which should allow for an efficient and accurate test of my theory, even if the population is not particularly representative. 


\section{Development of the IAT}

Qualitative analysis was an essential part of this project.  
It gave insight and data into the development of the IAT's used in the final part of the research project.
The methods used for qualitative analysis here involved a thematic analysis~\cite{braun2006using} of the interviews conducted was carried out to develop themes both for the repertory grid and for the IAT.

Qualitative research typically relates to the analysis of interviews and other texts derived from people. It differs fundamentally from quantitative analysis in that it aims for a deep understanding of particular individuals, while quantitative analysis aims for a broad understanding of the sample as a whole. 

The biggest problem with qualitative analysis is that it cannot scale to the level of large scale surveys, as it requires significant amounts of researcher time per participant while quantitative surveys have a cost of development in time, but the marginal cost of administering the survey to a new participant is essentially zero (assuming distribution over the internet).

The issue of reflexivity is crucial to qualitative research (and also appears in quantitative research, though rarely as openly)~\cite{rosenthal1967covert, rosenthal1969interpersonal}.

Reflexivity refers to the impact of the researcher's prior conceptions and approaches have on the course of the interviews~\cite{finlay2002outing}.

This is extremely obvious in the choice of the major questions to be asked in the interviews, but it can occur in subtle ways during the interviews also (for example in the use of language by the interviewer) and in the quality of communication or rapport experienced by the researcher in the course of the interview. Reflexivity is also critical during the analysis, as the researcher must be aware of their own biases and ensure that this affects the analysis as little as possible, or at least report where the problems arose for them. A process of thematic analysis was used in preliminary interview research for this thesis, and the output of this was used to inform the development of the repertory grid. 

% \subsection{Thematic Analysis}

% The thematic analysis's primary purpose was to look for common patterns in the conceptualisation of health, sickness and treatment. As such, it was felt that the best approach would be to develop the codes from the transcipts themselves. 
% This is called an inductive approach to coding~\cite{haberman1979analysis}. Following transcription, each interview was coded line by line by the primary researcher, and codes were developed throughout this process.

% After all the interviews had been transcribed, the codes were pruned and amalgamated to reduce redundancy, and this process was repeated. This second coding lead to a number of new codes and insights which had been missed the first time, and the text was again coded for a third time following the development of these new codes.

% Then, the document was coded a fourth time, but on this run through the aim was to look at higher level patterns that emerged from the text. Following this coding procedure, a process of chunking of codes was carried out. This involved looking at how codes fit together and grouping them under a number of thematic headings. The original texts and recordings were referred back to at this point to ensure that the themes were representative of the original data, and finally the themes were written up to record the results of this exercise.


\subsection{Repertory Grids}

The use of repertory grids in this research was as a bridge between the qualitative analysis carried out and the quantitative side of the research. 

Repertory grids were developed by George Kelly as an aid to therapy~\cite{kelly2003psychology} although it has been used in many diverse situations in the ensuing years. 
Repertory grids were developed out of Kelly's theory of cognitive consistency, an active area of research which fell out of favour following the discovery of cognitive dissonance by Festinger in 1947~\cite{greenwald2002}.

The premise of the technique is simple. Firstly, participants are supplied with a list of important people in their life, such as their mother, an older sibling and a teacher whom they liked or disliked. They write down the names they have chosen for each person, and then they compare the people in groups of 3. For each group (or sort), they are asked to describe how two of them are similar and also how one of them is different in a word or short phrase. These words or phrases can then be analysed both quantitatively or qualitatively.

The primary method of quantitative analysis was through factor analysis.

The approach taken in this project was as follows. In one of the surveys carried out on the UCC population (TCQ version 1) participants were asked to rank the most important people in their life who were related to health care. 
This data was then sorted and ranked, and a list of the most common people used was compiled into a health related repertory grid. 
This was then administered to a small sample (N=17) to test the instrument. 
 The results of this testing are described in Chapter~\ref{cha:devel-impl-meas}, in Section~\ref{sec:development-iat}.

\subsection{Development of IAT}
\label{sec:development-iat}

The plan for the development of the IAT was to use the constructs obtained from the repertory grids to develop useful stimuli for the IAT.\
Unfortunately, as Chapter~\ref{cha:devel-impl-meas} describes, this portion of the research did not lead to a successful outcome, for reasons described in that chapter.

Therefore the IAT was developed from both the important figures which arose from the repertory grid, the qualitative interviews, and to match the explicit measure of treatment credibility. The optimism IAT was developed along similar lines to most other Implicit Association tests, in that the survey for this measure was used as a base. 


\section{Quantitative Research Methodology}
\label{sec:quant-rese-meth}
This section  will describe the methods employed for each part of the thesis and provide a rationale for why these methods were used in the thesis.
A mostly quantitative approach was taken for the following reasons. Firstly, the placebo  is a very noisy phenomenon~\cite{Singer2005}, subject to many sources of error and bias (c.f. Section~\ref{sec:concept-placebo}). Secondly, in order to predict the placebo effect, there needs to be some kind of measure, and these are typically metered in numerical terms. 

This thesis consisted of three main parts. Firstly, following a thorough literature review, the major constructs associated with the placebo effect and implicit measures were identified. These constructs were then administered to large samples of the population from which the experimental participants were drawn. This procedure was carried out for two reasons. In the first case, this was so that the population means could be estimated more precisely and thus the experimental sample compared on these measures.  

The second reason was so that more sophisticated models could be developed for person responses (which typically require larger samples than are common to experimental studies) and could then be applied to the experimental sample. This approach marries two of the strengths of psychological research; the latent variable approach common in psychometric research; and  the use of rigorous experimental design to determine relationships between constructs (through Structural Equation Modelling). This thesis aimed to use both of these strengths in combination to gain insight into the causes underlying the response to placebo in healthy volunteers.

The second major part of the thesis was the development of the implicit association tests (IAT's) and the explicit measure of expectancies (treatment credibility questionnaire) used in the experimental portion of the research. 

The third, and final, part of the thesis was the testing of these measures in an experimental setting using a placebo analgesia design examining the response to ischemic pain in healthy volunteers. 


In this section, the statistical techniques used to answer the primary research questions shall be discussed. Only methodology used throughout the thesis will be described in this section. Therefore, methods of cross-validation will be described, as will the regression models utilised and finally Structural Equation Modeling shall be discussed. The methods employed only within one chapter are discussed in the appropriate chapter. 

\subsection{Problems of Sample Inference}
\label{sec:probl-sample-infer}
\subsubsection{The Problem}

In every statistical approach, the core is the development of inferential tools to reduce our uncertainty about the events under study~\cite{gelman2010philosophy}. Given that we typically lack infinite resources, sampling from populations in a randomised manner is used to approximate the quantities of interest~\cite{venables2002modern}.

However, we are rarely interested in the specific sample we have recruited; we tend to want to infer properties of the population  from which they are drawn.  In non technical terms, given a sample and some analyses, we develop a model which we hope will predict the behaviour of future samples (and indeed the population).

This approach toward inference is often operationalised in the creation of a model, whether based on the results of a linear regression or factor analysis. Typically, we aim to maximise the amount of the response variable  in a sample of size $n$ explained given some number of parameters. It is trivial to see that as the number of parameters ($p$) increases, so does the fit --- in the limit, this would involve the fitting of a model with a separate parameter for each observation. 

Clearly, such a model will violate the principles of parsimony and clarity that we aim for in our science. However, even when $p$ is less than $n$, we still run the risk of over fitting a model to our data. Over fitting is said to have occurred when we model features of the data that are essentially random~\cite{friedman2009elements}. 

Because factor analysis is lenient towards mis-specified models and tends to model error as well as signal, many psychometric theories have faltered on the rock of replication~\cite{fabrigar1999evaluating}. SEM is often used as a panacea for such problems. However, a Structural Equation Model is only as good as the data and theory behind it, and if the factor analysis models noise, so too will an SEM on the same data set (to a lesser extent, of course). % This may account for the poor replicability of psychological theories based on the results of factor analysis and SEM.\@

\subsubsection{A Solution}

The typical scientific approach to this problem is simple --- replication. Replication, preferably by independent researchers, is supposed to ensure that models eventually tend towards the minimum of parameters for the maximum of explanatory power.

Indeed, many fit indices penalise complex models over simple models. This suffices for some research. Replication can also be fraught with difficulties, as it takes time, effort and additionally makes the assumption that population quantities are stable over time. Nonetheless, it is the ideal solution. However, given that replication is not incentivised by the scientific community (and may be unethical in some situations) \cite{roediger2013psychology}, researchers in the field of machine learning have come up with a novel approach which appears to improve predictive accuracy and can also aid in the development of theoretical understanding~\cite{friedman2009elements}.

\subsubsection{Cross-Validation}

The solution proposed by those machine learning researchers is at once simple and elegant, while also extremely practical. The technique is known as cross-validation, and its application routine in commercial and scientific data-mining settings. However, it does not appear to have found much favour within psychology as of yet (with some notable exceptions ~\cite{dawes1979robust}).

The basic premises of the techniques are:
\begin{itemize}
\item All models are wrong;
\item Models are best tested on independent samples;
\item Independent samples are sometimes hard to come by;
\item Therefore, data sets should be split into training and test sets, where the model is developed on the training set, and its accuracy assessed on the test set.
\end{itemize}

This approach seems to improve predictive accuracy by an order of  magnitude, especially when applied to large data sets ~\cite{breiman2001statistical}.  

The principle of training and test sets has since been generalised to $k$-fold cross validation where the data set is split into $k$ random pieces, and all but one of these are used to estimate a model, while the other is used as a test. This procedure is repeated $k$ times, and the results are averaged to form the best model (for that sample of data, at least). Some authorities argue that this procedure should be repeated at least $k$ more times, to control for the effects of random sampling~\cite{friedman2009elements}.

Another variation on the central approach is leave-one-out cross validation, where given a sample of $n$ observations, fit a model on $n-1$ and test on the other, a total of n times. This approach, while taking the technique to its logical extreme is not of particular usefulness to us at this point, as large inter-personal variability between individual participants typically observed in psychological data would tend to reduce its efficacy\cite{friedman2009elements}. % However, given that it is the most efficient means of cross-validation for smaller samples, this method was employed in the experimental portion of the research. 

In essence, cross-validation is an extremely valuable technique which has been mostly ignored in psychology. It is the opinion of this researcher  that this technique is useful, and it will be applied consistently to this research.

\subsubsection{Regression Models}
\label{sec:regress-models}

At the heart of typical psychological modelling practice lies the general linear model.This model underlies such familiar techniques as correlation, regression and analysis of variance (ANOVA)\cite{gelman2007data}. These techniques are based on the idea of fitting a straight line (or plane, in the case of multiple predictor variables) to the observed data and using this line to make inferences about the relationships between variables of interest.

% The models are typically fitted by a least squares method. Least squares is a criterion which suggests that in order to fit the best line, the average vertical distance between the points should be minimised. Least squares is the heart of many social and physical science modelling techniques, and is formalised in the Gauss-Markov theorems which prove that if the assumptions of the model are met, then in the limit, the least squares approach is the most efficient unbiased estimator~\cite{friedman2009elements}.

% The assumptions of the general linear model are as follows:
% \begin{enumerate}
% \item The residuals of the model (the distances between the predicted line and the observed values) should be approximated by the standard normal distribution
% \item The variables should have constant variance across their entire range (homoscedasity)
% \item The residuals should be independent of the response variable

% \item The residuals should be uncorrelated with one another
% \end{enumerate}

The general linear model has been elaborated greatly over the last century, and has been applied to the approximation of relationships that do not meet all of the assumptions noted above\cite{gelman2007data}. The introduction of link functions (non linear functions designed to transform the response variable into a form suitable for the model)  led to the development of generalised linear models, which allow the same computational techniques to be used to fit and test models for data which does not fit the requirements of the standard linear model~\cite{mccullagh1989generalized}.

For example, logistic regression is a technique for prediction of binary
 variables using a link function. Poisson regression is used to model data
 which is bounded by zero and positive infinity. A number of quasi methods
 are available for both of these techniques which allow models to be fitted
 to  data which has an excess of zeros (so called zero inflated models)~\cite{gelman2007data,venables2002modern}. In another direction, the
 requirement for the residuals to be uncorrelated has been relaxed to allow for the development of multilevel or mixed models, which allow the residuals of particular groups to be correlated with one another~\cite{gelman2007data}. 

A number of major issues arise with the use of the general linear model with psychological data. Firstly, the requirement of normal errors can sometimes be difficult to satisfy. This follows from the manner in which the normal distribution appears to behave. When it was originally discovered (by Gauss)~\cite{stigler1986history} it was used to model the combination of many small, independent variables. It tends to work well as an approximation when there are many independent variables affecting the results of an analysis.

However, when psychological tests (such as self report instruments) are developed, the aim is to remove as many of these small influences as possible so that the measure taps one construct with clarity. This will often lead to a non-normal distribution of errors. Non parametric approaches are an alternative to the General Linear Model, but these often lack the power of their parametric alternatives.The central limit theorem assures us that, in the limit, any distribution of means will converge to a normal, and in practice perhaps as little as 100 observations may suffice for a t-test~\cite{venables2002modern}. 

A more serious problem (in terms of the impact on outcomes) is heteroscedasity, where the response variables (or the predictor) do not possess constant variance across their range\cite{gelman2007data}. This can seriously affect the models built as points with less variance will be much more influential than those with a higher variance. The assumption of homoscedasity can be checked by formal statistical tests, but often graphical tests of this assumption are much more revealing, as they can show where the model fails as well as whether or not it fails.

In this research, linear and logistic regression models were utilised throughout all of the quantitative research in order to test hypotheses. 

% \subsection{Treatment of Missing Data}


% Missing values can often cause a problem with large datasets. The
% traditional approach has been to delete either pairwise (dropping
% all variables that have missing values for a particular analysis)
% or listwise (dropping all variables that have any missing values)~\cite{graham2009missing}.
% Both of these methods are flawed. The pairwise method can cause issues
% with interpretation in that different analyses will be based on different
% samples and degrees of freedom. The list wise method is slightly better,
% but it does make the assumption that the missing values are missing
% completely at random (MCAR) which is often not tenable~\cite{graham2009missing}.

% The primary deficiency in the above methods is their granularity. Both methods discard information in a rather crude way. Another approach towards the treatment of missing data is mean imputation, where any missing value is recoded as the mean for the sample. Predictive mean matching is a similiar approach which substitutes the mean for the respective variable containing missing data instead.

% The major issue with mean (or median) imputation methods is that they artificially reduce the variance of the data. As more of the sample tends towards missing, this can lead to artificially centred variables which when used in model building, provide inaccurate standard errors and lead to poor inferences.

% A more principled approach to missing data was developed by Donald Rubin, a statistician concerned with the problem of missing data in large public datasets such as the census or longitudinal studies\cite{little1987statistical}. The basic idea is quite simple, and has been generalised far beyond its original goal of filling in data in large representative samples.

% The method is multiple imputation, and the process is as follows. Firstly, the missing values in the dataset are predicted given the information available in the non-missing values. Some noise is added (typically Gaussian, though bootstrapping from the observed distribution is often a better approach)~\cite{gelman2007data} in order to prevent the computer from predicting the same values again, and these new values are then used as the starting predictors for the next imputation. This process is done a number of times (3--15, depending on the proportion of missing data)~\cite{graham2009missing,little1987statistical} and then the the analysis is run on each of these datasets seperately, and the estimates are combined at the end of this process. The major advantage of this method is by looking at the variance of the estimated parameters, we can approximate the uncertainty which surrounds our method of imputation. In addition, the repeated draws ensure that chance features of the data do not lead to erronuous conclusions, as could easily be the case if a single imputation method were used.

% Throughout this research, multiple imputation was used where the amount of missing data was substantial. Typically, between five and fifteen simulated data sets were created, and models run seperately across all imputed datasets, with inferences combined at the end. The multiple imputations were then analysed and tested to ensure that the simulated data was an accurate reflection of the non-missing data, in line with best practice~\cite{abayomi2008diagnostics}.

\subsection{Psychometric Analyses}
\label{sec:psych-analys-meth}

The kinds of psychometric models employed in this thesis were threefold, Factor Analysis, Item Response Theory (IRT) and Structural Equation Modeling (SEM). The general approach taken was to use FA and IRT to develop models, and utilise SEM techniques to test these proposed structures. Below, these three methodologies are introduced, following which the general plan of analysis for Chapters~\ref{cha:health-for-thesis} and~\ref{cha:tcq-thesis} are described. 

\subsubsection{Factor Analysis}
\label{sec:factor-analysis}
Factor analysis has a long history in psychology, and is now over one hundred years old. It is the most commonly used latent variable modelling technique in psychology, and more pages of \textit{Psychometrika} have been devoted to it than to any other technique \cite{henson2006use}.  Despite this, there are still a number of issues and controversies which surround the technique \cite{sass2010comparative}.  Essentially, factor analysis is an attempt to approximate a correlation matrix with a smaller matrix of parameters.  These hypothesised latent variables tend to be called factors or components. As such, it is a dimension reduction technique, much like principal components or clustering. 

 % One of the first controversies surrounding factor analysis is the dispute between Factor Analysis proper and Principal Components Analysis\cite{henson2006use}. The major difference between Factor Analysis and Principal Components Analysis is that in Factor Analysis, only the variance common across observed response to items (or communalities) is analysed, while in PCA, all of the variance (including variance only found in one item, or unique variance) is analysed. PCA tends to work better for data reduction, and indeed this is the reason why it was developed \cite{borsboom2006attack}.

% Throughout this research only factor analytic methods were used for psychometric purposes, as Factor Analysis provides a true latent variable modelling approach, while Principal Components Analysis  (PCA) does not. 

The most critical issue surrounding factor analysis concerns determination of the number of factors to extract~\cite{zwick1986comparison}.  This is an important issue, as theory and practice are likely to be held back if an incorrect choice is made.  The issue is not that there are no criteria on which to base a principled decision, but rather that the different criteria often do not agree, and it is thus ultimately left to the informed opinion of the researcher which factor solution is to be preferred.  All of the decision criteria will be reviewed in turn, and their advantages and disadvantages will be discussed \cite{henson2006use}. 

\begin{quotation}
  "Solving the number of factors problem is
     easy, I do it everyday before breakfast.  But knowing the right
     solution is harder" (Kaiser, 1954).
\end{quotation}

The choice of criterion for retention of factors is extremely important in applied work. This is because if an incorrect number of factors are extracted, then the predictions for the experimental portion of the research will be biased, and thus will not prove as useful as the method could otherwise be. 

The first, and most popular, criterion is surprisingly the least useful \cite{zwick1986comparison}. This rule is called eigenvalues greater than one criterion and recommends keeping all factors whose eigenvalues are greater than one. The rationale behind this approach is that eigenvalues less than one explain less of the variance in the matrix than one item, and as such should not be retained.  More recent research appears to put the minimal criterion for retention of eigenvalues at approximately 0.7 \cite{henson2006use}. 

The second criterion often used is the scree plot technique, which was popularised by Raymond Cattell . This criterion recommends that the eigenvalues of all factors should be plotted against their number, and only factors before the drop off in eigenvalues should be used. As the process of factor analysis ensures that the first factor will have the largest eigenvalue, followed by the second and so forth, this criterion looks for the point where the eigenvalues are very close to one another.  This criterion has a number of advantages.  It is available in all statistical packages, it can be used without any special training and it tends to give results which are somewhat, if not totally accurate \cite{zwick1986comparison}.  Its major disadvantage is that it relies upon the interpretation of the researcher, but given the strong emphasis on interpretation throughout factor analytic literature this should not be regarded as too much of a handicap.

The next criterion which can be used is that of parallel analysis. Parallel analysis is a Monte Carlo (simulation)  technique which simulates a data matrix of equal size and shape to the matrix under study, and calculates the eigenvalues of these simulated matrix against those of the real matrix \cite{horn1965rationale}. All factors are retained up to the point where the simulated eigenvalues are greater than the true eigenvalues.  Parallel analysis is one of the better techniques for assessment of the number of factors to extract , and it can often give very accurate results \cite{zwick1986comparison}.

Its major disadvantage is that it tends not to be available in many statistical packages, and that it can often over factor the data-set. Additionally, many tools simulate the new data from a normal distribution \cite{micceri1989unicorn}.  It does produce some of the most accurate results in simulation studies so it is a useful tool in practice \cite{zwick1986comparison}. 

Another useful criterion is that of the Minimum Average Partial Criterion (MAP) which extracts factors from the data-set until only random variance is extracted \cite{revelle1979very}. Again, this is an accurate criterion  \cite{zwick1986comparison} which is little used as it is not available in popular statistical programs. The only problem that  has been found with this criterion is that it tends to under-extract factors. %However, with the use of this criterion as a lower bound, and parallel analysis as an upper bound, then the decision of how many factors to retain can be made much easier. [I found this, but the literature does not appear to have as many examples of it] This leads on to the major point and issue with much factor analytic research today, whereby one decision rule is used to the exclusion of all others. Many researchers have recommended the use of multiple decision criteria, but this does not appear to be am approach utilised by many in the literature \cite{henson2006use}\cite{sass2010comparative}. However, this is the approach which has been taken in this research. This work will use parallel analysis, the MAP criterion and examination of scree plots to ensure that all relevant factor solutions are examined.

However, the ultimate test of a factor solution (without using other methodologies, such as Structural Equation Modelling) \cite{joreskog1978structural} is its theoretical clarity and interpretability, and this will be the first test used for all proposed factor solutions.

Another area of dispute amongst researchers in the factor analytic field is which method of rotation to use\cite{sass2010comparative}. As the eigenvalues are only defined up to an arbitrary constant, these rotations do not have any substantive impact on the factor matrix, except that they can make it easier to interpret (which is normally very useful). 

Rotations are commonly applied to factor solutions in order to reduce items loading on multiple factors, and to aid in the discovery of simple structure \cite{henson2006use}. Rotations can be divided into two classes, orthogonal and oblique \cite{sass2010comparative}. Orthogonal rotations return uncorrelated factors, while oblique rotations allow the factors to be correlated. Given that most psychological measures are correlated with one another, one would expect oblique rotations to be more common. However, the default appears to be orthogonal rotations, as they are claimed to be easier to interpret \cite{henson2006use}. Oblique rotations were applied throughout this research, as if the factors are truly uncorrelated, then the oblique rotation will show that, while the converse is not true for orthogonal rotations.

In conclusion, factor analysis is an extremely useful technique which has been widely applied in psychology. It is available in most software packages, is typically easy to interpret and can be carried out with a small number of items (more than 200 is often sufficient). The major problems with factor analysis are the necessity of determining how many factors to extract, which is a difficult and often subjective decision, and additionally if scores are required then many methods exist, again without any clear rationale for choosing one over the other.  

\subsubsection{Item Response Theory}
\label{sec:item-response-theory}
Item response theory (IRT) is often called model-based measurement\cite{fischer1995rasch}, (also referred to as Rasch modelling), is a newer approach to analysing self report data, developed both by the Danish mathematician Rasch in work for the Danish army, and also separately by Lord and Novick in the US, while working for the Educational Testing Service (ETS) in the 1950's and 60's \cite{van1997handbook}.
The fundamental premise of IRT is that the properties and scores on a psychometric test can be modelled as functions of both the items on the test and the ability of the people taking the test.

The IRT approach suggests that conditional on both the ability of the person and the difficulty of the test, the responses of each participant can be predicted probabilistically. As the latent ability of the participant rises, they tend to choose alternative responses which are more reflective of this latent ability. One example of this might be an item for extraversion ``I am always the life and soul of the party``, those respondents who had a higher latent score on the extraversion construct would tend to choose the agree or strongly agree options (on a typical five point scale). 

For instance, if one was modelling extraversion using a set of ten items, the participants who scored highest in extraversion would be most likely to respond strongly agree to the items.  IRT also assumes a property called local independence, which states that conditional on the ability measured by the test, the scores of each participant are independent of one another.

There are a number of different approaches taken to IRT~\cite{van1997handbook,fischer1995rasch}. The  Rasch models are the simplest, and have a number of extremely appealing mathematical properties. These models assume that only one trait is measured by the items, that all items are equally predictive of the trait, and that there is no guessing~\cite{van1997handbook}.

Because of these assumptions, it is possible to separate out person abilities and item difficulties perfectly. However, another approach (normally referred to as a two parameter model, or IRT proper) claims that items are differentially predictive of the ability being measured, in a manner analogous to different strength of loadings of items on a construct in factor analysis. Another model the three parameter model~\cite{lord1968statistical}, allows for correct responses through a process of guessing, but this model is not normally applied to polytomous items~\cite{van1997handbook,Mair2010}.

% The Rasch model is the simplest of these three general forms of models, and will be discussed first, followed by a discussion of two parameter models, after which I discuss three parameter models. Finally, this section ends with a discussion about non-parametric item response theory and multidimensional IRT.

In general, IRT models are represented by the logistic function, and are estimated iteratively through procedures of numerical optimisation (maximum likelihood \cite{fischer1995rasch}). The function used to describe the data is a logistic one, where ability is estimated from the probability of answering the question correctly. % \footnote{a probit model was used for many years as the logistic function was harder to estimate, and the difference between these two functions after scaling are minimum}.

The difficulty of an item is conventionally defined as the ability of participants who answer the question with 50\% accuracy. The parameter $\alpha$ is defined as the difficulty of the item. In two parameter models, another parameter $\beta$ is defined and is used for the discrimination of the item (the slope of the curve). In the more complex three parameter model, $\theta$ is used to measure guessing (the probability of a correct answer given low ability)~\cite{van1997handbook}.

IRT was developed in the context of ability tests, and this leads to much of the vocabulary fitting uneasily within personality psychology. For instance, in the context of a credibility questionnaire about various treatments, the questions on homeopathy are categorised as most difficult (see Chapter \ref{cha:tcq-thesis}). This does not mean that they are harder to answer, just that the probability of a respondent endorsing them is lower than the probability of a participant endorsing a similar item on the efficacy of painkilling pills.

% Below, all of the models will be described in terms of their original use in ability testing, and the section concludes with a discussion of polytomous IRT, which is the methodology exclusively relied upon in this research as this research focused on personality  rather than ability testing.

It is important to note that the names of the models are slightly deceiving, while they are called 1, 2 \& 3 parameter models, they actually involve the estimation of 1,2, or 3 parameters for each item. This normally means that a test of ${1,2,3\ldots, n}$ items will require the estimation of either ${n, 2n, 3n}$ parameters. The parameters are estimated by an iterative maximum likelihood approach, as so issues of optimisation and ensuring that a global maximum has been found are often important in practical applications~\cite{gill2002bayesian}.


\subsubsection{Structural Equation Modelling}
\label{sec:struct-equat-model}

Structural equation modelling is regarded by many as an adjunct technique for evaluating the results of particular factor solutions~\cite{fabrigar1999evaluating}. However, it is actually a far more general techniques to test the relationships between both manifest and latent variables, and even to establish causality in some cases~\cite{pearl1998graphs}. The factor analytic procedure is full of interpretative procedures where no principled choice can be made, and structural equation modelling (hereafter SEM) is an attempt to compensate for some of these deficiencies.

SEM was developed by Joreskog in the 1970's~\cite{joreskog1978structural}. It provides a means of testing hypothesised relationships between latent and manifest variables. In practice, the result of a factor analysis is regarded as a measurement model of the data.

This is combined with a structural model (which describes how the latent variables relate to one another and to the manifest variables). The two of these models are then used to construct a covariance matrix which is then compared with the observed data, and a number of indices of model misfit are calculated. Foremost among these is the $\chi^2$, which estimates the degree of model misfit. The desired result is a p-value of greater than 0.05, which shows that the two matrices are not significantly different.

However, the $\chi^2$ is extremely sensitive to sample size, and tends to be rejected in almost every case~\cite{henson2006use}, given that the sample sizes needed for accurate factor analysis and structural equation modelling tend to be quite large. As a result of this, many other fit indices have been developed. Foremost amongst these are the Non Normed Fit Index (NNFI), which is also known as the Tucker Lewis Index~\cite{bentler1990comparative}, the Root Mean Square Error of Approximation (RMSEA)~\cite{rigdon1996cfi} and the Bayesian Information Criterion (BIC)~\cite{schwarz1978estimating} and An Information Criterion (AIC)~\cite{akaike1974new}. These all have different strengths and weaknesses and are typically used in a complementary way. All of these fit indices incorporate explicit penalisation, which aids in avoiding overly-complex models. The $\chi^2$ it also has some penalisation (based on degrees of freedom) but it is typically not strong enough to prevent over-fitting. Some authors argue that this focus on other fit measures apart from $\chi^2$ is a way to avoid determining better models, but such a view is controversial in the field at present~\cite{barrett2007structural}.

% There are also some requirements for SEM models which are typically not met for much psychological and psychometric data. These are as follows:
% \begin{itemize}
% \item The distribution must be approximated well by the first and second order moments.
% \item Sample size is required to be large (>300)
% \item The covariance matrix must be strictly positive definite across the entire parameter space.
% \end{itemize}

The multivariate normality assumption made by the procedure is often difficult to meet in practice. However, there are a number of distribution free methods in SEM, of which the most common is a Weighted Least Squares approach. This proceeds similarly to a Weighted Least Squares approach in linear regression, where points are assigned weights depending on how closely they meet the assumptions of the model. Sample size, by contrast, is typically easy to increase (at least for non-clinical populations).

Identification of the model is one issue in practice, though as Joreskog notes, this can often be achieved by fixing a number of parameters to 0 or 1 (the inter-factor variances are often scaled in this fashion)~\cite{joreskog1978structural}.

Another, more theoretical issue is that no set of data is uniquely determined by an SEM model. This is known as the problem of rotation in factor analysis~\cite{maccallum2000applications}. Given a covariance matrix $W$, and a set of data $D$, there are many solutions which provide the same fit indices of the model to the data. This can lead to a similar problem as occurs to factor analysis, where the researcher must make a choice between models which are quantitatively identical. One approach for resolving this problem was discussed above, in Section~\ref{sec:probl-sample-infer}.

In this thesis, SEM was applied extensively to test the theorised relationships between variables, and the experimental chapter includes a number of tests of theoretically derived models (c.f. Chapter~\ref{cha:primary-research}).

\subsection{General Analytical Approach}
\label{sec:gener-analyt-appr}




Factor solutions were extracted using principal axis methods primarily, and maximum likelihood methods where these did not converge.

Primarily, direct oblimin methods of rotations were utilised, but promax rotations were also applied to ensure that the proposed structure was not overly sensitive to the methods of rotation used.

After the various factor structures were obtained, they were plotted and analysed for interpretability. Communalities and uniquenesses were assessed to ensure that there was no over or under factoring in the solutions. %% Communalities were then graphed against the number of factors extracted and the methods of extraction to provide a simple graphical guide to the usefulness of each solution.

Following this procedure of extraction and interpretation, Structural Equation Modelling was applied to each of the proposed factor solutions using the OpenMx and lavaan packages for R. The optimal factor solution was chosen using the AIC of each fitted model, along with the RMSEA of the proposed solutions, and these solutions were then evaluated for their performance on unseen data. 


Following the investigation of structure with the methods of classical test theory, the scales were analysed using Rasch models and item response theory. Firstly, mokken analyses were run, in order to check the assumptions of monotonicity, local independence and to assess how many sub-scales the analysis should be carried out on. 

Following this, three successively more complicated IRT models were fitted to each sub-scale (Rasch, one parameter and two parameter). Graded Response Models were used for IRT estimation as these are appropriate for ordinal data. Additionally, the most successful IRT models were used as the basis for other models analysed using SEM. 


Linear regressions were run to examine the differential effects of each of the correlated variables. Step wise selection on the training set was carried out, but the p-values are always reported from the test set. The performance of each of these methods was then assessed on held-out data (from Sample One, using ten fold cross-validation as previously described). 

In the case of the first samples in Chapters~\ref{cha:health-for-thesis} and~\ref{cha:tcq-thesis}, some of the second sample was used as a held out data set. For the second sample, the entire data set was split into three or four splits, and the cross validation procedure carried out for each. The splits were kept quite large  to allow for psychometric models to be fit to each split separately, and then to be tested on the remaining data. In Chapter~\ref{cha:health-for-thesis} a back testing approach was used, whereby the most successful models from the second sample were assessed for their performance on the first sample. For Chapter~\ref{cha:tcq-thesis}, this was not possible due to the revision of the instrument, and so a boot-strapping procedure was used instead.


\section{Conclusions}
\label{sec:conclusions}


In this chapter, the theory underlying the work of this thesis has been explained, and the methods used to answer the primary research questions have been addressed. Additionally, due to the complexity of many psychometric models, cross-validation techniques were employed to provide unbiased estimates of their likelihood. % Multiple imputation was used on the survey data where the proportion of missing data was greater than 10\%, and
Structural Equation Modeling was employed pervasively throughout the thesis to test particular factor analytic and regression model structures.



% \section{Preliminary Research}

%  Sampling from a population can be difficult, especially as the requirement of randomness needs to be satisfied. However, even if the surveys and measures are sent to a random selection of participants, who responds will almost certainly not be random, as people may only respond to surveys which are salient to them, and ignore the others. This is especially true in a University environment where many surveys are sent out to either random samples or the entire student population regularly. Some of the issues and concerns around sampling for each of the different pieces of research are discussed below.




%%% Local Variables:
%%% TeX-master: "PlaceboMeasurementByMultipleMethods"
%%% End:



\chapter{Health Optimism and Mindfulness Data}
\label{cha:health-for-thesis}
\include{HealthforThesis}

\chapter{Treatment Credibility Questionnaire}
\label{cha:tcq-thesis}
\include{TCQThesis}

\chapter{Development of Implicit Measures}
\label{cha:devel-impl-meas}

\include{DevIATQuant}

\chapter{Measuring Placebo by Multiple Methods}
\label{cha:primary-research}
\include{Experiment}

\chapter{General Discussion and Conclusions}
\label{cha:general-discussion}


\section{General Discussion}

The central aim of this thesis was to examine the measurement of
treatment expectancies in the context of the placebo effect. This
thesis used explicit (self-report), implicit and physiological
variables to deliver greater understanding of these psychological
constructs and their relationship with the placebo effect.


The thesis was structured into a number of parts, each of which will
be discussed in turn, setting the discussions at the end of each
chapter into a fuller context. Following this, the overall aims and
objectives of the thesis will be reviewed and set into context. Next, the specific contributions of this research to the literature will be elucidated. 
Finally, some avenues for future research will be described.

The first section of the thesis was the review of the literature
around the placebo effect and implicit measures, along with any known
individual-level predictors of the effect. This review found that
placebo effects have been seen to be mediated by optimism (as
operationalised by the Life Orientation Test, Revised (LOT-R)
\cite{Scheier1994}) \cite{Geers2005,morton2009reproducibility} in both
pain and non-pain paradigms. This review also linked the constructs of
optimism, which can be defined as ``generalised outcome expectancies
around the future'' \cite{Carver2010}, with the construct of response
expectancies of Kirsch \cite{Kirsch1985,Kirsch1997}, which are
described as the ``expectation of a non-volitional response''. These
constructs both crucially rely on the participants' perception of the
likelihood of the event, and a secondary aim of the thesis developed
which was to examine the usefulness of these as two separate rather
than one combined predictor. The hypothesis was that these two
constructs should be significantly correlated, and that one of them
would mediate the impact of the other on the observed placebo response
\cite{Geers2005}.

Another important potential moderator variable came to light in the
literature review, which was the construct of mindfulness, as
operationalised by the Mindful Attention Absorption Scale (MAAS)
\cite{brown2003benefits}. This variable was found to moderate the
correlations between explicit and implicit measures in an experience
sampling study. Furthermore, implicit measures were found to be better
predictors of spontaneous responses, while explicit measures were
found to be better predictors of deliberate behaviour
\cite{Levesque2007}, a pattern known as the double-dissociation effect
and observed in IAT studies in a variety of domains
\cite{Asendorpf2002,Perugini2005,Grumm2007,Steffens2006}.

This research literature suggested that a good strategy would be to
replicate the LOT-R and MAAS on a number of samples from which the
eventual experimental sample would be drawn, both to develop better
models for these constructs' relationship (as they had not been
concurrently examined when this work was carried out) and to tailor
the measures towards the experimental sample, a strategy which was
hoped to increase the precision of estimation on the overall sample.
In this research, the RAND-MOS was used both to replicate the impact
of these primary constructs on health, and to provide a weak link to
health and treatment expectancies in preparation for the experimental
research.

One large problem with the development of an implicit measure of
treatment expectancies was that there existed no gold-standard measure
of explicit treatment expectancies, despite them having been shown to
materially affect the outcome of clinical trials
\cite{Linde2007,Bausell2005,Benedetti2005}. This was an issue both
because typical methods of IAT development rely on such a self-report
measure, and because without such a measure, the incremental
improvement (if any) granted by an explicit measure would not be
clear. Therefore, the development and testing of such a measure formed
part of the work of this thesis.

Another fact which became extremely clear during the course of the
literature review, is that current methods of both developing and
validating implicit measures (or more specifically, IATs), rely
extremely heavily on the existence of prior self-report instruments.
For the placebo effect which was the focus of primary investigation,
no such self-report measure existed, which meant that a new approach
needed to be tested. A review of methods in this space revealed the
work of George Kelly on personal construct theory \cite{Kelly1991},
and more specifically the repertory gird introduced in Book 1 of the
aforementioned reference. It was believed that such an approach might
prove useful in the current case, and might also prove useful with
other constructs of interest in the future\footnote{I am indebted to
Drs Jurek Kirakowski and Sean Hammond for originally suggesting this
idea.}.

Therefore, in order to develop these stimuli, it was decided to do
some qualitative research around health constructs with doctors,
alternative therapists and students, to gain an appreciation for how
they contextualised and understood issues around health, medicine and
sickness. The results of this procedure, along with a parallel
exercise asking participants to rate the five most important people or
qualities of people related to health, would then be used to develop
the Health Repertory Grid.

Finally, the development of the experiment was planned, and it was
decided to use a placebo analgesia design, inducing pain with the
sub-maximal torniquet technique \cite{moore1979submaximal}, as this
form of pain induction has been shown in a meta-analysis to produce
the largest effect sizes \cite{Sauro2005}.

Below, the results and findings from each of these sections are
described and placed in a context of larger research.

\section{Health, Optimism and Mindfulness}
\label{sec:health-optim-mindf}

This study consisted of approximately fifteen hundred ($N \approx
1500$) observations, of which 1100 were collected in an online
fashion, and four hundred were collected in pen and paper format.

The major aims of this part of the research was to collect useful
background data for the experimental sample, enabling psychometric
models to be built on large samples, which could then be applied to
the experimtal sample, as well as developing psychometric models which
could tease out the relationships between the health-related
constructs (RAND MOS), optimism (LOT-R) and mindfulness (MAAS). The
headline finding of this research, which was replicated across both
samples (and a further one which was not analysed for this thesis) was
that optimism was negatively related to both health and mindfulness, a
finding which has not been reported before.

This finding can be better understood in the light of some of the SEM
results on the same dataset. In these, it was determined that
emotional well-being moderated the impact of this suprising effect.


Additionally, this strange phenomenon replicated across both original
studies, and also in the pilot experiment. However, it did not
replicate in the main experimental study, raising the issue of what
the critical differences between the samples were. Upon examination of
the materials, the only difference between the setup for the
experimental study was the order in which the measures were
administered. In all the previous studies, the MAAS was filled out
before the LOT-R, while in the experimental study, the order was
opposite. There are not many examples of these kinds of effects in the psychological literature, but some survey research suggests that the order of administration of questions within a survey can have similar impacts~\cite{schwarz1999self}. 

This finding links in with the other finding that emotional well-being
(a subscale of the RAND-MOS) moderated the effect of optimism on
health. To see this, merely look at the questions which make up the
Emotional Well-Being scale, and note its similarity to the LOT-R. It
appears that what occurred is that participants who would typically
respond highly to optimism (as evidenced by their Emotional Well-Being
score) had their response patterns altered by the act of completing
the MAAS. This may have occurred because the MAAS operationalises
mindfulness in terms of negative mindlessness. This may have caused
the participants high in Optimism to have realised that some of their
optimism was unjustified, which therefore depressed their responses on
the LOT-R, leading to the unexpected negative correlation with health.
However, this is somewhat circumstantial given that this hypothesis
was not directly tested in this research, but it was supported by the
SEM carried out in Chapter \ref{cha:health-for-thesis} which pointed
towards mindfulness and emotional well being as mediating the strange
optimism-health relationship.

Additionally, the cross-validation approaches employed in this section
of the research showed an interesting pattern. The factor models which
were developed using the reduced scales from IRT analyses showed
better predictive ability on unseen data than did those which were
developed from factor analysis alone. This suggests that an
interesting approach to the development of scales would be to use IRT
(specifically Mokken analyses) to reduce the number of items, and use
this reduced scale in SEM to determine if this effect is widespread or
merely circumstantial to the samples collected here.

\section{Explicit Expectancy Measure}
\label{sec:expl-expect-meas}

The next step in the research was the development of an explicit
expectancy measure, which would capture the multi-dimensional nature
of treatment expectancies more fully \cite{Stone2005}. Again, this
measure was developed across two samples, a small first sample which
aimed to validate the measure in terms of reliability, and a second
larger study which incorporated the results of the first study and
utilised the Beliefs About Medicine Questionnaire (BMQ)
\cite{Horne1999} to provide evidence of convergent validity for the
measure.

The results of the first study showed that the measure had extremely
good split-half reliability ($ \alpha=0.9$). Another finding was that
the first instrument broke down in factor analytic terms in terms of
Conventional (Pills, Creams, Injections) and Alternative (Acupuncture)
treatments. This was obvious both from the intercorrelations of each
of the questions, and from the results of factor analytic and IRT
modelling. This suggested that two other alternative treatments should
be added to the questionnaire, to determine if this was a valid
structure and to provide balance.

The second study showed the expected correlations with the BMQ measure
(positive for alternative treatments, negative for conventional
treatments) which suggested that the measure did have convergent
validity. Additionally, the notion of conventional and alternative
treatments being separate factors was replicated across both samples,
and using factor analytic and IRT methods. Again, the backtesting
strategy employed in the previous chapter, using a combination of IRT
and FA approaches, performed better than either of these approaches
alone, suggesting that this may be a useful methodology in general.

%%insert some interesting SEM results here, when available.

\section{Repertory Grids and Experimental Pilot}
\label{sec:qual-rese-rep}

The final portion of research conducted before the main experimental
portion was the development of the repertory grid and the development
of the IAT measures and their piloting, along with the piloting of the
experimental procedure.

From the coding and analysis of the interviews (not reported), it was determined
that the major contrast between styles of healthcare (conventional and
alternative) was that the alternative therapists regarded themselves
as treating a person, whereas the GP's focused more on the symptoms
which this person presented with. Additionally, the GP's seemed to
focus more on external causes (pathogens, environment) whereas the
alternative therapist focused more on internal causes (body-mind
linkages, thought patterns). Note that both of the groups focused on
the nature of health as both a social and personal responsibility, but
that they disagreed on where the responsibility for changes lay. The
students in the sample provided either a moderately alternative
viewpoint, or a moderately conventional viewpoint, but in no case were
they as extreme as any of the practitioners. The major differential
theme appeared to be that of conventional versus alternative, and the
truth or falsity of health claims, and this was incorporated into the
development of the Health Repertory Grid (and later into the design of
the Treatment Credibility IAT).

The next step was the development of the Health Repertory Grid. This
was developed analogously to the original repertory grid, using health
related constructs taken both from the interviews and from the
questionnaire asking people to name the five most important
health-related people or qualities in their life. Unfortunately, it
became apparent in the course of piloting this instrument that
participants did not have enough examples of these constructs to
enable them to complete this task. Therefore, the treatment
credibility IAT was developed using the notion of true versus false
and conventional treatments versus alternative treatments, while the
optimism IAT was developed from the LOT-R, in line with current
practice in the field.

Next, the two implicit measures were administered to a small sample of
volunteers, and showed the expected correlations (small, but in the
right direction) with the explicit measures of the same constructs.

Additionally, when the experiment was piloted, there was a small
effect of the IAT's on the probability of placebo response in a
logistic regression. This pilot study also showed that the two IAT's
correlated quite well with the respective explicit measures, and that
a placebo response could be induced with the procedure. However, given
that a majority of the participants did not respond to placebo, it was
decided to use a priming procedure to increase the probability of
placebo response~\cite{Geers2005a}.

\section{Experimental Research}
\label{sec:exper-rese}

The final part of the thesis was the experimental research. One
hundred and eleven (N=111) participants were assigned to either a
Deceptive (told drug, got placebo), Open (told placebo, got placebo)
or No Treatment Condition. Additionally, participants were randomised
to either a Prime or No Prime condition. % The major finding in terms
% of condition was that those participants who were in the Open
% condition showed a greater placebo response than those who were in the
% Deceptive Condition. This finding was extremely unexpected, and has
% not been seen in the literature before. However, given that the
% suggestions focused on ``clinically proven'' (in the Open condition)
% and ``recently approved'' in the Deceptive condition, this may have
% had an impact. 
There was a significant Prime by Condition interaction,
suggesting that priming was differentially impacted by by the
condition (Open Placebo participants showed a greater propensity to
respond to suggestion after being primed). This would seem to have
potentially large implications for medical practice, if replicated.

With respect to the major hypothesis of the thesis, no direct impact
of the explicit or implicit measures was shown to be significant in a
logistic regression, using a step-wise approach with a training and
test set to ensure that the estimated p-values were unbiased. Note
that essentially, while this study provides some preliminary evidence
for an impact of the TCQ IAT on placebo response, this effect was not
significant on the test set (though the z-value was quite large), and
so further research would need to be a larger study to confirm or
refute this effect.

% However, a three way interaction was shown to be significant,
% However, this % % model suffered from issues of multi-collinearity and
% so should not be taken as accurate without replication.
Attempts were
made to utilise machine learning techniques for the prediction of
placebo response, and while some of these techniques were extremely
effective at predicting the lack of a placebo response, none performed
at above chance level at predicting positive placebo response.
However, it would be useful in terms of the design of clinical trials
if participants who would reliably respond to placebo could be
eliminated, and so this measure might prove useful if replicated.

Finally, the relationship between skin conductance and pain ratings
was examined. This analysis showed that for the two treatment
conditions, entirely different GSR responses were shown. This would
seem to indicate that the difference between certain and uncertain
expectancies can be examined using simple GSR equipment as opposed to
expensive fMRI equipment. There is very little earlier work on this in the literature, with the exception of one study~\cite{Fujita2000}. However, this study showed that the GSR measures changed when an active drug was administered, and did not when a placebo was administered, suggesting that these effects may be different. This finding is a contribution to the research in that it is novel, and if replicated could provide an interesting back-up of self-reported changes, and may help to elucidate the relationship between certainty of expectations and placebo response. 

In conclusion, there was no significant relationship shown between
implicit measures and placebo in standard analyses, suggesting that
any effect, if it exists, is likely to be quite small.

\section{Strengths and Limitations}
\label{sec:strengths-limit}

In order for this work to prove useful to future researchers, the strengths and weaknesses of the overall approach need to be outlined. Firstly, this research benefited from the following:


Background data was collected for all self-report measures from the sample population as the experimental sample was drawn. This is extremely useful for two reasons. Firstly, it allowed for the construction of psychometric models from the large samples which could then be applied to the smaller experimental samples. This approach was hoped to increase the predictive power of the self-report instruments used. While this approach did not pay dividends in this research, conceptually it seems like a useful addition to mixed survey and experimental design within psychological research. 

Secondly, this background data allowed for an assessment of how the experimental sample differed from the survey sample(s). This is extremely important given that the assumption underlying generalisation of the study is that the sample is random. Disregarding the impossibility of random selection from a self-selected audience (in all cases), it is still useful to examine how different samples from the same population collected for different purposes are similar and different from one another. These techniques are not commonly used in research, and as they proved useful in this research, it is the authors hope that they might prove useful to others. 

Another strength of this research (further outlined in Section~\ref{sec:meth-contr}) is the pervasive use of cross-validation and out of sample testing of all psychometric and regression models. This procedure improves test accuracy by an order of magnitude in some settings~\cite{friedman2009elements}, and is best at preventing errors when models grow complex, which many psychometric models do. Additionally, this process allowed for efficient use to be made of both survey and experimental data, which is something which makes this approach a useful one for future researchers to take. 

In terms of the overall approach taken to the experimental design, this study is one of the first to examine self-report, implicit and physiological variables and their relationship to the placebo response. This is a novel approach for the study of placebo, and one which is important given that the placebo effect straddles the boundaries of the physical and psychological. There were intriguing correlations between the condition and the physiological responses measured, which have not previously been reported in the literature (with the exception of some tangential research carried out by~\citet{Fujita2000}). This is a fruitful area of study, especially in the light of research suggesting that dopamine levels reported from neurological studies appear to co-vary with the certainty of reward. The only difference observed between the two conditions in this study was the level of uncertainty, and so it is justified as regarding this (until further evidence is acquired) as evidence of the same phenonmenon. 

In terms of limitations to this study, there are quite a few. The first, and most glaring is that all the research was carried out on students and staff from one university, and it is difficult to know how these results will generalise to other populations given the weirdness of the particular sample~\cite{henrich2010most}. Additionally, another limitation of this research is that the IAT measures were not administered to a large, representative  sample. This limitation is due to the unavailability of a computer-assisted methodology to administer these studies across the internet. 

Additionally, a further limitation with respect to the development of the IAT's is that the rep grid approach was not successful, which made it impossible to pursue the principled approach to IAT development proposed at the outset. 

A final limitation of this research was that it was only powered to detect medium sized effects between groups (in the schema of Cohen~\cite{cohen1988statistical}). Given the lack of significance of the main hypotheses in this sample, it may be that implicit measures have no predictive power, or that this research was not powerful enough to detect them. This is a determination which needs to be made by future research in the field. 


\section{Overall Contributions}
\label{sec:over-contr}

The overall contributions of this thesis can be divided into three distinct contributions: those which are applicable to clinical research, those which are applicable to experimental research, and those which are methodological. Each of these sets of contributions are outlined below, and in the next section further research within each of these areas will be proposed. 

\subsection{Implications for Clinical Research \& Practice}
\label{sec:impl-clin-rese}

The largest contribution for clinical practice may be the finding that semantic priming can impact the response to placebo. While this has previously been reported in the literature~\cite{Geers2005a,Jensen1991}, this study is one of the first to demonstrate that the effects of priming carry over to a situation where the participant is told that they will receive a placebo. This links up with the research of Kirsch \& Kaptchuk~\cite{kaptchuk2010placebos} where participants who were given open placebos in an RCT of irritable bowel syndrome showed significant improvements over a no-treatment group which was matched on all interventions except for the placebo. The effect sizes for this intervention were quite small, however, and this research indicates that through the use of priming, these effect sizes can be significantly increased. This would seem to provide a new way for physicians to ethically increase the response of patients to particular treatments, without breaking informed consent. 

One difficulty with this approach would be that it could be argued that there is no informed consent for the priming intervention, but if these kinds of semantic priming techniques could be generalised to be a routine part of clinical practice, (perhaps by being embedded in consultation documents), and the benefits were clear, then the benefits might seem to outweigh the risks. However, such a decision would be contingent on further research to outline both the benefits and risks of this approach in more naturalistic settings. A proposal for this research is outlined below.

Another important contribution of this research to clinical research may be that machine learning techniques were able to predict who would not respond to placebo in this study. This is an exciting finding, as pharmaceutical companies invest large sums of money into trials, and if extremely placebo-responsive patients could be removed using these models in the placebo washout phase of RCT's, then the success rate of trials would be improved and treatments with actual effectiveness would not be deemed useless due to large placebo responses in a subset of the placebo group. 

\subsection{Implications for Experimental Research}
\label{sec:impl-exper-rese}

There are also a number of implications for experimental research arising from the work reported in this thesis. Firstly, 

Additionally, the priming finding noted above in the clinical contributions section could also prove extremely useful in inducing more repeatable placebo effects with smaller samples. This is especially germane in the field of pain, given that more participants than necessary for the purposes of the research should not suffer noxious stimuli. Current practice in much of the field is to use a pre-conditioning approach, but this requires more expensive equipment and particular forms of pain induction. Should the priming findings be replicated more widely, this constraint would be removed, and both ethical and replicability concerns within the field could be alleviated. 



\subsection{Methodological Contributions}
\label{sec:meth-contr}

Another contribution of this thesis was in the general methodological approach taken. 
As described throughout the thesis, a number of methods were employed to develop the instruments 
in this thesis which are not commonly used in psychological research.

The first contribution of this part was the collection of survey samples of all the explicit
measures used in the experiment from the same population as the experimental participants were sampled from. This is novel in the literature, and provides a useful check on the generalisability of the measures to this population, and insight into how the experimental sample differed from the survey samples. 

Another contribution was the pervavise use of cross-validation throughout both the psychometric analyses and the regression modelling. This is a better approach in terms of minimising the number of false decisions made as a result of over-fitting, and its use throughout this work has ensured that all models were tested on unseen data, so it is somewhat more certain that they reflect a real process of interest rather than sample variability. 

These two contributions are important in that this thesis can provide a case study for future work which can reuse these tools. 

Additionally, the results of the optimism-mindfulness correlations across this thesis suggest that more attention should be paid to the order in which questionnaires are administered. Given that many surveys are delivered across the internet, randomising the order of complete instruments would seem to be a reasonable precaution to ensure that order effects do not lead to spurious correlations (and to assess if any previous studies have been impacted by this effect). 

\section{Further Research}
\label{sec:further-research}

There are a number of useful pieces of further research that would illuminate further some of the research conducted as part of this thesis. 

Firstly, the limits of semantic priming need to be assessed in an RCT. It seems logical, given the effects of expectancies~\cite{Bausell2005}, that there would be an impact, but the size of this effect would need to be assessed. The difficulty with such an RCT is that given the results of this study, the participants in the placebo group would tend to respond more than those in the treatment group, which might make this study more difficult to run (as no pharmaceutical company would be interested). Nonetheless, this would provide better insight into under what conditions semantic priming has an impact on response to verum treatment versus placebo. 

Another piece of further research would be to replicate the GSR-expectancies link described above. If placebo/verum drug conditions can be separated by the use of skin conductance measures, then this would greatly aid in the assessment both of the size and the duration of placebo effects. Additionally, this link may provide important insights into the physiological impacts of placebo. 

A final piece of further study which naturally arises from this research is the assessment of order of administration effects across particular self-report instruments. This is extremely important, as it is simply not known to what extent they can effect outcomes. While there has been some research into the norms surrounding questionnaires and the impact of the order of questions, there is much less research on the order effects between self-report instruments, and this research is important to ensure that one measure is not impacting the response on another in unforeseen and unwanted ways. 


\section{Conclusions}
\label{sec:concl-furth-rese}


This thesis set out to achieve the following:

\begin{itemize}
\item To develop and test an implicit measure(s) to predict the
placebo response;
\item To develop and test a more substantial explicit expectancy
measure for use in future placebo-related research;
\item To test a number of theoretical models around the relationships
between explicit, physiological and implicit measures and the response
to placebo;
\item To apply better models and methods to the experimental and survey data-sets used.
\end{itemize}

At this point of the thesis, it remains to assess to what extent these
goals have been achieved. Briefly, there was weak to some evidence for
a potential effect of implicit measures on the response to placebo.
This should be tempered with the fact that a direct significant effect
only occurred in a training sample, and did not replicate on unseen
data. Additionally, the impact was only significant in the presence of
interactions, and may as such represent random sample variability.

Next, the explicit measure developed in Chapter~\ref{cha:tcq-thesis}
and used in Chapters~\ref{cha:devel-impl-meas}
and~\ref{cha:primary-research} appeared to possess both face and
construct validity. However, in the experimental portion of the
research it did not appear to possess any predictive validity, which
would seem to suggest that this measure is not particularly useful in
the prediction of placebo response in healthy volunteers.


However, this measure could still prove useful in assessing
treatment-related expectancies in general. The instrument has been
revised in this thesis to have a common scale for all questions, and
additionally has been subjected to rigorous psychometric validation.
This should allow the measure to be re-used by researchers interested
in the manner in which treatment-related expectancies are
conceptualised by participants from the general population. It may be
that the measure itself would prove useful if the size of placebo
effects had been larger in the experimental study, and this is
something that could be tested by future research.

In terms of the testing of theoretical models, a model including both
implicit and explicit measures provided a better fit to the data than
did either alone. Importantly from a theoretical perspective, a model
with a generalised expectancy factor (from Kirsch) provided the best
fit to the data. This should be contextualised with the finding that
the LOT-R interacted with the TCQ IAT to achieve significance on the
training set.

An interesting and unexpected finding from this research was that the
GSR of participants appeared to be affected by the Condition in which
they were in. In fact, the cross-correlations between the pain ratings
and GSR were differentially directional between the Deceptive
(positive) and Open (negative) placebo conditions. This would seem to
suggest that certain versus uncertain expectancies can impact
physiological variables differently. This is an interesting finding,
and one which has not been reported before.

So, in conclusion, this thesis has provided a comprehensive test of
the predictive ability of implicit measures (IAT's) and a new
treatment credibility self-report item. On balance, neither of these
measures provided useful predictive power in a large study of placebo
analgesia in healthy volunteers.

While the major hypothesis of this study was not demonstrated to be
true, some other contributions to the literature have been made
\begin{itemize}
\item Priming exerts significant effects on the response to placebo;
\item GSR appears to correlate appropriately with response to placebo;
\item Models using both IRT and FA approaches perform better than
models trained with either method alone;
\item The relationship between optimism and mindfulness appears to
vary based on the order in which each of the instruments is delivered.
\item An approach using cross-validation and replication provides better evidence for the models tested
\end{itemize}

%%% Local Variables: 
%%% TeX-master:"PlaceboMeasurementByMultipleMethods"
%%% End:
\appendix
\chapter{Appendix 1: Supporting Tables and Figures}
\label{cha:append-1:-supp}
\include{Appendix}

\chapter{Appendix II:\ Qualitative Analysis}
\label{cha:append-ii:-qual}


\section{Analysis of Health Construct Interviews}

\subsection{Methodology}

Eight interviews were conducted in a semi-strctured format. The participants in the interviews included 3 alternative/complementary therapists, 2 GP's and 3 ordinary people with experience of healthcare, both western and complementary.

The questions which were asked were as follows:
\begin{enumerate}
\item What does the term health mean to you? 
\item Have you had any health problems in the past that you would like to discuss?
\item How do you believe that health can be promoted and sustained?
\item What kinds of treatments do you feel are effective? 
\item How are they effective?
\item What has been your experience with the official medical sector?
\item What has been your experience with the complementary medicine sector?
\item When you are sick, what types of help or treatment do you find most useful?
\item What do you think are the major causes of sickness (for you or others)?
\item What people or qualities do you associate with health?
\end{enumerate}


The Interviews were conducted between October 2009 and January 2010. 

\subsection{Analysis}

The interviews were first coded using an inductive coding procedure. In this, the codes were developed from the interview transcripts themselves, allowing the participants to define the nature of the analysis (in some sense). Following completion of the interviews and first coding, the interviews were then recoded using the entire database of codes which had been developed throughout the analysis procedure.
During this phase redundant codes were also removed. Following the coding procedure, codes weree combined into thematic units where appropriate. For example, the many codes relating to health were combined into an overall health code.

This analysis will look at the material collected and analysed in the following ways:

\begin{enumerate}
\item Analysis on the level of the group – GP's, ordinary people, and alternative therapists. All of these participants seemed to have different perspectives on the matters concerned, and this section will highlight the commonalities and differences between them
\item  Analysis of coding structures, across groups
\item Analysis of themes across the entire sample 
\end{enumerate}

\subsection{Overall theme}

In these interviews, a number of important themes emerged from the data. These are noted here and further developed throughout the test. These major themes were as follows:
\begin{enumerate}
\item Health – internal or externally determined. This theme seemed to stratify respondents into groups. Some respondents (alternative therapists mostly) seemed to look at the determinants of health as being mostly cognitive and emotional, while others (GP's mostly) seemed to look at more external determinants of health, such as pathogens and social status. The ordinary people interviewed in this research project tended to incorporate elements from both of these perspectives.
\item Doctors as prescribers, alternative therapists as facilitators – the alternative therapists tend to talk about facilitating people towards health, while the doctors tend to talk about curing people. 
\item Energy versus biology – doctors tend to talk about biology, alternative therapists tend to talk about energy. 

\end{enumerate}



\subsection{Group Analysis}

\subsubsection{Alternative/Complementary Practitioners}

\paragraph{Participant Information}

Three participants were interviewed as part of this group. One was a shaitsu and Reiki practitioner, one was a spiritual healer and the third was an acupuncturist. They were all recruited from the Cork area following email contact through an alternative therapy website. 

\subsection{Major Themes}
\begin{enumerate}
\item Health is caused by internal rather than external factors
\item Energy model of health – reference to vital forces, meridians and s on
\item They tend to talk about being facilitators rather than healers. They put more of the responsibility on the patient (self-healing model)
\end{enumerate}

\subsection{Coding Analysis and Examples}

\subsubsection{Food as medicine}

This was the most widely utilised code in the entire data-set, appearing over 17 times. This code only appeared in the transcripts of the ordinary people and the alternative therapists, which is why it appears here. 

An example of this code is below:

``in China, they have even the most common person on the street he's a peasant working in the fields are born and and have and inherent knowledge of how to eat in accordance with nature right in their body for a cold condition they know to eat something warm corn soup with ginger or rice casserole, meat and if they've a very very hot condition they know to eat know to eat, you know, cool foods and its very much putting nature into the body looking at external and balancing between the two like I was mentioning a while ago- Alternative Therapist No. 3''

As we can see in from this code, the conception (shared by folk healers across the world) regarding hot and cold foods exists here in his description of health. We can also see how he constructs this as a complement to Western styles of medicine, that it is common knowledge in other, non-traditional societies (where his method is drawn from) but is ignored in our prevalent system of medicine here in the west. 

Another example (from the same participant) follows below:

``I personally do like herbal medicine and I think that its its as close to food as you can get a herb generally grows either above or below ground the roots are usually stronger versions of it so just knowing how herbs work I will use preventative medicines and supplements to protect say if I find working too late head full of information and details, just go for a walk and just try and clear''

Here the participant is talking about what he does when he is sick, and we see how he legitimises herbal medicines by linking them with food, which he has previously described as the best method for promotion of health. 

``good food – healthy food organic food''. - Alternative Therapist No. 1
We can see here (this participant tended to give very short answers to questions) that the emphasis on food is common to the alternative therapists here. We can also note the linkage of health food, organic food as possibly indicating some disquiet with the technocractic nature of herbal medicine. 

\subsection{Linking Body and Mind}

Again, this code only appears in the transcripts of alternative practitioners and students, and is conspicuous by its absence throughout the transcripts of the General Practitioners. 

Some examples of this code follow below. 

`` like I would press certain acupuncture points, work certain energy channels...
I: mmmm
P: Through- which are directly applied on to the body so the focus of the person goes to those places – Alternative Therapist 1''

As one can see from this quote, this participant links physical manipulations of the body as having correlations to mental and emotional states. This is interesting as it is, in one sense, an extension of the idea of a healthy mind in a healthy body. 

``Yeah. Well, in in in again with every organ in the body there's there's associated emotions and there's pluses and minuses with that. So for example, we take an organ like the liver the liver ahhhh If we're totally overworked and ammm ehhh the liver can become very excitable and the energy of the liver it it it its a huge organ its it stores almost all of our blood – Alternative Therapist 3''

From this quote above, we can see a similar process to that observed from the first quote under this code heading. For this participant, emotional awareness can be correlated with the state of particular physical organs. Its interesting in that this is something which would probably never be done in psychology, as emotions would be regarded as being neural events rather than being distributed throughout the body. 

``Same thing with the heart If the heart is healthy, you experience joy 
I:Mmmmm
P:We find things funny, we like people, you know we think the best of every single person we meet 
I:Mmmm
P:When the heart is downtrodden and its there's so many expressions – my heart was broken or my heart sank when I heard that news so (pause) even in the lairs of that – Alternative Therapist 3''

Again, we see a cognate description here, where emotions are embodied in the organs and orificies of the body, rather than dissociated from these in the brain/mind. Its perhaps a radical dissolution of the traditional dualism that pervades Western european thinking. For this respondent the locus of emotion is embodied in the very physical structures that make up the body, rather than being something that sits on top of it, in the psychological or medical conceptions of emotion. 

\subsection{Health as balance}

Again, this code occurred quite frequently, but only in the transcripts of the alternative therapists. Some examples and discussions follow below. 

``So these are the processes in the mind, so again, health in respect to that, to emotions, mental energy very much keeping on a balanced level – Alternative Therapist 3''

We see here that the participants believes balance to be key to all health, on an emotional, physical and mental level. We can also note the focus on energy, which will be examined further below. We can also see here a focus on processes, that these elements are constrantly moving, which is borne out in the quote below. 

``The same thing with the emotions, keeping them moving, keeping them flowing, ammmm – Alternative Therapist 3''
Again, we can see the focus on movement, that emotional expressions should be facilitated, and we can imply conversely that stagnation of emotions is unhealthy. This is interesting in that there is much psychological evidence that suggests that secrets and unexpressed emotions can link into physical health problems. 

``P:Same thing, the imbalance-
I:mmmm
P:Of emotions, imbalance of mental thinking imbalance of sleep not sleep to much sleep ahhh and then what you're putting into the body obviously, and also what you're putting into the body if you're sitting in front of a computer for too long, if you're sitting in front of a TV too long, you're putting too much electromagnetic energy on the mobile phone for too long so generally, ammm so you've cardio-vascular health we need balance there- - Alternative Therapist 3''

We can see from the above quote that while health is conceived of as balance, sickness is thought to result from either too much or too little of a particular substance or process. We can note the focus on both mental and physical imbalance here, this linking of levels of existence is a theme that pervades the entire transcripts, and  will be examined in detail later. 

\subsection{Energy Model of Health}

Another code which occurs throughout the transcripts of the alternative therapists is this notion of \textit{elan vital}, or life force. This is one of those ideas with a strong resonance thoughout human history, as multiple cultures appear to have developed in independently. We'll look at this in more detail using excerpts from the transcripts below. 
``how do I think they work? Well....from doing different treatments....like shiatsu, so which works on the energetic plane – Alternative Therapist 1''

We note here, that when the respondent was asked how some of the treatments used by him worked, he focuses immediately on shiatsu (a form of Japanese massage) which is the treatment he uses most often in his professional practice. We can also see that he constructs this treatment as operating on an energetic plane, which presumably means that it effects the person on another level from the physical. This theme continues throughout the transcripts of the alternative therapists, and appears to be the explanatory model utilised by the majority of them. 

``you actually work energy channels that ammm...remind....the person to put energy in certain places. Yeah, like I would press certain acupuncture points, work certain energy channels...- Alternative Therapist 1''

As we can see from this excerpt, the practitioner seems to link this energy body to the physical body, such that changes in physical pressure on certain points, suggesting that these two parts of a person are interlinked and occupy the same space. 

``it is too easy to say mind body soul but roughly its pretty much that, physical energy, taking the energy from your food and translating that into nutrition into energy and accommodating that and then expelling energy and expelling waste products efficiently that's the physical side of things energy then – Alternative Therapist 3''

However, in this excerpt we have a very different interpretation of energy from this respondent. The energy she describes links up exactly with what we would consider energy to be (the breakdown of food into sugars), so this definition appears to differ quite substantially from the one given by Alternative Therapist 1 above, which suggests some conceptual confusion. Further excerpts will probably make this clear, however. 

``There's emmm very common ahhh which my professor used to say – if ahhh if you want to do exams, if you're studying, eat well. He says the spleen produces your blood your red blood cells – Alternative Therapist 3''

This extract is interesting, in that the energy model is extended here to apply to mental efforts also. We can also see the linkages between physical and mental health here. Again, we see examples of organs in the body being associated with particular states and traits of the mind. 

`` In france they use digestifs, and apertifs, and thats that's the same thing, you know the same as preparing the energetics to assimilate....transform and bring forth the energy from that food very very carefully, and if you do that, then you're going to have Chi and if you've Chi the the character for Chi in Chinese medicine, or sorry in Chinese language ammm the I'll draw for you is is similar to that and basically what it symbolising is a fire, a pot with rice in the pot and steam rising from it so the pot is sitting over the fire cooking the soup – Alternative Therapist 3''

Again here we see the linkage of physical nutrition and energy. This respondent links the physical act of eating to the life force describes as Chi, suggesting that this Chi terminology may be a descriptor for the sugars and other prodcuts needed to keep humans alive. Again, we see the notion that food also gives us this energy, which is seperate from the physical value of food, and that this is the energy which can impact our health and the lack of which causes us to become sick. 



``So again, when you say how do I think it works thats how I think it works as well manipulating energy
I:mmmm
P: ammm, the energy has to be there first day or first place, so someone is very depleted then often I have to make sure I that we get them to a place where they can that I can manipulate energy because if it's not there, I'm only going to deplete them much more – Alternative Therapist No 3''

Here we see a construction of this energy, or Chi in that it can become depleted, and that the task of the therapist is to increase this energy in order to allow their methods to work on the patient. 

``am manipulating pockets of either static or depleted for replete excessively replete some energy in the body and ahhh....
I: that effects change?
P: Yeah, exactly, yeah, yeah. So thats I suppose my limited way of understanding it and the rest then is down to patient cooperation, really, go and do your rest, we've told you enough here – Alternative Therapist 3''

In this excerpt we can see that the problem is constructed as one of imbalance, either there is too much energy, or too little and that the role of the therapist is to balance these energy pockets across the person to ensure that health returns. 

\subsection{Linking Levels}

This code is quite similar to the Linking Body and Mind code, but there are some important differences which we will explore below. 

``trouble with knees, have often ehhhh, a [pause] a relationship to, to to, the what direction you take in life – Alternative Therapist 1''

``your hands 
I: mmmm
P: what have you, how you handle things – Alternative Therapist 1''

``You know like your digestive system often is affected of how you digest life – Alternative Therapist''


We can see from these extracts that the respondent is linking problems of a physical nature to mental and emotional states. This excerpt tends to use argument by analogy which is not usually accepted as proof, but it does provide some insight into the construction of health by this participant. These quotes imply that the entire person is considered as one entity, and that any change in emotions, mental and physical issues will have an impact on all of the other areas. 

``breathe our emotions – Alternative Therapist 3''

``Breathing emotions and leaving them out – Alternative Therapist 3''

We see here again evidence of an extremely holistic approach to the person, where emotional expression can be linked to breathing. It is perhaps interesting in this sense that an old word for life energy was prana, which was also the word for breath, A similar observation can be made about the Greek pneuma, which means both spirit and breath. The Hebrews also used the same word for breath as for emotions (Ruach), all of which points to the commonality of these ideas throughout history. 
``probably equally important I presume it seems like kindof one feeds off the other if you're physically sick, chances are you're not going to be too well mentally, or if you're mentally you're probably not going to be too well physically – Alternative Therapist 3''

Here we can see a recognition of the interdependence of the differing parts of the human organism and the idea that one of them can throw the others off their functioning. 

\subsection{Emotional Health}

This code was predominantly used by the alternative therapists, but also appeared a number of times in one of the GP transcripts. We'll look at and discuss some of the examples from the alternative therapists section below. 

``P: Ammm...so....a non-stressful lifestyle you know? And a ...fun, laughter, that you do things that you like to do. - Alternative Therapist 1''

``o its more preventative, in its focus more and its all about balance being in
P: that's all the physical side
I: oh?
P: and then I try to definitely you know, meditate, 
I: mmmm, mmmm
P: and ehhhh, focus on my emotions and my thoughts – Alternative Therapist 1''

We can see from these quotes that for many of the alternative therapists, emotional well being is an important factor in their construction of health. They tend to construct health as being situated in a context whereby emotional well being is extremely important and that it forms an integral part of health. This links in with the research suggesting that optimism is associated with health, something which will be discussed later and which appears to be a construction made by almost all of the participants in this study. 

``P: Most causes are the emotional-
I:mmmm, mmmm-
P:and so the body is really reacting and that
I:mmmm, mmmm
P: So thats – thats why we say mind and body, you know that kind of a balance – Alternative Therapist 2''

For this respondent, the emotions are the primary driver of sickness or health. She seems to construct health as being something primarily determined by one's emotional outlook, rather than resulting from pathogens or the environment. Its interesting that this is exact opposite opinion offered by the GP's who were included in this study. 

``emotional energy has to be very well balanced we have to breathe our emotions, we cannot suppress them am they have to (pause) be (pause) tuned in and out of the body very carefully. - Alternative Therapist 3''

As we can see here, the construction of the importance of emotions runs throughout all of the transcripts of the alternative therapists. This respondent links the emotional expression with the breathing, that they have to be taken in and let out very regularly. He seems to construct them as something which can have great physical impact, and should be treated with respect.


``ahhh it it if used in itself it may not answer the whole problem because people come not only with a condition, they come with with a condition attached to their own bodies its got feelings, its got thoughts preconceptions, its got worries, etcetera- - Doctor 2''

Although this section is dealing with the transcripts of the alternative therapists, this quote is included here to gain a different perspective on emotional health. We can see from this quote that this respondent possesses a radically different construction of emotions and their relationship to health. For him, the body is the primary part, and he constructs this using depersonalising language (it) the body is the important part, but it has all of these messy emotions and thoughts attached to it which complicate matters. The contrasting quotes here nicely illustrate the differences in construction between the alternative therapists and the general practitioners. 


\subsection{Personal Effort}

Again, this code appeared mostly in the transcripts of the alternative therapists, although it also appeared in the transcripts of one of the doctors. 

``reflex points, If a person is not willing to work on themselves and to get better, you can do whatever you want, and they- Alternative Therapist 1''

``I: so it requires their willingness to take part in the procedure aswell - Alternative Therapist 1''

These quotes serve to illustrate a point made again and again by the alternative therapists and to a lesser extent by doctors. That point is that people are ultimately responsible for their own health, regardless of what help they get from a healthcare professional. 

``Yeah, but again it will need a change in lifestyle, change in diet, a willingness for the person to actually change – Alternative Therapist 1''

Here we can see a restatement of the points made above, that while the practitioner can give advice, the person must ultimately act upon it, and if they are not willing, then they will not change. The truly cynical could claim that this is because all of these therapies are nothing more than placebo, and this abdication of responsibility is nothing more than a device to conceal their lack of efficacious treatments. But we are not quite so cynical. 

``Yeah, I don't fix people, they have to fix themselves then with support and inspiration from me – Alternative Therapist 2''

This is perhaps the most telling quote in this section. This respondent constructs patients as being totally responsible for their own health, that he merely acts as a facilitator and allows them to heal themselves. This is a very empowering view of the healing process, and contrasts quite strongly with the more paternal constructions of the doctors. 

``how much they'll try to put practices that could be suggested in the clinic into into their lifestyle ammm, you can also I think tell if if they're \ldots you know you point something out to someone, that they can make this subtle  change that will make a huge difference to their bowel habits  
I:mmmm
P:Suggest something for them to eat-
I:mmmm
P: almost on hearing it, I could there could almost be a placebo effect would not be surprised when they come back in two weeks and tell me that their bowel habits have completely changed but I would often almost be able to tell with a patient who's going to do that or who's going to be...... a little bit more open to that or ammm.......how else do I think it works. I I I think a lot of the time with any form of healing I mean, - Alternative Therapist 3''

Again, we have a construction here that points out the differences between people's response to their health. Some other respondents suggest that health is not valued by many, and this respondent argues that some patients will do the work required to get better, while others will not, and that it is this factor which accounts for broadly divergent outcomes. 

``because people think that everything going right comes without any effort and in actual fact, to make things right requires huge effort – Doctor 1''

We have this quote from one of the GP's here, and it supports what the therapists have been saying. One can say that health is being constructed as something which needs to be valued by people if they are to enjoy good health. 

\subsection{Body Knowledge of Health}

Again, this code appeared only in the transcripts of the alternative therapists, and only in two of three of those. Below are some relevant examples with some commentaries on their constructions of health. 

``Because you should more talk to your own wisdom but if you really sick of course you have to go to the doctor – Alternative Therapist 2''

This is an interesting quote, in that it seems to construct an awareness on the part of the person as to the causes and nature of their sickness. The body appears imbued with understanding of the problems facing the person, and this can be consulted. It seems however, that this is only a first resort, as the quote continues to state that serious problems should be dealt with by a doctor.

``By paying attention to what is (pause) going on with yourself. We've simple basic needs. We need to sleep and we need to eat. Basically, those are the two most important we also need to breathe. Otherwise we can – we can name ammm – Alternative Therapist 3''

Here again there is a focus on intenal knowledge. This time, however, it appears constructed in terms of awareness of the body's needs at a more basic level, that of sleeping and eating. This constrasts with the unspecified wisdom referenced in the first quote. 


``P:And to reassure them you know that, everything is OK with them in in some fashion as well and its sustained by (pause) listening to yourself, and listening to nature – Alternative Therapist 3''

Here, from the same respondent we have an expansion of the previous statement. Where first listening to yourself was important, now we have a focus on listening to nature also. While we can construct this as listening to the natural world, this construction could prove quite problematic as there are many different `natures', and this quote does not allow us to distinguish between them. 

\subsection{Environmental Factors}

This next code is interesting, in that it appears  with about the same regularity in the transcripts of both doctors and alternative therapists, which does not occur particularly often throughout the interviews. 

The first example comes from the transcript of Alternative Therapist 1, where they refer to ``a healthy non-toxic environment'' as a major determinant of health. ``And ehhh, gives eh – a safe environment that the person feels safe and releaxed''. We can see here that the conception of the environment here is more of a reference to the sensations and feelings aroused by the environment. 

\section{Construction of Doctors}
\label{sec:construction-doctors}

One of the major themes that emerged from all of the interviews was the nature of doctors, even apart from a more general conception of health practitioners, all the participants focused on the role of doctors in society, and their positive and negative impacts. 

One of the ordinary participants focused on the relationship between Doctors and Alternative therapists, with this excerpt ``I think if you have like serious [pause] health problem, you will need both''. This stresses and indeed this participant stressed throughout the interview, that doctors and alternative therapists should be viewed as complementary rather than opposing. 

\begin{quotation}
  she mentioned this to her GP who (pause) became very angry and asked me to phone him immediately 
I:Mmmm
P:and this type of thing doesn't happen very often. I rang him, and he was quite set, saying I dont want this woman to have more and more treatments, you know and I want I want her to find the right treatment. And I said, so do I, and this is it, and this is what I deem most good for her at the moment and he asked about the treatment and I tried to explain a little – in simple language – and the end of the conversation he was Ok, let's do this. 

\end{quotation}

This quotation came from Alternative therapist 3, and describes the relationship between doctors and alternative therapists. Note that he constructs the two as in a collaborative relationship rather than competing, and shows that they were able to find a good balance between bothe of their particular forms of treatment. 

It is interesting in that one of the doctors focused on their role as agents of social change, particularly in this excerpt

\begin{quotation}
   I think as doctors we have a job maybe as advocates you know to point out the issues, and point out the we may not be have the solutions but we should be able to sortof say these are part of the problems – these are the problems, and these are some of the determinants and really these need sorting out 

\end{quotation}

This quotation focuses on the role of doctors in society rather than their role as a individual health practitioner, making the point that they have a responsibility to their patients to focus on the problems that are actually affecting their abilities to live healthy lives. 

Interestingly, many of the alternative practitioners did not subscribe to this viewpoint of doctors, instead regarding them as prescribers 
\begin{quotation}
  Ehhh, a lot of doctors just ah rely on on on on eh their pharmacuetical companies 

\end{quotation}


The implication here seems to be that the commercial relationships between doctors and pharmacuetical companies get in the way of healing. 

However, this point was actually raised by one of the doctors themselves, in a slightly different context: 

\begin{quotation}
  I think doctors actually should have a ethical obligation to say enough is enough you know and I get actually quite frustrated when I get sent a patient that I actually feel has nohting wrong with them 

\end{quotation}

Here we can see a doctor's frustration at the way in which they are expected to dispense medicines to patients even when they feel there is nothing wrong with them. It is worth noting that the doctors frame this as the patients demands, while to the alternative therapists, this is a relationship in which the doctors have power over the patients. 

The second GP also expressed frustration with this state of affairs, saying that 
\begin{quotation}
  even though you try to spend as much time as you can in terms of health prevention such as immunisation or health promotion by giving advice to people on healthy diets exercise giving up smoking etcetera we do tend to spend most of our time reaching for the pen, prescribing 

\end{quotation}

The sense is that they would like to be able to focus more on the health promotion effects that would actually change the persons state of health, but are corraled by the system into prescribing, as that is something that can be done within the confines of the 30 minutes doctors appointment. 

Also, one of the students expressed some annoyance with elements of the system: 
\begin{quotation}
  well as I say like, just that kind of stuff ammm like plenty of times i've been sick and gone to a doctor and he's given antibiotic and i've taken it and gone thats not helping me at all 

\end{quotation}

The notion is that antibiotics are a default option for many doctors and patients, and for some of the respondents, this is not actually particularly useful. 

\begin{quotation}
not unless you're actually particularly sick like, I think that an awful lot of the time people tend to be a small bit sick and go off to a doctor and get their antibiotics and go this didnt help at all, whereas there was probably no need to go to a doctor in the first place 

\end{quotation}

The quotation above referenced another students answer to the question how often do you go to the doctor. Note that given the age (early twenties) and social status (student) of this respondent this is perhaps not a surprising response.  The theme that comes through here is that doctors can be associated with unnecessary treatments and methods, and that this is a conception shared by all three groups in the sample. 


Another theme that emerged around doctors was that doctors focused more on illness. 

The extract (from Doctor 1) illustrates the point: 

\begin{quotation}
  I suppose sometimes health and ill health, you know, there's not different sides of the same coin
I:mmmm
P: but I suppose we're more set up in our training to deal with ill-health as a concept
I:yeah
P:- than health and particularly how the health services are structured 

\end{quotation}

The participant notes that ill-health gets far more attention than does health, perhaps linking back to the idea that ill-health is a breaking down of something that is normal (i.e. health). 

Another quote from the other Doctor (No 2) supports this reading:

\begin{quotation}
  we'd only get so far, and I think that we essentyially would be more into the disease ahhh ammm diagnosis and cure-

\end{quotation}

It is interesting to note that the doctor first says that they are more into disease, but then shifts their words to diagnosis and care, displaying the focus on ill-health referenced above. 


Perhaps an interesting theme which emerged from the interviews with ordinary people and alternative therapists was that doctors were not that important. The interesting point about this is that by making the point, they were in fact reinforcing the notion of doctors as important. 

\begin{quotation}
  So just to have a regular checkup and stuff checkup and ah, well everything was fine but yeah I don't really know so much about doctors. I think doctors are great if you need them but if you dont need them you shouldn't go to them.  -Alternative Therapist 1

\end{quotation} 

The respondent here proudly proclaims their ignorance of doctors, and introduces some tautologies in that doctors are only necessary if you ``need'' them, but does not define what exactly it is to need doctors. 

\begin{quotation}
  not unless you're actually particularly sick like, I think that an awful lot of the time people tend to be a small bit sick and go off to a doctor and get their antibiotics and go this didnt help at all, whereas there was probably no need to go to a doctor in the first place 

\end{quotation}

Here the respondent focuses on doctors only being important when you are really sick, and argues that people tend to go to doctors for issues which they would not consider important enough. 

The next theme around doctors came from Alternative Therapist 3, where they recall an incident where doctors came to see them. 

\begin{quotation}
  initially came for pain a third came for tiredness very interestingly they all asked to come after-hours 
I:Because they didnt want to be seen
P: By their patients, they know my system and that I do one patient per hour and I dont do two or three rooms like a lot of people do and they were fine with that ammm (pause) one a patient (pause) didnt want necessarily anything other than the main complaint which I think was stress ammm but I asked could I treat for other matters 

\end{quotation}

The interesting part here is that he constructs this incident as something out of the ordinary (whereas the converse is broadly accepted by all participants) and frames it as a series of differences between the doctor's typical practice versus that of the alternative practitioner. Note also that the doctor only wanted treatment for the main presenting symptom, rather than a more general process which the alternative therapist would have preferred. 

Another intriuging theme that came through from one of the doctors was the following: 

\begin{quotation}
  the herbalists, who have almost taken on a bio-medical model then a bio-medical sortof way of acting they see patients they diagnose they have a pharmacy of herbs-

\end{quotation}

The participant was responding to a question around whether or not they had any faith in alternative medicines and treatments. Note how the ``good'' alternative therapists are defined by their adherence to the bio-medical model, and note that the other salient features are diagnosis and pharmacies. 

However, note the contrast here with how one of the alternative therapists describes their relationship with doctors. 

\begin{quotation}
  I've become much more in happier terms today with it because of with the fertility side of things I need to rely on blood results a lot. I need people to go and have their hormones checked and and something like sperm analysis, I need, I need medical science completely for this these reasons.
- Alternative Therapist 3
\end{quotation}

This respondent regards their relationship with medical science as non-adversarial, but rather complementary. The contrast with the attitude of the doctor is quite striking. 


\begin{quotation}
  they just prescribe medication that has a lot of side effects and they don't really spend much time with you and you go to the doctor there's very few doctors which actually spend time and actually talk to people and really see where the problem comes from-
Alternative Therapist 1
\end{quotation}

Here, the conception of doctors is as seen earlier, distant prescrivers who do not actually spend time with a person. The implied contrast with their own practice is apparent. 

Even some of the Doctor (No 1) respondents argue in this point:

\begin{quotation}
  I would disagree with what my colleagues might do or i've occasionally been shocked ammm by what I would perceive to be a lack of care,

\end{quotation}

Here it is constructed more in terms of professional disagreement, and the occassional problem, whereas the alternative therapists are more fortright about this (but less so about the problems with their own profession). 

And from one of the Doctor participants: 

\begin{quotation}
  ammm, a bit, I'm not trained in them but obviously I prescribe St John's wort

\end{quotation}

Note the unassuming tone, a bit, expressions of ignorance but the use of obviously to describe the practice of prescribing St John's wort, as though it were the most obvious thing in the world. 
%%% Local Variables: 
%%% mode: latex
%%% TeX-master: "PlaceboMeasurementByMultipleMethods"
%%% End: 


\bibliography{placebo2,IAT2,statistics,healthoptmind2,probabilities2,TCQ}
\end{document}




%%% Local Variables:
%%% mode: latex
%%% TeX-master: "PlaceboMeasurementByMultipleMethods"
%%% End:
