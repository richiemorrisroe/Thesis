
\section{Introduction to this review}
\label{sec:intr-this-revi}

This review has three sections.  Firstly, the placebo effect is reviewed in greater detail than in the previous chapter, with a particular focus on the measurement of placebo expectancies and other factors which are thought to influence this effect.

Next, the implicit measure used in this research is reviewed (the Implicit Association Test)~\cite{Greenwald1998} with a particular focus on its psychometric properties and its relationship to explicit measures of the same construct. 

Finally, the evidence which links these two areas together is reviewed, and the major aims of this thesis are defined more completely. 


\section{The Concept of the Placebo}
\label{sec:concept-placebo}

The placebo effect has a long history in medicine, and some researchers believe that almost every treatment developed before the 20th Century may have relied primarily on this effect for their curative properties~\cite{Shapiro1997,Macedo2003}. Despite this, the concept is not particularly well defined, and different researchers use the same phrase to mean different things~\cite{Ernst1995b,hrobjartsson1996uncontrollable}. This section will examine some of these definitions, and look for markers towards which  a more comprehensive definition of the effect can be outlined.  

\subsection{History of the Concept} 
\label{sec:history-concept}

Placebos have a long and colourful past, some elements of which are still relevant today. The word itself is taken from the Latin for "I will please" and referred originally to the cries of paid mourners at funeral ceremonies \cite{Macedo2003}. Over time, it became associated with any medicine which was given more to please a patient than to actually relieve the symptoms \cite{Kaptchuk1998}. Placebos were considered ethically acceptable, if somewhat dubious, for many years. However, as more effective and tailored treatments were developed, the use of placebos declined \cite{Macedo2003}. Following the development of randomised double-blind trials, the  placebo increased in importance once more \cite{Kaptchuk1998}, as it became a required part of the process for establishing the efficacy of all new drugs.  

The first serious attempts at studying of the placebo as an effect in itself  were begun by Beecher's article ``The Powerful Placebo'' which argued that placebos were useful in many differing kinds of ailments and diseases, and that their effectiveness did not depend on the intelligence of the patient, as had previously been supposed~\cite{beecher1955powerful}.  This article aroused interest in the placebo, and is generally regarded as the first paper which focused on the effect in its own right, rather than as a device to disguise a doctor's lack of effective treatments~\cite{beecher1955powerful,Kaptchuk1998}. 

Beecher made many strong claims in his 1955 article, including an assertion that placebos were effective 35\% of the time. Later research has demonstrated that many of these claims around the efficacy of placebo did not account for all potential confounding factors~\cite{Kienle1998}. These researchers re-analysed the studies reported on by Beecher, and found that much of the reported improvement could have been due to natural history of the disease and regression to the mean. Additionally, later research has shown that placebo effects can be much more variable than 35\%~\cite{Turner1994}, especially when the median rather than the mean placebo response is considered \cite{McQuay1996}.   

Much of the lack of clarity found within the field~\cite{Macedo2003,Kaptchuk1998} results from the use of placebo in two contradictory situations. In the first, that of the randomised controlled trial (RCT)\footnote{typically consisting of an experimental and control group with a double-blind design}, placebos are a control for all effects of treatment not related to the substance or procedure under test~\cite{Vickers2000}. In this setting, the aim is that they should be minimised so as the impact of the active treatment can be assessed.  

In the second setting, that  of clinical practice, the placebo is imbued with all the authority of medicine and utilised in order to effect changes that may result from mindset or to placate a troublesome patient~\cite{Bootzin2003,Sherman2008}. Macedo and Kaptchuk argue that both of these approaches give too much power to the concept and contribute to the confusion surrounding its definition. 

Another confusing factor within placebo research results from the terms ``specific'' and ``non-specific'' effects, introduced by Grunbaum in 1981~\cite{grunbaum1981placebo}. These particularly confusing terms have lead to much agonising and debate over the years. Some have even suggested that these terms should be abandoned~\cite{Caspi2002}. 

The specific parts of a treatment are typically defined as the biologically or theoretically effective agent, while the non-specific factors are all other parts of the treatment. Price and Benedetti argue that there are three sources of non-specific effects --- the patient, the provider and the relationship between them~\cite{Price2008}, a distinction also made by other authors~\cite{Finniss2005,DiBlasi2002}. It is important to note that more of the conceptual confusion arises from the differences between a placebo and a placebo effect. In brief, the placebo is the device or treatment given to a person, while the placebo effect is the outcome of this intervention. The term placebo response is often used to describe the impact of such a treatment on a particular participant in a clinical or experimental study, as described in Section~\ref{sec:placebo}. 

% The non specific and inert criteria are problematic; given that placebos can exert extremely specific changes \cite{Amanzio2001,Caspi2002}, they are not non-specific effects, and to the extent that they do not exert changes in outcomes, then the concept is irrelevant \cite{Moerman2003,Barrett2006}.  

\subsection{Placebos in Randomised Trials}
\label{sec:plac-rand-trials}

Placebos are most well known for their use in a clinical trial setting. % The FDA (Food and Drug Administration) in the United States of America  requires all drugs to demonstrate usefulness over and above placebo in order to be licensed. This followed a number of scandals in the late nineteen fifties and early nineteen sixties where treatments which had been in use for many years showed little or no effect under double-blind conditions \cite{Moerman2000a}. 
In a typical randomised trial, neither patients nor physicians know whether a particular person receives drug or placebo.  If there is any significant changes in the placebo group (with respect to the primary outcome measure of the study), even in the absence of medication, this is called a placebo effect, and the inert procedure or pill used is a placebo.\footnote{Often, if the effect is negative in terms of treatment outcome, the response may be called a nocebo response} More technically, any mean improvement in the control group can be classified as a placebo effect. However this definition is not entirely accurate, as will be seen below.   


The problem with this definition is that the effects in a placebo arm of a clinical trial result from a combination of the placebo effect and other factors, such as regression to the mean~\cite{Morton2003}, demand characteristics of clinical trials~\cite{hrobjartsson2001,weber1972subject} and other factors such as the natural history of the disease. Regression to the mean is the tendency for an extreme score measured at time 1 to be closer to the mean at time 2. This tendency is a property of all measuring tools which are not perfectly reliable \cite{Morton2003}

Demand characteristics \cite{Fernandez1994,weber1972subject} refer to the tendency for participants in research to give the answer which they believe the researcher wishes to hear. This can result in an over-exaggeration of symptoms at the first assessment and a minimising tendency for the same symptoms at the end of the study \cite{Vase2005}.  

% , and can be controlled for by sampling from the general population, as using participants with high scores on the outcome variable to be measured tends to exacerbate the effect. 

% Sampling from the general population is not practical for many trials of new medicines, given the exclusion and inclusion criteria for clinical trials. These criteria normally require that participants in trials suffer from the condition, and have no other effective treatment \cite{Daugherty2008}.   



% Another factor which can effect the results of a clinical trial is unidentified parallel interventions \cite{Ernst1995b}. These occur when participants in a clinical trial change other factors as the result of being in the trial, thus biasing the results. One example of this could be if participants in a clinical trial for hypertension reduce their salt intake (potentially as a result of the increased salience of the risks associated with hypertension caused by the process of enrollment in the trial, where typically large amounts of information are provided regarding the condition under treatment). 

Without a no-treatment control group the effects of these confounders cannot be separated from the true placebo response.  The no treatment group serves this purpose as factors such as regression to the mean and natural history should apply equally to both the placebo and the no-treatment group.  The definition of the placebo typically used is:
\begin{quotation}
 the placebo response is the response to treatment in the placebo group less the response to treatment in the no-treatment group.  
\end{quotation}

A more precise definition in the context of clinical trials is given by Knipschild et al~\cite{Knipschild2005}. 

\begin{quote}
   the placebo effect \ldots [is] \ldots the difference in effect between the placebo group and the spontaneous course in a randomised clinical trial 
\end{quote}



This definition relies upon an understanding of the spontaneous course of an illness in a controlled trial, which can be operationalised as the progress of the no-treatment control group. It is somewhat limiting, but is clearly operationalised for a particular setting. % However, it does require the assumption that enrollment in a trial does not affect the spontaneous course of an illness, and this assumption is open to debate \cite{Moerman2000}.

\subsection{Problems with the clinical definitions of placebo}
\label{sec:plac-cogn-perf}

Within the confines of the clinical trial, the two definitions above are workable definitions of the placebo effect. The important part of a clinical trial is the test of the active medication, and the placebo is important only insofar as it relates to the testing of this medication. 

However, clinical trials are not the only context where placebos are administered, and in other situations these definitions run into problems. Consider some recent work of Oken et al \cite{Oken2008}. In this experiment, healthy seniors were administered placebos which they were told would improve cognitive performance. There were two active groups (given different instructions) and one no-treatment group. The participants were tested for cognitive abilities at the beginning, middle and end of the placebo treatment.  The seniors given the placebo pills showed significant increases in cognitive ability over the course of the study, and many were disappointed when debriefed and were told that they had been given placebos.

While the definitions given above can, at a stretch, account for these results, it does indicate a need to more carefully define the concept of placebo. Whether or not one accepts the clinical trial definition depends crucially on ones definition of treatment. 

For many, this is some device or procedure that restores the organism to optimal function. Alternatively, this view could be described as believing that treatment restores homoeostasis to the organism. Indeed, the Oxford English Dictionary agrees with this definition describing treatment as \textit{``medical care for an illness or injury''}~\cite{dictionary20101989}.

% In the Oken study above however, this was not the case, as the researchers were examining the positive effects of placebos, rather than attempting to cure a deficit. Treatment could be defined as something which improves the performance or health of an organism, and such a definition would not encounter these problems.

It is relatively easy to see how placebos can reduce pain, but is more difficult to see how such pills can improve cognitive performance. One can argue that decreases in pain result from a response bias \cite{Allan2002}, but the measurements of cognitive performance in the Oken et al study were not subject to these kinds of bias. 

%  One can argue that the expectancy of the participants for improvement led them to actually improve, but this argument begs the question as we then need to define how expectancies exert such effects. 

% A more plausible explanation for these findings is stereotype threat \cite{schmader2003converging,spencerclaude1999stereotype}. Stereotype threat occurs when a member of a particular group performs badly as the result of their worry surrounding being judged badly for their performance. 

% This phenomenon could have accounted for the results observed in the Oken et al study.  It may be that the belief that the pills were enabling improved performance compensated for the effects of stereotype threat. An interesting experiment would be to investigate the effects of placebo cognitive pill administration on the performance of women and African Americans in more traditional academic environments.  

Another study where placebo effects were demonstrated in a non clinical setting was that of Crum \cite{Crum2007}. In this study attendants in hotels were cluster randomised (using the hotel as the unit of sampling) and half of the attendants were informed of how many calories they burned by engaging in their roles as hotel attendants. One month later, the informed group had lost significantly more weight and had improved on both self report and externally measured dimensions related to weight and health. Again, this effect is difficult to conceptualise in terms of treatment.  Crum, in this paper defined the placebo effect as

\begin{quotation}
  any effect of treatment which is mediated more by the participants
  beliefs and expectancies rather than the physiological actions of
  the treatment.
\end{quotation}

This definition is more widely applicable than our first definition above, but at the cost of introducing two new terms which are not clearly defined, namely beliefs and expectancies. While the second of these terms has a number of specific meanings in psychological thought (for example, process, outcome and response expectancies) \cite{Bandura1977,Kirsch1985}, belief has not been so precisely defined. This definition also suffers from the issues surrounding the definition of treatment that were noted above, in Section \ref{sec:plac-rand-trials}. 



Another definition (which avoids the treatment definition problem) was quoted from Ross \& Olson, 1981 by Flaten and colleagues \cite{Flaten1999}
This definition is as follows:

\begin{quotation}
  A placebo or nocebo may be defined as an inactive
substance or a procedure that is administered with
suggestions that it will modify a symptom or sensation
\end{quotation}

This definition avoids the problem of defining treatment, and also avoids the use of the terms beliefs and expectancies. However, it can be seen from the definition that this is achieved only at the cost of introducing suggestion as a possible mediator of the placebo effect. On the face of it, this is perhaps not a term without merit. In studies of hypnosis, suggestion is commonly regarded as the driver of the observed effects~\cite{Kirsch1994} and hypnosis has been proposed as an ethical method to induce placebo responses in participating individuals~\cite{Raz2007a}. This definition is perhaps the best of those that have been examined so far, but it does require us to limit ourselves in the study of placebo to symptoms and sensations. Both the Crum \& Langer study and the Oken et al study indicate that placebos have a wider sphere of effect. 


Another definition of the placebo effects,  proposed by Price et al \cite{Price2008} is that

\begin{quotation}
  a placebo is any substance or procedure that simulates a treatment.
\end{quotation}

This definition fails to resolve our problem discussed above with relation to cognitive abilities. 
Kienle \& Keine give two definitions of placebo in their critique of Beecher:

\begin{quotation}
Placebo is defined in two separate ways, firstly as the imitation of a
therapy, and secondly as any self-healing effect. 
\end{quotation}

The first definition of imitation of a therapy is extremely similar to the Price definition quoted above, and the second, while it sounds plausible is far too vague to be of any use in research.

Another definition~\cite{DiBlasi2001}, does resolve the problems encountered with the Price \& Benedetti  definitions above. This definition is that

\begin{quotation}
  placebos are inert substances that have an effect due to context
\end{quotation}

This definition has some good points, in that it allows for placebo effects in any areas in which they are found, it allows for patient-provider effects and it does not prejudge the causes of the effect. 

One major difficulty which can be observed with this definition is that it does not account for active placebos, where a substance which is a medication in one context is administered for a condition where it is not expected to have any pharmacological effect~\cite{Kirsch1998} . 

These active placebos would contradict the definition of placebo as inert, and yet the research demonstrates that these active placebos can be as effective as the regular inert pill \cite{Flaten2004} and sometimes more effective \cite{Kirsch2002a}. 

% Another issue with this definition is that to the extent which a placebo induces specific changes, it is not inert, and therefore, using this definition would cease to be a placebo, which is clearly nonsensical \cite{Moerman2002b}.

A similar definition (in the context of clinical trials) is given by Knipschild et al \cite{Knipschild2005} who say that the placebo effect

\begin{quotation}
... is the effect of co-interventions in a
treatment study connected to the doctor--patient relationship.
\end{quotation}

Again, this definition is quite precisely operationalised, but it assumes that the active ingredient of placebo is the relationship between provider and patient, and this has not been demonstrated to be true as some research into minimising contact between providers and patients has still demonstrated a placebo effect \cite{Hyland2006}, indicating that the patient-provider relationship cannot account for the entirety of the effect.  



\subsection{Definition for this thesis}
\label{sec:defin-this-thes}


Perhaps the most useful definition for the purposes of this thesis is the definition of Di Blasi et al \cite{DiBlasi2001}, if we take account of the issue arising from active placebos. The definition would work better if we removed the word 'inert' and replaced with 'believed inert for the specific condition concerned', so that it reads

\begin{quotation}
  a placebo is a treatment believed inert for the specific condition
  concerned which has an effect due to context.
\end{quotation}

This definition would need to be supplemented with a more precise definition of context. One attempt at this would be that context is

\begin{quotation}
   the internal states, external environment and relationship of the individual to these states, environment and other individuals in their presence.
\end{quotation}





\section{Theories of Placebo Effects}
\label{sec:theor-plac-effects}

Following the discussion of the definitions of placebo and some problems arising from the different contexts within which it has been used and defined, the next step in this review is to examine the theories which have been proposed to account for this phenomenon. 

There are a few major theories, each of these will be described in turn  and examined for the ways in which they account for the effects, and those features of the effect which they fail to explain (or reject as being illegitimate). A useful run-down of all of these theories appeared in the past few years~\cite{Stewart-Williams2004b} and that review has helped to inform the arguments presented here. 

\subsection{Conditioning}
\label{sec:conditioning}

The first major theory which attempted to account for placebo effects was that of conditioning. Building off the demonstration of placebo effects in non-human mammals \cite{Herrnstein1962} the conditioning theory argued that placebo effects resulted from the learned association between a contingency in the environment (the doctor, pill or medical setting) and healing. This contingency lead to the activation of healing mechanisms based on previous experience with the pill~\cite{Vodouris1989,Voudouris1985}.

The conditioning theory has a number of advantages. Firstly, it can account for placebo effects in all mammals, as all seem to be capable of learning through reinforcement. Secondly, it is parsimonious, as it allows us to explain the placebo phenomenon without invoking any new processes or mechanisms. Thirdly, it appears to account for much of the observed effects. 

However, a major problem for the conditioning theory is that it cannot account for placebo effects from a product which a participant  has not experienced before. Given the nature of clinical trials, this rules out   conditioning as a sole explanation for all placebo effects.  One could assume that the observed placebo effects in clinical trials result from generalised associations with medical treatments more generally \cite{pearce1987model}. This does retain aspects of the theory, and provides a testable hypothesis, which is the following - to the extent that there is commonalities between the learning environment and the clinical trial environment, placebo responses will be observed.  

The conditioning theory rests on the demonstration that if pain is induced by means of a particular method, and the level of pain is surreptitiously reduced while the placebo is given, then participants typically show much stronger placebo responses \cite{Voudouris1985,Colloca2006}. Indeed, even though the conditioning theory has fallen from favour within the field, this methodology is still used in many cases to increase the size of observed placebo effects. 


There do appear to be some placebo responses which are totally mediated by conditioning (such as hormone secretions)~\cite{Amanzio1999}, but not all of them are~\cite{benedetti2003a}. Experimental research has elucidated some of the connections here, in that motor movement in participants with Parkinson's disease and pain can be modulated by expectancies while changes in hormone secretion appear to be modulated by conditioning exclusively~\cite{benedetti2003a}.

Strangely enough, even though conditioning appears to induce stronger placebo responses than does expectancies, nocebo suggestions can completely reverse the effects of positive placebo conditioning~\cite{Benedetti2008}\footnote{Possibly this occurs because of the impact of salience asymmetry}. Additionally, in one study, conditioned participants showed a greater placebo response when given neutral, rather than positive instructions~\cite{Klinger2007a}. 

Some authors have argued~\cite{Stewart-Williams2004b} that conditioning is not a theory of what causes placebo effects, but rather a mechanism through which other variables, such as expectancies, exert their influence. This is an interesting idea, and has some merit, but it does suggest that all placebo effects are mediated by expectancies, and this does not match up with other research showing that dependent on the context of the study, other variables may be significantly related while expectancies are not~\cite{Geers2005a,Hyland2006}.  

\subsection{Expectancies}
\label{sec:expectancies}

The competing theory to conditioning for the past few decades was the expectancy theory, as proposed by Kirsch \cite{Kirsch1985}. Kirsch coined the term ``response expectancy'' to describe what he called ``the expectation of a non-volitional response''. In a ten year review~\cite{Kirsch1997} suggests that this theory has applications in hypnosis and placebo effects. Recent research has shown that expectancies can also modulate sensory experience  \cite{Sterzer2008}. 

This theory competed with the conditioning theory for over a decade, but the issue was mostly resolved by a 1997 paper \cite{Montgomery1997}, which pitted the expectancy and conditioning explanations against one another. This study used the conditioning manipulation devised by Voudouris \cite{Voudouris1985} where the painful stimulus is reduced after application of a placebo cream to increase the size of the placebo effect. 

One group was told of the pain reduction, while the other was not. The group who were told showed no enhanced placebo response, which supported the expectancy theory. A multiple regression also carried out as part of the study indicated that the effects of conditioning were completely mediated by expectancies. 

This seemed to be convincing evidence in favour of the expectancy theory. However, it is worth noting that some authors \cite{Stewart-Williams2004a}  argue that conditioning is a mechanism, not a theory, and they claim that conditioning is one method through which expectancies are formed. This theory does not appear to be plausible given the existence of placebo responses which have only been demonstrated with conditioning mechanisms \cite{benedetti2003a}.   

\subsubsection{Expectancy Manipulations}
\label{sec:suggestion}

Suggestion (expectancy manipulation) is a feature which while prominent in explanations of hypnotic phenomena, is often neglected in studies of the placebo. This is despite the fact that often the placebo phenomenon is brought about by suggesting to participants that they have received an effective treatment. These suggestions are typically framed as expectancy manipulations; e.g. ``This treatment is a potent painkiller which will take effect quickly''. Recently, Kirsch proposed that placebos could be fruitfully considered in terms of suggestion rather than expectancies  \cite{Kirsch1999}. This viewpoint seems illuminating, as there are large differences in the size of placebo effects depending on the type of expectancy manipulation used. 

This line of research began with Kirsch \cite{kirsch1988double} when he looked at the effects of either telling participants that they would receive coffee, or that they might receive coffee. This clever design  mirrors the difference between placebo studies and double blind trials. This experimental study found that when coffee was deceptively administered, there was a much larger effect. Additionally, the physiological parameters measured showed effects in the opposite direction between these two conditions.  

This finding has been replicated in a clinical setting using analgesics following surgery, showing that deceptive administration produced larger effects than did double-blind administration of the same placebo \cite{Amanzio2001}, though another author failed to replicate this finding using student experimenters \cite{Walach2002}. In the Amanzio {\it et al} study, 30 surgical patients were enrolled in either a natural history condition, a double-blind condition or a deceptive administration condition. The outcome variable was the amount of painkillers requested over the recovery period. Each group was significantly different from each other, and the effect size from the ANOVA was f=0.92. Note that this effect size is for the difference between groups, and was computed assuming that the study possessed 80\% power. 

 These research findings argue in favour of  suggestion (or alternatively, expectancy manipulation) with regards to placebo being one of the major factors in driving the effect. Additionally, other authors have suggested that suggestion and placebo have much in common, and the lack of linkages between them may be due to lack of clarity in definitions \cite{DePascalis2002}. 

% One experimental study which shows the subtle effects of suggestion was the work of Levine \textit{et al}  on motion sickness \cite{Levine2006}.  In this study, participants were told that one placebo pill would reduce motion sickness (placebo suggestion) while the other would reduce spinning, but would increase the other effects (nocebo suggestion). Contrary to the hypothesis, participants in the nocebo condition showed the greatest reduction in symptoms. This was probably because of the caveat in the nocebo suggestion that spinning was felt by others to be the worst symptom. This example illustrates the need to be extremely careful when giving suggestions, lest an opposite result be obtained. % This point is further discussed in Chapter \ref{cha:primary-research}, and in Chapter \ref{cha:general-discussion}. 

Furthermore, some research in hypnosis indicates that even the features associated with hypnosis (lack of memory, lack of volition etc) are themselves the result of personal and cultural suggestions \cite{Kirsch1999}. This does seem to suggest that hypnosis and its effects are just as subject to suggestion as any other interpersonal phenomenon.  It may be that placebo effects are merely the result of suggestions (conscious or unconscious) which are given in the domain of health, while hypnosis merely refers to suggestions given in the context of hypnotic treatment or entertainment. 

Additionally, there is some meta-analytic evidence that in RCT's at least, expectancies are not always associated with improvement~\cite{DiBlasi2001}. This systematic review found that in only ten of the nineteen studies meeting the inclusion criteria were greater expectancies associated with larger placebo effects. The same author also found that many participants in clinical trials did not consider themselves to have expectancies around improvement at all, instead commenting about their hope. 

It is worth noting however, that the expectancy theory is a psychological construct mainly of interest to researchers in the field, and even if expectancies explain the entire variance in placebo response, then it would still not be necessary for participants to think about their hopes or fears in terms of expectancy in order for this construct to be  useful  in describing their response to treatment in clinical or experimental studies. 

% Suggestion also appears to be able to override conditioning in some situations \cite{Benedetti2008}. In this study, after pre-conditioning with ketrolac (a non-opioid painkiller), analgesia could be induced with positive suggestions, while if negative suggestions were given, then they were able to override the prior conditioning. This may be an effect of salience asymmetry \cite{Rothermund2004}, where negative stimuli (in this case, suggestion) are more salient, or it could be related to the non-opioid nature of the conditioning. A contradictory finding, discussed above, was that in a trial of placebo analgesia, conditioned participants demonstrated a greater response to placebo following neutral, rather than positive suggestions \cite{Klinger2007a}. 

However, in light of the research reviewed above, and following Stewart-Williams, it would seem most useful to conceptualise suggestion as a means through which particular sets of expectancies are generated, and this approach was taken throughout this thesis. 

In conclusion, expectancies are regarded as being an important factor in the response to placebo, and expectancy manipulations are used in many studies to induce placebo responses in participants. However, there is some research that suggests expectancies are not always important \cite{Geers2005a} (c.f. section below), and the measurement of expectancies is currently quite crude. It was this finding (discussed further in Chapter~\ref{cha:tcq-thesis}) that lead to this thesis focusing attention on the development of a more comprehensive measure. 

% In conclusion, the conceptual landscape around the response expectancy construct is quite confused, with elements of this construct being conflated with self-efficacy (through outcome expectancies), optimism (c.f. Chapter \ref{cha:health-for-thesis}) and suggestion (see above). This thesis contends that this confusion will not be resolved until after better measures of expectancies are developed, which will allow for a more accurate estimation of what unique explanatory power is possessed by this construct. 

\subsection{Motivational Theories}
\label{sec:behav-plac-motiv}

A competing perspective on the placebo has been advanced recently by Michael Hyland \cite{Hyland2006}. Hyland's theory is called motivational concordance, and it regards the behaviours which people engage in and the meanings that they attach to these as primary,  rather than the cognitive focus of response expectancy theory.  

His research seems to show that depending on how a particular therapy is framed, different variables can predict the placebo response.  In the study cited above, spirituality predicted the placebo response to Bach herbal essences, while expectancy was not an independent predictor. In further research \cite{Hyland2007} he established that this is only the case when flower essences were framed as a spiritual treatment, and not when they were described as motivational tools. These findings argue against the contention of Kirsch that expectancies mediate the effects of placebo on the body directly, or at least suggest that particular forms of expectancies are more effective than others. 

He also noted that when a placebo sleep therapy (which involved writing down things which participants were grateful for) was utilised, gratitude was the best predictor, and again, expectancy added nothing to the results (a finding also noted by Geers \cite{Geers2005}). 

These findings are quite interesting, as they imply that although (explicit) expectancies may provide a useful framework through which to understand placebo effect, they are not a sole explanation. 


One issue with the theory of Hyland is that it has not undergone extensive testing, and has never been examined under double-blind conditions, so it is unknown at this point to what extent it will generalise across the various conditions of placebo administration. For example, his results could be due to experimenter effects and demand characteristics. This is  unlikely as interactions between the researcher and the participants were minimised, but the possibility has yet to be examined. 

Another recent theory regarding placebo effects is that of Geers \cite{Geers2005a}. Geers et al notes that  motivational approaches to placebos were popular in the past.  He suggests that this perspective may prove fruitful for an analysis of placebo effects.   His research used priming techniques in order to influence the desires of  participants to respond to the treatment. 

The major finding of Geers \textit{et al} across a number of studies \cite{Geers2007,Geers2005a} was that placebo effects were significantly greater after participants had been primed with cooperative goals. His research also showed that expectancies had an impact, but again it was not independently significant after motivation was controlled for in a step-wise multiple regression procedure.\footnote{But such an approach is prone to over-fitting.} Effects of motivation were also demonstrated by Jenson and Karoly, \cite{Jensen1991}. The Jensen \textit{et al }   research found that motivation was a predictor of placebo response while expectancies were not, while the Geers \textit{et al } studies found that motivation and expectancies interacted to produce the observed effects.

Some neurological research does suggest that goal directed behaviour may be associated with endogenous dopamine release \cite{Scott2007a} which could provide a plausible mechanism through which  goals and motivation help to activate placebo effects. Some other evidence that would support the theory of Geers is that patients suffering from illness typically show much greater placebo responses to experimental pain than do healthy controls \cite{Klinger2007a}. The research of Geers \textit{et al } used various different priming manipulations to increase motivation to respond, which also suggests that implicit (or unconscious) motivations may be able to influence the response to placebo. 

% \subsection{Other theories of placebo}
% \label{sec:other-theor-plac}

% Some researchers, arguing from an anthropological perspective   have claimed that all of the current theories are far too cognitively focused, and argue for a conception of the placebo response as residing in embodied experience, rather than the constructs used to describe it currently. 

% This seems like an interesting hypothesis, but has not undergone much testing. It does however, fit with recent conceptions of cognition as embodied \cite{wilson2002six}, especially the conception that the body is intimately involved in cognition.  

% An experimental study \cite{Geers2006} demonstrated increased placebo responses when participants were asked to attend to bodily symptoms, which suggests some role for somatic awareness in the effect. This theory would also fit nicely with the recent meta-analysis \cite{Meissner2007} which noted large placebo effects on peripheral outcome parameters in organs, but very little in hormones. 

% This would fit the data as there are feedback mechanisms from organs to brain (through the central and peripheral nervous system), but the hormone system does not have as immediate feedback links to the parts of the brain involved in placebo effects, which would suggest that this theory deserves some credence. This theory is developed further and the implications expanded upon in Chapter \ref{cha:notes-towards-theory}.

% A final theory concerning the placebo effect is the framing of Daniel Moerman \cite{Moerman2000a,Moerman2003}, who conceptualises the effect as a meaning response, which is a useful idea as it brings awareness to the important intra-individual factors which underlie the observed placebo effects. 

% However, research designs and ways of distinguishing meaning from expectancies are sorely lacking, so at present this theory is little more than a clever name change for the same old effect.  This theory  could be tested by requiring participants in clinical trials to keep diaries of their experience and analysing them using a structured approach to determine the individual construction of experience which presumably underlies the construct of meaning. 


\section{Moderators of the Placebo Effect}
\label{sec:moder-plac-effect}

Moving on from theories about the nature of the effect, the next step is to examine factors which can moderate the placebo responses observed in research and clinical practice, covering both experimental data and the results of large meta-analyses. 


This section will address potential moderators of the effect.
These moderators will be divided into factors inherent to the participant, factors relating to the health care provider, and factors relating to the nature of the placebo and the study design. This review of placebo moderators is focused on individual-difference variables as this was the focus of this research. Note that optimism and its relationship to placebo is covered in Chapter \ref{cha:health-for-thesis}.


\subsection{Certain and Uncertain Expectations}
\label{sec:cert-uncert-expect}

Perhaps the most important feature of trial design affecting the response to placebo  is the influence of suggestion/expectancies. In most clinical trials, participants are informed that they will receive either active treatment or placebo. Some authors have suggested \cite{kirsch1988double}  that this process diminishes expectancies related to the treatment efficacy, which in turn reduces the effects \cite{Kleijnen1994}. The Kirsch study noted above looked at the effects of differing instructions on the results of ingesting placebo caffeine, and showed larger effects when participants were given placebo coffee with suggestions that it was real than when they were told there was a 50\% chance they would receive placebo. A replication attempt by Walach \textit{et al} did not confirm this finding, although this study was focused on experimenter effects on placebo, and was not powered to detect this difference. 

More recently, Amanzio and colleagues~\cite{Amanzio2001} replicated this finding with patients recovering from thoracic surgery. The principal finding was that those patients who believed they were getting a real medicine required much less analgesia than those who believed that they might receive placebo. Such an effect could account for the differences in effect sizes seen between experimental and clinical studies of placebo. Indeed, the results of  Amanzio~\textit{et al} suggest that variations in placebo response are responsible for much of the variability in the response to analgesics in general.

A second, related factor may be the use of suggestions in the experimental research. Participants are typically told that they will receive a powerful painkiller before placebo administration, whereas in the clinical trial, no such instructions are given. This finding regarding certain and uncertain expectations was also replicated in a test of a placebo sleep therapy by Geers~\cite{Geers2005a}.

Linking to the discussion above regarding certain and uncertain expectations, it may also be important to examine a paper by Ploghaus {\it et al\/} \cite{Ploghaus2003} where the authors argue that certain expectations of aversive events are associated with fear, while uncertain expectations are associated with anxiety. Anxiety is associated with both the nocebo effect and the production of the hormone CCK, which may be why uncertain expectancies appear to lead to lower placebo effects \cite{Colloca2008b}. These two emotions activate differing parts of the brain, and given the finding that dopamine systems are activated differentially by certain and uncertain expectancies \cite{Scott2007a}, this may point towards some important future avenues for research. This new focus on the brain and body correlates of placebo effects has contributed much to the field, as we will see below in section \ref{sec:neur-plac-effect} .

\subsection{Treatment Preference}
\label{sec:treatment-preference}


There is some evidence that benefits accruing from clinical trials may result from the patients expectancies about whether or not they have received the real treatment \cite{Bausell2005}. This study showed no difference between sham and real acupuncture but showed large differences between the outcomes of those who believed they received real treatment versus those who did not. This factor is typically ignored in clinical trials, although prominent commentators have argued that it should be taken more into account \cite{Benedetti2007}. This finding was later replicated \cite{Linde2007}  where four clinical trials of acupuncture were pooled. This study suggested that although real acupuncture showed similar improvements regardless of expectancies, the minimal acupuncture groups improvement was dependent on their expectancies around acupuncture and perceived treatment assignment. In another trial, belief in receiving real nicotine replacement therapy was an extremely good predictor of successfully stopping smoking \cite{Benedetti2008}. 

These results do suggest that pre-existing expectancies and treatment preferences and beliefs about assignment are an important moderator of the observed placebo responses. These results would seem to argue that placebo controlled trials are not valid unless these treatment beliefs are controlled for, as they could easily bias the outcome of a study. 

\subsection{Price of placebo}
\label{sec:price}

Another factor which may affect the placebo response is price~\cite{Shiv2005a}. A Shiv \textit{et al} study utilised an energy drink distributed to college students in an on campus gym under two conditions. In the first, they merely received the drink and were asked to solve a number of puzzles. In the second, they received the same drink, but were told that the price had been discounted (without being given a reason). The participants in the second condition solved significantly less puzzles than those in the first, suggesting an impact of perceived price on the effectiveness of the energy drink. 

This finding has been replicated in placebo analgesia~\cite{Waber2008}, which is more relevant to this thesis. This is extremely topical, given the nature of the patenting process on pharmaceutical drugs and the proliferation of generic drugs following the expiry of the original patent. This finding probably reflects cultural associations of price with value, and one could hypothesise that in other cultures, items perceived as being of greater value would invoke similar effects.  

It is worth noting that in neither study were the participants actually required to pay for the drugs, and as such, inferences cannot be made directly regarding the real world effects of these results. Such a study would have much greater external validity and relevance to health care policy makers. 

This research also ties into the classic paper by Branthwaite on branded and unbranded pills for the treatment of headaches \cite{Branthwaite1981}. This experimental study, using an extremely large sample, found that branded placebos were more effective than unbranded placebos, suggesting that either advertising or prior learning can affect the effectiveness of two identical preparations. 

Again, the findings discussed in this section can be interpreted in terms of  expectancies. Price is typically taken as a signal for quality in Western societies, and particular brands of pharmaceuticals and medicines can be associated with relief. That being said, the branding experiments are equally conducive to being explained in terms of conditioning, while the price findings are certainly expectancy driven. Older research did not find any effects of familiarity, which would suggest that the effects of branding have either increased  in recent times or that familiarity alone is not enough \cite{Morris1974}.  


\subsection{Patient/Participant Characteristics}
\label{sec:psych-char}

The next  kind of moderators of placebo which will be reviewed are psychological characteristics of the individuals under study. Both state and trait variables may be involved here, though most of the research has focused on traits, as they tend to be easier to measure. 

\subsubsection{Somatic Focus}

Somatic focus, or the focus on internal bodily sensations (the proprioceptive sense) appears to have an impact on the response to placebo, though this finding has only been demonstrated in a small number of studies. 

This finding arises from the work of Geers \textit{et al}~\cite{Geers2006}  on somatic focus and its effect on the placebo response. In summary, this experimental study asked half the participants to attend to their somatic sensations following placebo administration, and gave the other half no such instructions. A similar finding was made by Rainville \textit{et al} with regard to hypnotic suggestions~\cite{Price2008}. 

The participants who focused on their bodily sensations showed an increased placebo effect, which is an interesting finding for many reasons. Firstly, it suggests that the effectiveness of a treatment can be increased by asking participants to pay attention.% \footnote{this may also be a potential mechanism through which MBSR exerts its beneficial health effects}
and this notion of somatic focus is one of the ways in which mindfulness is typically conceptualised. 

Secondly, it links in with an explanation given for differences in placebo response across treatments following a meta-analytic review \cite{Meissner2007} where the authors provide evidence that placebo effects are not common where the outcome measure is a hormone level, while they are common where the outcome measure is a peripheral disease parameter. They suggest that this occurs because nervous system feedback loops are available for the second kind of outcome, but not for the first. This finding is discussed further in Chapter \ref{cha:methodology}, particularly Section \ref{sec:embod-cogn-plac}. 

\subsection{Provider Factors}

Another factor which is believed to be of importance in placebo effects is the patient-provider relationship. The classic study in this field was performed by Thomas~\cite{Thomas1994}  trial where  patients suffering from unclear symptoms were given either a positive or negative consultation.  The results of this study showed that 2/3rds of the patients given a positive consultation improved, while only 1/3rd of the patients given a neutral consultation had. This study has been called into question by more recent research~\cite{Knipschild2005}, but there is some meta-analytic evidence that provider effects may account for a significant portion of the response to placebo~\cite{DiBlasi2001}.



In conclusion,the placebo response is a complex phenomenon and can be impacted by internal  participant factors, features of the patient-provider relationship, features of the treatment itself, and also features of the setting in which the treatment is administered. Very few studies control all of these factors, and this may contribute to some of the confusion and controversy surrounding the construct. 

The above review of the  research relating to features associated with response to placebo has established the following. 
\begin{itemize}
\item Placebo effects appear to be related to expectancies and through these, optimism (cf. Chapter~\ref{cha:health-for-thesis});
\item The perceived value of a placebo (as expressed through price and branding) effects the response (potentially through expectancies);
\item The relationship between patient and provider appears to drive some of the observed placebo effects (also potentially through expectancies);
\item Treatment preference and beliefs about assignment to treatment appear to be as important as is actual treatment assignment (c.f. Section~\ref{sec:beli-about-treatm}).
\end{itemize}

\section{Effect Sizes for Placebo Studies}
\label{sec:effect-sizes-placebo}
The placebo is only amenable to prediction if it can be induced in a given sample. One of the critical requirements for inducing placebo effects is therefore to have some estimate of the size of the effects so that appropriately powered studies can be designed. 

One well known meta-analysis suggested that the benefits of placebo were negligible in most areas \cite{hrobjartsson2001}, with the exception of pain trials. While this meta-analysis has been critiqued for its ignorance of psychological studies of placebo \cite{Evans2003,Stewart-Williams2004b}  and for the combining of placebo effects across 200 plus treatments \cite{Wickramasekera2001}, there is no denying its large effects on the field. 

One of the major reasons for the popularity of pain studies in placebo research is probably the large effect sizes, as measured by Cohen's d. While effects in some areas range from about $d=0.15-0.25$, the effect sizes in pain studies tend to be much larger, ranging from $d=0.45-0.95$~\cite{Vase2002}. Given that the $d$ measure  expresses effect sizes in terms of standard deviations, an effect of between a half and one standard deviation is quite respectable, and allows for smaller studies to examine effects of interest.  However, it has been argued that these effect sizes are illusory, and result from lack of blinding, inadequate controls and poor randomisation~\cite{hrobjartsson2003,Kienle1997}. 



These two viewpoints can be reconciled, at least in the opinion of Vase \textit{et al}~\cite{Vase2002}. In this meta-analysis, Vase and colleagues looked at the sample of placebo pain trials in both the clinical  and  the experimental areas. What they found was that effect sizes tended to be small when the placebo was used in a clinical trial, and much larger in experimental studies of the placebo effect. This analysis was disputed by Hrobjarrtson and Goetzche~\cite{hrobjartsson2003} who noted problems with the methods of analysis chosen by Vase et al. Even using the more conservative estimates of Hrobjarrtsson and Goetzche, the effect size from experimental research (d=0.5) is still twice as large as those observed in clinical trials. There are a number of factors which differ in these two contexts which could be responsible for these observed differences.  

Sauro~\textit{et al}, in a review of endogenous opioids and the placebo effect did find a significant difference between the effect sizes in post-operative and experimental pain, with post-operative pain showing an average effect size of $d=0.65, 95~CI(0.37-0.87)$ while the experimental studies showed an average effect size of from $d=0.53, 95~CI(0.02-1.04)$ for shock induced pain, to $d=0.72, 95~CI(0.34-1.16)$ for capsaician induced pain, to $d=1.23, 95~CI(1.00-1.46)$ for ischemic pain~\cite{Sauro2005}, suggesting that ischemic pain is the best way to invoke a substantial placebo effect.


If there is a gap in the literature, it probably results from a paucity of meta-analytic studies on non clinical trials of placebo - with some notable exceptions~\cite{Wampol2007,Vase2002}, but this is a matter that could be easily addressed by future research. It is however, a matter worth addressing, as what little studies we do have indicate that there are a number of major differences between the data revealed by each of these methodologies. 






\section{Physiology of the Placebo Effect}
\label{sec:neur-plac-effect}


The placebo effect is an interesting phenomenon in that it straddles the boundaries of psychological and physical. This section will examine the research demonstrating the effects of placebo on a neurological level, and then examine other physiological impacts and correlates of placebo administration. While there are a large number of recent studies examining the fMRI correlates of placebo response in individuals~\cite{Benedetti2005a}, these were not essential to this research, and as such are not reported here. 


\subsection{Opioids and Placebo}
\label{sec:opiods-placebo}

The biochemical history of placebo begins with Levine~\cite{Levine1978a} and the demonstration that naloxone blocks many placebo pain responses. Induced from this is the notion that placebo pain relief is mediated by the endogenous opioid system. 

This finding has been qualified by research over the past thirty years, suggesting that both opioid and non-opioid systems can be involved in the placebo pain response depending on the the method of inducing placebo responses and the biological system involved~\cite{Amanzio2001,benedetti2003a}. The lasting contribution of this research is that it paved the way for the placebo to come in from the fringes of medical science.

In this area, the work of Benedetti and his colleagues has been instrumental in unveiling the biochemical pathways through which placebos exert their effects, and much of this work is summarised in his book.  It appears that both the opioid and dopaminergic systems are involved in the placebo effect.  Benedetti and colleagues have demonstrated that respiratory depression can be induced by placebo administration~\cite{Benedetti1999a}. 

A further discovery with regard to placebo analgesia is that it can be directed at specific sites in the body~\cite{Benedetti1999}. This study induced expectancies of placebo responses at either the right or the left hand, and demonstrated the expected placebo effects. These effects were completely antagonised by naloxone, which suggests that they were mediated by the endogenous opioid system. 

This finding is interesting as it suggests that the opioid systems can be activated at specific parts of the body, and not just globally as some former theorists have claimed. A more recent finding~\cite{Watson2006} found that perhaps 50\% of participants in a placebo analgesia study generalised a placebo response across both arms, even though cream was only applied to one arm for each person. 
This study would suggest that the placebo analgesia phenomenon is quite malleable and subject to individual interpretation (i.e. moderated by ``descending'' pathways from the brain \cite{Goffaux2007}). 

Further research on the blockade of opioid receptors by naloxone has established that proglumide can be used to increase the size of placebo analgesic effects~\cite{Benedetti1995}. Additionally, CCK, a chemical which tends to produce anxiety in human participants, has been shown to increase the size of the nocebo effect \cite{Benedetti1996}. 

A recent meta-analytic review \cite{Sauro2005} seems to argue that placebo effects in pain are quite large (d=.89) and that naloxone is quite effective in reducing them (d=.55), pointing towards an interpretation of placebo effects in pain being substantially mediated by endogenous opioids. 

% Unfortunately, Sauro~\textit{et al} do not report what kinds of receptors these endogenous opioids bound to, as this was not the primary focus of their meta-analysis. 
 % Note that all $d$ measures above, are Cohen's, and are weighted effect sizes based on a the inverse of the variance in the sample from which the estimates were drawn. 

% The issue of which receptors are implicated in placebo analgesia is important, as they have different effects. Up to eighteen types have been reported, but there are three main types, the mu, kappa and delta receptors. These receptors have differing sites of action, and information regarding which ones are activated by placebo analgesia will presumably improve our understanding of how and when this phenomena is likely to occur.  The anterior cingulate cortex, as discusssed above appears to be involved in placebo analgesia, and is also involved in the opioid system in the brain. This area is rich in mu receptors, which would seem to suggest that this kind of receptor is important in placebo analgesia.

One study which looked at patients suffering from IBS found that naloxone did not reduce the size of placebo effects, which would suggest that these were not opioid mediated \cite{Vase2005}. It remains to be determined why placebo effects in IBS are not opioid mediated, and understanding this may give some insight into the phenomenon. 

% The ACC area of the brain is involved in the brain opioid system, and additionally is also associated with nitrous oxide (see Section \ref{sec:nitr-oxide-plac}). 

% \subsubsection{Dopamine and Placebo}
% \label{sec:dopamine-placebo}

% While Benedetti and colleagues have done much of the research into the opioid system, De La Fuente Fernandez~\cite{DeLaFuente-Fernandez2002} has published a large amount of research looking at the dopaminergic system.It has been observed that the dopamine system activates not just to reward, but rather the expectancy of reward, and that this release varies as a factor of the certainty of the expectancies~\cite{Scott2007}. In one study, the activation of the dopaminergic systems during placebo analgesia was correlated with activity observed during a monetary reward task, suggesting that the mechanisms of reward are a common feature of placebo effects~\cite{Scott2007a}.  This finding also may provide a mechanism through which certain and uncertain expectancies exert their effects (see Section~\ref{sec:cert-uncert-expect}).   

% De La Fuente Fernandez has argued that there is a descending link from the OFC to the Periaqueductal Gray Area (PAG) and from here to the amygdala, and that this link is responsible for the observed placebo effects~\cite{Fuente-Fernandez2002}. These areas (along with the substantia nigra, which has also been linked with placebo response), produce large amounts of dopamine, and this may play a large role in the mediation of placebo effects and the physical body.Additionally, Fuente-Fernandez has argued that all of the parts of the brain that activate during the response to placebo receive dopaminergic projections~\cite{Fuente-Fernandez2002}.  

% Other research has shown that the expectation of receiving a drug triggers large releases of dopamine in the brains of patients with parkinson's disease, and this release of dopamine directly causes improvement~\cite{Pollo2002}. The really interesting question that arises from this research is why, if patients with Parkinson's can release this dopamine when treatment is expected, do their brains not release these levels of dopamine naturally? Additionally, this study also found that the amount of dopamine released was related to the amount of improvement, in a dose-response fashion~\cite{Fuente-Fernandez2002}. 

% This finding should be tempered with other research that indicated that there was no correlation between the amount of dopamine released and the size of the placebo response~\cite{Scott2007a}. % However, correlations only test for linear relations, and many relationships in the body are non-linear, suggesting that this finding may not be particularly robust. 
% Also the Scott \textit{et al} study used quite a small sample, and so these results would benefit from replication with larger samples \cite{vul2009puzzlingly}.

% Also, the expectancies surrounding motor improvement have been found to correlate with actual motor improvement in both real and placebo sub-cranial thalamic stimulation for Parkinson's disease~\cite{Benedetti2004a}. 

% Expectations of receiving caffeine have been associated with dopamine release, which would seem to provide further evidence that dopaminergic systems are involved in the difference in outcomes between certain and uncertain expectations Kassien (2004), cited in \cite{Beauregard2007a}. 

% Additionally, there is some evidence that the mid-brain dopamine cells associated with addiction and reward project on to areas which are involved with motor and emotional function \cite{DeLaFuente-Fernandez2002}.  This ties in with the effects that CCK has on the nocebo effect (as it additionally induces anxiety), and the involvement of motor function again ties in with a conception of the placebo as a much more embodied phenomenon than is currently thought. The implications of these findings will be examined in Chapter \ref{cha:methodology}. 

% It is important to realise that dopamine and opioid systems may interact, and the limbic system (which the PAG is a part of) within the brain appears to be the site where they do so \cite{Fuente-Fernandez2002}. This section of the brain is associated with both mood and movement, and it may be here that the effects of CCK are exerted, along with those of proglumide, either reducing or increasing the size of the placebo response. -

% \subsubsection{Nitrous Oxide and Placebo}
% \label{sec:nitr-oxide-plac}

% It has also been hypothesised \cite{Stefano2001,Fricchione2005} that Nitrous Oxide (NO) is involved in the placebo effect. These authors argue that the placebo effect is similar to the relaxation response, and they present a substantial amount of evidence that links Nitrous Oxide to various health promoting systems in the body. However, all of their evidence is quite circumstantial and no empirical study has tested the involvement of the NO system in the placebo effect.  

% NO does regulate the production of both dopamine and norepiphedrine, and also disinhibits the actions of striatal neurons, which have been associated with placebo effects in Parkinson's and in placebo analgesia \cite{Fricchione2005,Fuente-Fernandez2001}. Additionally, mu opioid receptors are activated by NO synthesis, additionally suggesting that NO may play an important role in the placebo response. 

%  The rostral anterior cingulate cortex has been associated with placebo analgesia in many studies, and this part of the brain does produce large quantities of  NO, which may suggest that nitrous oxide does have some input to placebo analgesia. The findings noted above with regards to the rACC also include that this area of the brain is associated with hypnotic analgesia also. This may be because both placebo and hypnotic analgesia are phenomena of suggestion, and thus fall under the control of the generalised expectancy network described above.  


% One intriguing finding, noted above is that conditioning can affect hormonal and endocrine responses, while expectancies cannot \cite{benedetti2003a}. This finding is important, as hormonal and endocrine responses were those found by Meissner  \textit{et al} to not show significant placebo effects in a review of randomised controlled trials. This would seem to suggest that the only placebo effects which are controlled for in clinical trials are those related to expectancies. 

% \subsubsection{Placebos and the heart}
% \label{sec:placebos-heart}

% Heart rate and patterns across heart beats are altered by opioids, and also by endogenous opioids released during the placebo effect. While heart rate and the placebo have not been studied as extensively in recent times as neural or endocrine correlates of the placebo, there has been some work done, mostly in an experimental setting. 

% One study found that the heart rate was reduced by placebo analgesia, and this response was blocked by naloxone, suggesting that this effect was mediated by endogenous opioids~\cite{Benedetti2008}. Matre and colleagues found no effects of placebo administration on blood pressure or heart rate~\cite{Matre2006a}, but they appear to have conducted a between groups ANOVA, which is an inefficient means of examining the changes evoked by placebo (as these variables would have changed continuously over time, and would probably be best modeled as time series). Indeed, this problem is common in placebo releated research, and the effects of this problem and some possible resolutions are discussed in Chapter~\ref{cha:primary-research}. 

% Additionally, one of the studies investigated earlier showed an effect of placebo administration on blood pressure~\cite{Shiv2005a}, but only when participants were motivated to solve puzzles (which was the task in that study). This particular effect seems more likely as it was examining effects of a energy drink placebo, and increased blood pressure is a side effect of consuming caffeinated drinks. Flaten in a 1999 study, found no effect of placebo on any of the physiological outcome variables studied~\cite{Flaten1999}

% Heart rate variability is one measure which has been examined in some studies. Alasken \textit{et al } (2008) argued that placebo administration would decrease the ratio of high frequency to low frequency waves, and this hypothesis was supported in their experiment~\cite{Aslaksen2008}. 

\subsubsection{Placebo and Skin Conductance}
\label{sec:plac-skin-cond}



Skin conductance is another measure sometimes used in placebo experiments. Some authors have reported no difference between groups on this measure~\cite{Flaten1999}, but these authors also examined differences between groups using an ANOVA method. One extremely interesting study claimed that pain ratings could be derived from the measurement of skin conductance, and that active drugs changed the response patterns, while placebo administration did not~\cite{Fujita2000} (and c.f. Chapter~\ref{cha:primary-research}).

The Crum et al exercise study discussed earlier also showed physiological effects of their exercise ``placebo'', arguing that suggestion and expectancy effects can have sizeable impacts on physiological outcomes~\cite{Crum2007}.

Placebos have been noted to effect hormone levels also, but these placebo responses appear to be completely insensitive to expectancies, and can only be induced by conditioning~\cite{benedetti2003a}. 

These kinds of physiological markers of response to placebo are extremely useful as they can be used to determine if a physiological placebo effect is occurring, or if the change in self rated pain is driven by more cognitive re-appraisals of the situation. If more cognitively driven, it would be expected that these changes would lag the changes in self reported pain, whereas if the placebo were mediated locally then it would be expected that the physiological changes would occur in advance of a reported drop in self rated pain. One cautionary point is that biological mediators and correlates of placebo are not the cause of placebo effects, rather they are examples of them, and while the study of physiological correlates of placebo can help to understand how they are mediated, it will not remove the need for social and psychological explanations for how they occur~\cite{Stewart-Williams2004b}. 


\section{Review of the Placebo Effect}
\label{sec:revi-plac-effect}

In this section, the major themes which have emerged from the literature review of the placebo effect are recapped. There are a number of major points to bear in mind. 

Firstly, the placebo effect is a difficult phenomenon to define precisely, and there is wide variability in what is considered to constitute a placebo effect. Perhaps the best conceptualisation of the effect is that proposed by Goetzche~\cite{Gotzsche1995}, where it is argued that placebo effects should be broken down into three parts: those attributable to the patient-provider interaction, those attributable to the administration of the medicine, and those attributable to the context in which the treatment was delivered, and these three factors have been taken into account in the definition given in Section~\ref{sec:defin-this-thes}. 

The next major theme to emerge from this review is that that there is one theoretical perspective shared by most researchers in the field, that of response expectancies~\cite{Kirsch1997,Kirsch1985}. However, this theory, while still core to the conceptualisation of the effect, has been contextualised by a number of demonstrations that expectancies are not always the best predictor of placebo~\cite{Hyland2006,Geers2005a}. This theory also suffers from the lack of clear terminology and instruments with which to measure expectancies, a deficiency which this research hopes to remedy. 

The major factors relating to context appear to be the setting in which placebo is administered (clinical trial or experimental), along with the participants beliefs about treatment assignment. Another major contextual factor appears to be the rationale given for the placebo's effects.  In addition, provider factors appear to include the level of suggestion given (certain versus uncertain) and the charisma and authority of the provider. Patient characteristics which have been shown to effect the response to placebo include dispositional optimism, somatic focus, and in some situations, gratitude and spirituality.  


% A further theme which emerges from this review is that the various placebo responses observed in clinical and non-clinical populations appear to recruit a number of neurobiological systems, at the very least the opioid and dopamine systems, and potentially the serotonin system also. It also appears that placebo effects can display a high level of specific action at particular parts of the body, and involve both the central and peripheral nervous system. 

The meta-analytic evidence, though conflicting, appears to indicate that placebo effects occur when the outcome variable is under the control of the central nervous system, and do not occur nearly as much in the endocrine system. However, this finding is based on clinical trial data and contradicts the successful conditioning of humans to respond to endocrine placebos~\cite{benedetti2003a}. The size of placebo effects is also a matter of some dispute, and appears to significantly differ as a function of study design, as noted above in Section~\ref{sec:effect-sizes-placebo}. 

In conclusion, the placebo effect is a complex phenomenon which appears to provide a link between the psychological and physiological  experience of the world, and which is associated  with some psychological and physiological variables. 

This review has established the following:

\begin{itemize}
\item The placebo effect seems to occur in many situations, in both experimental and clinical trials;
\item Expectancies and optimism have been connected to the placebo response;
\item As yet, there is no multi-item scale for assessing the treatment-related expectancies associated with placebo response;
\item The placebo effect is typically associated with a lack of awareness relating to the treatments veracity.
\end{itemize}

This thesis contributes a measure of treatment expectancies, and an investigation of both measuring the placebo effect with an implicit measure, and the relative usefulness of both expectancies and optimism to the prediction of this construct. 

\section{The Implicit Association Test}
\label{sec:impl-assoc-test}

A brief introduction to implicit measures in general, and the IAT specifically was given in Chapter~\ref{cha:introduction}. Given that the focus of this work was on the development of an IAT to measure treatment credibility, this section focuses on the psychometric features of the instrument, its predictive validity and the settings in which it has been used. 


\section{Psychometric Analysis of the IAT}
\label{sec:uses-psych-feat}

The method has become very popular, and has been applied to many areas of social psychology, such as attitudes towards fatness \cite{Ahern2008}, towards disability \cite{Pruett2006} and towards smoking \cite{Kahler2007}. However, regardless of popularity, a measure is only useful to the extent that it possesses the following three properties:


\begin{enumerate}
\item Validity - it measures what it purports to measure
\item Reliability - the measurements are consistent across time
\item Predictive power - unless the measure predicts some outcome which were are interested in, it is essentially useless.
\end{enumerate}

In this section, the validity and reliability, along with the predictive capability of the IAT will be examined to determine if this measure is likely to prove useful in this thesis. 

\subsection{Convergent Validity}
\label{sec:convergent-validity}

One of the major elements used to define and assess the validity of a measure is that of convergent validity, which assesses the strength and directionality of relationships between the measure and other constructs which are theoretically related to the measure or construct. In this section, the relationships of the IAT to other explicit measures and implicit measures will be examined. 


\subsubsection{Relationships with Explicit Measures}
\label{sec:relat-with-expl}



IAT scores are typically weakly correlated with explicit measures of similar attitudes. These correlations average ($\bar r=0.39$) \cite{Nosek2005},and so one could be justified as regarding the two as distinct constructs \cite{Nosek2007a} and this is the approach taken by many of the originators and early workers in the field \cite{Greenwald2000,Nosek2007a}. 

Additionally, they tend to reveal stronger associations than explicit measures when the topic is politicised or controversial~\cite{Greenwald2009}, but tend to provide less predictive power in less controversial situations, such as soda choice~\cite{Karpinski2006}.

% Work on self-esteem has suggested that implicit and explicit self-esteem have a synergistic effect, in that participants with high implicit and explicit self esteem show much better resistance to a forced failure task~\cite{Meagher2004}.
% This relationship for the constructs under study in  this thesis was examined in this thesis, to assess whether or not it was more general than self-esteem (see Chapter~\ref{cha:primary-research}). 

It has been proposed that implicit measures should be more resistant to change than explicit measures \cite{Greenwald1995a,Greenwald1998}. Recent research has shown this not to be the case~\cite{Meagher2004,Gschwendner2008}. In these experiments, IAT scores were shown to change much more than explicit measures in response to experimental manipulations. This may suggest that the measure captures state rather than trait variance, a position which will be discussed further throughout this thesis. 

Dasgupta~\cite{Dasgupta2001} determined that implicit evaluations could be significantly changed by displaying either positive or negative exemplars of the construct under study. Dasgupta \textit{et al} suggested that this may mean that implicit associations are more effected by shallow processing techniques, which is a theory entirely consistent with the notion that they capture state variance. 

One major requirement for convergent validity is that the measure should correlate with other measures known to tap similar constructs. This requirement has been fulfilled for the IAT. One study showed that mindfulness (as measured by the MAAS) was a mediator of the relationship between explicit and implicit attitude scores (in the particular study concerned, attitudes towards autonomy)~\cite{Levesque2007}. Additionally, private self consciousness, an analogous construct was found to also act as a moderator of the relationship between implicit and explicit attitudes~\cite{Gschwendner2006}. Introspection was negatively related to the correlation between implicit and explicit measures~\cite{Hofmann2005}, while spontaneity was positively associated with this correlation. 

Another feature which moderates the relationship between implicit and explicit attitudes is that of attitude importance. As attitudes increase in importance, the correlation between explicit and implicit attitudes increases, suggesting that implicit attitudes tap awareness es that can be, but are not necessarily, conscious~\cite{Karpinski2005}. 

% Additionally, attitude elaboration increases the size of the correlation between implicit and explicit measures \cite{Karpinski2005}, this was demonstrated in a study of soda choice where participants asked to write about their attitudes towards soda showed a higher correlation between their explicit and implicit attitudes towards the topic. 

It does appear that IAT's may predict spontaneous behaviour better than do explicit measures~\cite{Asendorpf2002,Richetin2007,Perugini2005}. This has been demonstrated in a number of domains, and appears to suggest a so-called double-dissociation model, whereby (in a food example) IAT measures predict spontaneous food choice while explicit measures predict food diary measures. 

The double dissociation model has been tested using Structural Equation Modelling~\cite{Nosek2007a,Perugini2005} and appears to provide a better fit than a pure explicit or pure implicit measure. % Again, this form of modelling will be applied in this thesis to examine the measures developed and used in both the preliminary and primary research. 

The form of the explicit measure also appears to affect the relationship between explicit and implicit measures. It appears from a meta-analysis that relative explicit measures tend to have a far greater correlation with implicit measures than do explicit ones \cite{Hofmann2005}. This makes sense given that the IAT is a relative measure. For this reason, the measure developed in Chapter~\ref{cha:tcq-thesis} is not a relative measure, in order to attempt to maximise the incremental variance explained between the explicit and implicit measures. 

% \subsubsection{Relationship with other implicit measures}
% \label{sec:relat-with-other}


% IAT scores tend to reveal similar effects to other techniques such as semantic priming, although IAT's seem to be more sensitive to variations within the construct \cite{Wittenbrink2007a}.

% It does appear that implicit measures may be capturing different facets of the implicit construct under study as some research suggests that two implicit measures of self esteem (the IAT and the self-apperception test) did not correlate \cite{Meagher2004}. However, when all potential confounders were controlled for, measures of implicit self esteem did converge. The major question for future research in this area is to determine why the correlations for measures ostensibly tapping the same construct are so low.

% This lack of relationship between implicit measures of the ``same'' construct is extremely worrying, and suggests that either there is substantial method variance which obscures the relationships, they are actually measuring different constructs or that none (or perhaps only one) of these measures actually has any predictive power. 

% In light of the research described below in Section \ref{sec:pred-valid-iat}, explanation three does not appear to be likely. The different constructs explanation, while comforting to researchers in the field, has little to no evidence in its favour. The first explanation, that of method variance/extraneous factors seems to be the most likely. In light of the relatively recent development of these methods it is quite plausible that many extraneous factors influence the results obtained using these measures, and much more research will be needed to tease out these effects. 


\subsection{Predictive Validity of the IAT}
\label{sec:pred-valid-iat}


The major proposed advantage for the use of implicit measures is that they would act as better predictors of behaviour, or allow for more insight into hidden cognitions that were associated with behaviours yet either not accessible or not reported by participants \cite{Greenwald1998}. This section examines the extent to which these hopes have been fulfilled. While the IAT does not appear to be a better predictor of behaviour overall, it does possess some ability to predict behaviours which are typically hard to predict using self report measures \cite{Greenwald2009}. 

The classic demonstration of the difference in prediction between implicit and explicit measures relates to~\cite{Asendorpf2002} who investigated the attitude of shyness. In this study, spontaneous shy behaviour was predicted by implicit associations, while controlled shy behaviour was predicted by explicit attitudes. This pattern has become known as double dissociation, and has been observed in a number of studies~\cite{Perugini2005}, and  then supported by  a meta-analysis~\cite{Hofmann2005}. 

A recent review of implicit measures of self esteem suggests that implicit and explicit self esteem are entirely distinct constructs~\cite{Rudolph2008}. Implicit self esteem has been shown to predict response to success or failure~\cite{Greenwald2000}. 
The research on implicit and explicit self esteem seems to indicate that individuals can be classified as having particular types of self esteem based on their relative levels of implicit and explicit self esteem, where participants who have high levels of both implicit and explicit self esteem are classified as having genuine self esteem~\cite{Meagher2004}. 

These participants tended to be more resilient and suffer less negative outcomes following a false feedback manipulation designed to reduce self esteem. Although the effect size for the explicit measures was much higher in this study, the implicit measure (the Self Apperception Test and the IAT) appeared to be more sensitive to the emotional tone of the feedback. 


In the domain of personality, implicit measures of all Big 5 traits have been correlated with spontaneous behaviour which reflected these traits \cite{Steffens2006}. 
The Steffens \textit{et al} study predicted behaviour based on the Big 5 personality traits. There was no effect for extraversion, but the experimental manipulation may have been poorly designed. This findings lends more credence to the double dissociation theory \cite{Asendorpf2002} which has been shown to apply in a wide variety of tasks since then \cite{Perugini2005,Conner2005}. 

% In another study, Extraversion and Neuroticism IAT's were found to converge with the Extraversion and Conscientiousness traits as assessed by the NEO-FFI~\cite{Grumm2007}.%  However, another recent article, which attempte
% d to assess trait anxiousness and angriness, found that there was a significant interaction effect between the order of administration of the IAT's \cite{Schnabel2006}. 

% When the angriness IAT was administered first, the results of the anxiousness IAT were highly correlated. The converse, however, did not occur. The authors suggest that this may be because the participants applied a coding strategy to the first IAT which they then generalised to the second. This problem may also have arisen because Schnabel \textit{et al} administered the two IAT's directly after one another, rather than with other tasks in between as was done in the Grumm \textit{et al} paper. 

One paper~\cite{Boldero2007} which used the Go/No Go Association Test (GNAT) showed that implicit Extraversion and Neuroticism were able to predict reaction time in the experiment. More generally, the implicit attitudes were able to predict scores on the explicit attitude measure. 

However, the Big 5 traits predicted were different from the findings of other researchers, suggesting that there may be some method variance involved. This method variance has also caused problems for the IAT \cite{Mierke2003,Greenwald2003a}. Again, this highlights the issue that these new implicit measures may have confounding factors within them which can only be highlighted by further detailed research.

% It does appear that implicit personality measurement can predict spontaneous behaviours related to  personality traits \cite{Steffens2006}. 


Gender IAT scores have been shown to correlate with ratings of prejudiced behaviour in a simulated job interview task~\cite{Rudman1999a}. Additionally, poor interactions with an opposite race experimenter have been associated with high scores on the Race IAT~\cite{McConnell2001}. 
There is some evidence that IAT measured preference for males over females  can result in prejudicial behaviour against females in a simulated interview setting~\cite{Greenwald2000,Heider2007}. The magnitude of the IAT scores was correlated with the observer-reported prejudicial behaviour scores. 

Some have critiqued these findings as both IAT and behavioural assessments were carried out in the same session, and this may falsely inflate the attitude-behaviour correlation. Another study \cite{McConnell2001} showed substantial correlations between IAT assessed bias and ratings of friendliness given to each participant in a scripted interaction. More recent research has demonstrated that even when separated by a week, attitude assessments using the IAT are significant predictors of verbal and non verbal friendliness with a compatriot of opposite race \cite{Heider2007}. 

% There is some evidence that IAT's can predict spontaneous behaviour better than explicit measures \cite{Conner2005,Perugini2005,Grumm2007} ,In the Conner \textit{et al} (2005) study, using an experience sampling methodology, the IAT measured attitudes predicted how the participants felt on a day to day basis far better than the explicit (self-report) measures, but the explicit measures predicted global ratings better. 

% The extra predictive power afforded by the IAT  was only demonstrated for negative affect,  while the explicit measures were equally predictive for both positive and negative affect.  This is an interesting finding, as it suggests that there are differences in the effects of implicit attitudes on emotion dependent on the valence. As this pattern was not observed with the explicit measures, it may be an avenue for future research aimed at determining how implicit associations predict experience and behaviour. 

A further study on explicit moderators of the predictive validity of the IAT showed that participants scores on the Self Reported Habit Index moderated the predictive validity of the IAT \cite{Conner2007}. This finding, along with the relationship between mindfulness, introspection and other similar constructs and IAT scores provides evidence that IAT scores are differentially predictive based on moderator variables. 



The predictive validity of the IAT may also vary as a function of domain. An Italian study \cite{Arcuri2008} showed that the IAT was able to predict the future voting behaviour of people based on the results of an IAT measuring attitudes towards left and right wing candidates. This finding is quite impressive, given the difficulties political scientists normally find in predicting the behaviour of undecided voters. However, this should be qualified with the fact that it was self reported voting behaviour was measured, as opposed to actual voting patterns. Although the Arcuri findings are interesting, other research on  the relationship between IAT scores and voting behaviour showed that IAT scores had no incremental validity over explicit measures in prediction of vote choice \cite{Karpinski2005}. The Karpinski \textit{et al } study did not focus on undecided voters, however, which may be responsible for the differences in the results. 

% Another study showed that 57\% of the variance in course choices at University could be predicted on the basis of IAT measured preference for mathematics versus the arts \cite{Perugini2007}. However, this study suffered from a number of flaws, most prominent amongst them that the students had been enrolled in the college for one year at the time of study. This would suggest that the results could easily have occurred as a result of their enrollment in the course, rather than acting as a cause of this. Noentheless, this finding would be extremely easy to replicate correctly, and one can hope that this replication will be carried out. 

Other research has indicated that IAT scores are more predictive when participants are under strain or otherwise tasked for resources \cite{Hofmann2008a} These findings could suggest that IAT scores typically have quite large amounts of state variance if they are malleable in terms of experimental manipulations such as this. An alternative theory is that the IAT responses can be inhibited by awareness, which causes the differences observed in the studies reported above.  % Additionally, IAT scores have been found to increase their predictive validity under conditions of self-activation \cite{Perugini2007}. 

% Linking to the notion that the predictive power of the IAT varies across domains is a recent meta-analysis \cite{Greenwald2009} which compared the predictive validity of explicit and implicit measures across a large number of domains. In general, explicit measures appeared to be more predictive, except in the cases of inter-group behaviour and race attitudes. While this meta-analysis appeared to use some unclear selection criteria for studies, in general it was well conducted and the results show that the IAT has good predictive validity, especially in domains where there are social or awareness difficulties with the use of explicit measures.  

A study which demonstrated a link between chronic pain and implicit measures of pain and identification with pain was reported in Grumm \textit{et al} \cite{Grumm2008}. This study followed a group of chronic pain patients as they took part in a mindfulness-based intervention for pain reduction. The use of pre-post design allowed for the authors to assess that the intervention caused a significant drop both in self reported pain and medication use, and a reduction in self-pain associations as measured by an IAT. This study is one of the first to map an IAT to a health-related construct, and to actually take behavioural measures into account. 

To conclude, there is substantial evidence for the predictive power of the IAT across a number of disparate domains, but there is much more work to be done given that predictive power seems to be quite malleable across experimental studies and their is some conflicting evidence, especially in the area of political attitudes. 

\subsection{Reliability of the IAT}
\label{sec:reliability-iat}

The reliability of the IAT has been a major issue throughout its existence. Test-retest seems to average around ($r=.49$) which, while permissible in a psychometric instrument for theoretical purposes, is far too low for making clinical or legal judgments~\cite{Greenwald2000, Blanton2006d}. It is worth noting however, that the test-retest reliabilities do not drop much farther than this, even over periods as long as one year~\cite{Egloff2005}. 
Split-half reliabilities are typically higher, averaging around ($r=0.80$),  

This may suggest that the IAT response is composed of both state and trait portions, and that the trait portion of the measure is relatively invariant across temporal distance. Some authors have argued that this low test-retest reliability is due to both error variance and person by situation interactions~\cite{Gschwendner2008}.  Manipulation of accessibility of the constructs measured in the IAT has been shown to improve the temporal stability of the IAT scores, suggesting that particular situations may make the constructs more or less likely to be expressed~\cite{Gschwendner2008}. 

% Additionally, some research has suggested that the picture of implicit attitudes as developed early in life and resistant to change, may be incorrect \cite{Gschwendner2008}. % This, and the evidence given below in Section \ref{sec:iat-contr-faking} may point towards implicit measures as being more state based than early theory surrounding dual-process models would suggest. 




\section{Moderators of the IAT}
\label{sec:moderators-iat}



\subsection{Contextual Moderators of IAT Effects}
\label{sec:cont-moder-iat}



Some contextual factors have been noted to affect IAT measures, even though these were some of the problems which the measure were designed to avoid. A study asking participants to complete the IAT under conditions when they believed that the experimenter would or would not know their scores (the so-called bogus pipeline) \cite{Boysen2006} showed a notable diminution in IAT effects in a measure of attitudes towards homosexuals when participants believed that the experimenter would know their scores. % Further investigations revealed that this did not arise because of social desirability issues as the effects were similar under a bogus pipeline condition.
Another study examining the attitudes of Italian students to Turkish immigrants replicated this finding, with IAT scores being reduced when in the presence of others \cite{Castelli2008}. 

\subsection{Cognitive Moderators of IAT effects}
\label{sec:cogn-moder-iat}



Another factor which appears to moderate the observed IAT effects is that of memory resources. A study looking at attitudes towards Blacks and Turks \cite{Hofmann2008a} found that the IAT acted as a far better predictor of behaviour when participants had been asked to remember a list of words than when they were untaxed. This finding was replicated by the same authors using a different sample and IAT, suggesting that it is quite robust.  This suggests that the attitudes measured by the IAT are the result of more automatic processes, and will have predictive power to the extent that the matter involved is not the subject of deep processing \cite{Kahneman2002}. 

The Hofmann (2008) study described above involved interaction with an experimenter of the out-group measured in the IAT, and the second study separated the IAT from the behaviour assessment by one week, so these results are both good measures of behaviour and unaffected by  issues of attitude-behaviour consistency . 

% One recent study \cite{Perugini2007} showed that the predictive ability of the IAT increases when under conditions of self activation. This was operationalised in these experiments by asking participants to either circle self or non-self related words in a passage of text. This finding was replicated across four domains and was found to be valid in all of them. In one case the correlation between IAT and behaviour was raised from .36 to .76 under conditions of self activation, which is a huge change. 

Some research \cite{Dasgupta2001} suggests that showing exemplars of groups typically the subject of negative associations on the IAT measures (such as Black Americans or Females) can reduce the size of these associations. However, as described below, difficulty in recalling such exemplars can lead to larger IAT effects. 

It also appears that IAT effects are influenced by ease of retrieval mechanisms \cite{Kahneman2002} . This extremely well conducted study examined a number of implicit measures and the mechanisms through which they are influenced by context \cite{Gawronski2005}. 

Gawronski \textit{et al} class the IAT as a response compatibility measure, and argue that these measures are affected by ease of retrieval from memory. In support of this, participants who generally liked African Americans showed higher levels of implicit preference against this group when asked to generate a high number of either liked or disliked African Americans. Conversely, participants who generally disliked African Americans showed lower levels of prejudice when they generated a lower number of exemplars. The authors explain this effect in terms of ease of retrieval. The subjective difficulty of generating the exemplars seems to alter the attitudes which the participants report~\cite{Kahneman2002}.


% This is probably affected by attitude-behaviour consistency effects, so it would interesting to examine whether or not these changes remain stable over time.  

% The IAT also appears to be affected by attitude importance \cite{Karpinski2005}. In this study, the authors demonstrated that both scores on a Republican/Democrat IAT and a Coke/Pepsi IAT became more predictive of self reported behaviour as attitude importance increased. This would suggest that the measure is more useful in matters where the participants have a investment into the attitude being measured. However, this study used two domains where explicit measures are normally better predictors than implicit measures \cite{Nosek2007d}, so the findings here should be treated with some caution. 

Another study \cite{Levesque2007} looked at the effects of mindfulness on expression of implicit attitudes and argued that high mindfulness can stop implicit attitudes from being expressed and over time, causes them to become more in tune with self reported attitudes. This study used an experience sampling methodology and examined attitudes towards autonomy and heteronomy. The findings suggested that participants high in mindfulness seemed to show higher levels of autonomy in general, and that mindfulness could act as a protective factor against the expression of unwanted implicit attitudes.  This finding has also been supported by other recent research \cite{Gschwendner2006}, when they noted that Private Self Consciousness seemed to correlate with the expression of implicit attitudes towards Germans and Turks. In addition, the self reported habit scale, a measure which has been found to be negatively correlated with mindfulness, was found to be associated with stronger implicit attitudes and less congruence between explicit and implicit attitudes \cite{Conner2007}. 





\section{Conclusions}
\label{sec:conclusions}


In summation, the Implicit Association Test is a new and useful measure. It has been embroiled in a number of spirited debates since its publication, and much progress has been made in the understanding of the measure as a result of this. 

The measure can reliably differentiate between groups possessing different attitudes, and seems to produce results which while correlating somewhat with explicit attitudes, appear to reflect a different underlying process~\cite{Nosek2007a}. 

It can predict behaviour quite well in certain situations, especially in matters of importance to the participants and when they are working with limited cognitive resources. It appears to predict spontaneous behaviour much better than explicit measures, which is certainly of interest. This prediction of spontaneous behaviour, coupled with low test-retest reliabilities, would seem to suggest that there are major state components to the measure. 

However, the measure still has its problems. There is still no generally accepted theoretical rationale for its effects, it can be contaminated by such issues as task switching costs\cite{Klauer2005}, processing speed\cite{Blanton2006} and the context in which it is administered \cite{Boysen2006}. 

The D algorithm seems to do a good job of controlling for task-switching costs. Secondly, valence and familiarity need to be controlled for in order to avoid contaminations with general processing speed and salience asymmetries. Thirdly, the measure does not have a theoretical foundation, and this is something that needs to be addressed. 

One can argue that if it works then we should use it, but a cogent theoretical account of the IAT would allow for more detail and precision in experimental design  than is the case at present, and for this reason such a development is essential if the IAT is to become a permanent part of research in experimental psychology.



\section{Placebo: Implicit and Explicit Expectancies}
\label{sec:measurement-placebo}



% \section{Implicit Measures and Placebo}
% \label{sec:impl-meas-plac}

So, having reviewed the placebo effect and the implicit association test above, we are now in a position to examine the main experimental hypothesis of this thesis: that the placebo effect can be predicted by a combination of self reported and implicitly measured (explicit and implicit expectancies). Below, the reasons for believing this shall be set out, and in the next section  research evidence shall be presented which supports this original hypothesis. 

Individual level predictors of the placebo response have proved elusive in experimental work for over 50 years. Many psychological variables have been tested to see if they can predict the response, but almost all have failed to replicate~\cite{Shapiro1997}. Indeed, some argue that placebo responders do not even exist~\cite{Kaptchuk2008a}. My thesis is that the lack of results in the prediction of the placebo effect is an artefact of the methods used rather than the unpredictability of the effect. One of the contentions of this thesis is that the Implicit Association Test is a useful method with which to predict the response. 


\subsection{The research evidence for the combination of the two measures}
\label{sec:rese-base-comb}
% The placebo is a badly defined but widely used concept in medicine \cite{Kaptchuk1998} . Many theories have been put forth to account for its effects, including expectancy, conditioning, emotional change and meaning \cite{Stewart-Williams2004b}. However, none of these theories can account for all of the effects observed in research. The conditioning approach gained a following at first \cite{Voudouris1985}  but was quickly opposed by the response expectancy theory of Kirsch \cite{Kirsch1985,Kirsch1997}.

Research on placebo focusing on either conditioning or expectancy explanations lead to a number of findings which suggest that implicit measures may be a way in which placebo responses can be predicted. In this section, I shall first review the expectancy versus conditioning debate insofar as it relates to my main point, which is to examine whether or not the use of implicit measures of expectancies will allow us to predict the placebo response in individuals with greater accuracy. Then I shall review some other lines of evidence which support my case, and finally sum up the evidence in favour of this hypothesis.

The general format developed by Voudouris and still in use today in the research of Benedetti \cite{Benedetti2006c} and others, consists of three stages. 

Firstly, the participant has their pain thresholds calibrated and is given a number of blocks of painful stimuli. Secondly, the participants are either given the same stimuli again following the application of placebo cream or the pain is reduced for the second stage after the application of the cream. The second group are then said to be conditioned by this stage. In the third block, pain is increased again for the conditioned participants, and they typically show a much larger placebo response than those who are merely given verbal suggestions of analgesia.  

These results were believed by Voudouris to argue in favour of a conditioned approach to the placebo effect. However, an experiment by Montgomery and Kirsch \cite{Montgomery1997} demonstrated that these conditioned effects resulted solely from expectancies,and after a regression in which expectancies were partialled out, there were no significant effects arising from conditioning. This would seem to argue that expectancies are the prime method through which conditioning has an impact, and indeed this is the position of Kirsch. 

Kirsch's \cite{Kirsch1985,Kirsch1997} theory of response expectancies specifies that these are the expectation of a non-volitional response, and he argues that they play a role in placebos and hypnosis. This theory  relies upon the measurement of self report expectancies to determine this, and this seems  to be somewhat incompatible to what happened in the Montgomery and Kirsch research described in the paragraph above. In the experiment that confirmed the effects of expectancies, there were two groups who received lowered stimuli to condition them. One of these groups was informed of this pairing, and the other was not. 

Contrary to the predictions of the conditioned response model, those in the informed pairing group did not show an enhanced placebo effect, while those in the uninformed pairing group did. This seems to indicate that the effects of the conditioning procedure were inhibited by awareness. In other words, what occurred here could be described as an example of implicit learning (learning without conscious awareness).

This conclusion is further reinforced by the work of Shiv and Carmon \cite{Shiv2005a} in the placebo effects of energy drinks given to participants at a lowered price. This research used an outcome measure of the number of problems solved in a specified period, and it demonstrated that those who believed that the price had been discounted solved less puzzles than those who received the drink at the normal price. The interesting feature of this research (for the purposes of this thesis) is that when participants attention was drawn to the discounted price, the difference between the groups was reduced. This would seem to argue in favour of an interpretation involving learning without conscious awareness. 

From the research into the IAT, it appears that the measure captures state variance, and given the irreproducibility of the placebo response across participants and conditions \cite{Whalley2008,Shapiro1997}, this would also seem to be a factor in the placebo response. Secondly, the research on the IAT suggests that it is likely to be more predictive in contexts where participants are depleted of resources - such as when they are sick, or in pain, which are common conditions in which placebo effects are observed. 



So, reviewing these research findings, we can note that in some cases, the placebo effect seems to emerge without conscious awareness. The Shiv \textit{et al} (2005) study noted above showed that when participants awareness was drawn to the discounted price, the effects disappeared. A similar phenomenon occurred in the Montgomery and Kirsch (1997) \cite{Montgomery1997} study. These research findings suggest that the placebo response is at least partially determined by factors outside conscious awareness, and as such, it would make more sense to use implicit measures, of which the IAT is the most prominent. 

% Implicit learning is a phenomenon which has come to the attention of psychologists in the past three decades. The research mostly focused on learning of patterns in random letter fragments and in the effects of sub-threshold sounds and pictures upon participants in a research setting. The investigation of these implicit attitudes was given a huge leap forward by the development of the Implicit Association Test \cite{Greenwald1998}. This computer administered instrument requests the participants to classify words into either trait (race, gender etc) or evaluative (self versus other, pleasant versus unpleasant) categories. This categorisation is performed separately at first, and then one of each type is paired together. This pairing is then reversed in the final step. The test works primarily based on reaction time (in milliseconds) and the assumption underlying the test is that items which are associated are easier to classify together, and that the difference in mean response latencies for classifications of each concept reflect the attitude towards that object. 


One line of evidence which supports this thesis is the finding that the IAT is better at predicting spontaneous behaviour than explicit measures \cite{Conner2005,Hofmann2005}. The IAT outperforms explicit measures in some domains \cite{Greenwald2009} and these domains tend to be where there is little conscious deliberation or reflection upon the matter concerned. The placebo effect is the example par excellence of a un-deliberated and spontaneous  type of phenomenon, as no one chooses to have such a response and it seems dependent upon the lack of awareness of a participant that the treatment which they are receiving is not an active one. Thus, we can take this as supporting evidence that implicit measures should predict the placebo response more effectively. 

% A final line of evidence is suggestive of this link, and it arises from the neurological patterns of activation associated with the placebo effects and the implicit association test. These patterns have been studied greatly over the last decade in placebo research, and the major findings are that a network involving the dorsolateral prefrontal cortex and the rostral anterior cingulate cortex appear to be activated during placebo related cognitions \cite{Mayberg2002,Zubieta2006} (cf Section \ref{sec:neur-plac-effect}). While neurological research into the IAT has not been as prominent, two  studies  \cite{Knutson2007,Knutson2006} has shown that both of these areas are activated during the association task that is the IAT. While this cannot be used to prove that the two phenomena are related, it does suggest that there may be some common neurological substrate underlying the kinds of cognitions involved in both of these effects. 

In conclusion, the evidence reviewed in this chapter seems to show that the expectancies underlying the placebo effect can be measured by means of the Implicit Association Test for the following reasons. 
\begin{itemize}
\item Firstly, some placebo effects seem to require a lack of conscious awareness in order to occur \cite{Shiv2005a,Geers2005a}.
\item  Secondly, the placebo effect seems to be best modelled as a spontaneous phenomenon and the IAT has been shown to predict these kinds of behaviours better~\cite{Asendorpf2002,Richetin2007}.
% \item  Thirdly, the placebo and the IAT task seem to share some common patterns of neurological activation. 
\end{itemize}

For these reasons, the development of an IAT which can measure these expectancies is a worthy contribution to human knowledge. 



%%% Local Variables:
%%% TeX-master: "PlaceboMeasurementByMultipleMethods"
%%% End:




